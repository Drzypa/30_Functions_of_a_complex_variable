\documentclass[12pt]{article}
\usepackage{pmmeta}
\pmcanonicalname{HarnacksPrinciple}
\pmcreated{2013-03-22 14:57:35}
\pmmodified{2013-03-22 14:57:35}
\pmowner{Mathprof}{13753}
\pmmodifier{Mathprof}{13753}
\pmtitle{Harnack's principle}
\pmrecord{14}{36657}
\pmprivacy{1}
\pmauthor{Mathprof}{13753}
\pmtype{Theorem}
\pmcomment{trigger rebuild}
\pmclassification{msc}{30F15}
\pmclassification{msc}{31A05}

% this is the default PlanetMath preamble.  as your knowledge
% of TeX increases, you will probably want to edit this, but
% it should be fine as is for beginners.

% almost certainly you want these
\usepackage{amssymb}
\usepackage{amsmath}
\usepackage{amsfonts}

% used for TeXing text within eps files
%\usepackage{psfrag}
% need this for including graphics (\includegraphics)
%\usepackage{graphicx}
% for neatly defining theorems and propositions
%\usepackage{amsthm}
% making logically defined graphics
%%%\usepackage{xypic}

% there are many more packages, add them here as you need them

% define commands here
\begin{document}
If the functions\, $u_1(z)$, $u_2(z)$, \ldots\, are \PMlinkname{harmonic}{HarmonicFunction} in the domain\, $G \subseteq\mathbb{C}$\, and
    $$u_1(z) \le u_2(z) \le \cdots$$
in every point of $G$, then\, $\lim_{n\to\infty}u_n(z)$\, either is infinite in every point of the domain or it is finite in every point of the domain, in both cases \PMlinkname{uniformly}{UniformConvergence} in each \PMlinkname{closed}{ClosedSet} subdomain of $G$.\, In the latter case, the function\, $u(z) = \lim_{n\to\infty}u_n(z)$\, is harmonic in the domain $G$ (cf. limit function of sequence).
%%%%%
%%%%%
\end{document}
