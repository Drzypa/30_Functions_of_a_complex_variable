\documentclass[12pt]{article}
\usepackage{pmmeta}
\pmcanonicalname{TeichmullerSpace}
\pmcreated{2013-03-22 14:19:48}
\pmmodified{2013-03-22 14:19:48}
\pmowner{jirka}{4157}
\pmmodifier{jirka}{4157}
\pmtitle{Teichm\"uller space}
\pmrecord{8}{35799}
\pmprivacy{1}
\pmauthor{jirka}{4157}
\pmtype{Definition}
\pmcomment{trigger rebuild}
\pmclassification{msc}{30F60}
\pmdefines{Teichm\"uller metric}
\pmdefines{Teichm\"uller equivalence}
\pmdefines{Teichm\"uller equivalent}

\endmetadata

% this is the default PlanetMath preamble.  as your knowledge
% of TeX increases, you will probably want to edit this, but
% it should be fine as is for beginners.

% almost certainly you want these
\usepackage{amssymb}
\usepackage{amsmath}
\usepackage{amsfonts}

% used for TeXing text within eps files
%\usepackage{psfrag}
% need this for including graphics (\includegraphics)
%\usepackage{graphicx}
% for neatly defining theorems and propositions
\usepackage{amsthm}
% making logically defined graphics
%%%\usepackage{xypic}

% there are many more packages, add them here as you need them

% define commands here
\theoremstyle{theorem}
\newtheorem*{thm}{Theorem}
\newtheorem*{lemma}{Lemma}
\newtheorem*{conj}{Conjecture}
\newtheorem*{cor}{Corollary}
\newtheorem*{example}{Example}
\newtheorem*{prop}{Proposition}
\theoremstyle{definition}
\newtheorem*{defn}{Definition}
\theoremstyle{remark}
\newtheorem*{rmk}{Remark}
\begin{document}
\begin{defn}
Let $S_0$ be a Riemann surface.  Consider all pairs $(S,f)$ where
$S$ is a Riemann surface and $f$ is a sense-preserving quasiconformal
mapping of $S_0$ onto $S$.  We say $(S_1,f_1) \sim (S_2,f_2)$ if $f_2 \circ
f_1^{-1}$ is homotopic to a conformal mapping of $S_1$ onto $S_2$.  In this case we say that $(S_1,f_1)$ and $(S_2,f_2)$ are {\em Teichm\"uller equivalent}.  The space of equivalence classes under this relation is called the {\em Teichm\"uller space} $T(S_0)$ and $(S_0,I)$ is called the initial
point of $T(S_0)$.  The equivalence relation is called {\em Teichm\"uller equivalence}.
\end{defn}

\begin{defn}
There exists a natural {\em Teichm\"uller metric} on $T(S_0)$, where the distance
between $(S_1,f_1)$ and $(S_2,f_2)$ is $\log K$ where $K$ is the smallest maximal dilatation of a mapping homotopic to $f_2 \circ f_1^{-1}$.
\end{defn}

There is also a natural isometry between $T(S_0)$ and $T(S_1)$ defined by
a quasiconformal mapping of $S_0$ onto $S_1$.  The mapping
$(S,f) \mapsto (S,f \circ g)$ induces an isometric mapping of $T(S_1)$ onto $T(S_0)$.  So we could think of $T(\cdot)$ as a contravariant functor from
the category of Riemann surfaces with quasiconformal maps to the category of
Teichm\"uller spaces (as a subcategory of metric spaces).

\begin{thebibliography}{9}
\bibitem{ahlfors}
L.\@ V.\@ Ahlfors.  \emph{\PMlinkescapetext{Lectures on Quasiconformal
Mappings}}.  Van Nostrand-Reinhold, Princeton, New Jersey, 1966
\end{thebibliography}
%%%%%
%%%%%
\end{document}
