\documentclass[12pt]{article}
\usepackage{pmmeta}
\pmcanonicalname{ExamplesOfInfiniteProducts}
\pmcreated{2013-03-22 14:02:32}
\pmmodified{2013-03-22 14:02:32}
\pmowner{mathcam}{2727}
\pmmodifier{mathcam}{2727}
\pmtitle{examples of infinite products}
\pmrecord{5}{35391}
\pmprivacy{1}
\pmauthor{mathcam}{2727}
\pmtype{Example}
\pmcomment{trigger rebuild}
\pmclassification{msc}{30E20}
\pmrelated{ComplexTangentAndCotangent}

\usepackage{amssymb}
\usepackage{amsmath}
\usepackage{amsfonts}
\begin{document}
\PMlinkescapeword{connection}
\PMlinkescapeword{decomposition}
A classic example is the Riemann zeta function.
For $\Re(z)>1$ we have
$$\zeta(z)=\sum_{n=1}^\infty\frac{1}{n^z}=
\prod_{p\text{ prime}}\frac{1}{1-p^{-z}}\;.$$
With the help of a Fourier series, or in other ways, one can prove
this infinite product expansion of the sine function:
\begin{equation} \label{eq:sin}
\sin z=z\prod_{n=1}^\infty\left(1-\frac{z^2}{n^2\pi^2}\right)
\end{equation}
where $z$ is an arbitrary complex number.
Taking the logarithmic derivative (a frequent move in connection with
infinite products) we get a decomposition
of the cotangent into partial fractions:
\begin{equation} \label{eq:1}
\pi\cot\pi z=\frac{1}{z}+\sum_{n=1}^\infty
\left(\frac{1}{z+n}+\frac{1}{z-n}\right)\;.
\end{equation}
The equation \eqref{eq:1}, in turn, has some interesting uses, e.g. to get
the Taylor expansion of an Eisenstein series, or to evaluate
$\zeta(2n)$ for positive integers $n$.
%%%%%
%%%%%
\end{document}
