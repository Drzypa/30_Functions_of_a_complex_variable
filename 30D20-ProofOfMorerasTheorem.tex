\documentclass[12pt]{article}
\usepackage{pmmeta}
\pmcanonicalname{ProofOfMorerasTheorem}
\pmcreated{2013-03-22 18:53:34}
\pmmodified{2013-03-22 18:53:34}
\pmowner{Ziosilvio}{18733}
\pmmodifier{Ziosilvio}{18733}
\pmtitle{proof of Morera's theorem}
\pmrecord{10}{41740}
\pmprivacy{1}
\pmauthor{Ziosilvio}{18733}
\pmtype{Proof}
\pmcomment{trigger rebuild}
\pmclassification{msc}{30D20}

% this is the default PlanetMath preamble.  as your knowledge
% of TeX increases, you will probably want to edit this, but
% it should be fine as is for beginners.

% almost certainly you want these
\usepackage{amssymb}
\usepackage{amsmath}
\usepackage{amsfonts}

% used for TeXing text within eps files
%\usepackage{psfrag}
% need this for including graphics (\includegraphics)
%\usepackage{graphicx}
% for neatly defining theorems and propositions
%\usepackage{amsthm}
% making logically defined graphics
%%%\usepackage{xypic}

% there are many more packages, add them here as you need them

% define commands here

\begin{document}
We provide a proof of Morera's theorem
under the hypothesis that
\begin{math}
\int_\Gamma f(z)dz=0
\end{math}
for any circuit $\Gamma$ contained in $G$.
This is apparently more restrictive, but actually equivalent,
to supposing
\begin{math}
\int_{\partial\Delta} f(z)dz=0
\end{math}
for any triangle $\Delta\subseteq G$,
provided that $f$ is continuous in $G$.

The idea is to prove that $f$ has an antiderivative $F$ in $G$.
Then $F$, being holomorphic in $G$,
will have derivatives of any order in $G$;
but $F^{(n)}(z)=f^{(n-1)}(z)$ for all $z\in G$, $n\in\mathbb{N}$, $n\geq 1$.

First, suppose $G$ is connected.
Then $G$, being open, is also pathwise connected.

Fix $z_0\in G$.
For any $z\in G$ define $F(z)$ as
\begin{equation} \label{eq:antid}
F(z) = \int_{\gamma(z_0,z)}f(w)dw\;,
\end{equation}
where $\gamma(z_0,z)$ is a path entirely contained in $G$
with initial point $z_0$ and final point $z$.

The function $F:G\to\mathbb{C}$ is well defined.
In fact, let $\gamma_1$ and $\gamma_2$
be any two paths entirely contained in $G$
with initial point $z_0$ and final point $z$;
define a circuit $\Gamma$ by joining $\gamma_1$ and $-\gamma_2$,
the path obtained from $\gamma_2$ by
``reversing the parameter direction''.
Then by linearity and additivity of integral
\begin{equation} \label{eq:wd}
\int_\Gamma f(w)dw
=\int_{\gamma_1}f(w)dw+\int_{-\gamma_2}f(w)dw
=\int_{\gamma_1}f(w)dw-\int_{\gamma_2}f(w)dw\;;
\end{equation}
but the left-hand side is 0 by hypothesis,
thus the two integrals on the right-hand side are equal.

We must now prove that $F'=f$ in $G$.
Given $z\in G$, there exists $r>0$
such that the ball $B_r(z)$ of radius $r$ centered in $z$
is contained in $G$.
Suppose $0<|\Delta{z}|<r$:
then we can choose as a path from $z$ to $z+\Delta{z}$
the segment $\gamma:[0,1]\to G$ parameterized by $t\mapsto z+t\Delta{z}$.
Write $f=u+iv$ with $u,v:G\to\mathbb{R}$:
by additivity of integral and the mean value theorem,
%%\begin{equation} \label{eq:ir}
%%\frac{F(z+\Delta{z})-F(z)}{\Delta{z}}
%%=\frac{1}{\Delta{z}}\int_{\gamma}f(w)dw
%%=\frac{1}{\Delta{z}}\int_0^1f(z+t\Delta{z})\Delta{z}dt
%%=u(z+\theta_u\Delta{z})+iv(z+\theta_v\Delta{z})
%%\end{equation}
\begin{eqnarray*}
\frac{F(z+\Delta{z})-F(z)}{\Delta{z}}
& = & \frac{1}{\Delta{z}}\int_{\gamma}f(w)dw \\
& = & \frac{1}{\Delta{z}}\int_0^1f(z+t\Delta{z})\Delta{z}dt \\
& = & u(z+\theta_u\Delta{z})+iv(z+\theta_v\Delta{z})
\end{eqnarray*}
for some $\theta_u,\theta_v\in(0,1)$.
Since $f$ is continuous, so are $u$ and $v$, and
\begin{displaymath}
\lim_{\Delta{z}\to 0} \frac{F(z+\Delta{z})-F(z)}{\Delta{z}}
= u(z) + iv(z) = f(z) \,.
\end{displaymath}
In the general case, we just repeat the procedure
once for each connected component of $G$.

%%%%%
%%%%%
\end{document}
