\documentclass[12pt]{article}
\usepackage{pmmeta}
\pmcanonicalname{MergelyansTheorem}
\pmcreated{2013-03-22 14:23:59}
\pmmodified{2013-03-22 14:23:59}
\pmowner{jirka}{4157}
\pmmodifier{jirka}{4157}
\pmtitle{Mergelyan's theorem}
\pmrecord{7}{35897}
\pmprivacy{1}
\pmauthor{jirka}{4157}
\pmtype{Theorem}
\pmcomment{trigger rebuild}
\pmclassification{msc}{30E10}
\pmrelated{RungesTheorem}

% this is the default PlanetMath preamble.  as your knowledge
% of TeX increases, you will probably want to edit this, but
% it should be fine as is for beginners.

% almost certainly you want these
\usepackage{amssymb}
\usepackage{amsmath}
\usepackage{amsfonts}

% used for TeXing text within eps files
%\usepackage{psfrag}
% need this for including graphics (\includegraphics)
%\usepackage{graphicx}
% for neatly defining theorems and propositions
\usepackage{amsthm}
% making logically defined graphics
%%%\usepackage{xypic}

% there are many more packages, add them here as you need them

% define commands here
\theoremstyle{theorem}
\newtheorem*{thm}{Theorem}
\newtheorem*{lemma}{Lemma}
\newtheorem*{conj}{Conjecture}
\newtheorem*{cor}{Corollary}
\newtheorem*{example}{Example}
\theoremstyle{definition}
\newtheorem*{defn}{Definition}
\begin{document}
\begin{thm}[Mergelyan]
Let $K \subset {\mathbb{C}}$ be a compact subset of the complex plane such that
${\mathbb{C}} \backslash K$ (the complement of $K$) is connected, and let
$f\colon K \to {\mathbb{C}}$ be a continuous function which is also holomorphic
on the interior of $K.$  Then $f$ is the uniform limit on $K$ of holomorphic
polynomials (polynomials in one complex variable).
\end{thm}

So for any $\epsilon > 0$ one can find a polynomial $p(z) = \sum_{j=1}^n a_j z^j$
such that $\lvert f(z) - p(z) \rvert < \epsilon$ for all $z \in K.$

Do note that this theorem is not a weaker version of Runge's theorem.  Here, we do not
need $f$ to be holomorphic on a neighbourhood of $K,$ but just on the interior of $K.$  For example, if the interior of $K$ is empty, then $f$ just needs to be continuous on $K.$  Further, it could be that the closure of the interior of $K$
might not be all of $K.$  Consider $K = D \cup [-10,10],$ where $D$
is the closed unit disc.  Then $K$ has two lines coming out of either end of the disc and $f$ needs to only be continuous there.

Also note that this theorem is distinct from the Stone-Weierstrass theorem.  The point here is that the polynomials are
holomorphic in Mergelyan's theorem.

\begin{thebibliography}{9}
\bibitem{Conway:complexI}
John~B. Conway.
{\em \PMlinkescapetext{Functions of One Complex Variable I}}.
Springer-Verlag, New York, New York, 1978.
\bibitem{Rudin:realcomplex}
Walter Rudin.
{\em \PMlinkescapetext{Real and Complex Analysis}}.
McGraw-Hill, Boston, Massachusetts, 1987.
\end{thebibliography}
%%%%%
%%%%%
\end{document}
