\documentclass[12pt]{article}
\usepackage{pmmeta}
\pmcanonicalname{ProofOfCasoratiWeierstrassTheorem}
\pmcreated{2013-03-22 13:32:40}
\pmmodified{2013-03-22 13:32:40}
\pmowner{pbruin}{1001}
\pmmodifier{pbruin}{1001}
\pmtitle{proof of Casorati-Weierstrass theorem}
\pmrecord{4}{34144}
\pmprivacy{1}
\pmauthor{pbruin}{1001}
\pmtype{Proof}
\pmcomment{trigger rebuild}
\pmclassification{msc}{30D30}
\pmrelated{PicardsTheorem}

% this is the default PlanetMath preamble.  as your knowledge
% of TeX increases, you will probably want to edit this, but
% it should be fine as is for beginners.

% almost certainly you want these
\usepackage{amssymb}
\usepackage{amsmath}
\usepackage{amsfonts}

% used for TeXing text within eps files
%\usepackage{psfrag}
% need this for including graphics (\includegraphics)
%\usepackage{graphicx}
% for neatly defining theorems and propositions
%\usepackage{amsthm}
% making logically defined graphics
%%%\usepackage{xypic}

% there are many more packages, add them here as you need them

% define commands here
\begin{document}
Assume that $a$ is an essential singularity of $f$.  Let $V\subset U$
be a punctured neighborhood of $a$, and let $\lambda\in\mathbb{C}$.
We have to show that $\lambda$ is a limit point of $f(V)$.  Suppose it
is not, then there is an $\epsilon>0$ such that
$|f(z)-\lambda|>\epsilon$ for all $z\in V$, and the function
$$
g:V\to\mathbb{C}, z\mapsto\frac{1}{f(z)-\lambda}
$$
is bounded, since $|g(z)|=\frac{1}{|f(z)-\lambda|}<\epsilon^{-1}$
for all $z\in V$.  According to Riemann's removable singularity
theorem, this implies that $a$ is a removable singularity of $g$, so
that $g$ can be extended to a holomorphic function $\bar
g:V\cup\{a\}\to\mathbb C$.  Now
$$
f(z)=\frac{1}{\bar g(z)}-\lambda
$$
for $z\neq a$, and $a$ is either a removable singularity of $f$ (if
$\bar g(z)\neq 0$) or a pole of order $n$ (if $\bar g$ has a zero of
order $n$ at $a$).  This contradicts our assumption that $a$ is an
essential singularity, which means that $\lambda$ must be a limit
point of $f(V)$.  The argument holds for all $\lambda\in\mathbb{C}$,
so $f(V)$ is dense in $\mathbb{C}$ for any punctured neighborhood $V$
of $a$.

To prove the converse, assume that $f(V)$ is dense in $\mathbb{C}$ for
any punctured neighborhood $V$ of $a$.  If $a$ is a removable
singularity, then $f$ is bounded near $a$, and if $a$ is a pole,
$f(z)\to\infty$ as $z\to a$.  Either of these possibilities
contradicts the assumption that the image of any punctured
neighborhood of $a$ under $f$ is dense in $\mathbb C$, so $a$ must be
an essential singularity of $f$.
%%%%%
%%%%%
\end{document}
