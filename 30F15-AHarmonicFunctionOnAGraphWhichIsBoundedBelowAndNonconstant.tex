\documentclass[12pt]{article}
\usepackage{pmmeta}
\pmcanonicalname{AHarmonicFunctionOnAGraphWhichIsBoundedBelowAndNonconstant}
\pmcreated{2013-03-22 12:44:26}
\pmmodified{2013-03-22 12:44:26}
\pmowner{drini}{3}
\pmmodifier{drini}{3}
\pmtitle{a harmonic function on a graph which is bounded below and nonconstant}
\pmrecord{6}{33042}
\pmprivacy{1}
\pmauthor{drini}{3}
\pmtype{Example}
\pmcomment{trigger rebuild}
\pmclassification{msc}{30F15}
\pmclassification{msc}{31C05}
\pmclassification{msc}{31B05}
\pmclassification{msc}{31A05}

\endmetadata

% this is the default PlanetMath preamble.  as your knowledge
% of TeX increases, you will probably want to edit this, but
% it should be fine as is for beginners.

% almost certainly you want these
\usepackage{amssymb}
\usepackage{amsmath}
\usepackage{amsfonts}

% used for TeXing text within eps files
%\usepackage{psfrag}
% need this for including graphics (\includegraphics)
%\usepackage{graphicx}
% for neatly defining theorems and propositions
%\usepackage{amsthm}
% making logically defined graphics
%%%\usepackage{xypic}

% there are many more packages, add them here as you need them

% define commands here

\newcommand{\Prob}[2]{\mathbb{P}_{#1}\left\{#2\right\}}
\newcommand{\norm}[1]{\left\|#1\right\|}

% Some sets
\newcommand{\Nats}{\mathbb{N}}
\newcommand{\Ints}{\mathbb{Z}}
\newcommand{\Reals}{\mathbb{R}}
\newcommand{\Complex}{\mathbb{C}}


\newcommand{\tree}{\mathcal{T}_3}
\begin{document}
There exists no harmonic function on all of the $d$-dimensional grid~$\Ints^d$ which is bounded below and nonconstant. This categorises a particular property of the grid; below we see that other graphs can admit such harmonic functions.

Let~$\tree=(V_3,E_3)$ be a 3-regular tree.  Assign ``levels'' to the vertices of~$\tree$ as follows:  Fix a vertex~$o\in V_3$, and let~$\pi$ be a branch of~$\tree$ (an infinite simple path) from~$o$.  For every vertex~$v\in V_3$ of~$\tree$ there exists a \emph{unique} shortest path from~$v$ to a vertex of~$\pi$; let~$\ell(v)=\left|\pi\right|$ be the length of this path.

Now define~$f(v)=2^{-\ell(v)} > 0$.  Without loss of generality, note that the three neighbours $u_1,u_2,u_3$ of~$v$ satisfy $\ell(u_1)=\ell(v)-1$ (``$u_1$ is the parent of~$v$''), $\ell(u_2)=\ell(u_3)=\ell(v)+1$ (``$u_2, u_3$ are the siblings of~$v$'').  And indeed,
$\frac{1}{3}\left(2^{\ell(v)-1}+2^{\ell(v)+1}+2^{\ell(v)+1}\right)
= 2^{\ell(v)}$.

So $f$ is a positive nonconstant harmonic function on~$\tree$.
%%%%%
%%%%%
\end{document}
