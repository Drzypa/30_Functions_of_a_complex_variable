\documentclass[12pt]{article}
\usepackage{pmmeta}
\pmcanonicalname{QuasiperiodicFunction}
\pmcreated{2013-03-22 14:40:38}
\pmmodified{2013-03-22 14:40:38}
\pmowner{rspuzio}{6075}
\pmmodifier{rspuzio}{6075}
\pmtitle{quasiperiodic function}
\pmrecord{13}{36280}
\pmprivacy{1}
\pmauthor{rspuzio}{6075}
\pmtype{Definition}
\pmcomment{trigger rebuild}
\pmclassification{msc}{30A99}
\pmrelated{ComplexTangentAndCotangent}
\pmrelated{CounterperiodicFunction}
\pmdefines{quasiperiod}
\pmdefines{quasiperiodicity}
\pmdefines{period}
\pmdefines{periodic function}
\pmdefines{periodic}
\pmdefines{periodicity}

% this is the default PlanetMath preamble.  as your knowledge
% of TeX increases, you will probably want to edit this, but
% it should be fine as is for beginners.

% almost certainly you want these
\usepackage{amssymb}
\usepackage{amsmath}
\usepackage{amsfonts}

% used for TeXing text within eps files
%\usepackage{psfrag}
% need this for including graphics (\includegraphics)
%\usepackage{graphicx}
% for neatly defining theorems and propositions
%\usepackage{amsthm}
% making logically defined graphics
%%%\usepackage{xypic}

% there are many more packages, add them here as you need them

% define commands here
\begin{document}
\PMlinkescapeword{connection}
\PMlinkescapeword{even}
\PMlinkescapeword{meets}
\PMlinkescapeword{order}
\PMlinkescapeword{strictly}
\PMlinkescapeword{term}

A function $f$ is said to have a \emph{quasiperiod} $p$ if there exists a function $g$ such that
 $$f(z + p) = g(z) f(z).$$

In the special case where $g$ is identically equal to $1$, we call $f$ a \emph{periodic function}, and we say that $p$ is a \emph{period} of $f$ or that $f$ has \emph{periodicity} $p$.

Except for the special case of periodicity noted above, the notion of quasiperiodicity is somewhat loose and fuzzy.  Strictly speaking, many functions could be regarded as quasiperiodic if one defines $g(z) = f(z+p) / f(z)$.  In order for the term ``quasiperiodic'' not to be trivial, it is customary to reserve its use for the case where the function $g$ is, in some vague, intuitive sense, simpler than the function $f$.  For instance, no one would call the function $f(z) = z^2 + 1$ quasiperiodic even though it meets the criterion of the definition if we set $g(z) = (z^2 + 2z + 2) / (z^2 + 1)$ because the rational function $g$ is ``more complicated'' than the polynomial $f$.  On the other hand, for the gamma function, one would say that $1$ is a quasiperiod because $\Gamma (z+1) = z \Gamma(z)$ and the function $g(z) = z$ is a ``much simpler'' function than the gamma function.

Note that the every complex number can be said to be a quasiperiod of the exponential function.  The term ``quasiperiod'' is most frequently used in connection with theta functions.

Also note that almost periodic functions are quite a different affair than quasiperiodic functions --- there, one is dealing with a precise notion.
%%%%%
%%%%%
\end{document}
