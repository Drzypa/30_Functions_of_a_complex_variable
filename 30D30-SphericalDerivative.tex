\documentclass[12pt]{article}
\usepackage{pmmeta}
\pmcanonicalname{SphericalDerivative}
\pmcreated{2013-03-22 14:18:36}
\pmmodified{2013-03-22 14:18:36}
\pmowner{jirka}{4157}
\pmmodifier{jirka}{4157}
\pmtitle{spherical derivative}
\pmrecord{7}{35773}
\pmprivacy{1}
\pmauthor{jirka}{4157}
\pmtype{Definition}
\pmcomment{trigger rebuild}
\pmclassification{msc}{30D30}

% this is the default PlanetMath preamble.  as your knowledge
% of TeX increases, you will probably want to edit this, but
% it should be fine as is for beginners.

% almost certainly you want these
\usepackage{amssymb}
\usepackage{amsmath}
\usepackage{amsfonts}

% used for TeXing text within eps files
%\usepackage{psfrag}
% need this for including graphics (\includegraphics)
%\usepackage{graphicx}
% for neatly defining theorems and propositions
\usepackage{amsthm}
% making logically defined graphics
%%%\usepackage{xypic}

% there are many more packages, add them here as you need them

% define commands here
\theoremstyle{theorem}
\newtheorem*{thm}{Theorem}
\newtheorem*{lemma}{Lemma}
\newtheorem*{conj}{Conjecture}
\newtheorem*{cor}{Corollary}
\newtheorem*{example}{Example}
\newtheorem*{prop}{Proposition}
\theoremstyle{definition}
\newtheorem*{defn}{Definition}
\theoremstyle{remark}
\newtheorem*{rmk}{Remark}
\begin{document}
Let $G \subset {\mathbb{C}}$ be a domain.

\begin{defn}
Let $f \colon G \to {\mathbb{C}}$ be a meromorphic function, then the {\em spherical
derivative} of $f$, denoted $f^\sharp$ is defined as
\begin{equation*}
f^\sharp(z) := \frac{2\lvert f'(z) \rvert}{1+\lvert f(z) \rvert^2}
\end{equation*}
for $z$ where $f(z) \not= \infty$ and when $f(z) = \infty$ define
\begin{equation*}
f^\sharp(z) = \lim_{\zeta \to z} f^\sharp(\zeta) .
\end{equation*}
\end{defn}

The second definition makes sense since a meromorphic functions has
only isolated poles, and thus $f^\sharp(\zeta)$ is defined by the first equation when we are close to $z$.  Some basic properties of the spherical derivative
are as follows.

\begin{prop}
If $f \colon G \to {\mathbb{C}}$ is a meromorphic function
then
\begin{itemize}
\item $f^\sharp$ is a continuous function,
\item $f^\sharp(z) < \infty$ for all $z \in G$. 
\end{itemize}
\end{prop}

Note that sometimes the spherical derivative is also denoted
as $\mu(f)(z)$ rather then $f^\sharp(z)$.

\begin{thebibliography}{9}
\bibitem{Conway:complexI}
John~B. Conway.
{\em \PMlinkescapetext{Functions of One Complex Variable I}}.
Springer-Verlag, New York, New York, 1978.
\bibitem{Gamelin:complex}
Theodore~B.\@ Gamelin.
{\em \PMlinkescapetext{Complex Analysis}}.
Springer-Verlag, New York, New York, 2001.
\end{thebibliography}
%%%%%
%%%%%
\end{document}
