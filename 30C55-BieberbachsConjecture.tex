\documentclass[12pt]{article}
\usepackage{pmmeta}
\pmcanonicalname{BieberbachsConjecture}
\pmcreated{2013-03-22 14:24:07}
\pmmodified{2013-03-22 14:24:07}
\pmowner{jirka}{4157}
\pmmodifier{jirka}{4157}
\pmtitle{Bieberbach's conjecture}
\pmrecord{7}{35899}
\pmprivacy{1}
\pmauthor{jirka}{4157}
\pmtype{Theorem}
\pmcomment{trigger rebuild}
\pmclassification{msc}{30C55}
\pmclassification{msc}{30C45}
\pmsynonym{Bieberbach conjecture}{BieberbachsConjecture}

\endmetadata

% this is the default PlanetMath preamble.  as your knowledge
% of TeX increases, you will probably want to edit this, but
% it should be fine as is for beginners.

% almost certainly you want these
\usepackage{amssymb}
\usepackage{amsmath}
\usepackage{amsfonts}

% used for TeXing text within eps files
%\usepackage{psfrag}
% need this for including graphics (\includegraphics)
%\usepackage{graphicx}
% for neatly defining theorems and propositions
\usepackage{amsthm}
% making logically defined graphics
%%%\usepackage{xypic}

% there are many more packages, add them here as you need them

% define commands here
\theoremstyle{theorem}
\newtheorem*{thm}{Theorem}
\newtheorem*{lemma}{Lemma}
\newtheorem*{conj}{Conjecture}
\newtheorem*{cor}{Corollary}
\theoremstyle{definition}
\newtheorem*{defn}{Definition}
\begin{document}
The following theorem is known as the Bieberbach conjecture, even though it has
now been proven.  Bieberbach proposed it in 1916 and it was finally proven in 1984 by Louis de Branges.

Firstly note that if $f \colon {\mathbb{D}} \to {\mathbb{C}}$ is a schlicht function (univalent, $f(0) = 0$ and $f'(0) = 1$) then $f$ has a power series representation
as
\begin{equation*}
f(z) = z + a_2 z^2 + a_3 z^3 + \cdots = z + \sum_{k=2}^\infty a_k z^k .
\end{equation*}

\begin{thm}[Bieberbach]
Suppose that $f$ is a schlicht function, then $\lvert a_k \rvert \leq k$ for
all $k \geq 2$ and furthermore
if there is some integer $k$ such that $\lvert a_k \rvert = k$, then $f$ is some rotation of the Koebe function.
\end{thm}

In fact if $f$ is a rotation of the Koebe function then $\lvert a_k \rvert = k$
for all $k$.

\begin{thebibliography}{9}
\bibitem{Conway:complexII}
John~B. Conway.
{\em \PMlinkescapetext{Functions of One Complex Variable II}}.
Springer-Verlag, New York, New York, 1995.
\end{thebibliography}
%%%%%
%%%%%
\end{document}
