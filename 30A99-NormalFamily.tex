\documentclass[12pt]{article}
\usepackage{pmmeta}
\pmcanonicalname{NormalFamily}
\pmcreated{2013-03-22 14:17:49}
\pmmodified{2013-03-22 14:17:49}
\pmowner{jirka}{4157}
\pmmodifier{jirka}{4157}
\pmtitle{normal family}
\pmrecord{7}{35753}
\pmprivacy{1}
\pmauthor{jirka}{4157}
\pmtype{Definition}
\pmcomment{trigger rebuild}
\pmclassification{msc}{30A99}

\endmetadata

% this is the default PlanetMath preamble.  as your knowledge
% of TeX increases, you will probably want to edit this, but
% it should be fine as is for beginners.

% almost certainly you want these
\usepackage{amssymb}
\usepackage{amsmath}
\usepackage{amsfonts}

% used for TeXing text within eps files
%\usepackage{psfrag}
% need this for including graphics (\includegraphics)
%\usepackage{graphicx}
% for neatly defining theorems and propositions
\usepackage{amsthm}
% making logically defined graphics
%%%\usepackage{xypic}

% there are many more packages, add them here as you need them

% define commands here
\theoremstyle{theorem}
\newtheorem*{thm}{Theorem}
\newtheorem*{lemma}{Lemma}
\newtheorem*{conj}{Conjecture}
\newtheorem*{cor}{Corollary}
\newtheorem*{example}{Example}
\theoremstyle{definition}
\newtheorem*{defn}{Definition}
\begin{document}
\begin{defn}
A set (sometimes called a family) ${\mathcal{F}}$
of continuous functions $f \colon X \to Y$ for some (\PMlinkname{complete}{Complete})
metric spaces $X$ and $Y$
is called {\em \PMlinkescapetext{normal}} if each sequence of functions in ${\mathcal{F}}$
contains a subsequence which converges uniformly on compact subsets of $X$ to a 
continuous function from $X$ to $Y$.
\end{defn}

This definition is often used in complex analysis for spaces of holomorphic functions.  It turns out that a sequence of holomorphic functions that converges uniformly on compact sets converges to a holomorphic function.  So you can replace
$X$ with a region in the complex plane, $Y$ with the complex plane itself
and every instance of ``continuous'' with ``holomorphic'' and you get a version of the definition most used in complex analysis.

Another space where this is often used is the space of meromorphic functions.
This is similar to the holomorphic case, but instead of using the standard
metric for convergence we must use the spherical metric.  That is if $\sigma$
is the spherical metric, then want $f_n(z) \to f(z)$ uniformly on compact
subsets to mean that $\sigma(f_n(z),f(z))$ goes to 0 uniformly on compact
subsets.

Note that this is a classical definition that, while very often used, is not really \PMlinkescapetext{consistent}
with modern naming.  In more modern \PMlinkescapetext{language},
one would give a metric
on the space of continuous (holomorphic) functions that corresponds to 
convergence on compact subsets and then you'd say ``precompact set of functions'' in such a metric space instead of saying ``\PMlinkescapetext{normal} family of continuous (holomorphic) functions''.  This added generality however
makes it more cumbersome to use since one would need to define the metric mentioned above.

\begin{thebibliography}{9}
\bibitem{Conway:complexI}
John~B. Conway.
{\em \PMlinkescapetext{Functions of One Complex Variable I}}.
Springer-Verlag, New York, New York, 1978.
\end{thebibliography}
%%%%%
%%%%%
\end{document}
