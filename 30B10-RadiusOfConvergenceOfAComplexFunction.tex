\documentclass[12pt]{article}
\usepackage{pmmeta}
\pmcanonicalname{RadiusOfConvergenceOfAComplexFunction}
\pmcreated{2013-03-22 14:40:33}
\pmmodified{2013-03-22 14:40:33}
\pmowner{rspuzio}{6075}
\pmmodifier{rspuzio}{6075}
\pmtitle{radius of convergence of a complex function}
\pmrecord{6}{36278}
\pmprivacy{1}
\pmauthor{rspuzio}{6075}
\pmtype{Theorem}
\pmcomment{trigger rebuild}
\pmclassification{msc}{30B10}

% this is the default PlanetMath preamble.  as your knowledge
% of TeX increases, you will probably want to edit this, but
% it should be fine as is for beginners.

% almost certainly you want these
\usepackage{amssymb}
\usepackage{amsmath}
\usepackage{amsfonts}

% used for TeXing text within eps files
%\usepackage{psfrag}
% need this for including graphics (\includegraphics)
%\usepackage{graphicx}
% for neatly defining theorems and propositions
%\usepackage{amsthm}
% making logically defined graphics
%%%\usepackage{xypic}

% there are many more packages, add them here as you need them

% define commands here
\begin{document}
Let $f$ be an analytic function defined in a disk of radius $R$ about a point $z_0 \in \mathbb{C}$.  Then the radius of convergence of the Taylor series of $f$ about $z_0$ is at least $R$.

For example, the function $a(z) = 1 / (1 - z)^2$ is analytic inside the disk $|z| < 1$.  Hence its the radius of covergence of its Taylor series about $0$ is at least $1$.  By direct examination of the Taylor series we can see that its radius of convergence is, in fact, equal to $1$.

Colloquially, this theorem is stated in the sometimes imprecise but memorable form ``The radius of convergence of the Taylor series is the distance to the nearest singularity.''
%%%%%
%%%%%
\end{document}
