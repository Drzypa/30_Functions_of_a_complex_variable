\documentclass[12pt]{article}
\usepackage{pmmeta}
\pmcanonicalname{MittagLefflersTheorem}
\pmcreated{2013-03-22 13:15:15}
\pmmodified{2013-03-22 13:15:15}
\pmowner{Koro}{127}
\pmmodifier{Koro}{127}
\pmtitle{Mittag-Leffler's theorem}
\pmrecord{4}{33732}
\pmprivacy{1}
\pmauthor{Koro}{127}
\pmtype{Theorem}
\pmcomment{trigger rebuild}
\pmclassification{msc}{30D30}
\pmrelated{WeierstrassFactorizationTheorem}

% this is the default PlanetMath preamble.  as your knowledge
% of TeX increases, you will probably want to edit this, but
% it should be fine as is for beginners.

% almost certainly you want these
\usepackage{amssymb}
\usepackage{amsmath}
\usepackage{amsfonts}

% used for TeXing text within eps files
%\usepackage{psfrag}
% need this for including graphics (\includegraphics)
%\usepackage{graphicx}
% for neatly defining theorems and propositions
%\usepackage{amsthm}
% making logically defined graphics
%%%\usepackage{xypic}

% there are many more packages, add them here as you need them

% define commands here
\begin{document}
Let $G$ be an open subset of $\mathbb{C}$, let $\{a_k\}$ be a sequence of distinct points in $G$ which has no limit point in $G$. For each $k$, let 
$A_{1k},\dots,A_{m_kk}$ be arbitrary complex coefficients, and define
\[S_k(z) = \sum_{j=1}^{m_k} \frac{A_{jk}}{(z-a_k)^j}.\]
Then there exists a meromorphic function $f$ on $G$ whose poles are exactly the points $\{a_k\}$ and such that the singular part of $f$ at $a_k$ is $S_k(z)$, for each $k$.
%%%%%
%%%%%
\end{document}
