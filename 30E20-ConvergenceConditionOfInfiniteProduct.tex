\documentclass[12pt]{article}
\usepackage{pmmeta}
\pmcanonicalname{ConvergenceConditionOfInfiniteProduct}
\pmcreated{2013-03-22 14:37:22}
\pmmodified{2013-03-22 14:37:22}
\pmowner{pahio}{2872}
\pmmodifier{pahio}{2872}
\pmtitle{convergence condition of infinite product}
\pmrecord{16}{36202}
\pmprivacy{1}
\pmauthor{pahio}{2872}
\pmtype{Theorem}
\pmcomment{trigger rebuild}
\pmclassification{msc}{30E20}
%\pmkeywords{Cauchy sequence}
\pmrelated{OrderOfFactorsInInfiniteProduct}
\pmrelated{NecessaryConditionOfConvergence}
\pmdefines{infinite product}
\pmdefines{value of infinite product}

% this is the default PlanetMath preamble.  as your knowledge
% of TeX increases, you will probably want to edit this, but
% it should be fine as is for beginners.

% almost certainly you want these
\usepackage{amssymb}
\usepackage{amsmath}
\usepackage{amsfonts}

% used for TeXing text within eps files
%\usepackage{psfrag}
% need this for including graphics (\includegraphics)
%\usepackage{graphicx}
% for neatly defining theorems and propositions
 \usepackage{amsthm}
% making logically defined graphics
%%%\usepackage{xypic}

% there are many more packages, add them here as you need them

% define commands here
\theoremstyle{definition}
\newtheorem*{thmplain}{Theorem}
\begin{document}
Let us think the sequence \,$u_1,\,u_1u_2,\,u_1u_2u_3,\,\ldots$\, In the complex analysis, one often uses the definition of the convergence of an {\em infinite product}\, $\displaystyle\prod_{k = 1}^{\infty}u_k$\, where the case\, $\displaystyle\lim_{k\to\infty}u_1u_2 \ldots u_k = 0$\, is excluded.\, Then one has the

\begin{thmplain}\, The infinite product $\displaystyle\prod_{k = 1}^{\infty}u_k$ of the non-zero complex numbers\, $u_1$, $u_2$, ... is convergent iff for every positive number $\varepsilon$ there exists a positive number $n_\varepsilon$ such that the condition
 $$\vert u_{n+1}u_{n+2} \ldots u_{n+p}-1 \vert < \varepsilon \quad \forall \,p\in\mathbb{Z}_+$$
is true as soon as\, $n \geqq n_\varepsilon$.
\end{thmplain}

\textbf{Corollary.}\, If the infinite product converges, then we necessarily have\, $\displaystyle\lim_{k\to\infty}u_k = 1$. (Cf. the necessary condition of convergence of series.)

When the infinite product converges, we say that the {\em value of the infinite product} is equal to $\displaystyle\lim_{k\to\infty} u_1u_2 \ldots u_k$.
%%%%%
%%%%%
\end{document}
