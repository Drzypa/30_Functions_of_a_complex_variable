\documentclass[12pt]{article}
\usepackage{pmmeta}
\pmcanonicalname{ProofOfLimitRuleOfProduct}
\pmcreated{2013-03-22 17:52:22}
\pmmodified{2013-03-22 17:52:22}
\pmowner{pahio}{2872}
\pmmodifier{pahio}{2872}
\pmtitle{proof of limit rule of product}
\pmrecord{6}{40352}
\pmprivacy{1}
\pmauthor{pahio}{2872}
\pmtype{Proof}
\pmcomment{trigger rebuild}
\pmclassification{msc}{30A99}
\pmclassification{msc}{26A06}
%\pmkeywords{limit rule of product}
\pmrelated{ProductOfFunctions}
\pmrelated{TriangleInequality}
\pmrelated{ProductAndQuotientOfFunctionsSum}

% this is the default PlanetMath preamble.  as your knowledge
% of TeX increases, you will probably want to edit this, but
% it should be fine as is for beginners.

% almost certainly you want these
\usepackage{amssymb}
\usepackage{amsmath}
\usepackage{amsfonts}

% used for TeXing text within eps files
%\usepackage{psfrag}
% need this for including graphics (\includegraphics)
%\usepackage{graphicx}
% for neatly defining theorems and propositions
 \usepackage{amsthm}
% making logically defined graphics
%%%\usepackage{xypic}

% there are many more packages, add them here as you need them

% define commands here

\theoremstyle{definition}
\newtheorem*{thmplain}{Theorem}

\begin{document}
Let $f$ and $g$ be \PMlinkname{real}{RealFunction} or complex functions having the limits
$$\lim_{x\to x_0}f(x) = F \quad \mbox{and} \quad \lim_{x\to x_0}g(x) = G.$$
Then also the limit $\displaystyle\lim_{x\to x_0}f(x)g(x)$ exists and equals $FG$.\\

{\em Proof.}\, Let $\varepsilon$ be any positive number.\, The assumptions imply the existence of the positive numbers $\delta_1,\,\delta_2,\,\delta_3$ such that
\begin{align}
|f(x)-F| < \frac{\varepsilon}{2(1+|G|)}\;\;\mbox{when}\;\;0 < |x-x_0| < \delta_1
\end{align}
\begin{align}
|g(x)-G| < \frac{\varepsilon}{2(1+|F|)}\;\;\mbox{when}\;\;0 < |x-x_0| < \delta_2,
\end{align}
\begin{align}
|g(x)-G| < 1\;\;\mbox{when}\;\;0 < |x-x_0| < \delta_3.
\end{align}
According to the condition (3) we see that
$$|g(x)| = |g(x)\!-\!G\!+\!G| \leqq |g(x)\!-\!G|+|G| < 1\!+\!|G|\;\;\mbox{when}\;\;0 < |x-x_0| < \delta_3.$$
Supposing then that\, $0 < |x-x_0| < \min\{\delta_1,\,\delta_2,\,\delta_3\}$\, and using (1) and (2) we obtain
\begin{align*}
|f(x)g(x)-FG|\;& = |f(x)g(x)-Fg(x)+Fg(x)-FG|\\
               & \leqq |f(x)g(x)\!-\!Fg(x)|+|Fg(x)\!-\!FG|\\
               & = |g(x)|\cdot|f(x)\!-\!F|+|F|\cdot|g(x)\!-\!G|\\
               & < (1\!+\!|G|)\frac{\varepsilon}{2(1\!+\!|G|)}+(1\!+\!|F|)\frac{\varepsilon}{2(1\!+\!|F|)}\\ 
               & = \varepsilon
\end{align*}
This settles the proof.



%%%%%
%%%%%
\end{document}
