\documentclass[12pt]{article}
\usepackage{pmmeta}
\pmcanonicalname{SquareOfSum}
\pmcreated{2013-03-22 15:32:03}
\pmmodified{2013-03-22 15:32:03}
\pmowner{pahio}{2872}
\pmmodifier{pahio}{2872}
\pmtitle{square of sum}
\pmrecord{9}{37427}
\pmprivacy{1}
\pmauthor{pahio}{2872}
\pmtype{Topic}
\pmcomment{trigger rebuild}
\pmclassification{msc}{30-00}
\pmclassification{msc}{26-00}
\pmclassification{msc}{11-00}
\pmrelated{SquareRootOfPolynomial}
\pmrelated{DifferenceOfSquares}
\pmrelated{HeronianMeanIsBetweenGeometricAndArithmeticMean}
\pmrelated{ContraharmonicMeansAndPythagoreanHypotenuses}
\pmrelated{CompletingTheSquare}
\pmrelated{TriangleInequalityOfComplexNumbers}

% this is the default PlanetMath preamble.  as your knowledge
% of TeX increases, you will probably want to edit this, but
% it should be fine as is for beginners.

% almost certainly you want these
\usepackage{amssymb}
\usepackage{amsmath}
\usepackage{amsfonts}

% used for TeXing text within eps files
%\usepackage{psfrag}
% need this for including graphics (\includegraphics)
%\usepackage{graphicx}
% for neatly defining theorems and propositions
 \usepackage{amsthm}
% making logically defined graphics
%%%\usepackage{xypic}

% there are many more packages, add them here as you need them

% define commands here

\theoremstyle{definition}
\newtheorem*{thmplain}{Theorem}
\begin{document}
The well-known \PMlinkescapetext{formula} for squaring a sum of two numbers or \PMlinkescapetext{terms} is
\begin{align}
(a\!+\!b)^2 \;=\; a^2\!+\!2ab\!+\!b^2.
\end{align}
It may be derived by multiplying the binomial $a\!+\!b$ by itself.

Similarly one can get the squaring \PMlinkescapetext{formula} for a sum of three summands:
\begin{align}
(a\!+\!b\!+\!c)^2 \;=\; a^2\!+\!b^2\!+\!c^2\!+\!2bc\!+\!2ca\!+\!2ab
\end{align}
Its contents may be expressed as the

\textbf{Rule.}\, The square of a sum is equal to the sum of the squares of all the summands plus the sum of all the double products of the summands in twos:
$$\left(\sum_ia_i\right)^2 \;=\; \sum_ia_i^2+2\!\sum_{i<j}a_ia_j.$$\\

This is true for any number of summands.\, The rule may be formulated also as
\begin{align}
(a\!+\!b\!+\!c+...)^2 \;=\;
 (a)a+(2a\!+\!b)b+(2a\!+\!2b\!+\!c)c+...
\end{align}
which in the case of four summands is
\begin{align}
(a\!+\!b\!+\!c\!+\!d)^2 \;=\;
 (a)a+(2a\!+\!b)b+(2a\!+\!2b\!+\!c)c+(2a\!+\!2b\!+\!2c\!+\!d)d.
\end{align}
One can use the idea of (3) to find the \PMlinkescapetext{{\em square root of a polynomial}}, when one tries to arrange the polynomial into the form of the right hand \PMlinkname{side}{Equation} of (3).
%%%%%
%%%%%
\end{document}
