\documentclass[12pt]{article}
\usepackage{pmmeta}
\pmcanonicalname{Antiholomorphic}
\pmcreated{2014-11-06 12:07:50}
\pmmodified{2014-11-06 12:07:50}
\pmowner{pahio}{2872}
\pmmodifier{pahio}{2872}
\pmtitle{antiholomorphic}
\pmrecord{9}{41476}
\pmprivacy{1}
\pmauthor{pahio}{2872}
\pmtype{Definition}
\pmcomment{trigger rebuild}
\pmclassification{msc}{30A99}
\pmsynonym{antiholomorphic function}{Antiholomorphic}
\pmrelated{ComplexConjugate}

% this is the default PlanetMath preamble.  as your knowledge
% of TeX increases, you will probably want to edit this, but
% it should be fine as is for beginners.

% almost certainly you want these
\usepackage{amssymb}
\usepackage{amsmath}
\usepackage{amsfonts}

% used for TeXing text within eps files
%\usepackage{psfrag}
% need this for including graphics (\includegraphics)
%\usepackage{graphicx}
% for neatly defining theorems and propositions
 \usepackage{amsthm}
% making logically defined graphics
%%%\usepackage{xypic}

% there are many more packages, add them here as you need them

% define commands here

\theoremstyle{definition}
\newtheorem*{thmplain}{Theorem}

\begin{document}
\PMlinkescapeword{iff}

A complex function \,$f\!: D \to \mathbb{C}$,\, where $D$ is a domain of the complex plane, having the derivative 
$$\frac{df}{d \overline{z}}$$
in each point $z$ of $D$, is said to be {\em antiholomorphic} in $D$. \\

The following conditions are \PMlinkname{equivalent}{Equivalent3}:
\begin{itemize}

\item $f(z)$ is antiholomorphic in $D$.

\item \, $\overline{f(z)}$\, is holomorphic in $D$. 

\item $f(\overline{z})$ is holomorphic in\, $\overline{D} \,:=\, \{\overline{z}\;\vdots\;\, z \in D\}$.

\item $f(z)$ may be \PMlinkescapetext{expanded} to a power series $\sum_{n=0}^\infty a_n(\overline{z}-u)^n$ at each\, $u \in D$.

\item The real part \,$u(x,\,y)$\, and the imaginary part 
\,$v(x,\,y)$\, of the function $f$ satisfy the equations
$$\frac{\partial u}{\partial x} 
\;=\; -\frac{\partial v}{\partial y}, \qquad 
  \frac{\partial u}{\partial y} \;=\; \frac{\partial v}{\partial x}.$$
N.B. the \PMlinkescapetext{place} of minus; cf. the \PMlinkname{Cauchy--Riemann equations}{CauchyRiemannEquations}.\\

\end{itemize}

\textbf{Example.}\, The function\, $\displaystyle z \mapsto \frac{1}{\overline{z}}$ is antiholomorphic in\, 
$\mathbb{C}\!\smallsetminus\!\{0\}$.\, One has 
$$f(z) \;=\; \frac{z}{|z|^2} 
       \;=\; \underbrace{\frac{x}{x^2\!+\!y^2}}_{u}+i\underbrace{\frac{y}{x^2\!+\!y^2}}_{v}$$
and thus         
$$\frac{\partial u}{\partial x} 
\;=\; \frac{y^2\!-\!x^2}{(x^2\!+\!y^2)^2},  \qquad
\frac{\partial v}{\partial y}
\;=\; \frac{x^2\!-\!y^2}{(x^2\!+\!y^2)^2},  \qquad 
\frac{\partial u}{\partial y} 
\;=\; -\frac{2xy}{(x^2\!+\!y^2)^2},  \qquad
\frac{\partial v}{\partial x} \;=\; -\frac{2xy}{(x^2\!+\!y^2)^2}.$$


%%%%%
%%%%%
\end{document}
