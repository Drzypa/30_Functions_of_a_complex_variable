\documentclass[12pt]{article}
\usepackage{pmmeta}
\pmcanonicalname{MaximumPrinciple}
\pmcreated{2013-03-22 12:46:06}
\pmmodified{2013-03-22 12:46:06}
\pmowner{mathcam}{2727}
\pmmodifier{mathcam}{2727}
\pmtitle{maximum principle}
\pmrecord{5}{33078}
\pmprivacy{1}
\pmauthor{mathcam}{2727}
\pmtype{Theorem}
\pmcomment{trigger rebuild}
\pmclassification{msc}{30C80}
\pmclassification{msc}{31A05}
\pmclassification{msc}{31B05}
\pmclassification{msc}{30F15}
\pmsynonym{maximal modulus principle}{MaximumPrinciple}
\pmsynonym{maximum principle for harmonic functions}{MaximumPrinciple}
\pmrelated{HadamardThreeCircleTheorem}
\pmrelated{PhragmenLindelofTheorem}

\endmetadata

% this is the default PlanetMath preamble.  as your knowledge
% of TeX increases, you will probably want to edit this, but
% it should be fine as is for beginners.

% almost certainly you want these
\usepackage{amssymb}
\usepackage{amsmath}
\usepackage{amsfonts}

% used for TeXing text within eps files
%\usepackage{psfrag}
% need this for including graphics (\includegraphics)
%\usepackage{graphicx}
% for neatly defining theorems and propositions
%\usepackage{amsthm}
% making logically defined graphics
%%%\usepackage{xypic}

% there are many more packages, add them here as you need them

% define commands here

\newcommand{\Prob}[2]{\mathbb{P}_{#1}\left\{#2\right\}}
\newcommand{\Expect}{\mathbb{E}}
\newcommand{\norm}[1]{\left\|#1\right\|}

% Some sets
\newcommand{\Nats}{\mathbb{N}}
\newcommand{\Ints}{\mathbb{Z}}
\newcommand{\Reals}{\mathbb{R}}
\newcommand{\Complex}{\mathbb{C}}



%%%%%% END OF SAVED PREAMBLE %%%%%%
\begin{document}
\begin{description}

\item[Maximum principle]
Let $f:U\to \Reals$ (where $U\subseteq \Reals^d$) be a harmonic function.  Then $f$ attains its extremal values on any compact $K\subseteq U$ on the boundary $\partial K$ of $K$.  If $f$ attains an extremal value anywhere in the \emph{interior} of $K$, then it is constant.


\item[Maximal modulus principle]
Let $f:U\to\Complex$ (where $U\subseteq \Complex$) be a holomorphic function.  Then $|f|$ attains its maximal value on any compact $K\subseteq U$ on the boundary $\partial K$ of $K$.  If $|f|$ attains its maximal value anywhere on the \emph{interior} of $K$, then it is constant.

\end{description}
%%%%%
%%%%%
\end{document}
