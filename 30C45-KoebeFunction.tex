\documentclass[12pt]{article}
\usepackage{pmmeta}
\pmcanonicalname{KoebeFunction}
\pmcreated{2013-03-22 14:23:30}
\pmmodified{2013-03-22 14:23:30}
\pmowner{jirka}{4157}
\pmmodifier{jirka}{4157}
\pmtitle{Koebe function}
\pmrecord{6}{35888}
\pmprivacy{1}
\pmauthor{jirka}{4157}
\pmtype{Definition}
\pmcomment{trigger rebuild}
\pmclassification{msc}{30C45}
\pmsynonym{K\"obe function}{KoebeFunction}
\pmdefines{rotation of the Koebe function}
\pmdefines{rotation of the K\"obe function}

\endmetadata

% this is the default PlanetMath preamble.  as your knowledge
% of TeX increases, you will probably want to edit this, but
% it should be fine as is for beginners.

% almost certainly you want these
\usepackage{amssymb}
\usepackage{amsmath}
\usepackage{amsfonts}

% used for TeXing text within eps files
%\usepackage{psfrag}
% need this for including graphics (\includegraphics)
%\usepackage{graphicx}
% for neatly defining theorems and propositions
\usepackage{amsthm}
% making logically defined graphics
%%%\usepackage{xypic}

% there are many more packages, add them here as you need them

% define commands here
\theoremstyle{theorem}
\newtheorem*{thm}{Theorem}
\newtheorem*{lemma}{Lemma}
\newtheorem*{conj}{Conjecture}
\newtheorem*{cor}{Corollary}
\newtheorem*{example}{Example}
\theoremstyle{definition}
\newtheorem*{defn}{Definition}
\begin{document}
\begin{defn}
The analytic function
\begin{equation*}
f(z) := \frac{z}{(1-z)^2}
\end{equation*}
on the unit disc in the complex plane is called the {\em Koebe function}.  For some $\lvert \alpha \rvert = 1$, the functions
\begin{equation*}
f_\alpha(z) := \frac{z}{(1- \alpha z)^2}
\end{equation*}
are called {\em rotations of the Koebe function}.
\end{defn}

Firstly note that $f_1 = f$, and
next
note that $f$ is a map from the open unit disc onto ${\mathbb{C}} \backslash (-\infty,-1/4]$.  The maps $f_\alpha (z)$ can be also given as
$f_\alpha(z) = \bar{\alpha} f_1 (\alpha z )$.
Further note that the power series representation of these
functions is given by
\begin{equation*}
f_\alpha(z) = \frac{z}{(1-\alpha z)^2}
= \sum_{n=1}^\infty n \alpha^{n-1} z^n .
\end{equation*}

Also note that these functions belong to the class of Schlicht functions.

\begin{thebibliography}{9}
\bibitem{Conway:complexII}
John~B. Conway.
{\em \PMlinkescapetext{Functions of One Complex Variable II}}.
Springer-Verlag, New York, New York, 1995.
\end{thebibliography}
%%%%%
%%%%%
\end{document}
