\documentclass[12pt]{article}
\usepackage{pmmeta}
\pmcanonicalname{HeavisideStepFunction}
\pmcreated{2013-03-22 13:46:14}
\pmmodified{2013-03-22 13:46:14}
\pmowner{Koro}{127}
\pmmodifier{Koro}{127}
\pmtitle{Heaviside step function}
\pmrecord{8}{34476}
\pmprivacy{1}
\pmauthor{Koro}{127}
\pmtype{Definition}
\pmcomment{trigger rebuild}
\pmclassification{msc}{30-00}
\pmclassification{msc}{26A06}
\pmsynonym{Heaviside function}{HeavisideStepFunction}
\pmrelated{SignumFunction}
\pmrelated{DelayTheorem}
\pmrelated{TelegraphEquation}

\endmetadata

% this is the default PlanetMath preamble.  as your knowledge
% of TeX increases, you will probably want to edit this, but
% it should be fine as is for beginners.

% almost certainly you want these
\usepackage{amssymb}
\usepackage{amsmath}
\usepackage{amsfonts}

% used for TeXing text within eps files
%\usepackage{psfrag}
% need this for including graphics (\includegraphics)
%\usepackage{graphicx}
% for neatly defining theorems and propositions
%\usepackage{amsthm}
% making logically defined graphics
%%%\usepackage{xypic}

% there are many more packages, add them here as you need them

% define commands here

\newcommand{\sR}[0]{\mathbb{R}}
\newcommand{\sC}[0]{\mathbb{C}}
\newcommand{\sN}[0]{\mathbb{N}}
\newcommand{\sZ}[0]{\mathbb{Z}}
\begin{document}
The \emph{Heaviside step function} is the function $H:\sR\to \sR$ defined as
 \begin{eqnarray*}
 H(x) &=& \left\{ \begin {array}{ll} 0 & \mbox{when}\,\, x< 0, \\
 1/2 & \mbox{when}\,\, x= 0,\\
 1 & \mbox{when}\,\, x> 0.\\
 \end{array} \right.
 \end{eqnarray*}
Here, there are many conventions for the value at $x=0$. The
motivation for setting $H(0)=1/2$ is that we can then write
$H$ as a function of the signum function (see
\PMlinkname{this page}{SignumFunction}). In applications, such as
the Laplace transform, where the Heaviside function is used extensively,
the value of $H(0)$ is irrelevant.
The Fourier transform of heaviside function is 
$$\mathcal{F}_0 H(t)=\frac{1}{2}\left(\delta(t)-\frac{i}{\pi t}\right)$$
where $\delta$ denotes the Dirac delta centered at $0$.
The function is named after Oliver Heaviside (1850-1925)
\cite{heaviside_bib}. However, the function was already used by
Cauchy\cite{cauchy_bib}, who defined the function as
$$ u(t) = \frac{1}{2}\big( 1 + t/\sqrt{t^2}\big)$$
and called it a \emph{coefficient limitateur} \cite{hoskins}.

\begin{thebibliography}{9}
\bibitem{heaviside_bib}
 The MacTutor History of Mathematics archive,
 \PMlinkexternal{Oliver Heaviside}{http://www-gap.dcs.st-and.ac.uk/~history/Mathematicians/Heav
iside.html}.
\bibitem{cauchy_bib}
 The MacTutor History of Mathematics archive,
 \PMlinkexternal{Augustin Louis Cauchy}{http://www-gap.dcs.st-and.ac.uk/~history/Mathematicians/Cauc
hy.html}.
\bibitem{hoskins}
 R.F. Hoskins, \emph{Generalised functions},
 Ellis Horwood Series: Mathematics and its applications,
 John Wiley \& Sons, 1979.
\end{thebibliography}
%%%%%
%%%%%
\end{document}
