\documentclass[12pt]{article}
\usepackage{pmmeta}
\pmcanonicalname{EvaluatingTheGammaFunctionAt12}
\pmcreated{2013-03-22 16:57:13}
\pmmodified{2013-03-22 16:57:13}
\pmowner{CWoo}{3771}
\pmmodifier{CWoo}{3771}
\pmtitle{evaluating the gamma function at 1/2}
\pmrecord{4}{39223}
\pmprivacy{1}
\pmauthor{CWoo}{3771}
\pmtype{Derivation}
\pmcomment{trigger rebuild}
\pmclassification{msc}{30D30}
\pmclassification{msc}{33B15}
\pmrelated{AreaUnderGaussianCurve}
\pmrelated{LaplaceTransformOfPowerFunction}

\endmetadata

% this is the default PlanetMath preamble.  as your knowledge
% of TeX increases, you will probably want to edit this, but
% it should be fine as is for beginners.

% almost certainly you want these
\usepackage{amssymb}
\usepackage{amsmath}
\usepackage{amsfonts}

% used for TeXing text within eps files
%\usepackage{psfrag}
% need this for including graphics (\includegraphics)
%\usepackage{graphicx}
% for neatly defining theorems and propositions
%\usepackage{amsthm}
% making logically defined graphics
%%%\usepackage{xypic}

% there are many more packages, add them here as you need them

% define commands here

\begin{document}
In the entry on the gamma function it is mentioned that $\Gamma(1/2) =
\sqrt{\pi}$.  In this entry we reduce the proof of this claim to the
problem of computing the area under the bell curve.  First note that by
definition of the gamma function,
\begin{align*}
\Gamma(1/2) 
&= \int_0^{\infty} e^{-x} x^{-1/2}\,dx \\
&=  2\int_0^{\infty} e^{-x} \frac{1}{2\sqrt{x}}\,dx.
\end{align*}
Performing the substitution $u = \sqrt{x}$, we find that $du =
\frac{1}{2\sqrt{x}}\,dx$, so
\[
\Gamma(1/2) = 2\int_0^{\infty} e^{-u^2}\,du = \int_{\mathbb{R}} e^{-u^2}\,du,
\]
where the last equality holds because $e^{-u^2}$ is an even function.
Since the area under the bell curve is $\sqrt{\pi}$, it follows that
$\Gamma(1/2) = \sqrt{\pi}$.

\PMlinkescapeword{equality}
%%%%%
%%%%%
\end{document}
