\documentclass[12pt]{article}
\usepackage{pmmeta}
\pmcanonicalname{Orientation}
\pmcreated{2013-03-22 12:56:09}
\pmmodified{2013-03-22 12:56:09}
\pmowner{CWoo}{3771}
\pmmodifier{CWoo}{3771}
\pmtitle{orientation}
\pmrecord{6}{33292}
\pmprivacy{1}
\pmauthor{CWoo}{3771}
\pmtype{Definition}
\pmcomment{trigger rebuild}
\pmclassification{msc}{30A99}
\pmrelated{SensePreservingMapping}

\endmetadata

% this is the default PlanetMath preamble.  as your knowledge
% of TeX increases, you will probably want to edit this, but
% it should be fine as is for beginners.

% almost certainly you want these
\usepackage{amssymb}
\usepackage{amsmath}
\usepackage{amsfonts}

% used for TeXing text within eps files
%\usepackage{psfrag}
% need this for including graphics (\includegraphics)
%\usepackage{graphicx}
% for neatly defining theorems and propositions
%\usepackage{amsthm}
% making logically defined graphics
%%%\usepackage{xypic}

% there are many more packages, add them here as you need them

% define commands here
\begin{document}
Let $\alpha$ be a rectifiable, Jordan curve in $\mathbb{R}^{2}$ and $z_{0}$ be a point in $\mathbb{R}^{2} - \operatorname{Im}(\alpha)$ and let $\alpha$ have a winding number $W [ \alpha : z_{0} ]$. Then $W [ \alpha : z_{0} ] = \pm 1$; all points inside $\alpha$ will have the same index and we define the \textbf{orientation} of a Jordan curve $\alpha$ by saying that $\alpha$ is \textbf{positively oriented} if the index of every point in $\alpha$ is $+1$ and \textbf{negatively oriented} if it is $-1$.
%%%%%
%%%%%
\end{document}
