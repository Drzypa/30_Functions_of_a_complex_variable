\documentclass[12pt]{article}
\usepackage{pmmeta}
\pmcanonicalname{LogarithmicIntegral}
\pmcreated{2013-03-22 17:03:05}
\pmmodified{2013-03-22 17:03:05}
\pmowner{pahio}{2872}
\pmmodifier{pahio}{2872}
\pmtitle{logarithmic integral}
\pmrecord{14}{39341}
\pmprivacy{1}
\pmauthor{pahio}{2872}
\pmtype{Definition}
\pmcomment{trigger rebuild}
\pmclassification{msc}{30E20}
\pmclassification{msc}{33E20}
\pmclassification{msc}{26A36}
\pmsynonym{Li}{LogarithmicIntegral}
\pmrelated{SineIntegral}
\pmrelated{PrimeNumberTheorem}
\pmrelated{PrimeCountingFunction}
\pmrelated{LaTeXSymbolForCauchyPrincipalValue}
\pmrelated{ConvergenceOfIntegrals}
\pmdefines{logarithmic integral}
\pmdefines{logarithmus integralis}
\pmdefines{Eulerian logarithmic integral}

% this is the default PlanetMath preamble.  as your knowledge
% of TeX increases, you will probably want to edit this, but
% it should be fine as is for beginners.

% almost certainly you want these
\usepackage{amssymb}
\usepackage{amsmath}
\usepackage{amsfonts}

% used for TeXing text within eps files
%\usepackage{psfrag}
% need this for including graphics (\includegraphics)
%\usepackage{graphicx}
% for neatly defining theorems and propositions
 \usepackage{amsthm}
% making logically defined graphics
%%%\usepackage{xypic}

% there are many more packages, add them here as you need them

% define commands here
\newcommand{\Li}{\operatorname{Li}}
\newcommand{\li}{\operatorname{li}}

\theoremstyle{definition}
\newtheorem*{thmplain}{Theorem}

\begin{document}
\PMlinkescapeword{expansion}

The European or Eulerian version of {\em logarithmic integral} (in Latin {\em logarithmus integralis}) is defined as
\begin{align}
  \Li{x} := \int_2^x\frac{dt}{\ln{t}},
\end{align}
and the American version is
\begin{align}
  \li{x} := \int_0^x\frac{dt}{\ln{t}},
\end{align}
The integrand $\displaystyle\frac{1}{\ln{t}}$ has a singularity\, $t = 1$,\, 
and for\, $x > 1$\, the latter definition is interpreted as 
the Cauchy principal value
$$\li{x} = 
\lim_{\varepsilon\to 0+}\left(\int_0^{1-\varepsilon}\!\frac{dt}{\ln{t}}
+\int_{1+\varepsilon}^x\frac{dt}{\ln{t}}\right).$$
The connection between (1) and (2) is
$$\Li{x} = \li{x}-\li{2}.$$
The logarithmic integral appears in some physical problems 
and in a formulation of the prime number theorem ($\Li{x}$\, gives 
a slightly better approximation for the prime counting function than\, $\li{x}$).

One has the asymptotic series expansion
$$\Li{x} = \frac{x}{\ln{x}}\sum_{n=0}^\infty\frac{n!}{(\ln{x})^n}.$$

The definition of the logarithmic integral may be extended to the whole 
complex plane, and one gets the analytic function \, $\Li{z}$\, having 
the branch point\, $z = 1$\, and the derivative \,$\displaystyle\frac{1}{\log{z}}$.
%%%%%
%%%%%
\end{document}
