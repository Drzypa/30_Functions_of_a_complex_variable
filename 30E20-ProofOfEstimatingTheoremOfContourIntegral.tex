\documentclass[12pt]{article}
\usepackage{pmmeta}
\pmcanonicalname{ProofOfEstimatingTheoremOfContourIntegral}
\pmcreated{2013-03-22 15:46:02}
\pmmodified{2013-03-22 15:46:02}
\pmowner{cvalente}{11260}
\pmmodifier{cvalente}{11260}
\pmtitle{proof of estimating theorem of contour integral}
\pmrecord{22}{37723}
\pmprivacy{1}
\pmauthor{cvalente}{11260}
\pmtype{Proof}
\pmcomment{trigger rebuild}
\pmclassification{msc}{30E20}
\pmclassification{msc}{30A99}

% this is the default PlanetMath preamble.  as your knowledge
% of TeX increases, you will probably want to edit this, but
% it should be fine as is for beginners.

% almost certainly you want these
\usepackage{amssymb}
\usepackage{amsmath}
\usepackage{amsfonts}

% used for TeXing text within eps files
%\usepackage{psfrag}
% need this for including graphics (\includegraphics)
%\usepackage{graphicx}
% for neatly defining theorems and propositions
%\usepackage{amsthm}
% making logically defined graphics
%%%\usepackage{xypic}

% there are many more packages, add them here as you need them

% define commands here
\begin{document}
WLOG consider $g(t):\mathbb{R} \to \mathbb{C}$ a parameterization of the $\gamma$ curve along which the integral is evaluated with $|g'(t)|=1$. This amounts to a canonical parameterization and is always possible.
Since the integral is independent of re-parameterization\footnote{apart from a possible sign change due to exchange of orientation of the path} the result will be completely general.

With this in mind, the contour integral can be explicitly written as

\begin{equation}
\label{integral}
\int_\gamma f(z) dz = \int_0^L f(g(t)) g'(t) dt
\end{equation}

where $L$ is the arc length of the curve $\gamma$.

Consider the set of all continuous functions $[0,L]\to \mathbb{C}$ as a vector space\footnote{axioms are trivial to verify}, we can define an inner product in it via

\begin{equation}
\langle f,g\rangle = \int_0^L f(t)\bar{g}(t) dt
\end{equation}

The axioms are easy to verify:

\begin{itemize}
\item $\langle k_1 a_1+k_2 a_2,a_3\rangle = \int_0^L (k_1 a_1(t) + k_2 a_2(t))\bar{a_3}(t) dt = k_1\langle a_1,a_3 \rangle +k_2\langle a_2,a_3 \rangle$

\item $\langle a,b \rangle = \int_0^L a(t)\bar{b}(t) dt = \int_0^L \overline{b(t)\bar{a}(t)}dt = \overline{\int_0^L b(t)\bar{a}(t) dt} = \overline{\langle b,a \rangle}$

\item $\langle a,a \rangle = \int_0^L a(t)\bar{a}(t) dt = \int_0^L |a(t)|^2 dt \ge 0$ since the integrand is a non-negative (real) function, and $0$ iff $|a|^2=0$ everywhere in the interval, that is: $\langle a,a \rangle=0 \iff a=0$

\end{itemize}

With all this in mind, equation \ref{integral} can be written as

\begin{equation}
\int_\gamma f(z) dz = \langle f\circ g, \bar{g}' \rangle
\end{equation}

Where by definition $\| f \| = \sqrt{<f,f>}$ is the norm associated with the inner product defined previously.

Using Cauchy-Schwarz inequality we can write that

\begin{equation}
|\langle f\circ g, \bar{g}' \rangle| \le \|f\circ g\| \|\bar{g}'\|
\end{equation}

But since by assumption the parameterization $g$ is canonic, $\|\bar{g}'\|=\|g'\|=\sqrt{\int_0^L 1 dt} = \sqrt{L}$.

On the other hand $\|f\circ g\|=\sqrt{\int_0^L f(g(t)) \bar{f}(g(t))dt} \le \sqrt{\int_0^L M^2 dt} = M\sqrt{L}$, where $|f(g(t))|\le M$ for every point on $\gamma$.

The previous paragraphs imply that

\begin{equation}
\left|\int_\gamma f(z) dz\right|\le ML
\end{equation}

which is the result we aimed to prove.

Cauchy-Schwarz inequality says more, it also says that $|\langle a,b \rangle |=\|a\|\|b\| \iff a=\lambda b$ where $\lambda$ is a constant.

So if $|\langle f\circ g, \bar{g}' \rangle =\|f\circ g\|\|\bar{g}'\|$ then $f\circ g = \lambda \bar{g}'$, where $\lambda \in \mathbb{C}$ is a constant.
If $g$ is a canonical parameterization $|g'| = 1$ and we get the absolute modulus  $|\lambda| = |f\circ g|$ (which must be constant) and all that remains is to find the phase of $\lambda$ which must also be constant.

%%%%%
%%%%%
\end{document}
