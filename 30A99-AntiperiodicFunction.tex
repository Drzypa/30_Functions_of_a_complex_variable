\documentclass[12pt]{article}
\usepackage{pmmeta}
\pmcanonicalname{AntiperiodicFunction}
\pmcreated{2015-12-16 15:19:14}
\pmmodified{2015-12-16 15:19:14}
\pmowner{pahio}{2872}
\pmmodifier{pahio}{2872}
\pmtitle{antiperiodic function}
\pmrecord{12}{40324}
\pmprivacy{1}
\pmauthor{pahio}{2872}
\pmtype{Definition}
\pmcomment{trigger rebuild}
\pmclassification{msc}{30A99}
\pmrelated{PeriodicFunctions}
\pmrelated{QuasiperiodicFunction}
\pmdefines{antiperiodicity}
\pmdefines{antiperiodic}
\pmdefines{antiperiod}

% this is the default PlanetMath preamble.  as your knowledge
% of TeX increases, you will probably want to edit this, but
% it should be fine as is for beginners.

% almost certainly you want these
\usepackage{amssymb}
\usepackage{amsmath}
\usepackage{amsfonts}

% used for TeXing text within eps files
%\usepackage{psfrag}
% need this for including graphics (\includegraphics)
%\usepackage{graphicx}
% for neatly defining theorems and propositions
 \usepackage{amsthm}
% making logically defined graphics
%%%\usepackage{xypic}

% there are many more packages, add them here as you need them

% define commands here

\theoremstyle{definition}
\newtheorem*{thmplain}{Theorem}

\begin{document}
A special case of the \PMlinkname{quasiperiodicity}{Period3} of functions is the {\em antiperiodicity}.  
An {\em antiperiodic} function $f$ satisfies for a certain constant $p$ the equation
$$f(z+p) \;=\; -f(z)$$
for all values of the variable $z$.\, The constant $p$ is the {\em antiperiod} of $f$.  Then, $f$ has also other antiperiods,  e.g. $-p$, and generally $(2n\!+\!1)p$ with any\, $n \in \mathbb{Z}$.

The antiperiodic function $f$ is always as well periodic with period $2p$, since
$$f(z+2p) \;=\; f((z+p)+p) \;=\; -f(z+p) \;=\; -(-f(z)) \;=\;
f(z).$$
Naturally, then there are all periods $2np$ with\, $n \in \mathbb{Z}$.

Not all periodic functions are antiperiodic.\\

For example, the sine and cosine functions are antiperiodic with\, $p = \pi$, which is their absolutely least antiperiod:
$$\sin(z+\pi) \;=\; -\sin{z}, \qquad \cos(z+\pi) \;=\; -\cos{z}$$
The \PMlinkname{tangent}{Trigonometry} and cotangent functions are not antiperiodic although they are periodic (with the prime period $\pi$; see complex tangent and cotangent).

The exponential function is antiperiodic with the antiperiod $i\pi$ (see Euler relation):
$$e^{z+i\pi}\;=\; e^z e^{i\pi} \;=\; -e^z$$
%%%%%
%%%%%
\end{document}
