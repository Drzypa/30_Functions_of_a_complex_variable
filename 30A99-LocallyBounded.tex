\documentclass[12pt]{article}
\usepackage{pmmeta}
\pmcanonicalname{LocallyBounded}
\pmcreated{2013-03-22 14:17:47}
\pmmodified{2013-03-22 14:17:47}
\pmowner{jirka}{4157}
\pmmodifier{jirka}{4157}
\pmtitle{locally bounded}
\pmrecord{9}{35752}
\pmprivacy{1}
\pmauthor{jirka}{4157}
\pmtype{Definition}
\pmcomment{trigger rebuild}
\pmclassification{msc}{30A99}
\pmclassification{msc}{54-00}

\endmetadata

% this is the default PlanetMath preamble.  as your knowledge
% of TeX increases, you will probably want to edit this, but
% it should be fine as is for beginners.

% almost certainly you want these
\usepackage{amssymb}
\usepackage{amsmath}
\usepackage{amsfonts}

% used for TeXing text within eps files
%\usepackage{psfrag}
% need this for including graphics (\includegraphics)
%\usepackage{graphicx}
% for neatly defining theorems and propositions
\usepackage{amsthm}
% making logically defined graphics
%%%\usepackage{xypic}

% there are many more packages, add them here as you need them

% define commands here
\theoremstyle{theorem}
\newtheorem*{thm}{Theorem}
\newtheorem*{lemma}{Lemma}
\newtheorem*{conj}{Conjecture}
\newtheorem*{cor}{Corollary}
\newtheorem*{example}{Example}
\theoremstyle{definition}
\newtheorem*{defn}{Definition}
\begin{document}
Suppose that $X$ is a topological space and $Y$ a metric space.  

\begin{defn}
A set ${\mathcal{F}}$
of functions $f\colon X \to Y$ is said to be {\em locally bounded} if for
every $x \in X$, there exists a neighbourhood $N$ of $x$ such that ${\mathcal{F}}$ is uniformly bounded on $N$.
\end{defn}

In the special case of functions on the complex plane where it
is often used, the definition can be given as follows.

\begin{defn}
A set ${\mathcal{F}}$ of functions $f\colon G \subset {\mathbb{C}} \to {\mathbb{C}}$ is said to be {\em locally bounded} if for every $a \in G$
there exist constants $\delta > 0$ and $M > 0$ such that for all
$z \in G$ such that $\lvert z-a \rvert < \delta$, $\lvert f(z) \rvert < M$ for all $f \in {\mathcal{F}}$.
\end{defn}

As an example we can look at the set ${\mathcal{F}}$ of entire functions where
$f(z) = z^2 + t$ for any $t \in [0,1]$.  Obviously each such $f$ is unbounded
itself, however if we take a small neighbourhood around any point we can
bound all $f \in {\mathcal{F}}$.  Say on an open ball $B(z_0,1)$ we can show
by triangle inequality that $\lvert f(z) \rvert \leq (\lvert z_0 \rvert +1)^2 + 1$
for all $z \in B(z_0,1)$.  So this set of functions is locally bounded.

Another example would be say the set of all analytic functions from
some region $G$ to the unit disc.  All those functions are bounded by 1,
and so we have a uniform bound even over all of $G$.

As a counterexample suppose the we take the constant functions $f_n(z) = n$ for
all natural numbers $n$.  While each of these functions is itself bounded,
we can never find a uniform bound for all such functions.

\begin{thebibliography}{9}
\bibitem{Conway:complexI}
John~B. Conway.
{\em \PMlinkescapetext{Functions of One Complex Variable I}}.
Springer-Verlag, New York, New York, 1978.
\end{thebibliography}
%%%%%
%%%%%
\end{document}
