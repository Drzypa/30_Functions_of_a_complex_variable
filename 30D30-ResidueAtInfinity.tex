\documentclass[12pt]{article}
\usepackage{pmmeta}
\pmcanonicalname{ResidueAtInfinity}
\pmcreated{2013-03-22 19:15:00}
\pmmodified{2013-03-22 19:15:00}
\pmowner{pahio}{2872}
\pmmodifier{pahio}{2872}
\pmtitle{residue at infinity}
\pmrecord{9}{42175}
\pmprivacy{1}
\pmauthor{pahio}{2872}
\pmtype{Definition}
\pmcomment{trigger rebuild}
\pmclassification{msc}{30D30}
\pmrelated{Residue}
\pmrelated{RegularAtInfinity}

\endmetadata

% this is the default PlanetMath preamble.  as your knowledge
% of TeX increases, you will probably want to edit this, but
% it should be fine as is for beginners.

% almost certainly you want these
\usepackage{amssymb}
\usepackage{amsmath}
\usepackage{amsfonts}

% used for TeXing text within eps files
%\usepackage{psfrag}
% need this for including graphics (\includegraphics)
%\usepackage{graphicx}
% for neatly defining theorems and propositions
 \usepackage{amsthm}
% making logically defined graphics
%%%\usepackage{xypic}

% there are many more packages, add them here as you need them

% define commands here

\theoremstyle{definition}
\newtheorem*{thmplain}{Theorem}

\begin{document}
\PMlinkescapeword{order}

If in the Laurent expansion
\begin{align}
f(z) \;=\; \sum_{k=-\infty}^\infty c_kz^k
\end{align}
of the function $f$, the coefficient $c_n$ is distinct from zero ($n > 0$) and\, $c_{n+1} = c_{n+2} = \ldots = 0$,\, then there exists the numbers $M$ and $K$ such that
$$|z^{-n}f(z)| \;<\; M \quad \mbox{always when} \quad |z| \;>\; K.$$
In this case one says that $\infty$ is a \emph{pole of order} $n$ of the function $f$ (cf. zeros and poles of rational function).

If there is no such positive integer $n$, (1) \PMlinkescapetext{contains} infinitely many positive powers of $z$, and one may say that $\infty$ is an \emph{essential singularity} of $f$.\\

In both cases one can define for $f$ the \emph{residue at infinity} as
\begin{align}
\frac{1}{2i\pi}\!\oint_C\!f(z)\,dz \;=\; c_{-1},
\end{align}
where the integral is taken along a closed contour $C$ which goes once anticlockwise around the origin, i.e. once clockwise around the point \,$z = \infty$ (see the Riemann sphere).

Then the usual form 
$$\frac{1}{2i\pi}\!\oint_C\!f(z)\,dz \;=\; \sum_j\mbox{Res}(f;\,a_j)$$
of the residue theorem may be expressed as follows:\\

\emph{The sum of all residues of an analytic function having only a finite number of points of singularity is equal to zero.}


\begin{thebibliography}{8}
\bibitem{lindelof}{\sc Ernst Lindel\"of}: {\em Le calcul des r\'esidus et ses applications \`a la th\'eorie des fonctions}.\, Gauthier-Villars, Paris (1905).
\end{thebibliography}
%%%%%
%%%%%
\end{document}
