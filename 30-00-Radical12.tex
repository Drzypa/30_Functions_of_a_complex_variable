\documentclass[12pt]{article}
\usepackage{pmmeta}
\pmcanonicalname{Radical12}
\pmcreated{2013-03-22 17:06:00}
\pmmodified{2013-03-22 17:06:00}
\pmowner{Wkbj79}{1863}
\pmmodifier{Wkbj79}{1863}
\pmtitle{radical}
\pmrecord{22}{39396}
\pmprivacy{1}
\pmauthor{Wkbj79}{1863}
\pmtype{Definition}
\pmcomment{trigger rebuild}
\pmclassification{msc}{30-00}
\pmclassification{msc}{12D99}
\pmclassification{msc}{00A05}
\pmsynonym{radical symbol}{Radical12}
\pmrelated{Radical5}
\pmrelated{EvenEvenOddRule}
\pmrelated{NthRoot}
\pmdefines{radicand}
\pmdefines{index}
\pmdefines{radical expression}

\endmetadata

\usepackage{amssymb}
\usepackage{amsmath}
\usepackage{amsfonts}

\usepackage{psfrag}
\usepackage{graphicx}
\usepackage{amsthm}
%%\usepackage{xypic}

\begin{document}
The \PMlinkescapetext{symbol}\; $\sqrt{\hspace{3mm}}$\; is called a \emph{radical}.  The \PMlinkescapetext{origin} of the radical symbol is the first letter ``{\em r}'' of the Latin \PMlinkescapetext{word} {\em radix}, which \PMlinkescapetext{means} ``\PMlinkname{root}{NthRoot}''.

There is other terminology that is related to radicals:  In the expression $\sqrt[3]{x}$, $x$ is the \emph{radicand} and $3$ is the \emph{index}.

The index of a radical is usually an integer at least as large as two; however, if the index is two, it is usually not written.  Moreover, if a radical does not have a written index, it is understood to be two, in which case the radical denotes a square root.

A \emph{radical expression} is an expression that involves a radical.

One can use fractional powers instead of the radical symbol via $\sqrt[n]{x} = x^{\frac{1}{n}}$.  This practice of avoiding the radical symbol may be \PMlinkescapetext{necessary} due to typesetting issues.
%%%%%
%%%%%
\end{document}
