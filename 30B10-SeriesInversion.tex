\documentclass[12pt]{article}
\usepackage{pmmeta}
\pmcanonicalname{SeriesInversion}
\pmcreated{2013-03-22 15:39:09}
\pmmodified{2013-03-22 15:39:09}
\pmowner{stevecheng}{10074}
\pmmodifier{stevecheng}{10074}
\pmtitle{series inversion}
\pmrecord{8}{37585}
\pmprivacy{1}
\pmauthor{stevecheng}{10074}
\pmtype{Derivation}
\pmcomment{trigger rebuild}
\pmclassification{msc}{30B10}

\endmetadata

\usepackage{amssymb}
\usepackage{amsmath}
\usepackage{amsfonts}
\begin{document}
The method of \emph{series inversion} allows us to derive the power series
of an inverse function $f^{-1}$ given the power series of $f$.

Clearly, since we are representing function $f$ and $f^{-1}$ by power series,
the function $f$ must necessarily be holomorphic; that is, it is differentiable
as a complex-valued function on an open subset of the complex plane.
(It follows that $f^{-1}$ must also be holomorphic.)
Holomorphic functions include the elementary functions studied in calculus
such as $\sin$, $\cos$, $\tan$ and $\exp$.

For the method to work smoothly, it is best to assume that we want
to invert $f$ at around the origin, and its value there is zero.
There is no loss of generality, since
if $f(b) = a$, we can apply series inversion to the function $g$ defined
by $f(z) = a + g(z-b)$.
Then $f^{-1}(w) = g^{-1}(w-a) + b$;
we obtain the power series for $f^{-1}$ centred at $a$.

Also, it must be true that $f'(0) \neq 0$, otherwise $f$ will not even
be invertible around the origin.

\section*{An example}

We explain the method by an example, for $f(z) = \arctan z$.
In the following, we will consistently use the notation $O(z^n)$
to denote a holomorphic function $h(z)$ whose power series expansion begins
with $z^n$.  And similarly when $z$ is replaced with the variable $w$.

First, we start with the well-known power series expansion for $w = \arctan z$:
\begin{equation}
\label{eq:arctan}
w = z - \frac{z^3}{3} + \frac{z^5}{5} + O(z^7)\,.
\end{equation}
The number of explicit terms in the power series expansion determines
the number of terms that we will be able to obtain
in the power series expansion of $f^{-1}$.
So in this case, we are going to seek an expansion
of $f^{-1}(w) = \tan w$ up to (but excluding) the $w^7$ term.

A simple rearrangement of \eqref{eq:arctan} gives
\begin{equation}
\label{eq:z}
z = w + \frac{z^3}{3} - \frac{z^5}{5} + O(z^7)
\end{equation}

Now we \emph{substitute equation \eqref{eq:z} into itself}.
Of course, usually when we substitute an equation into itself
we do not get anything, but here it works because we can 
\emph{perform simplications using the $O$ notation}.
So for instance, in the following,
in second term $z^3/3$ on the right of equation \eqref{eq:z}, we put in
equation \eqref{eq:z} simplified to $z = w + O(z^3)$.
Why we should choose this simplication will be clear in a moment:
\begin{align}
\label{eq:z3}
z &= w + \frac{1}{3} \Bigl( w + O(z^3) \Bigr)^3 - \frac{z^5}{5} + O(z^7) \\
&= w + \frac{1}{3} \Bigl( w^3 + O(z^3) (w^2 + \dotsb) \Bigr) - \frac{z^5}{5} + O(z^7) \\
\label{eq:Ow3}
&= w + \frac{1}{3} \Bigl( w^3 + O(w^3) \cdot O(w^2) \Bigr) + O(w^5)\,.
\end{align}
In equation \eqref{eq:Ow3} we used the fact that the expansion
for $z = f^{-1}(w)$ \emph{must begin} with a $w$ term, i.e.
$f^{-1}(0) = 0$.  Also note that we are guaranteed that the $w$ and $z$ terms
have non-zero coefficients, because $f'(0) \neq 0$.  (Otherwise
we would not be able to isolate $z$ in equation \eqref{eq:z}.)

Now equation \eqref{eq:Ow3} simplifies to
\begin{equation}
\label{eq:Ow5}
z = w + \frac{w^3}{3} + O(w^5)\,,
\end{equation}
which is already an achievement, because we have identified
exactly what the $w^3$ term must be.

We substitute \eqref{eq:Ow5} into the $z^3$ and $z^5$ terms of \eqref{eq:z},
and obtain:
\begin{align}
z &= w + \frac{1}{3} \left( w + \frac{w^3}{3} + O(w^5) \right)^3 - \frac{1}{5} \left( w + \frac{w^3}{3} + O(w^5) \right)^5 + O(z^7) \\
&= w + \frac{1}{3} \left( w^3 + \binom{3}{1} \frac{w^3}{3} w^2 + O(w^7) \right) - \frac{1}{5} \left( w^5 + O(w^7) \right) + O(w^7)  \\
&= w + \frac{1}{3} w^3 + \frac{2}{15} w^5 + O(w^7)\,.
\end{align}
And this gives our desired expansion of $z = \tan w$ 
of degree $<7$.

\section*{Summary}

To summarize the procedure in general,
we start with the expansion
\begin{equation}
\label{eq:f}
w = f(z) = a_1 z + a_2 z^2 + a_3 z^3 + \dotsb \,, \quad a_1 \neq 0\,,
\end{equation}
and rearrange it to,
\begin{equation}
\label{eq:z-general}
z = b_1 \Bigl( w - a_2 z^2 - a_3 z^3 - \dotsb - a_n z^n \Bigr) + O(w^{n+1})\,, \quad b_1 = \frac{1}{a_1}\,.
\end{equation}
So we know that $z = b_1 w + O(w^2)$,
and we can substitute
this into the term $z^2$ of equation \eqref{eq:z-general}.
At the end we will get an equation of the form
$z = b_1 w + b_2 w^2 + O(w^3)$, and we can substitute this
into the terms $z^2$ and $z^3$ of \eqref{eq:z-general}.
And what ever results we will substitute back into the 
terms $z^2$, $z^3$, $z^4$ of equation \eqref{eq:z-general}.
We can repeat this process until 
we have all the terms of $z = f^{-1}(w)$ that we need.

We probably should normalize
the functions so that $a_1 = b_1 = 1$
to make the computations easier.

\begin{thebibliography}{XXXXX}

\bibitem{Ahlfors}
Lars V. Ahlfors. {\it Complex Analysis}. McGraw-Hill, 1979.

\end{thebibliography}
%%%%%
%%%%%
\end{document}
