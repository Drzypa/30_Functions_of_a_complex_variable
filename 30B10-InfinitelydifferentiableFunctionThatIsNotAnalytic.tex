\documentclass[12pt]{article}
\usepackage{pmmeta}
\pmcanonicalname{InfinitelydifferentiableFunctionThatIsNotAnalytic}
\pmcreated{2013-03-22 12:46:15}
\pmmodified{2013-03-22 12:46:15}
\pmowner{ariels}{338}
\pmmodifier{ariels}{338}
\pmtitle{infinitely-differentiable function that is not analytic}
\pmrecord{5}{33081}
\pmprivacy{1}
\pmauthor{ariels}{338}
\pmtype{Example}
\pmcomment{trigger rebuild}
\pmclassification{msc}{30B10}
\pmclassification{msc}{26A99}

% this is the default PlanetMath preamble.  as your knowledge
% of TeX increases, you will probably want to edit this, but
% it should be fine as is for beginners.

% almost certainly you want these
\usepackage{amssymb}
\usepackage{amsmath}
\usepackage{amsfonts}

% used for TeXing text within eps files
%\usepackage{psfrag}
% need this for including graphics (\includegraphics)
%\usepackage{graphicx}
% for neatly defining theorems and propositions
%\usepackage{amsthm}
% making logically defined graphics
%%%\usepackage{xypic}

% there are many more packages, add them here as you need them

% define commands here

\newcommand{\Prob}[2]{\mathbb{P}_{#1}\left\{#2\right\}}
\newcommand{\Expect}{\mathbb{E}}
\newcommand{\norm}[1]{\left\|#1\right\|}

% Some sets
\newcommand{\Nats}{\mathbb{N}}
\newcommand{\Ints}{\mathbb{Z}}
\newcommand{\Reals}{\mathbb{R}}
\newcommand{\Complex}{\mathbb{C}}



%%%%%% END OF SAVED PREAMBLE %%%%%%
\usepackage{amsthm}
\begin{document}
If $f\in\mathcal{C}^{\infty}$, then we can certainly \emph{write} a Taylor series for $f$.  However, analyticity requires that this Taylor series actually converge (at least across some radius of convergence) to $f$.  It is not necessary that the power series for $f$ converge to $f$, as the following example shows.

\newcommand{\e}{e^{-\frac{1}{x^2}}}
Let
$$
f(x)=\begin{cases}
\e & x \ne 0 \\
0 & x = 0
\end{cases}.
$$
Then $f\in \mathcal{C}^{\infty}$, and for any $n\ge 0$, $f^{(n)}(0)=0$ (see below).  So the Taylor series for $f$ around 0 is 0; since $f(x)>0$ for all $x\ne 0$, clearly it does not converge to $f$.

\subsection*{Proof that $f^{(n)}(0)=0$}

Let $p(x), q(x)\in \Reals[x]$ be polynomials, and define
$$
g(x)=\frac{p(x)}{q(x)} \cdot f(x).
$$
Then, for $x\ne 0$,
$$
g'(x) = \frac{(p'(x) + p(x)\frac{2}{x^3})q(x) - q'(x)p(x)}{q^2(x)} \cdot\e.
$$
Computing (e.g. by applying \PMlinkname{L'H\^opital's rule}{LHpitalsRule}), we see that $g'(0)=\lim_{x\to 0}g'(x)=0$.

Define $p_0(x)=q_0(x)=1$.  Applying the above inductively, we see that we may write $f^{(n)}(x)=\frac{p_n(x)}{q_n(x)}f(x)$.  So $f^{(n)}(0)=0$, as required.
%%%%%
%%%%%
\end{document}
