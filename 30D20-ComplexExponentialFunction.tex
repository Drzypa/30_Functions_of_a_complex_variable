\documentclass[12pt]{article}
\usepackage{pmmeta}
\pmcanonicalname{ComplexExponentialFunction}
\pmcreated{2013-03-22 14:43:08}
\pmmodified{2013-03-22 14:43:08}
\pmowner{pahio}{2872}
\pmmodifier{pahio}{2872}
\pmtitle{complex exponential function}
\pmrecord{22}{36341}
\pmprivacy{1}
\pmauthor{pahio}{2872}
\pmtype{Definition}
\pmcomment{trigger rebuild}
\pmclassification{msc}{30D20}
\pmclassification{msc}{32A05}
\pmrelated{ExponentialFunctionDefinedAsLimitOfPowers}
\pmrelated{ExponentialFunction}
\pmrelated{ComplexSineAndCosine}
\pmrelated{ProofOfEquivalenceOfFormulasForExp}
\pmrelated{DerivativeOfExponentialFunction}
\pmrelated{ConvergenceOfRiemannZetaSeries}
\pmdefines{exponential function}
\pmdefines{prime period}

% this is the default PlanetMath preamble.  as your knowledge
% of TeX increases, you will probably want to edit this, but
% it should be fine as is for beginners.

% almost certainly you want these
\usepackage{amssymb}
\usepackage{amsmath}
\usepackage{amsfonts}

% used for TeXing text within eps files
%\usepackage{psfrag}
% need this for including graphics (\includegraphics)
%\usepackage{graphicx}
% for neatly defining theorems and propositions
%\usepackage{amsthm}
% making logically defined graphics
%%%\usepackage{xypic}

% there are many more packages, add them here as you need them

% define commands here
\begin{document}
The {\em complex exponential function}\,\, $\exp:\,\mathbb{C}\to \mathbb{C}$\, may be defined in many equivalent ways:\, Let\, $z = x\!+\!iy$\, where\, $x,\,y\in\mathbb{R}$.
\begin{itemize}
 \item $\displaystyle\exp{z} \;:=\; e^x(\cos{y}+i\sin{y})$
 \item $\displaystyle\exp{z} \;:=\; \lim_{n\to\infty}\left(1+\frac{z}{n}\right)^n$
 \item $\displaystyle\exp{z} \;:=\; \sum_{n = 0}^\infty\frac{z^n}{n!}$
\end{itemize}
The complex exponential function is usually denoted in power form:
                          $$e^z \;:=\; \exp{z},$$
where $e$ is the Napier's constant.\, It also coincincides with the real exponential function when $z$ is real (choose\,  $y = 0$).\, It has  all the properties of power, e.g.\, $e^{-z} = \frac{1}{e^z}$;\, these are consequences of the addition formula
                     $$e^{z_1+z_2} \;=\; e^{z_1}e^{z_2}$$
of the complex exponential function.

The function gets all complex values except 0 and is \PMlinkname{periodic}{PeriodicityOfExponentialFunction} having the \PMlinkescapetext{{\em prime period}} (the \PMlinkescapetext{period} with least non-zero modulus) $2\pi i$.\, The $\exp$ is holomorphic, its derivative 
               $$\frac{d}{dz}e^z \;=\; e^z,$$
which is obtained from the series form via termwise differentiation, is similar as in $\mathbb{R}$.

So we have a fourth way to define 
\begin{itemize}
 \item $\exp{z} \;:=\; w(z)$ 
\end{itemize}
with $w$ the solution of the differential equation \,$\displaystyle\frac{dw}{dz} = w$\, under the initial condition\, $w(0) = 1$.

\textbf{Some formulae:}
 $$|e^z| \;=\; e^x, \quad \arg{e^z} \;=\; y+2n\pi\quad(n = 0,\,\pm1,\,\pm2,\,\ldots),$$
 $$\mbox{Re}(e^z) \;=\; e^x\cos{y}, \quad \mbox{Im}(e^z) \;=\; e^x\sin{y}$$
%%%%%
%%%%%
\end{document}
