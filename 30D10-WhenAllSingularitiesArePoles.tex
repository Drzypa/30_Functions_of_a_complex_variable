\documentclass[12pt]{article}
\usepackage{pmmeta}
\pmcanonicalname{WhenAllSingularitiesArePoles}
\pmcreated{2014-11-21 21:30:22}
\pmmodified{2014-11-21 21:30:22}
\pmowner{pahio}{2872}
\pmmodifier{pahio}{2872}
\pmtitle{when all singularities are poles}
\pmrecord{17}{40043}
\pmprivacy{1}
\pmauthor{pahio}{2872}
\pmtype{Theorem}
\pmcomment{trigger rebuild}
\pmclassification{msc}{30D10}
\pmclassification{msc}{30C15}
\pmclassification{msc}{30A99}
%\pmkeywords{poles only}
\pmrelated{RiemannSphere}
\pmrelated{ZeroesOfAnalyticFunctionsAreIsolated}
\pmrelated{Meromorphic}

\endmetadata

% this is the default PlanetMath preamble.  as your knowledge
% of TeX increases, you will probably want to edit this, but
% it should be fine as is for beginners.

% almost certainly you want these
\usepackage{amssymb}
\usepackage{amsmath}
\usepackage{amsfonts}

% used for TeXing text within eps files
%\usepackage{psfrag}
% need this for including graphics (\includegraphics)
%\usepackage{graphicx}
% for neatly defining theorems and propositions
 \usepackage{amsthm}
% making logically defined graphics
%%%\usepackage{xypic}

% there are many more packages, add them here as you need them

% define commands here

\theoremstyle{definition}
\newtheorem*{thmplain}{Theorem}

\begin{document}
In the \PMlinkname{parent entry}{ZerosAndPolesOfRationalFunction} we see that a rational function has as its only singularities a finite set of poles.  It is also valid the converse

\textbf{Theorem.}  Any single-valued analytic function, which has in the whole closed complex plane no other singularities than poles, is a rational function.\\

{\em Proof.}  Suppose that\, $z\mapsto w(z)$\, is such an analytic function.  The number of the poles of $w$ must be finite, since otherwise the set of the poles would have in the closed complex plane an accumulation point which is neither a \PMlinkname{point of regularity}{Holomorphic} nor a pole.  Let $b_1,\,b_2,\,\ldots,\,b_k$ and possibly $\infty$ be the poles of the function $w$.

For every\, $i = 1,\,2,\,\ldots,\,k$,\, the function has at the pole $b_i$ with the order $n_i$, the Laurent expansion of the form
\begin{align}
  w(z) = \frac{c_{-n_i}}{(z-b_i)^{n_i}}+\frac{c_{-n_i+1}}{(z-b_i)^{n_i-1}}+\ldots+c_0+c_1(z-b_i)+\ldots
\end{align}
This is in \PMlinkescapetext{force} in the greatest open disc containing no other poles.  We write (1) as
\begin{align}
  w(z) = F_{n_i}\!\left(\frac{1}{z-b_i}\right)+P(z-b_i),
\end{align}
where the first addend is the principal part of (1), i.e. consists of the terms of (1) which become infinite in\, $z = b_i$.

If we think a circle having center in the origin and containing all the finite poles $b_i$ (an annulus\, 
$\varrho < |z| < \infty$), then $w(z)$ has outside it the Laurent series expansion
$$w(z) = d_mz^m+d_{m-1}z^{m-1}+\ldots+d_0+\frac{d_{-1}}{z}+\ldots,$$
which we write, corresponding to (2), as
\begin{align}
  w(z) = G_m(z)+Q\!\left(\frac{1}{z}\right)\!,
\end{align}
where $G_m(z)$ is a polynomial of $z$ and $Q\left(\frac{1}{z}\right)$ a power series in $\frac{1}{z}$.  Then the equation
$$R(z)\, := \sum_{i=1}^kF_{n_i}\!\left(\frac{1}{z-b_i}\right)+G_m(z)$$
defines a rational function having the same poles as $w$.  Therefore the function defined by
$$f(z)\, :=\, w(z)-R(z)$$
is \PMlinkname{analytic}{Analytic} everywhere except possibly at the points\, $z = b_i$\, and\, $z = \infty$.\, If we write
$$f(z) = 
\left[w(z)-F_{n_i}\!\left(\frac{1}{z-b_i}\right)\right]-\sum_{j\neq i}F_{n_j}\!\left(\frac{1}{z-b_j}\right)-G_m(z),$$
we see that $f(z)$ is bounded in a neighbourhood of the point $b_i$ and is analytic also in this point ($i = 1,\,2,\,\ldots,\,k$).  But then again, the \PMlinkescapetext{presentation} 
$$f(z) = 
\left[w(z)-G_m(z)\right]-\sum_{j=1}^kF_{n_j}\!\left(\frac{1}{z-b_j}\right)$$
shows that $f$ is analytic in the \PMlinkname{infinity}{RiemannSphere}, too.  Thus $f$ is analytic in the whole closed complex plane.  By Liouville's theorem, $f$ is a constant function.\, We conclude that\, $R(z)+f(z) = w(z)$\, is a rational function. Q.E.D.\\

The theorem implies, that if a meromorphic function is regular at infinity or has there a pole, then it is a rational function.

\begin{thebibliography}{9}
\bibitem{NP}{\sc R. Nevanlinna \& V. Paatero}: {\em Funktioteoria}.\, Kustannusosakeyhti\"o Otava, Helsinki (1963).
\end{thebibliography}

%%%%%
%%%%%
\end{document}
