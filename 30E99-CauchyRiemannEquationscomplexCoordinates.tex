\documentclass[12pt]{article}
\usepackage{pmmeta}
\pmcanonicalname{CauchyRiemannEquationscomplexCoordinates}
\pmcreated{2013-03-22 14:24:28}
\pmmodified{2013-03-22 14:24:28}
\pmowner{jirka}{4157}
\pmmodifier{jirka}{4157}
\pmtitle{Cauchy-Riemann equations (complex coordinates)}
\pmrecord{6}{35909}
\pmprivacy{1}
\pmauthor{jirka}{4157}
\pmtype{Definition}
\pmcomment{trigger rebuild}
\pmclassification{msc}{30E99}
\pmrelated{CauchyRiemannEquations}
\pmrelated{Holomorphic}

% this is the default PlanetMath preamble.  as your knowledge
% of TeX increases, you will probably want to edit this, but
% it should be fine as is for beginners.

% almost certainly you want these
\usepackage{amssymb}
\usepackage{amsmath}
\usepackage{amsfonts}

% used for TeXing text within eps files
%\usepackage{psfrag}
% need this for including graphics (\includegraphics)
%\usepackage{graphicx}
% for neatly defining theorems and propositions
\usepackage{amsthm}
% making logically defined graphics
%%%\usepackage{xypic}

% there are many more packages, add them here as you need them

% define commands here
\begin{document}
Let $f \colon G \subset {\mathbb{C}} \to {\mathbb{C}}$ be a continuously differentiable function in the real sense, using ${\mathbb{R}}^2$ instead of
${\mathbb{C}}$, identifying $f(z)$ with $f(x,y)$ where $z = x + iy$ and we also write $\bar{z} = x - iy$ (the complex conjugate).  Then we have the following partial derivatives:
\begin{align*}
\frac{\partial f}{\partial z} & :=
\frac{1}{2} \left(
\frac{\partial f}{\partial x} - i \frac{\partial f}{\partial y}
\right) ,
\\
\frac{\partial f}{\partial \bar{z}} & :=
\frac{1}{2} \left(
\frac{\partial f}{\partial x} + i \frac{\partial f}{\partial y}
\right) .
\end{align*}
Sometimes these are written as $f_z$ and $f_{\bar{z}}$ respectively.

The classical Cauchy-Riemann equations are equivalent to
\begin{equation*}
\frac{\partial f}{\partial \bar{z}} = 0 .
\end{equation*}
This can be seen if we write $f = u+iv$ for real valued $u$ and $v$ and
then the differentials become
\begin{align*}
\frac{\partial f}{\partial z} & =
\frac{1}{2} \left(
\frac{\partial u}{\partial x} + \frac{\partial v}{\partial y}
\right)
+
\frac{i}{2} \left(
\frac{\partial v}{\partial x} - \frac{\partial u}{\partial y}
\right) ,
\\
\frac{\partial f}{\partial \bar{z}} & =
\frac{1}{2} \left(
\frac{\partial u}{\partial x} - \frac{\partial v}{\partial y}
\right)
+
\frac{i}{2} \left(
\frac{\partial v}{\partial x} + \frac{\partial u}{\partial y}
\right) .
\end{align*}


In several complex dimensions, for a function
$f \colon G \subset {\mathbb{C}}^n \to {\mathbb{C}}$ which maps
$(z_1,\ldots,z_n) \mapsto f(z_1,\ldots,z_n)$ where $z_j = x_j + i y_j$ we generalize simply by
\begin{align*}
\frac{\partial f}{\partial z_j} & :=
\frac{1}{2} \left(
\frac{\partial f}{\partial x_j} - i \frac{\partial f}{\partial y_j}
\right) ,
\\
\frac{\partial f}{\partial \bar{z}_j} & :=
\frac{1}{2} \left(
\frac{\partial f}{\partial x_j} + i \frac{\partial f}{\partial y_j}
\right) .
\end{align*}

Then the Cauchy-Riemann equations are given by
\begin{equation*}
\frac{\partial f}{\partial \bar{z}_j} = 0 \qquad \text{for all $1 \leq j \leq n$} .
\end{equation*}
That is, $f$ is holomorphic if and only if it satisfies the above equations.

\begin{thebibliography}{9}
\bibitem{Krantz:several}
Steven~G.\@ Krantz.
{\em \PMlinkescapetext{Function Theory of Several Complex Variables}},
AMS Chelsea Publishing, Providence, Rhode Island, 1992.
\end{thebibliography}
%%%%%
%%%%%
\end{document}
