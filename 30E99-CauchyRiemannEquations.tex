\documentclass[12pt]{article}
\usepackage{pmmeta}
\pmcanonicalname{CauchyRiemannEquations}
\pmcreated{2013-03-22 12:55:36}
\pmmodified{2013-03-22 12:55:36}
\pmowner{rmilson}{146}
\pmmodifier{rmilson}{146}
\pmtitle{Cauchy-Riemann equations}
\pmrecord{5}{33281}
\pmprivacy{1}
\pmauthor{rmilson}{146}
\pmtype{Definition}
\pmcomment{trigger rebuild}
\pmclassification{msc}{30E99}
\pmrelated{Holomorphic}

\endmetadata

\usepackage{amsmath}
\usepackage{amsfonts}
\usepackage{amssymb}
\newcommand{\reals}{\mathbb{R}}
\newcommand{\natnums}{\mathbb{N}}
\newcommand{\cnums}{\mathbb{C}}
\newcommand{\znums}{\mathbb{Z}}
\newcommand{\lp}{\left(}
\newcommand{\rp}{\right)}
\newcommand{\lb}{\left[}
\newcommand{\rb}{\right]}
\newcommand{\supth}{^{\text{th}}}
\newtheorem{proposition}{Proposition}
\newtheorem{definition}[proposition]{Definition}
\newcommand{\nl}[1]{{\PMlinkescapetext{{#1}}}}
\newcommand{\pln}[2]{{\PMlinkname{{#1}}{#2}}}
\begin{document}
The following system of partial differential
equations 
$$
\frac{\partial u}{\partial x} = \frac{\partial v}{\partial y},\quad
\frac{\partial u}{\partial y} = -\frac{\partial v}{\partial x},
$$
where $u(x,y), v(x,y)$ are real-valued functions defined on some
open subset of $\reals^2$, was introduced by Riemann[1] as a
definition of a holomorphic function.  Indeed, if $f(z)$ satisfies the
standard definition of a holomorphic function, i.e. if the
complex derivative
$$f'(z) = \lim_{\zeta\rightarrow 0} \frac{f(z+\zeta)-f(z)}{\zeta}$$
exists in the domain of definition, then the real and imaginary parts
of $f(z)$
satisfy the Cauchy-Riemann equations.
Conversely, if $u$ and $v$ satisfy the Cauchy-Riemann equations, and if their
partial derivatives are continuous, then the complex valued function
$$f(z) = u(x,y) + i v(x,y),\quad z=x+i y,$$
possesses a continuous complex derivative.

\paragraph{References}

\begin{enumerate}
\item D. Laugwitz, \emph{Bernhard Riemann, 1826-1866:
 Turning points in the Conception of
 Mathematics}, translated by Abe Shenitzer. Birkhauser, 1999.
\end{enumerate}
%%%%%
%%%%%
\end{document}
