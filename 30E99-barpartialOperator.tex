\documentclass[12pt]{article}
\usepackage{pmmeta}
\pmcanonicalname{barpartialOperator}
\pmcreated{2013-03-22 15:10:39}
\pmmodified{2013-03-22 15:10:39}
\pmowner{jirka}{4157}
\pmmodifier{jirka}{4157}
\pmtitle{$\bar{\partial}$ operator}
\pmrecord{7}{36931}
\pmprivacy{1}
\pmauthor{jirka}{4157}
\pmtype{Definition}
\pmcomment{trigger rebuild}
\pmclassification{msc}{30E99}
\pmclassification{msc}{32A99}
\pmsynonym{d bar operator}{barpartialOperator}
\pmsynonym{d-bar operator}{barpartialOperator}
\pmdefines{$\partial$ operator}

\endmetadata

% this is the default PlanetMath preamble.  as your knowledge
% of TeX increases, you will probably want to edit this, but
% it should be fine as is for beginners.

% almost certainly you want these
\usepackage{amssymb}
\usepackage{amsmath}
\usepackage{amsfonts}

% used for TeXing text within eps files
%\usepackage{psfrag}
% need this for including graphics (\includegraphics)
%\usepackage{graphicx}
% for neatly defining theorems and propositions
\usepackage{amsthm}
% making logically defined graphics
%%%\usepackage{xypic}

% there are many more packages, add them here as you need them

% define commands here
\theoremstyle{theorem}
\newtheorem*{thm}{Theorem}
\newtheorem*{lemma}{Lemma}
\newtheorem*{conj}{Conjecture}
\newtheorem*{cor}{Corollary}
\newtheorem*{example}{Example}
\newtheorem*{prop}{Proposition}
\theoremstyle{definition}
\newtheorem*{defn}{Definition}
\theoremstyle{remark}
\newtheorem*{rmk}{Remark}
\begin{document}
Let $G \subset {\mathbb{C}}^n$ be a domain and let
$f \colon G \to {\mathbb{C}}$ be a $C^1$
function (continuously differentiable)
$(z^1,\ldots,z^n) \mapsto f(z^1,\ldots,z^n)$ where $z^j = x^j + i y^j$.
We can think of $G$ as a subset of ${\mathbb{R}}^{2n}$.
We therefore
have the following partial derivatives for all $1 \leq j \leq n$,
\begin{align*}
\frac{\partial f}{\partial z^j} & :=
\frac{1}{2} \left(
\frac{\partial f}{\partial x^j} - i \frac{\partial f}{\partial y^j}
\right) ,
\\
\frac{\partial f}{\partial \bar{z}^j} & :=
\frac{1}{2} \left(
\frac{\partial f}{\partial x^j} + i \frac{\partial f}{\partial y^j}
\right) .
\end{align*}
Now let $d$ be the standard exterior derivative on
${\mathbb{R}}^{2n}$ and the $dx^j$ and $dy^j$ the standard basis of cotangent
vectors.  Then if we define
\begin{align*}
dz^j & := dx^j + i dy^j , \\
d\bar{z}^j & := dx^j - i dy^j ,
\end{align*}
then we can define two new operators acting on $C^1$ functions on $G$
giving 1-forms by
\begin{align*}
\partial f & := \sum_{j=1}^n \frac{\partial f}{\partial z^j} dz^j , \\
\bar{\partial} f & := \sum_{j=1}^n \frac{\partial f}{\partial \bar{z}^j}
d\bar{z}^j .
\end{align*}
By direct calculation we immediately see that
\begin{equation*}
df = \partial f + \bar{\partial} f .
\end{equation*}

Similarly we now define $\partial$ and $\bar{\partial}$
on arbitrary differential form
$\omega = \sum_{\alpha,\beta} f_{\alpha,\beta} dz^\alpha \wedge
d\bar{z}^\beta$, where $\alpha$ and $\beta$ range over all multi-indices with
elements less then $n$, where if $\alpha = (\alpha_1,\ldots,\alpha_k)$
then $dz^\alpha = dz^{\alpha_1} \wedge \ldots \wedge dz^{\alpha_k}$,
and $f_{\alpha,\beta}$ is a $C^1$, complex valued function
on $G$.
\begin{align*}
\partial \omega
& :=
\sum_{\alpha,\beta} \frac{\partial f_{\alpha,\beta}}{\partial z^j} dz^j
\wedge
dz^\alpha \wedge d\bar{z}^\beta
, \\
\bar{\partial} \omega
& :=
\sum_{\alpha,\beta} \frac{\partial f_{\alpha,\beta}}{\partial \bar{z}^j}
d\bar{z}^j
\wedge
dz^\alpha \wedge d\bar{z}^\beta .
\end{align*}
Again a direct calculation shows that $d = \partial + \bar{\partial}$.

The Cauchy-Riemann equations are then given by
\begin{equation*}
\bar{\partial} f = 0
\end{equation*}
That is, $f$ is holomorphic if and only if it satisfies the above equations.
Note that this only applies to functions.  If $\bar{\partial}\omega = 0$
for a differential form, then the coefficients in the standard basis
need not be holomorphic.

\begin{prop}
$\bar{\partial}$ and $\partial$ satisfy the following properties
\begin{itemize}
\item $\bar{\partial}$ and $\partial$ are linear,
\item $\bar{\partial}^2 = \bar{\partial} \bar{\partial} = 0$ and $\partial^2 = \partial \partial = 0$,
\item $\bar{\partial} \partial - \partial \bar{\partial} = 0$.
\end{itemize} 
\end{prop}

While $\bar{\partial} u = 0$ is our condition for $u$ to be a
holomorphic function it turns out that it is more important to solve the inhomogeneous
$\bar{\partial}u = f$ equation, as that allows us to construct holomorphic
objects from nonholomorphic ones.

\begin{thebibliography}{9}
\bibitem{Hormander:several}
Lars H\"ormander.
{\em \PMlinkescapetext{An Introduction to Complex Analysis in Several
Variables}},
North-Holland Publishing Company, New York, New York, 1973.
\bibitem{Krantz:several}
Steven~G.\@ Krantz.
{\em \PMlinkescapetext{Function Theory of Several Complex Variables}},
AMS Chelsea Publishing, Providence, Rhode Island, 1992.
\end{thebibliography}
%%%%%
%%%%%
\end{document}
