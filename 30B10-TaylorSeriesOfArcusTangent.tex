\documentclass[12pt]{article}
\usepackage{pmmeta}
\pmcanonicalname{TaylorSeriesOfArcusTangent}
\pmcreated{2013-03-22 16:33:37}
\pmmodified{2013-03-22 16:33:37}
\pmowner{pahio}{2872}
\pmmodifier{pahio}{2872}
\pmtitle{Taylor series of arcus tangent}
\pmrecord{7}{38748}
\pmprivacy{1}
\pmauthor{pahio}{2872}
\pmtype{Derivation}
\pmcomment{trigger rebuild}
\pmclassification{msc}{30B10}
\pmclassification{msc}{26A24}
\pmclassification{msc}{41A58}
\pmrelated{GregorySeries}
\pmrelated{TaylorSeriesOfArcusSine}
\pmrelated{SubstitutionNotation}
\pmrelated{ExamplesOnHowToFindTaylorSeriesFromOtherKnownSeries}
\pmrelated{CyclometricFunctions}
\pmrelated{LogarithmSeries}

\endmetadata

% this is the default PlanetMath preamble.  as your knowledge
% of TeX increases, you will probably want to edit this, but
% it should be fine as is for beginners.

% almost certainly you want these
\usepackage{amssymb}
\usepackage{amsmath}
\usepackage{amsfonts}

% used for TeXing text within eps files
%\usepackage{psfrag}
% need this for including graphics (\includegraphics)
%\usepackage{graphicx}
% for neatly defining theorems and propositions
 \usepackage{amsthm}
% making logically defined graphics
%%%\usepackage{xypic}

% there are many more packages, add them here as you need them

% define commands here
\newcommand{\sijoitus}[2]%
{\operatornamewithlimits{\Big/}_{\!\!\!#1}^{\,#2}}

\theoremstyle{definition}
\newtheorem*{thmplain}{Theorem}

\begin{document}
The derivative of the \PMlinkname{arcus tangent}{CyclometricFunctions} function, $\frac{1}{1+x^2}$, can be expanded as a geometric series
$$\frac{1}{1\!+\!x^2} = 1\!-\!x^2\!+\!x^4\!-\!x^6\!+-\ldots,$$
the radius of convergence of which is $1$.\, The series \PMlinkescapetext{expansion} is valid only on the open interval\, $-1 < x < 1$,\, because the series apparently diverges for\, $x = \pm 1$.\, The power series may be integrated termwise on its interval of convergence, giving
$$\int_0^x\!\frac{dt}{1\!+\!t^2} = \sijoitus{0}{\quad x}\!\arctan{t} =\arctan{x} = 
x\!-\!\frac{x^3}{3}\!+\!\frac{x^5}{5}\!-+\ldots \qquad (|x| < 1).$$
We can show that this Taylor series of arcus tangent is valid also for the end points\, $x = \pm 1$\, of the interval.

We start from the identical equation
$$\frac{1}{1\!+\!t^2} = 
1\!-\!t^2\!+\!t^4\!-+\ldots+\!(-1)^{n-1}t^{2n-2}+(-1)^n\frac{t^{2n}}{1+t^2},$$
which can be verified by performing the division $1\!:\!(1\!+\!t^2)$.\, Integrating both sides from $0$ to an arbitrary $x$, we obtain
$$\arctan{x} = 
\underbrace{x\!-\!\frac{x^3}{3}\!+\!\frac{x^5}{5}\!-+\ldots+\!(-1)^{n-1}\frac{x^{2n-1}}{2n\!-\!1}}_{S_{2n-1}}+
\underbrace{(-1)^n\int_0^x\frac{t^{2n}}{1+t^2}dt}_{R_{2n-1}}.$$
We estimate $R_{2n-1}$:
$$|R_{2n-1}| = \int_0^{|x|}\frac{t^{2n}}{1+t^2}dt \leqq 
\int_0^{|x|}t^{2n}\,dt = \frac{|x|^{2n-1}}{2n\!-\!1}\,\to\, 0 
\quad\mathrm{as}\,\,n\to\infty$$
for\, $x = \pm 1$.\, Accordingly, when\, $x = \pm 1$,\, we see that
$$S_{2n-1} = \arctan{x}\!-\!R_{2n-1}\,\to\, \arctan{x}$$
as\, $n\to\infty$.\, This \PMlinkescapetext{means} that
$$\arctan{(\pm 1)} = \pm\frac{\pi}{4} =
\pm\left(1\!-\!\frac{1}{3}\!+\!\frac{1}{5}\!-+\ldots\right).$$
%%%%%
%%%%%
\end{document}
