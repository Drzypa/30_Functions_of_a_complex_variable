\documentclass[12pt]{article}
\usepackage{pmmeta}
\pmcanonicalname{ProofOfLimitOfNthRootOfN}
\pmcreated{2014-02-28 7:21:31}
\pmmodified{2014-02-28 7:21:31}
\pmowner{rspuzio}{6075}
\pmmodifier{rspuzio}{6075}
\pmtitle{proof of limit of nth root of n}
\pmrecord{19}{40113}
\pmprivacy{1}
\pmauthor{rspuzio}{6075}
\pmtype{Proof}
\pmcomment{improving proof}
\pmclassification{msc}{30-00}
\pmclassification{msc}{12D99}

% this is the default PlanetMath preamble.  as your knowledge
% of TeX increases, you will probably want to edit this, but
% it should be fine as is for beginners.

% almost certainly you want these
\usepackage{amssymb}
\usepackage{amsmath}
\usepackage{amsfonts}

% used for TeXing text within eps files
%\usepackage{psfrag}
% need this for including graphics (\includegraphics)
%\usepackage{graphicx}
% for neatly defining theorems and propositions
\usepackage{amsthm}
% making logically defined graphics
%%%\usepackage{xypic}

% there are many more packages, add them here as you need them

% define commands here
\newtheorem{lemma}{Lemma}
\newtheorem{theorem}{Theorem}
\begin{document}
In this entry, we present a self-contained, elementary
proof of the fact that\, $\lim_{n \to \infty} n^{1/n} = 1$.
We begin by with inductive proofs of two integer 
inequalities --- real numbers will not enter until
the very end.

\begin{lemma}
For all integers $n$ greater than or equal to $5$,
\[
 2^n > n^2
\]
\end{lemma}

\begin{proof}
We begin with a few easy observations.  First,
a bit of arithmetic:
\[
 2^5 = 32 > 25 = 5^2
\]
Second, some algebraic manipulation of the inequality $n > 4$:
\begin{align*}
 n - 1 &> 3 \\
 (n - 1)^2 &> 9 \\
 (n - 1)^2 &> 2 \\
 n^2 - 2n + 1 &> 2 \\
 2n^2 &> n^2 + 2n + 1 \\
 2n^2 &> (n + 1)^2
\end{align*}

These observations provide us with the makings of an
inductive proof.  Suppose that $2^n > n^2$ for some
integer $n \ge 5$.  Using the inequality we just showed,
\[
 2^{n+1} = 2 \cdot 2^n > 2 n^2 > (n+1)^2.
\]
Snce $2^5 > 5^2$ and  $2^n > n^2$ implies that $2^{n+1} > 
(n + 1)^2$ when $n \ge 5$ we conclude that $2^n > n^2$ dor
all $n \ge 5$.
\end{proof}

\begin{lemma}
For all integers $n$ greater than or equal to $3$,
\[
 n^{n+1} > (n+1)^n
\]
\end{lemma}

\begin{proof}
We begin by noting that
\[
 3^4 = 81 > 64 = 4^3.
\]
Next, we make assume that
\[
 (n-1)^n > n^{(n-1)}.
\]
for some $n$.
Multiplying both sides by $n$:
\[
 n (n-1)^n > n^n.
\]
Multiplying both sides by $(n+1)^n$ and making 
use of the identity $(n+1)(n-1) = n^2 - 1$,
\[
 n (n^2 - 1)^n > n^n (n +1)^n.
\]
Since $n^2 > n^2 - 1$, the left-hand side is 
less than $n^{2n+1}$, hence
\[
 n^{2n+1} > n^n (n +1)^n.
\]
Canceling $n^n$ from both sides,
\[
 n^{(n+1)} > (n+1)^n.
\]
Hence, by induction, $n^{(n+1)} > (n+1)^n$
for all $n \ge 3$.
\end{proof}

\begin{theorem}
\[
 \lim_{n \to \infty} n^{1/n} = 1
\]
\end{theorem}

\begin{proof}
Consider the subsequence where $n$ is a power of $2$.
We then have
\[
 (2^m)^{(1/2^m)} =
 2^{m/2^m}.
\]
By lemma 1, $m/2^m < 1/m$ when $m \ge 5$.  Hence,
$(2^m)^{1/2^m} < 2^{1/m}$.  Since $\lim_{m \to 0}
2^{1/m} = 1$, and $(2^m)^{1/2^m)} > 1$, we conclude
by the squeeze rule that
\[
 \lim_{m \to 0} (2^m)^{1/2^m} = 1.
\]

By lemma 2, the sequence $\{n^{1/n}\}$ is decreasing.  It 
is clearly bounded from below by $1$.  Above, we exhibited
a subsequence which tends towards $1$.   Thus it follows that
\[
 \lim_{n \to \infty} n^{1/n} = 1.
\]

\end{proof}
%%%%%
%%%%%
\end{document}
