\documentclass[12pt]{article}
\usepackage{pmmeta}
\pmcanonicalname{ProofOfCriterionForConformalMappingOfRiemannianSpaces}
\pmcreated{2013-03-22 16:22:03}
\pmmodified{2013-03-22 16:22:03}
\pmowner{rspuzio}{6075}
\pmmodifier{rspuzio}{6075}
\pmtitle{proof of criterion for conformal mapping of Riemannian spaces}
\pmrecord{7}{38505}
\pmprivacy{1}
\pmauthor{rspuzio}{6075}
\pmtype{Proof}
\pmcomment{trigger rebuild}
\pmclassification{msc}{30E20}

% this is the default PlanetMath preamble.  as your knowledge
% of TeX increases, you will probably want to edit this, but
% it should be fine as is for beginners.

% almost certainly you want these
\usepackage{amssymb}
\usepackage{amsmath}
\usepackage{amsfonts}

% used for TeXing text within eps files
%\usepackage{psfrag}
% need this for including graphics (\includegraphics)
%\usepackage{graphicx}
% for neatly defining theorems and propositions
%\usepackage{amsthm}
% making logically defined graphics
%%%\usepackage{xypic}

% there are many more packages, add them here as you need them

% define commands here

\begin{document}
In this attachment, we prove that the a mapping $f$ of Riemannian (or
pseudo-Riemannian) spaces $(M,g)$ and $(N,h)$ is conformal if and only if 
$f^* h = s g$ for some scalar field $s$ (on $M$).

The key observation is that the angle $A$ between curves $S$ and $T$ which 
intersect at a point $P$ is determined by the tangent vectors to these two curves 
(which we shall term $s$ and $t$) and the metric at that point, like so:
 \[ \cos A = {g(s,t) \over \sqrt{g(s,s)} \sqrt{g(t,t)}}\]
Moreover, given any tangent vector at a point, there will exist at least one
curve to which it is the tangent.  Also, the tangent vector to the image of
a curve under a map is the pushforward of the tangent to the original curve
under the map; for instance, the tangent to $f(S)$ at $f(P)$ is $f^* s$.  Hence, 
the mapping $f$ is conformal if and only if
 \[{g(u,v) \over \sqrt{g(u,u)} \sqrt{g(v,v)}} = {h(f^* u, f^* v) \over 
\sqrt{h(f^* u, f^* u)} \sqrt{h(f^* v, f^* v)}}\]
for all tangent vectors $u$ and $v$ to the manifold $M$.  By the way pushforwards 
and pullbacks work, this is equivalent to the condition that
 \[{g(u,v) \over \sqrt{g(u,u)} \sqrt{g(v,v)}} = {(f^*  h)(u, v) \over 
\sqrt{(f^*  h)(u, u)} \sqrt{(f^* h)(v, v)}}\]
for all tangent vectors $u$ and $v$ to the manifold $N$.  Now, by elementary
algebra, the above equation is equivalent to the requirement that there
exist a scalar $s$ such that, for all $u$ and $v$, it is the case that 
$g(u,v) = s h^* (u,v)$ or, in other words, $f^* h = s g$ for some scalar 
field $s$.
%%%%%
%%%%%
\end{document}
