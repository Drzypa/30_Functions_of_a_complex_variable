\documentclass[12pt]{article}
\usepackage{pmmeta}
\pmcanonicalname{SchwarzReflectionPrinciple}
\pmcreated{2013-03-22 14:17:58}
\pmmodified{2013-03-22 14:17:58}
\pmowner{jirka}{4157}
\pmmodifier{jirka}{4157}
\pmtitle{Schwarz reflection principle}
\pmrecord{7}{35757}
\pmprivacy{1}
\pmauthor{jirka}{4157}
\pmtype{Theorem}
\pmcomment{trigger rebuild}
\pmclassification{msc}{30C35}
\pmsynonym{Schwarz reflection theorem}{SchwarzReflectionPrinciple}
\pmsynonym{reflection principle}{SchwarzReflectionPrinciple}
\pmdefines{symmetric region}

\endmetadata

% this is the default PlanetMath preamble.  as your knowledge
% of TeX increases, you will probably want to edit this, but
% it should be fine as is for beginners.

% almost certainly you want these
\usepackage{amssymb}
\usepackage{amsmath}
\usepackage{amsfonts}

% used for TeXing text within eps files
%\usepackage{psfrag}
% need this for including graphics (\includegraphics)
%\usepackage{graphicx}
% for neatly defining theorems and propositions
\usepackage{amsthm}
% making logically defined graphics
%%%\usepackage{xypic}

% there are many more packages, add them here as you need them

% define commands here
\theoremstyle{theorem}
\newtheorem*{thm}{Theorem}
\newtheorem*{lemma}{Lemma}
\newtheorem*{conj}{Conjecture}
\newtheorem*{cor}{Corollary}
\newtheorem*{example}{Example}
\newtheorem*{prop}{Proposition}
\theoremstyle{definition}
\newtheorem*{defn}{Definition}
\begin{document}
For a region $G \subset {\mathbb{C}}$ define $G^* := \{ z : \bar{z} \in G \}$ (where $\bar{z}$ is the complex conjugate of $z$).  If $G$ is a {\em symmetric region}, that is $G = G^*$, then we define
$G_+ := \{ z \in G : \operatorname{Im} z > 0 \}$,
$G_- := \{ z \in G : \operatorname{Im} z < 0 \}$ and
$G_0 := \{ z \in G : \operatorname{Im} z = 0 \}$.

\begin{thm}
Let $G \subset {\mathbb{C}}$ be a region such that $G = G^*$ and suppose that
$f \colon G_+ \cup G_0 \to {\mathbb{C}}$ is a continuous functions that is
analytic on $G_+$ and further that $f(x)$ is real for $x \in G_0$ (that is
for real $x$), then there is an analytic function $g : G \to {\mathbb{C}}$
such that $g(z) = f(z)$ for $z \in G_+ \cup G_0$.
\end{thm}

That is you can ``reflect'' an analytic function across the real axis.  Note that by composing with various conformal mappings you could generalize the above to reflection across an analytic curve.
So loosely stated, the theorem says that if an analytic function is defined in a region with some ``nice'' boundary and the function behaves ``nice'' on this boundary, then we can extend the function to a larger domain.  Let us make this statement precise with the following generalization.

\begin{thm}
Let $G, \Omega \subset {\mathbb{C}}$ be regions and let $\gamma$ and $\omega$
be free analytic boundary arcs in $\partial G$ and $\partial \Omega$.  Suppose
that $f \colon G \cup \gamma \to {\mathbb{C}}$ is a continuous function that
is analytic on $G$, $f(G) \subset \Omega$ and $f(\gamma) \subset \omega$, then for any compact set $\kappa \subset \gamma$, $f$ has an analytic continuation to an open set containing $G \cup \kappa$.
\end{thm}

\begin{thebibliography}{9}
\bibitem{Conway:complexI}
John~B. Conway.
{\em \PMlinkescapetext{Functions of One Complex Variable I}}.
Springer-Verlag, New York, New York, 1978.
\bibitem{Conway:complexII}
John~B. Conway.
{\em \PMlinkescapetext{Functions of One Complex Variable II}}.
Springer-Verlag, New York, New York, 1995.
\end{thebibliography}
%%%%%
%%%%%
\end{document}
