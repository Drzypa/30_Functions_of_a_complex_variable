\documentclass[12pt]{article}
\usepackage{pmmeta}
\pmcanonicalname{PlemeljFormulas}
\pmcreated{2013-03-22 16:02:02}
\pmmodified{2013-03-22 16:02:02}
\pmowner{perucho}{2192}
\pmmodifier{perucho}{2192}
\pmtitle{Plemelj formulas}
\pmrecord{5}{38079}
\pmprivacy{1}
\pmauthor{perucho}{2192}
\pmtype{Definition}
\pmcomment{trigger rebuild}
\pmclassification{msc}{30D10}

% this is the default PlanetMath preamble.  as your knowledge
% of TeX increases, you will probably want to edit this, but
% it should be fine as is for beginners.

% almost certainly you want these
\usepackage{amssymb}
\usepackage{amsmath}
\usepackage{amsfonts}

% used for TeXing text within eps files
%\usepackage{psfrag}
% need this for including graphics (\includegraphics)
%\usepackage{graphicx}
% for neatly defining theorems and propositions
%\usepackage{amsthm}
% making logically defined graphics
%%%\usepackage{xypic}

% there are many more packages, add them here as you need them

% define commands here

\begin{document}
Let $\psi(\zeta)$ be a density function of a complex variable satisfying the H\"older condition (the Lipschitz condition of order $\alpha$){\footnote{A function $f(\zeta)$ satisfies the H\"older condition on a smooth curve $C$ if for every $\zeta_1,\zeta_2\in C$ $|f(\zeta_2)-f(\zeta_1)|\leq M|\zeta_2-\zeta_1|^\alpha$, $M>0$, $0<\alpha\leq 1$. It is clear that the H\"older condition is a weaker restriction than a bounded derivative for $f(\zeta)$.}} on a smooth closed contour $C$ in the integral
\begin{align}
\Psi(z)=\frac{1}{2\pi i}\int_C\frac{\psi(\zeta)}{\zeta-z}d\zeta,
\end{align}
then the limits $\Psi^+(t)$ and $\Psi^-(t)$ as $z$ approaches an arbitrary point $t$ on $C$ from the interior and the exterior of $C$, respectively, are
\begin{align}
\left\{ \begin{array}{ll}
\Psi^+(t) \equiv \frac{1}{2}\psi(t)+
\frac{1}{2\pi i}\int_C\frac{\psi(\zeta)}{\zeta-t}d\zeta, \\ 
\Psi^-(t) \equiv -\frac{1}{2}\psi(t)+
\frac{1}{2\pi i}\int_C\frac{\psi(\zeta)}{\zeta-t}d\zeta.
\end{array}
\right.
\end{align}
These are the Plemelj\cite{cite:Plemelj} formulas {\footnote{cf.\cite{cite:Musk}, where restrictions that Plemelj made, were relaxed.}}  and the improper integrals in (2) must be interpreted as Cauchy's principal values.
\begin{thebibliography}{99}
\bibitem{cite:Plemelj}
J. Plemelj, {\em Monatshefte f\"ur Mathematik und Physik,} vol. 19, pp. 205- 210, 1908.
\bibitem{cite:Musk}
N. I. Muskhelishvili, {\em Singular Integral Equations,} Groningen: Noordhoff (based on the second Russian edition published in 1946), 1953.
\end{thebibliography} 
	


  
%%%%%
%%%%%
\end{document}
