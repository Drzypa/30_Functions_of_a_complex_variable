\documentclass[12pt]{article}
\usepackage{pmmeta}
\pmcanonicalname{BiquadraticEquation}
\pmcreated{2013-03-22 17:52:45}
\pmmodified{2013-03-22 17:52:45}
\pmowner{pahio}{2872}
\pmmodifier{pahio}{2872}
\pmtitle{biquadratic equation}
\pmrecord{11}{40360}
\pmprivacy{1}
\pmauthor{pahio}{2872}
\pmtype{Topic}
\pmcomment{trigger rebuild}
\pmclassification{msc}{30-00}
\pmclassification{msc}{12D99}
\pmrelated{BiquadraticExtension}
\pmrelated{BiquadraticField}
\pmrelated{EulersDerivationOfTheQuarticFormula}
\pmrelated{IrreduciblePolynomialsObtainedFromBiquadraticFields}
\pmrelated{LogicalOr}
\pmdefines{biquadratic equation}

\endmetadata

% this is the default PlanetMath preamble.  as your knowledge
% of TeX increases, you will probably want to edit this, but
% it should be fine as is for beginners.

% almost certainly you want these
\usepackage{amssymb}
\usepackage{amsmath}
\usepackage{amsfonts}

% used for TeXing text within eps files
%\usepackage{psfrag}
% need this for including graphics (\includegraphics)
%\usepackage{graphicx}
% for neatly defining theorems and propositions
 \usepackage{amsthm}
% making logically defined graphics
%%%\usepackage{xypic}

% there are many more packages, add them here as you need them

% define commands here

\theoremstyle{definition}
\newtheorem*{thmplain}{Theorem}

\begin{document}
A {\em biquadratic equation} (in a narrower sense) is the special case of the \PMlinkname{quartic equation}{QuarticFormula} containing no odd degree terms:
\begin{align}
ax^4+bx^2+c = 0
\end{align}
Here, $a$, $b$, $c$ are known real or complex numbers and\, $a \neq 0$.

For solving a biquadratic equation (1) one does not need the \PMlinkname{quartic formula}{QuarticFormula} since the equation may be thought a quadratic equation with respect to $x^2$, i.e.
$$a(x^2)^2+bx^2+c = 0,$$
whence
$$x^2 = \frac{-b\pm\sqrt{b^2-4ac}}{2a}$$
(see quadratic formula or \PMlinkname{quadratic equation in $\mathbb{C}$}{QuadraticEquationInMathbbC}).\, Taking square roots of the values of $x^2$ (see taking square root algebraically), one obtains the four \PMlinkname{roots}{Equation} of (1).\\

\textbf{Example.}\, Solve the biquadratic equation
\begin{align}
x^4+x^2-20 = 0.
\end{align}
We have
\begin{align}
x^2 = \frac{-1\pm\sqrt{1^2-4\cdot1\cdot(-20)}}{2\cdot1} = \frac{-1\pm9}{2},
\end{align}
i.e.\, $x^2 = 4$\, or\, $x^2 = -5$.\, The solution is
\begin{align}
x = \pm2 \quad \lor \quad x = \pm i\sqrt{5}.
\end{align}

\textbf{Remark.}\, In one wants to form of rational numbers a polynomial equation with rational coefficients and most possibly low degree by using two square root operations, then one gets always a biquadratic equation.\, A couple of examples:

1) $x = 1+\sqrt2+\sqrt3$\\
$(x-1)^2 = 2+2\sqrt{6}+3$\\
$y^2-5 = 2\sqrt{6}$\\
$y^4-10y^2+1 = 0$\, (one has \PMlinkname{substituted}{TchirnhausTransformations}\, $x-1 := y$)\\

2) $x = \sqrt{\sqrt{2}-1}$\\
$x^2 = \sqrt{2}-1$\\
$(x^2+1)^2 = 2$\\
$x^4+2x^2-1 = 0$


%%%%%
%%%%%
\end{document}
