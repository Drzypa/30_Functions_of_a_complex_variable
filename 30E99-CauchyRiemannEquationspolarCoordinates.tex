\documentclass[12pt]{article}
\usepackage{pmmeta}
\pmcanonicalname{CauchyRiemannEquationspolarCoordinates}
\pmcreated{2013-03-22 14:03:58}
\pmmodified{2013-03-22 14:03:58}
\pmowner{Daume}{40}
\pmmodifier{Daume}{40}
\pmtitle{Cauchy-Riemann equations (polar coordinates)}
\pmrecord{8}{35423}
\pmprivacy{1}
\pmauthor{Daume}{40}
\pmtype{Definition}
\pmcomment{trigger rebuild}
\pmclassification{msc}{30E99}
\pmrelated{TangentialCauchyRiemannComplexOfCinftySmoothForms}
\pmrelated{ACRcomplex}

\endmetadata

% this is the default PlanetMath preamble.  as your knowledge
% of TeX increases, you will probably want to edit this, but
% it should be fine as is for beginners.

% almost certainly you want these
\usepackage{amssymb}
\usepackage{amsmath}
\usepackage{amsfonts}

% used for TeXing text within eps files
%\usepackage{psfrag}
% need this for including graphics (\includegraphics)
%\usepackage{graphicx}
% for neatly defining theorems and propositions
%\usepackage{amsthm}
% making logically defined graphics
%%%\usepackage{xypic} 

% there are many more packages, add them here as you need them

% define commands here
\begin{document}
Suppose $A$ is an open set in $\mathbb{C}$ and $f(z)=f(re^{i\theta})=u(r,\theta)+iv(r,\theta): A\subset\mathbb{C} \to \mathbb{C}$ is a function.  If the derivative of $f(z)$ exists at $z_0=(r_0,\theta_0)$.  Then the functions $u$, $v$ at $z_0$ satisfy:
\begin{eqnarray*}
\frac{\partial u}{\partial r} & = & \frac{1}{r}\frac{\partial v}{\partial \theta}\\
\frac{\partial v}{\partial r} & = & -\frac{1}{r}\frac{\partial u}{\partial \theta}
\end{eqnarray*}
which are called \emph{Cauchy-Riemann equations} in polar form.\\\\
%%%%%
%%%%%
\end{document}
