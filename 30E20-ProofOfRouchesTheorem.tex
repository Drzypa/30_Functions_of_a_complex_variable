\documentclass[12pt]{article}
\usepackage{pmmeta}
\pmcanonicalname{ProofOfRouchesTheorem}
\pmcreated{2013-03-22 14:34:26}
\pmmodified{2013-03-22 14:34:26}
\pmowner{rspuzio}{6075}
\pmmodifier{rspuzio}{6075}
\pmtitle{proof of Rouch\'e's theorem}
\pmrecord{6}{36132}
\pmprivacy{1}
\pmauthor{rspuzio}{6075}
\pmtype{Proof}
\pmcomment{trigger rebuild}
\pmclassification{msc}{30E20}

\endmetadata

% this is the default PlanetMath preamble.  as your knowledge
% of TeX increases, you will probably want to edit this, but
% it should be fine as is for beginners.

% almost certainly you want these
\usepackage{amssymb}
\usepackage{amsmath}
\usepackage{amsfonts}

% used for TeXing text within eps files
%\usepackage{psfrag}
% need this for including graphics (\includegraphics)
%\usepackage{graphicx}
% for neatly defining theorems and propositions
%\usepackage{amsthm}
% making logically defined graphics
%%%\usepackage{xypic}

% there are many more packages, add them here as you need them

% define commands here
\begin{document}
Consider the integral
 $$N(\lambda) = {1 \over 2 \pi i} \oint_C {f'(z) + \lambda g'(z) \over f(z) + \lambda g(z)} dz$$
where $0 \le \lambda \le 1$.  By the hypotheses, the function $f + \lambda g$ is non-singular on $C$ or on the interior of $C$ and has no zeros on $C$.  Hence, by the argument principle, $N(\lambda)$ equals the number of zeros (counted with multiplicity) of $f + \lambda g$ contained inside $C$.  Note that this means that $N(\lambda)$ must be an integer.

Since $C$ is compact, both $|f|$ and $|g|$ attain minima and maxima on $C$.  Hence there exist positive real constants $a$ and $b$ such that
 $$|f(z)| > a > b > |g(z)|$$
for all $z$ on $C$.  By the triangle inequality, this implies that $|f(z) + \lambda g(z)| > a - b$ on $C$.  Hence $1/(f + \lambda g)$ is a continuous function of $\lambda$ when $0 \le \lambda \le 1$ and $z \in C$.  Therefore, the integrand is a continuous function of $C$ and $\lambda$.  Since $C$ is compact, it follows that $N(\lambda)$ is a continuous function of $\lambda$.

Now there is only one way for a continuous function of a real variable to assume only integer values -- that function must be constant.  In particular, this means that the number of zeros of $f + \lambda g$ inside $C$ is the same for all $\lambda$.  Taking the extreme cases $\lambda = 0$ and $\lambda = 1$, this means that $f$ and $f+g$ have the same number of zeros inside $C$.
%%%%%
%%%%%
\end{document}
