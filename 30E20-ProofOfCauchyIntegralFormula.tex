\documentclass[12pt]{article}
\usepackage{pmmeta}
\pmcanonicalname{ProofOfCauchyIntegralFormula}
\pmcreated{2013-03-22 12:47:23}
\pmmodified{2013-03-22 12:47:23}
\pmowner{rmilson}{146}
\pmmodifier{rmilson}{146}
\pmtitle{proof of Cauchy integral formula}
\pmrecord{15}{33105}
\pmprivacy{1}
\pmauthor{rmilson}{146}
\pmtype{Proof}
\pmcomment{trigger rebuild}
\pmclassification{msc}{30E20}

\usepackage{amsmath}
\usepackage{amsfonts}
\usepackage{amssymb}
\newcommand{\reals}{\mathbb{R}}
\newcommand{\natnums}{\mathbb{N}}
\newcommand{\cnums}{\mathbb{C}}
\newcommand{\znums}{\mathbb{Z}}
\newcommand{\lp}{\left(}
\newcommand{\rp}{\right)}
\newcommand{\lb}{\left[}
\newcommand{\rb}{\right]}
\newcommand{\supth}{^{\text{th}}}
\newtheorem{proposition}{Proposition}
\newtheorem{definition}[proposition]{Definition}
\newcommand{\nl}[1]{{\PMlinkescapetext{{#1}}}}
\newcommand{\pln}[2]{{\PMlinkname{{#1}}{#2}}}
\begin{document}
Let $D=\{z\in\cnums:\Vert z-z_0\Vert< R\}$
be a disk in the
complex plane, $S\subset D$ a finite subset, and $U\subset\cnums$ an
 open domain that contains the closed disk  $\overline{D}$. Suppose that
\begin{itemize}
\item $f:U\backslash S\rightarrow \cnums$ is holomorphic, and that
\item $f(z)$ is bounded near all $z\in D\backslash S$.  
\end{itemize}
Hence, by a straightforward compactness argument we also have that 
$f(z)$ is bounded on $\overline{D}\backslash S$, and hence bounded on
$D\backslash S$.  

Let $z\in D\backslash
S$ be given, and set
$$g(\zeta) = \frac{f(\zeta)-f(z)}{\zeta-z},\quad \zeta\in D\backslash
S',$$
where $S'=S\cup \{ z\}$.  Note that $g(\zeta)$ is holomorphic and
bounded on $ D\backslash S'$.
The second assertion is true, because
$$g(\zeta)\rightarrow f'(z),\;\mbox{as}\; \zeta\rightarrow z.$$
Therefore, by the Cauchy integral theorem
$$
\oint_C g(\zeta)\, d\zeta=0,
$$
where $C$ is the counterclockwise circular contour parameterized by
$$\zeta = z_0 + R e^{it},\; 0\leq t\leq 2\pi.$$
Hence,
\begin{equation}
  \label{eq:1}
\oint_C \frac{f(\zeta)}{\zeta-z}\, d\zeta = \oint_C
\frac{f(z)}{\zeta-z}\, d\zeta.
\end{equation}

\noindent
$\mathbf{Lemma}$
 If  $z\in\cnums$ is such that $\Vert z\Vert\neq 1$, then
$$
\oint_{\Vert\zeta\Vert=1} \frac{d\zeta}{\zeta-z} =
\begin{cases}
  0 &\text{if } \Vert z\Vert>1\\
  2\pi i &\text{if } \Vert z\Vert<1\\
\end{cases}
$$

The proof is a fun exercise in elementary integral calculus, an
application of the half-angle trigonometric substitutions.

Thanks to the Lemma, the  right hand side of \eqref{eq:1} evaluates to 
$2\pi i f(z).$
Dividing through by $2\pi i$, we obtain 
$$
f(z) = \frac{1}{2\pi i} \oint_C \frac{f(\zeta)}{\zeta-z}\, d\zeta,
\quad z\in D,
$$
as desired.

Since a circle is a compact set, the defining limit for the derivative
$$\frac{d}{dz} \frac{f(\zeta)}{\zeta-z}= 
\frac{f(\zeta)}{(\zeta-z)^2},\quad z\in D$$
converges uniformly for $\zeta\in \partial D$.  Thanks to the uniform
convergence, the order of the derivative and the integral operations 
can be interchanged.  In this way we obtain the second formula:
$$
f'(z) = \frac{1}{2\pi i} \frac{d}{dz} \oint_C
\frac{f(\zeta)}{\zeta-z}\, d\zeta  =  \frac{1}{2\pi i} \oint_C
\frac{f(\zeta)}{(\zeta-z)^2}\, d\zeta,\quad z\in D.$$
%%%%%
%%%%%
\end{document}
