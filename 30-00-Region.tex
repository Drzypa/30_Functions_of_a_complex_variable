\documentclass[12pt]{article}
\usepackage{pmmeta}
\pmcanonicalname{Region}
\pmcreated{2013-03-22 11:56:19}
\pmmodified{2013-03-22 11:56:19}
\pmowner{Wkbj79}{1863}
\pmmodifier{Wkbj79}{1863}
\pmtitle{region}
\pmrecord{11}{30670}
\pmprivacy{1}
\pmauthor{Wkbj79}{1863}
\pmtype{Definition}
\pmcomment{trigger rebuild}
\pmclassification{msc}{30-00}
%\pmkeywords{Complex Analysis}
\pmrelated{Complex}
\pmrelated{Domain2}

\usepackage{amssymb}
\usepackage{amsmath}
\usepackage{amsfonts}
\usepackage{graphicx}
%%%%\usepackage{xypic}

\begin{document}
\PMlinkescapeword{domain}

A \emph{region} is a nonempty open subset of $\mathbb{C}$.  Note that this definition is a \PMlinkescapetext{restriction} of that of \PMlinkname{domain}{Domain2} (as defined in complex analysis) to the complex plane.  Some people prefer to use ``region'' instead of ``domain'' to avoid confusion with other mathematical definitions of domain.  (The set theoretic definition of \PMlinkname{domain}{Domain} is also used in complex analysis.)

Regions play a major role in complex analysis since every nonempty open subset of $\mathbb{C}$ is the union of countably many connected components, each of which is a region.
%%%%%
%%%%%
%%%%%
%%%%%
\end{document}
