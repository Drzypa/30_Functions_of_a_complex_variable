\documentclass[12pt]{article}
\usepackage{pmmeta}
\pmcanonicalname{ZeroAsContourIntegral}
\pmcreated{2013-03-22 16:46:42}
\pmmodified{2013-03-22 16:46:42}
\pmowner{rspuzio}{6075}
\pmmodifier{rspuzio}{6075}
\pmtitle{zero as contour integral}
\pmrecord{5}{39008}
\pmprivacy{1}
\pmauthor{rspuzio}{6075}
\pmtype{Corollary}
\pmcomment{trigger rebuild}
\pmclassification{msc}{30E20}

% this is the default PlanetMath preamble.  as your knowledge
% of TeX increases, you will probably want to edit this, but
% it should be fine as is for beginners.

% almost certainly you want these
\usepackage{amssymb}
\usepackage{amsmath}
\usepackage{amsfonts}

% used for TeXing text within eps files
%\usepackage{psfrag}
% need this for including graphics (\includegraphics)
%\usepackage{graphicx}
% for neatly defining theorems and propositions
%\usepackage{amsthm}
% making logically defined graphics
%%%\usepackage{xypic}

% there are many more packages, add them here as you need them

% define commands here

\begin{document}
Suppose that $f$ is a complex function which is defined in some open
set $D \subseteq \mathbb{C}$ which has a simple zero at some point
$p \in D$.  Then we have
\[
p = {1 \over 2 \pi i} \oint_C {z f'(z) \over f(z)} \, dz
\]
where $C$ is a closed path in $D$ which encloses $p$ but does not enclose or
pass through any other zeros of $f$.

This follows from the Cauchy residue theorem.  We have that the poles of
$f'/f$ occur at the zeros of $f$ and that the residue of a pole of
$f'/f$ is $1$ at a simple zero of $f$.  Hence, the residue of
$z f'(z) / f(z)$ at $p$ is $p$, so the above \PMlinkescapetext{formula} follows from the residue theorem. 
%%%%%
%%%%%
\end{document}
