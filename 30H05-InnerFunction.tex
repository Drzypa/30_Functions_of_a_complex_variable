\documentclass[12pt]{article}
\usepackage{pmmeta}
\pmcanonicalname{InnerFunction}
\pmcreated{2013-03-22 15:36:20}
\pmmodified{2013-03-22 15:36:20}
\pmowner{jirka}{4157}
\pmmodifier{jirka}{4157}
\pmtitle{inner function}
\pmrecord{6}{37522}
\pmprivacy{1}
\pmauthor{jirka}{4157}
\pmtype{Definition}
\pmcomment{trigger rebuild}
\pmclassification{msc}{30H05}
\pmrelated{FactorizationTheoremForHinftyFunctions}
\pmdefines{singular inner function}
\pmdefines{outer function}

\endmetadata

% this is the default PlanetMath preamble.  as your knowledge
% of TeX increases, you will probably want to edit this, but
% it should be fine as is for beginners.

% almost certainly you want these
\usepackage{amssymb}
\usepackage{amsmath}
\usepackage{amsfonts}

% used for TeXing text within eps files
%\usepackage{psfrag}
% need this for including graphics (\includegraphics)
%\usepackage{graphicx}
% for neatly defining theorems and propositions
\usepackage{amsthm}
% making logically defined graphics
%%%\usepackage{xypic}

% there are many more packages, add them here as you need them

% define commands here
\theoremstyle{theorem}
\newtheorem*{thm}{Theorem}
\newtheorem*{lemma}{Lemma}
\newtheorem*{conj}{Conjecture}
\newtheorem*{cor}{Corollary}
\newtheorem*{example}{Example}
\newtheorem*{prop}{Proposition}
\theoremstyle{definition}
\newtheorem*{defn}{Definition}
\theoremstyle{remark}
\newtheorem*{rmk}{Remark}
\begin{document}
If $f \colon \mathbb{D} \to \mathbb{C}$ is an analytic function on the unit disc, we denote by
$f^*(e^{i\theta})$ the radial limit of $f$ where it exists, that is
\begin{equation*}
f^*(e^{i\theta}) := \lim_{r\to 1, r<1} f(re^{i\theta}) .
\end{equation*}
A bounded analytic function on the disc will have radial limits almost everywhere (with respect to the Lebesgue measure on the $\partial \mathbb{D}$).

\begin{defn}
A bounded analytic function $f$ is called an \emph{inner function} if $\lvert f^*(e^{i\theta}) \rvert = 1$ almost everywhere.  If $f$ has no zeros on the unit disc, then $f$ is called a \emph{singular inner function}. 
\end{defn}

\begin{thm}
Every inner function can be written as
\begin{equation*}
f(z) := \alpha B(z) \exp \left( - \int \frac{e^{i\theta}+z}{e^{i\theta}-z}d\mu(e^{i\theta}) \right) ,
\end{equation*}
where $\mu$ is a positive singular measure on $\partial \mathbb{D}$, $B(z)$
is a Blaschke product and $\lvert \alpha \rvert = 1$ is a constant.
\end{thm}

Note that all the zeros of the function come from the Blaschke product.

\begin{defn}
Let
\begin{equation*}
f(z) := \exp \left(\int \frac{e^{i\theta}+z}{e^{i\theta}-z}h(e^{i\theta})dm(e^{i\theta}) \right) ,
\end{equation*}
where $h$ is a real valued Lebesgue integrable function on the unit circle and $m$ is the Lebesgue measure.  Then $f$ is called
an \emph{outer function}.
\end{defn}

The significance of these definitions is that every bounded holomorphic function can be written as an inner function times an outer function.  See the \PMlinkname{factorization theorem for $H^\infty$ functions}{FactorizationTheoremForHinftyFunctions}.

\begin{thebibliography}{9}
\bibitem{Conway:complexII}
John~B. Conway.
{\em \PMlinkescapetext{Functions of One Complex Variable II}}.
Springer-Verlag, New York, New York, 1995.
\end{thebibliography}
%%%%%
%%%%%
\end{document}
