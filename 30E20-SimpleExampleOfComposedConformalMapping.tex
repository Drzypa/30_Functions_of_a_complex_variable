\documentclass[12pt]{article}
\usepackage{pmmeta}
\pmcanonicalname{SimpleExampleOfComposedConformalMapping}
\pmcreated{2013-03-22 16:47:25}
\pmmodified{2013-03-22 16:47:25}
\pmowner{pahio}{2872}
\pmmodifier{pahio}{2872}
\pmtitle{simple example of composed conformal mapping}
\pmrecord{8}{39023}
\pmprivacy{1}
\pmauthor{pahio}{2872}
\pmtype{Example}
\pmcomment{trigger rebuild}
\pmclassification{msc}{30E20}
\pmclassification{msc}{53A30}

% this is the default PlanetMath preamble.  as your knowledge
% of TeX increases, you will probably want to edit this, but
% it should be fine as is for beginners.

% almost certainly you want these
\usepackage{amssymb}
\usepackage{amsmath}
\usepackage{amsfonts}

% used for TeXing text within eps files
%\usepackage{psfrag}
% need this for including graphics (\includegraphics)
\usepackage{graphicx}
% for neatly defining theorems and propositions
 \usepackage{amsthm}
% making logically defined graphics
%%%\usepackage{xypic}

% there are many more packages, add them here as you need them

% define commands here

\theoremstyle{definition}
\newtheorem*{thmplain}{Theorem}

\begin{document}
Let's consider the mapping
$$f\colon \mathbb{C}\to\mathbb{C} \quad \mathrm{with}\quad f(z) = az\!+\!b,$$
where $a$ and $b$ are complex \PMlinkescapetext{constants} and\, $a \neq 0$.

Because\, $f'(z) \equiv a \neq 0$,\, the mapping is conformal in the whole 
$z$-plane.\, Denote\, $\displaystyle a := \varrho e^{i\alpha}$ (where\, $\varrho,\,\alpha \in\mathbb{R}$) and
\begin{align}
z_1 := \varrho z,
\end{align}
\begin{align}
z_2 := e^{i\alpha}z_1,
\end{align}
\begin{align}
w := z_2\!+\!b.
\end{align}
Then the mapping\, $z\mapsto z_1$\, means a dilation in the complex plane, the mapping\, $z_1\mapsto z_2$\, a rotation by the angle $\alpha$ and the mapping\, $z_2\mapsto w$\, a translation determined by the vector from the origin to the point $b$.\, Thus $f$ is composed of these three consecutive mappings which all are conformal.

\begin{figure}
\begin{center}
\includegraphics{simple1}
\end{center}
\caption{The mapping from $z$ to $z_1$}
\end{figure}

\begin{figure}
\begin{center}
\includegraphics{simple2}
\end{center}
\caption{The mapping from $z_1$ to $z_2$}
\end{figure}

\begin{figure}
\begin{center}
\includegraphics{simple3}
\end{center}
\caption{The mapping from $z_2$ to $w$}
\end{figure}

%%%%%
%%%%%
\end{document}
