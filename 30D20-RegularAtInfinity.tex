\documentclass[12pt]{article}
\usepackage{pmmeta}
\pmcanonicalname{RegularAtInfinity}
\pmcreated{2013-03-22 17:37:30}
\pmmodified{2013-03-22 17:37:30}
\pmowner{pahio}{2872}
\pmmodifier{pahio}{2872}
\pmtitle{regular at infinity}
\pmrecord{8}{40045}
\pmprivacy{1}
\pmauthor{pahio}{2872}
\pmtype{Definition}
\pmcomment{trigger rebuild}
\pmclassification{msc}{30D20}
\pmclassification{msc}{32A10}
\pmsynonym{analytic at infinity}{RegularAtInfinity}
%\pmkeywords{regular}
\pmrelated{RegularFunction}
\pmrelated{ClosedComplexPlane}
\pmrelated{VanishAtInfinity}
\pmrelated{ResidueAtInfinity}

% this is the default PlanetMath preamble.  as your knowledge
% of TeX increases, you will probably want to edit this, but
% it should be fine as is for beginners.

% almost certainly you want these
\usepackage{amssymb}
\usepackage{amsmath}
\usepackage{amsfonts}

% used for TeXing text within eps files
%\usepackage{psfrag}
% need this for including graphics (\includegraphics)
%\usepackage{graphicx}
% for neatly defining theorems and propositions
 \usepackage{amsthm}
% making logically defined graphics
%%%\usepackage{xypic}

% there are many more packages, add them here as you need them

% define commands here

\theoremstyle{definition}
\newtheorem*{thmplain}{Theorem}

\begin{document}
When the function $w$ of one complex variable is regular in the annulus
               $$\varrho \;<\; |z| \;<\; \infty,$$
it has a Laurent expansion
\begin{align}
  w(z) \;=\; \sum_{n=-\infty}^{\infty}c_nz^n.
\end{align}
If especially the coefficients $c_1,\, c_2,\,\ldots$ vanish, then we have
   $$w(z) \;=\; c_0+\frac{c_{-1}}{z}+\frac{c_{-2}}{z^2}+\ldots$$
Using the inversion \,$z = \frac{1}{\zeta}$,\, we see that the function
     $$w\!\left(\frac{1}{\zeta}\right) \;=\; c_0+c_{-1}\zeta+c_{-2}\zeta^2+\ldots$$
is regular in the disc \,$|\zeta| < \varrho$.\, Accordingly we can define that the function $w$ is {\em regular at infinity} also.

For example,\, $\displaystyle w(z) \;:=\; \frac{1}{z}$\, is regular at the point \,$z = \infty$\, and\, $w(\infty) = 0$.\, Similarly, $e^{\frac{1}{z}}$ is regular at $\infty$ and has there the value 1.

%%%%%
%%%%%
\end{document}
