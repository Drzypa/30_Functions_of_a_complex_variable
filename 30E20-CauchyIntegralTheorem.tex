\documentclass[12pt]{article}
\usepackage{pmmeta}
\pmcanonicalname{CauchyIntegralTheorem}
\pmcreated{2013-03-22 12:54:12}
\pmmodified{2013-03-22 12:54:12}
\pmowner{rmilson}{146}
\pmmodifier{rmilson}{146}
\pmtitle{Cauchy integral theorem}
\pmrecord{16}{33253}
\pmprivacy{1}
\pmauthor{rmilson}{146}
\pmtype{Theorem}
\pmcomment{trigger rebuild}
\pmclassification{msc}{30E20}
\pmsynonym{fundamental theorem of complex analysis}{CauchyIntegralTheorem}
\pmrelated{ClosedCurveTheorem}
\pmrelated{CauchyResidueTheorem}
\pmdefines{Goursat's Theorem}

\usepackage{amsmath}
\usepackage{amsfonts}
\usepackage{amssymb}
\newcommand{\reals}{\mathbb{R}}
\newcommand{\natnums}{\mathbb{N}}
\newcommand{\cnums}{\mathbb{C}}
\newcommand{\znums}{\mathbb{Z}}
\newcommand{\lp}{\left(}
\newcommand{\rp}{\right)}
\newcommand{\lb}{\left[}
\newcommand{\rb}{\right]}
\newcommand{\supth}{^{\text{th}}}
\newtheorem{proposition}{Proposition}
\newtheorem{definition}[proposition]{Definition}
\newtheorem{theorem}[proposition]{Theorem}
\newcommand{\nl}[1]{{\PMlinkescapetext{{#1}}}}
\newcommand{\pln}[2]{{\PMlinkname{{#1}}{#2}}}
\begin{document}
\begin{theorem}
Let $U\subset\cnums$ be an open, simply connected domain, and let
$f:U\rightarrow \cnums$ be a function whose complex derivative, that is
$$\lim_{w\rightarrow z} \frac{f(w)-f(z)}{w-z},$$
exists for all $z\in U$.
Then, the \PMlinkname{integral}{Integral2} around  every closed contour
$\gamma\subset U$ vanishes; in symbols
$$\oint_\gamma f(z)\, dz = 0.$$
\end{theorem}

We also have the following, technically important generalization
involving  removable singularities.
\begin{theorem}
  Let $U\subset\cnums$ be an open, simply connected domain, and
  $S\subset U$ a finite subset. Let $f:U\backslash S \rightarrow
  \cnums$ be a function whose complex derivative exists for all $z\in
  U\backslash S$, and that is bounded near all $z\in S$.  Let $\gamma\subset    
   U\backslash S$ be a closed contour 
  that avoids the exceptional points.  Then, the integral of $f$ around $\gamma$ vanishes.
\end{theorem}


Cauchy's theorem is an essential stepping stone in the theory of
complex analysis. It is required for the proof of the Cauchy integral
formula, which in turn is required for the proof that the existence of
a complex derivative implies a power series representation.

The original version of the theorem, as stated by Cauchy in the early
1800s, requires that the derivative $f'(z)$ exist and be continuous.
The existence of $f'(z)$ implies the Cauchy-Riemann equations, which
in turn can be restated as the fact that the complex-valued
differential $f(z)\, dz$ is closed. The original proof makes use of
this fact, and calls on Green's Theorem to conclude that the contour
integral vanishes. The proof of Green's theorem, however, involves an
interchange of order in a double integral, and this can only be
justified if the integrand, which involves the real and imaginary
parts of $f'(z)$, is assumed to be continuous. To this date, many
authors prove the theorem this way, but erroneously fail to mention
the continuity assumption.

In the latter part of the $19\supth$ century E. Goursat found a proof
of the integral theorem that merely required that $f'(z)$ exist.
Continuity of the derivative, as well as the existence of all higher
derivatives, then follows as a consequence of the Cauchy integral
formula. Not only is Goursat's version a sharper result, but it is
also more elementary and self-contained, in that sense that it is does
not require Green's theorem. Goursat's argument makes use of
rectangular contour (many authors use triangles though), but the
extension to an arbitrary simply-connected domain is relatively
straight-forward.

\begin{theorem}[Goursat]
Let $U$ be an open domain containing a rectangle
$$R = \{ x+iy\in\cnums: a\leq x\leq b\,, c\leq y\leq d\}.$$
If the complex derivative of a function $f:U\rightarrow \cnums$ exists
at all points of $U$, then the contour integral of $f$ around the
boundary of $R$ vanishes; in symbols
$$\oint_{\partial R} f(z)\,dz = 0.$$
\end{theorem}

\paragraph{Bibliography.}
\begin{itemize}
\item  Ahlfors, L., Complex Analysis. McGraw-Hill, 1966.
\end{itemize}
%%%%%
%%%%%
\end{document}
