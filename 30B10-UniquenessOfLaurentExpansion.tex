\documentclass[12pt]{article}
\usepackage{pmmeta}
\pmcanonicalname{UniquenessOfLaurentExpansion}
\pmcreated{2013-03-22 19:14:12}
\pmmodified{2013-03-22 19:14:12}
\pmowner{pahio}{2872}
\pmmodifier{pahio}{2872}
\pmtitle{uniqueness of Laurent expansion}
\pmrecord{11}{42161}
\pmprivacy{1}
\pmauthor{pahio}{2872}
\pmtype{Theorem}
\pmcomment{trigger rebuild}
\pmclassification{msc}{30B10}
\pmrelated{CoefficientsOfLaurentSeries}
\pmrelated{UniquenessOfFourierExpansion}
\pmrelated{UniquenessOfDigitalRepresentation}

% this is the default PlanetMath preamble.  as your knowledge
% of TeX increases, you will probably want to edit this, but
% it should be fine as is for beginners.

% almost certainly you want these
\usepackage{amssymb}
\usepackage{amsmath}
\usepackage{amsfonts}

% used for TeXing text within eps files
%\usepackage{psfrag}
% need this for including graphics (\includegraphics)
%\usepackage{graphicx}
% for neatly defining theorems and propositions
 \usepackage{amsthm}
% making logically defined graphics
%%%\usepackage{xypic}

% there are many more packages, add them here as you need them

% define commands here

\theoremstyle{definition}
\newtheorem*{thmplain}{Theorem}

\begin{document}
\PMlinkescapeword{expansion} \PMlinkescapeword{expansions}

The Laurent series expansion of a function $f(z)$ in an annulus \, $r < |z\!-\!z_0| < R$\, is unique.\\

\emph{Proof.}\, Suppose that $f(z)$ has in the annulus two Laurent expansions:
$$f(z) \;=\; \sum_{n=-\infty}^\infty\!a_n(z\!-\!z_0)^n 
       \;=\; \sum_{n=-\infty}^\infty\!b_n(z\!-\!z_0)^n$$
It follows that
$$f(z)(z\!-\!z_0)^{-\nu-1}\;=\; \sum_{n=-\infty}^\infty\!a_n(z\!-\!z_0)^{n-\nu-1} 
                          \;=\; \sum_{n=-\infty}^\infty\!b_n(z\!-\!z_0)^{n-\nu-1}$$
where $\nu$ is an integer.\, Let now $\gamma$ be an arbitrary closed contour in the annulus, going once around $z_0$.\, Since $\gamma$ is a compact set of points, those two Laurent series \PMlinkname{converge uniformly}{UniformConvergence} on it and therefore they can be \PMlinkname{integrated termwise}{SumFunctionOfSeries} along $\gamma$, i.e.
\begin{align}
\sum_{n=-\infty}^\infty\!a_n\oint_\gamma(z\!-\!z_0)^{n-\nu-1}\,dz \;=\; 
\sum_{n=-\infty}^\infty\!b_n\oint_\gamma(z\!-\!z_0)^{n-\nu-1}\,dz.
\end{align}
But
\begin{align*}
 \oint_\gamma(z\!-\!z_0)^{n-\nu-1}\,dz \;=\;
   \begin{cases}
     2i\pi \quad \mbox{if} \;\; n \;=\; \nu, \\
     0    \qquad \mbox{if} \;\; n \;\neq\; \nu, \\
   \end{cases}
\end{align*}
when integrated anticlockwise (see calculation of contour integral).\, Thus (1) reads 
$$2i\pi a_\nu \;=\; 2i\pi b_\nu,$$
i.e.\, $a_\nu = b_\nu$,\, for any integer $\nu$, whence both expansions are identical.


%%%%%
%%%%%
\end{document}
