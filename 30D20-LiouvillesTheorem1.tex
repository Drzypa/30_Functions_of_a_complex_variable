\documentclass[12pt]{article}
\usepackage{pmmeta}
\pmcanonicalname{LiouvillesTheorem}
\pmcreated{2013-03-22 12:04:31}
\pmmodified{2013-03-22 12:04:31}
\pmowner{djao}{24}
\pmmodifier{djao}{24}
\pmtitle{Liouville's theorem}
\pmrecord{8}{31145}
\pmprivacy{1}
\pmauthor{djao}{24}
\pmtype{Theorem}
\pmcomment{trigger rebuild}
\pmclassification{msc}{30D20}

\usepackage{amssymb}
\usepackage{amsmath}
\usepackage{amsfonts}
\usepackage{graphicx}
%%%\usepackage{xypic}
\begin{document}
A bounded entire function is constant. That is, a bounded complex function $f: \mathbb{C} \to \mathbb{C}$ which is holomorphic on the entire complex plane is always a constant function.

More generally, any holomorphic function $f: \mathbb{C} \to \mathbb{C}$ which satisfies a polynomial bound condition of the form
$$
|f(z)| < c \cdot |z|^n
$$
for some $c \in \mathbb{R}$, $n \in \mathbb{Z}$, and all $z \in \mathbb{C}$ with $|z|$ sufficiently large is necessarily equal to a polynomial function.

Liouville's theorem is a vivid example of how stringent the holomorphicity condition on a complex function really is. One has only to compare the theorem to the corresponding statement for real functions (namely, that a bounded differentiable real function is constant, a patently false statement) to see how much stronger the complex differentiability condition is compared to real differentiability.

Applications of Liouville's theorem include proofs of the fundamental theorem of algebra and of the partial fraction decomposition theorem for rational functions.
%%%%%
%%%%%
%%%%%
\end{document}
