\documentclass[12pt]{article}
\usepackage{pmmeta}
\pmcanonicalname{ClosurePropertiesOfCauchyRiemannEquations}
\pmcreated{2013-03-22 17:44:20}
\pmmodified{2013-03-22 17:44:20}
\pmowner{rspuzio}{6075}
\pmmodifier{rspuzio}{6075}
\pmtitle{closure properties of Cauchy-Riemann equations}
\pmrecord{14}{40189}
\pmprivacy{1}
\pmauthor{rspuzio}{6075}
\pmtype{Theorem}
\pmcomment{trigger rebuild}
\pmclassification{msc}{30E99}
\pmrelated{TangentialCauchyRiemannComplexOfCinftySmoothForms}
\pmrelated{ACRcomplex}

% this is the default PlanetMath preamble.  as your knowledge
% of TeX increases, you will probably want to edit this, but
% it should be fine as is for beginners.

% almost certainly you want these
\usepackage{amssymb}
\usepackage{amsmath}
\usepackage{amsfonts}

% used for TeXing text within eps files
%\usepackage{psfrag}
% need this for including graphics (\includegraphics)
%\usepackage{graphicx}
% for neatly defining theorems and propositions
\usepackage{amsthm}
% making logically defined graphics
%%%\usepackage{xypic}

% there are many more packages, add them here as you need them

% define commands here

\newtheorem{thm}{Theorem}
\begin{document}
The set of solutions of the Cauchy-Riemann equations is closed under
a surprisingly large number of operations.  For convenience, let
us introduce the notational conventions that $f$ and $g$ are complex
functions with $f(x+iy) = u(x,y) + i v(x,y)$ and 
$g(x + iy) = p(x,y) + i q (x,y)$.  Let $D$  and $D'$ denote
open subsets of the complex plane.

\begin{thm}
If $f \colon D \to \mathbb{C}$ and 
$g \colon D \to \mathbb{C}$ 
satisfy the Cauchy-Riemann equations, so does $f + g$.  Furthermore, if $z\in \mathbb{C}$, then $zf$ satisfies the Cauchy-Riemann equations.
\end{thm}

\begin{proof}
This is an immediate consequence of the linearity of derivatives.
\end{proof}

\begin{thm}
If $f \colon D \to \mathbb{C}$ and 
$g \colon D \to \mathbb{C}$ satisfy the
Cauchy-Riemann equations, so does $f \cdot g$.
\end{thm}

\begin{proof}
Letting $h$ and $k$ denote the real and imaginary parts of $f \cdot g$
respectively, we have
\begin{align*}
 {\partial h \over \partial x} - {\partial k \over \partial y} &=
 {\partial \over \partial x} \left( up - vq \right) -
 {\partial \over \partial y} \left( uq + vp \right) \\ &= 
 u {\partial p \over \partial x} + p {\partial u \over \partial x} -
 v {\partial q \over \partial x} - q {\partial v \over \partial x} -
 u {\partial q \over \partial y} - q {\partial u \over \partial y} -
 v {\partial p \over \partial y} - p {\partial v \over \partial y} \\ &=
 u \left( {\partial p \over \partial x} - {\partial q \over \partial y} \right) -
 v \left( {\partial p \over \partial y} + {\partial q \over \partial x} \right) +
 p \left( {\partial u \over \partial x} - {\partial v \over \partial y} \right) -
 q \left( {\partial u \over \partial y} + {\partial v \over \partial x} \right) = 0
\end{align*}
and
\begin{align*}
 {\partial h \over \partial y} + {\partial k \over \partial x} &=
 {\partial \over \partial y} \left( up - vq \right) +
 {\partial \over \partial x} \left( uq + vp \right) \\ &= 
 u {\partial p \over \partial y} + p {\partial u \over \partial y} -
 v {\partial q \over \partial y} - q {\partial v \over \partial y} +
 u {\partial q \over \partial x} + q {\partial u \over \partial x} +
 v {\partial p \over \partial x} + p {\partial v \over \partial x} \\ &=
 u \left( {\partial p \over \partial y} + {\partial q \over \partial x} \right) +
 v \left( {\partial p \over \partial x} - {\partial q \over \partial y} \right) +
 p \left( {\partial u \over \partial y} + {\partial v \over \partial x} \right) +
 q \left( {\partial u \over \partial x} - {\partial v \over \partial y} \right) = 0 .
\end{align*}
\end{proof}

\begin{thm}
If $f \colon D \to D'$ and 
$g \colon D' \to \mathbb{C}$ satisfy the
Cauchy-Riemann equations, so does $f \circ g$.
\end{thm}

\begin{proof}
Letting $h$ and $k$ denote the real and imaginary parts of $f \circ g$
respectively, we have
\begin{align*}
 {\partial h \over \partial x} - {\partial k \over \partial y} &=
 {\partial \over \partial x} u(p(x,y),q(x,y)) -
 {\partial \over \partial y} v(p(x,y),q(x,y)) \\ &=
 {\partial u \over \partial p} {\partial p \over \partial x} +
 {\partial u \over \partial q} {\partial q \over \partial x} -
 {\partial v \over \partial p} {\partial p \over \partial y} -
 {\partial v \over \partial q} {\partial q \over \partial y} \\ &=
 {\partial u \over \partial p} \left( {\partial p \over \partial x} -
                                      {\partial q \over \partial y} \right) +
 {\partial q \over \partial y} \left( {\partial u \over \partial p} -
                                      {\partial v \over \partial q} \right) +
 {\partial u \over \partial q} \left( {\partial p \over \partial y} +
                                      {\partial q \over \partial x} \right) -
 {\partial p \over \partial y} \left( {\partial u \over \partial q} +
                                      {\partial v \over \partial p} \right) = 0
\end{align*}
and
\begin{align*}
 {\partial h \over \partial y} + {\partial k \over \partial x} &=
 {\partial \over \partial y} u(p(x,y),q(x,y)) +
 {\partial \over \partial x} v(p(x,y),q(x,y)) \\ &=
 {\partial u \over \partial p} {\partial p \over \partial y} +
 {\partial u \over \partial q} {\partial q \over \partial y} +
 {\partial v \over \partial p} {\partial p \over \partial x} +
 {\partial v \over \partial q} {\partial q \over \partial x} \\ &=
 {\partial u \over \partial p} \left( {\partial p \over \partial y} +
                                      {\partial q \over \partial x} \right) -
 {\partial q \over \partial x} \left( {\partial u \over \partial p} -
                                      {\partial v \over \partial q} \right) -
 {\partial u \over \partial q} \left( {\partial p \over \partial x} -
                                      {\partial q \over \partial y} \right) +
 {\partial p \over \partial x} \left( {\partial u \over \partial q} +
                                      {\partial v \over \partial p} \right) = 0
\end{align*}

\end{proof}

\begin{thm}
If $f \colon D \to \mathbb{C}$ satisfies the Cauchy-Riemann equations, 
and has non-vanishing Jacobian, then $f^{-1}$
also satisfies the Cauchy-Riemann equations.
\end{thm}

\begin{proof}
Let us denote the real and imaginary parts of $f^{-1}$ as $h$ and $k$, respectively.
Then, by definition of inverse function, we have
\begin{align*}
 u(h(x,y), k(x,y)) &= x \\
 v(h(x,y), k(x,y)) &= y .
\end{align*}
Taking derivatives,
\begin{align*}
 {\partial u \over \partial h} {\partial h \over \partial x} +
 {\partial u \over \partial k} {\partial k \over \partial x} &= 1 \\
 {\partial u \over \partial h} {\partial h \over \partial y} +
 {\partial u \over \partial k} {\partial k \over \partial y} &= 0 \\
 {\partial v \over \partial h} {\partial h \over \partial x} +
 {\partial v \over \partial k} {\partial k \over \partial x} &= 0 \\
 {\partial v \over \partial h} {\partial h \over \partial y} +
 {\partial v \over \partial k} {\partial k \over \partial y} &= 1
\end{align*}
By the Cauchy-Riemann equations, $\partial u / \partial h =
\partial v / \partial k$ and $\partial u / \partial k = -
\partial v / \partial h$.  Using these relations to re-express the
derivatives of $u$ as derivatives of $v$, then subtracting the 
fourth equation form the first equation and adding the second and 
third equations, we obtain
\begin{align*}
 {\partial u \over \partial h}
  \left(
   {\partial h \over \partial x} -
   {\partial k \over \partial y} 
  \right) +
 {\partial u \over \partial k}
  \left(
   {\partial h \over \partial y} +
   {\partial k \over \partial x} 
  \right) &= 0 \\
 {\partial u \over \partial h}
  \left(
   {\partial h \over \partial y} +
   {\partial k \over \partial x}
  \right) -
 {\partial u \over \partial k}
  \left(
   {\partial h \over \partial x} -
   {\partial k \over \partial y} 
  \right) &= 0 .
\end{align*}
With a little algebraic manipulation, we may conclude
\begin{align*}
 \left(
  \left( 
   {\partial u \over \partial h}
  \right)^2 +
  \left( 
   {\partial u \over \partial k}
  \right)^2
 \right)
  \left(
   {\partial h \over \partial y} +
   {\partial k \over \partial x}
  \right) &= 0 \\
 \left(
  \left( 
   {\partial u \over \partial h}
  \right)^2 +
  \left( 
   {\partial u \over \partial k}
  \right)^2
 \right)
  \left(
   {\partial h \over \partial x} -
   {\partial k \over \partial y} 
  \right) &= 0 .
\end{align*}
Note that, by the Cauchy-Riemann equations, the Jacobian of $f$
equals the common prefactor of these equations:
\[
 {\partial (u,v) \over \partial (h,k)} =
 {\partial u \over \partial h} {\partial v \over \partial k} -
 {\partial u \over \partial k} {\partial v \over \partial h} =
 \left( 
   {\partial u \over \partial h}
 \right)^2 +
 \left( 
   {\partial u \over \partial k}
 \right)^2
\]
Hence, by assumptions, this quantity differs from zero and we may 
cancel it to obtain the Cauchy-Riemann equations for $f^{-1}$.
\end{proof}

%%%%%
%%%%%
\end{document}
