\documentclass[12pt]{article}
\usepackage{pmmeta}
\pmcanonicalname{SecondFormOfCauchyIntegralTheorem}
\pmcreated{2013-03-22 15:19:39}
\pmmodified{2013-03-22 15:19:39}
\pmowner{pahio}{2872}
\pmmodifier{pahio}{2872}
\pmtitle{second form of Cauchy integral theorem}
\pmrecord{9}{37139}
\pmprivacy{1}
\pmauthor{pahio}{2872}
\pmtype{Theorem}
\pmcomment{trigger rebuild}
\pmclassification{msc}{30E20}
\pmsynonym{equivalent form of Cauchy integral theorem}{SecondFormOfCauchyIntegralTheorem}
%\pmkeywords{analytic}
%\pmkeywords{holomorphic}
\pmrelated{CauchyIntegralTheorem}
\pmdefines{example of non-analytic function}

\endmetadata

% this is the default PlanetMath preamble.  as your knowledge
% of TeX increases, you will probably want to edit this, but
% it should be fine as is for beginners.

% almost certainly you want these
\usepackage{amssymb}
\usepackage{amsmath}
\usepackage{amsfonts}

% used for TeXing text within eps files
%\usepackage{psfrag}
% need this for including graphics (\includegraphics)
%\usepackage{graphicx}
% for neatly defining theorems and propositions
 \usepackage{amsthm}
% making logically defined graphics
%%%\usepackage{xypic}

% there are many more packages, add them here as you need them

% define commands here

\theoremstyle{definition}
\newtheorem*{thmplain}{Theorem}
\begin{document}
\begin{thmplain}
Let the complex function $f$ be analytic in a simply connected open domain $U$ of the complex plane, and let $a$ and $b$ be any two points of $U$.\, Then the contour integral
\begin{align}
\int_\gamma f(z)\,dz
\end{align}
is independent on the path $\gamma$ which in $U$ goes from $a$ to $b$.
\end{thmplain}


\textbf{Example.}\, Let's consider the integral (1) of the real part function defined by
$$f(z) := \mbox{Re}(z)$$
with the path $\gamma$ going from the point\, $O = (0,\,0)$\, to the point\, $Q = (1,\,1)$.\, If $\gamma$ is the line segment $OQ$, we may use the substitution
    $$z := (1\!+\!i)t,\quad dz = (1\!+\!i)\,dt,\quad 0 \leqq t \leqq 1,$$
and (1) equals 
        $$\int_0^1t\!\cdot\!(1\!+\!i)\,dt = \frac{1}{2}\!+\!\frac{1}{2}i.$$
Secondly, we choose for $\gamma$ the broken line $OPQ$ where\, $P = (1,\,0)$.\, Now (1) is the sum
      $$\int_{OP}\mbox{Re}(z)\,dz+\int_{PQ}\mbox{Re}(z)\,dz 
   = \int_0^1x\,dx+\int_0^1i\,dy = \frac{1}{2}\!+\!i.$$
Thus, the integral (1) of the function depends on the path between the two points.\, This is explained by the fact that the \PMlinkescapetext{real part} function $f$ is not analytic --- its real part $x$ and imaginary part 0 do not satisfy the Cauchy-Riemann equations.
%%%%%
%%%%%
\end{document}
