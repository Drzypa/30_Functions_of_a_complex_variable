\documentclass[12pt]{article}
\usepackage{pmmeta}
\pmcanonicalname{ComplexFunction}
\pmcreated{2014-02-23 10:20:21}
\pmmodified{2014-02-23 10:20:21}
\pmowner{Wkbj79}{1863}
\pmmodifier{pahio}{2872}
\pmtitle{complex function}
\pmrecord{12}{36226}
\pmprivacy{1}
\pmauthor{Wkbj79}{2872}
\pmtype{Definition}
\pmcomment{trigger rebuild}
\pmclassification{msc}{30A99}
\pmclassification{msc}{03E20}
\pmrelated{RealFunction}
\pmrelated{Meromorphic}
\pmrelated{Holomorphic}
\pmrelated{Entire}
\pmrelated{IndexOfSpecialFunctions}
\pmrelated{ValuesOfComplexCosine}
\pmdefines{real part}
\pmdefines{imaginary part}
\pmdefines{function theory}
\pmdefines{complex analysis}

% this is the default PlanetMath preamble.  as your knowledge
% of TeX increases, you will probably want to edit this, but
% it should be fine as is for beginners.

% almost certainly you want these
\usepackage{amssymb}
\usepackage{amsmath}
\usepackage{amsfonts}

% used for TeXing text within eps files
%\usepackage{psfrag}
% need this for including graphics (\includegraphics)
%\usepackage{graphicx}
% for neatly defining theorems and propositions
%\usepackage{amsthm}
% making logically defined graphics
%%%\usepackage{xypic}

% there are many more packages, add them here as you need them

% define commands here
\begin{document}
A {\em complex function} is a function $f$ from a subset $A$ of 
$\mathbb{C}$ to $\mathbb{C}$.

For every\, $z = x+iy\in A\,\,\,(x,\,y \in \mathbb{R})$\, the 
complex value $f(z)$ can be split into its real and imaginary 
parts $u$ and $v$, respectively, which can be considered as 
real functions of two real variables:
\begin{align}
f(z) \;=\; u(x,y)+iv(x,y)
\end{align}
The functions $u$ and $v$ are called the {\em real part} and 
the {\em imaginary part} of the complex function $f$, 
respectively.\, Conversely, any two functions $u(x,y)$ and 
$v(x,y)$ defined in some subset of $\mathbb{R}^2$ determine via 
(1) a complex function $f$.\\

If $f(z)$ especially is defined as a polynomial 
of $z$, then both $u(x,y)$ and $v(x,y)$ are polynomials of $x$ and 
$y$ with real coefficients.\\

Following are the notations for $u$ and $v$ that
are used most commonly (the parentheses around $f(z)$ may be 
omitted):

$$u(x,y) \;=\; \mbox{Re}\left(f(z)\right) = \Re\left(f(z)\right)$$
$$v(x,y) \;=\; \mbox{Im}\left(f(z)\right) = \Im\left(f(z)\right)$$

The \PMlinkescapetext{branch} of mathematics concerning differentiable complex functions is called {\em function theory} or {\em complex analysis}.
%%%%%
%%%%%
\end{document}
