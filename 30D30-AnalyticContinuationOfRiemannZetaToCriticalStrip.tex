\documentclass[12pt]{article}
\usepackage{pmmeta}
\pmcanonicalname{AnalyticContinuationOfRiemannZetaToCriticalStrip}
\pmcreated{2015-08-22 13:33:33}
\pmmodified{2015-08-22 13:33:33}
\pmowner{pahio}{2872}
\pmmodifier{pahio}{2872}
\pmtitle{analytic continuation of Riemann zeta to critical strip}
\pmrecord{23}{39150}
\pmprivacy{1}
\pmauthor{pahio}{2872}
\pmtype{Example}
\pmcomment{trigger rebuild}
\pmclassification{msc}{30D30}
\pmclassification{msc}{30B40}
\pmclassification{msc}{11M41}
\pmrelated{RiemannZetaFunction}
\pmrelated{AnalyticContinuation}
\pmrelated{MeromorphicExtension}
\pmrelated{CriticalStrip}
\pmrelated{AnalyticContinuationOfRiemannZetaUsingIntegral}
\pmrelated{FormulaeForZetaInTheCriticalStrip}
\pmrelated{GammaFunction}
\pmdefines{alternating zeta function}

\endmetadata

% this is the default PlanetMath preamble.  as your knowledge
% of TeX increases, you will probably want to edit this, but
% it should be fine as is for beginners.

% almost certainly you want these
\usepackage{amssymb}
\usepackage{amsmath}
\usepackage{amsfonts}

% used for TeXing text within eps files
%\usepackage{psfrag}
% need this for including graphics (\includegraphics)
%\usepackage{graphicx}
% for neatly defining theorems and propositions
 \usepackage{amsthm}
% making logically defined graphics
%%%\usepackage{xypic}

% there are many more packages, add them here as you need them

% define commands here

\theoremstyle{definition}
\newtheorem*{thmplain}{Theorem}

\begin{document}
The \PMlinkescapetext{terms}\, $\frac{1}{n^s} = e^{-s\log{n}}$\, (see the general power) of the series
\begin{align}
\sum_{n=1}^\infty\frac{1}{n^s} \;=\;
 1+\frac{1}{2^s}+\frac{1}{3^s}+\frac{1}{4^s}+\ldots,
\end{align}
defining the Riemann zeta function $\zeta(s)$ for\, 
$\Re{s} > 1$,\, are holomorphic in the whole $s$-plane and the 
series converges uniformly in any closed disc of the 
half-plane\, $\Re{s} > 1$ ((let\, $s = \sigma+it$\, with 
$\sigma,\, t\in\mathbb{R}$\, and $\sigma > 1$;\, 
then $|\frac{1}{n^s}| = \frac{1}{n^\sigma} \le \frac{1}{n^{1+d}}$\,
for a positive $d$ for all\, $n = 1,\,2,\,\ldots$;\, 
the series\, $\sum_{n=1}^\infty\frac{1}{n^{1+d}}$ converges 
since\, $1\!+\!d > 1$;\, thus the series (1) converges uniformly 
in the closed half-plane\, $\Re{s} \ge 1\!+\!d$,\, by the 
\PMlinkname{Weierstrass criterion}{WeierstrassCriterionOfUniformConvergence})).\, 
Therefore we can infer (see 
\PMlinkname{theorems on complex function series}{TheoremsOnComplexFunctionSeries}) 
that the sum $\zeta(s)$ of (1) is holomorphic in the domain\, 
$\Re{s} > 1$.

We use also the fact that the series
\begin{align}
\sum_{n=1}^\infty\frac{(-1)^{n-1}}{n^s} \;=\;
 1-\frac{1}{2^s}+\frac{1}{3^s}-\frac{1}{4^s}+-\ldots
\end{align}

defining the Dirichlet eta function $\eta(s)$, a.k.a. the {\it alternating zeta function}, is convergent for\, 
$\Re{s} > 0$\, and its sum is holomorphic in this half-plane.


If we multiply the series (1) by the difference 
$\displaystyle 1\!-\!\frac{2}{2^s}$, every other \PMlinkescapetext{term} of the series changes its sign and we get the series (2).\, So we can write
\begin{align}
\zeta(s) \;=\; \frac{\eta(s)}{1-\frac{2}{2^s}},
\end{align}
which is valid when the denominator does not vanish and\, $\Re{s} > 1$.\, The zeros of the denominator are obtained from\, $2^s = 2$,\, i.e. from
$$e^{s\log{2}} \;=\; e^{\log{2}}.$$
This gives\, $s\log{2}-\log{2} = n\cdot2i\pi$ (see the periodicity of exponential function), i.e.
\begin{align}
s \;=\; 1\!+\!n\!\cdot\!\frac{2\pi i}{\ln{2}}\quad(n\in\mathbb{Z}).
\end{align}
Thus the zeros of the denominator of (3) are on the line \,$\Re{s} = 1$.

Now the function on the right hand side of (3) is holomorphic in the set
$$D \;:=\; \{s\in\mathbb{C}\,\vdots\,\,\, \Re{s} > 0\}\smallsetminus 
      \{1\!+\!n\!\cdot\!\frac{2\pi i}{\ln{2}}\,\vdots\,\,\, n\in\mathbb{Z}\}$$
and the values of this function coincide with the values of zeta function in the half-plane\, $\Re{s} > 1$.

This result means that, via the equation (3), the zeta function has been \PMlinkname{analytically continued}{AnalyticContinuation} to the domain $D$, as far as to the imaginary axis.\\

\textbf{Remark.}\, In reality, all points (4) except\, $s = 1$\, are removable singularities of $\zeta(s)$ given by (3), due to the fact that they are also zeros of $\eta(s)$.\, The fact is considered in the entry zeros of Dirichlet eta function.\\

Charles Hermite has shown that the zeta function may be analytically continued to the whole $s$-plane except for a simple pole at\, $s = 1$,\, by using the equation
\begin{align}
\zeta(s) \;=\; \frac{1}{\Gamma(s)}\int_0^\infty\!\frac{x^{s-1}}{e^x-1}\,dx.
\end{align}
See \PMlinkname{this article}{AnalyticContinuationOfRiemannZetaUsingIntegral}.
%%%%%
%%%%%
\end{document}
