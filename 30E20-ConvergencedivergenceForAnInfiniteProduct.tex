\documentclass[12pt]{article}
\usepackage{pmmeta}
\pmcanonicalname{ConvergencedivergenceForAnInfiniteProduct}
\pmcreated{2013-03-22 13:36:11}
\pmmodified{2013-03-22 13:36:11}
\pmowner{aoh45}{5079}
\pmmodifier{aoh45}{5079}
\pmtitle{convergence/divergence for an infinite product}
\pmrecord{13}{34230}
\pmprivacy{1}
\pmauthor{aoh45}{5079}
\pmtype{Definition}
\pmcomment{trigger rebuild}
\pmclassification{msc}{30E20}
\pmrelated{AbsoluteConvergenceImpliesConvergenceForAnInfiniteProduct}

\endmetadata

% almost certainly you want these
\usepackage{amssymb}
\usepackage{amsmath}
\usepackage{amsfonts}
\begin{document}
Consider $\prod_{n=1}^{\infty} p_n$. We say that this infinite product converges iff the finite products $P_m = \prod_{n=1}^{m} p_n \longrightarrow P$ converge. Otherwise the infinite product is called divergent.

% or for at most a finite number of terms 
%$p_{n_{k}}=0 \hspace{2mm}, \hspace{2mm} k=1,\ldots,K$. 

%% Possibly incorrect note:
%% Note: The infinite product vanishes only if a factor is zero.
%%%%%
%%%%%
\end{document}
