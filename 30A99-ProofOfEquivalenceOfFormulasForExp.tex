\documentclass[12pt]{article}
\usepackage{pmmeta}
\pmcanonicalname{ProofOfEquivalenceOfFormulasForExp}
\pmcreated{2013-03-22 15:22:52}
\pmmodified{2013-03-22 15:22:52}
\pmowner{stevecheng}{10074}
\pmmodifier{stevecheng}{10074}
\pmtitle{proof of equivalence of formulas for exp}
\pmrecord{11}{37210}
\pmprivacy{1}
\pmauthor{stevecheng}{10074}
\pmtype{Proof}
\pmcomment{trigger rebuild}
\pmclassification{msc}{30A99}
\pmrelated{ComplexExponentialFunction}
\pmrelated{ExponentialFunction}
\pmrelated{MatrixExponential}

% this is the default PlanetMath preamble.  as your knowledge
% of TeX increases, you will probably want to edit this, but
% it should be fine as is for beginners.

% almost certainly you want these
\usepackage{amssymb}
\usepackage{amsmath}
\usepackage{amsfonts}
\usepackage{amsthm}

% used for TeXing text within eps files
%\usepackage{psfrag}
% need this for including graphics (\includegraphics)
%\usepackage{graphicx}
% for neatly defining theorems and propositions
%\usepackage{amsthm}
% making logically defined graphics
%%%\usepackage{xypic}

% there are many more packages, add them here as you need them

% define commands here

\newcommand{\nat}{\mathbb{N}}
\providecommand{\abs}[1]{\lvert#1\rvert}
\providecommand{\absB}[1]{\Bigl\lvert#1\Bigr\rvert}
\providecommand{\absW}[1]{\left\lvert#1\right\rvert}
\begin{document}
We present an elementary proof that:
\begin{align*}
\sum_{k=0}^\infty \frac{z^k}{k!} = \lim_{n\to\infty} \left( 1 + \frac{z}{n} \right)^n\,.
\end{align*}
There are of course other proofs, but this one has the advantage that it carries verbatim for the matrix exponential and the operator exponential. 

At the outset, we observe that 
$\sum_{k=0}^\infty z^k/k!$ converges by the ratio test. 
For definiteness, the notation $e^z$ below will refer to exactly this series.

\begin{proof}
We expand the right-hand \PMlinkescapetext{side} in the straightforward manner:
\begin{align*}
\left( 1 + \frac{z}{n} \right)^n &= \sum_{k=0}^n \binom{n}{k} \left(\frac{z}{n}\right)^k \\
&= \sum_{k=0}^n \frac{n \cdot (n-1) \dotsm (n-k+1) }{n^k} \frac{z^k}{k!} 
= \sum_{k=0}^n \pi(k, n) \, \frac{z^k}{k!}\,,
\end{align*}
where $\pi(k, n)$ denotes the coefficient
\begin{align*}
1 \left(1 - \frac{1}{n}\right) \cdot \left(1 -\frac{2}{n} \right) \dotsm \left(1 - \frac{k-1}{n} \right)\,.
\end{align*}
Let $\abs{z} \leq M$.
Given $\epsilon > 0$, there is a $N \in \nat$ such that whenever $n \geq N$, then
$\sum_{k=n+1}^\infty M^k/k! < \epsilon/2$, 
since the sum is the tail of the convergent series $e^M$.

Since $\lim_{n \to \infty} \pi(k,n) = 1$ for \PMlinkescapetext{fixed} $k$, there is also a $N' \in \nat$, with $N' \geq N$, so that whenever $n \geq N'$ and $0 \leq k \leq N$, then
$\abs{\pi(k, n) - 1} < \epsilon/(2e^{M})$.
(Note that $k$ is chosen only from a \emph{finite} set.)

Now, when $n \geq N'$, we have
\begin{align*}
\absW{ \sum_{k=0}^n \pi(k,n) \frac{z^k}{k!} \,-\, \sum_{k=0}^\infty \frac{z^k}{k!} } 
&= \absW{
\sum_{k=0}^n (\pi(k,n)-1) \frac{z^k}{k!} \,-\, \sum_{k=n+1}^\infty \frac{z^k}{k!} } \\
&\leq 
\sum_{k=0}^n \abs{\pi(k,n) -1} \, \frac{M^k}{k!} + \sum_{k=n+1}^\infty \frac{M^k}{k!} \\
&=
\sum_{k=0}^{N} \abs{\pi(k,n) -1} \, \frac{M^k}{k!} + \sum_{k=N+1}^n
\abs{\pi(k,n) -1} \, \frac{M^k}{k!} 
+ \sum_{k=n+1}^\infty \frac{M^k}{k!} \\
&< \frac{\epsilon}{2e^M} \sum_{k=0}^{N} \frac{M^k}{k!} + \sum_{k=N + 1}^n \frac{M^k}{k!} + \sum_{k=n+1}^\infty \frac{M^k}{k!} \\
\intertext{(In the middle sum, we use the bound $\abs{\pi(k,n) -1} = 1 - \pi(k,n) \leq 1$ for all $k$ and $n$.)} &< \frac{\epsilon}{2e^M} \cdot e^M + \frac{\epsilon}{2} = \epsilon\,. \qedhere
\end{align*}
\end{proof}

In fact, we have proved uniform convergence of $\lim_{n\to\infty} \left( 1 + \frac{z}{n} \right)^n$
over $\abs{z} \leq M$.
Exploiting this fact we can also show:
\begin{align*}
\left( 1 + \frac{z}{n} + o\left(\frac{1}{n}\right) \right)^n
= 
\left( 1 + \frac{z + o(1)}{n} \right)^n
 \to 
\sum_{k=0}^\infty \frac{z^k}{k!} \quad \textrm{(pointwise, as $n \to \infty$)}
\end{align*}

\begin{proof}
\PMlinkescapetext{Fix} $\abs{z} < M$.
Given $\epsilon > 0$, for large enough $n$, we have
\begin{align*}
\absW{ \left( 1 + \frac{w}{n} \right)^n - e^w } < \epsilon/2 \quad
\textrm{uniformly for all $\abs{w} \leq M$.}
\end{align*}
Since $o(1) \to 0$, for large enough $n$ we can set $w = z+o(1)$ above.
Since the exponential is continuous\footnote{follows from uniform convergence on bounded subsets of either expression for $e^z$}, for large enough $n$ we also have $\abs{e^{z+o(1)} - e^z} < \epsilon/2$.  Thus
\begin{align*}
\absW{\left( 1 + \frac{z + o(1)}{n} \right)^n - e^z}
\leq 
\absW{ \left( 1 + \frac{z + o(1)}{n} \right)^n
- e^{z+o(1)}} + \abs{e^{z+o(1)} - e^z} < \epsilon\,.
\qedhere
\end{align*}
\end{proof}
%%%%%
%%%%%
\end{document}
