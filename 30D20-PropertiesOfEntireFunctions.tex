\documentclass[12pt]{article}
\usepackage{pmmeta}
\pmcanonicalname{PropertiesOfEntireFunctions}
\pmcreated{2013-03-22 14:52:09}
\pmmodified{2013-03-22 14:52:09}
\pmowner{pahio}{2872}
\pmmodifier{pahio}{2872}
\pmtitle{properties of entire functions}
\pmrecord{19}{36545}
\pmprivacy{1}
\pmauthor{pahio}{2872}
\pmtype{Result}
\pmcomment{trigger rebuild}
\pmclassification{msc}{30D20}
\pmrelated{RationalFunction}
\pmrelated{AlgebraicFunction}
\pmrelated{BesselsEquation}
\pmdefines{entire rational function}
\pmdefines{entire transcendental function}

% this is the default PlanetMath preamble.  as your knowledge
% of TeX increases, you will probably want to edit this, but
% it should be fine as is for beginners.

% almost certainly you want these
\usepackage{amssymb}
\usepackage{amsmath}
\usepackage{amsfonts}

% used for TeXing text within eps files
%\usepackage{psfrag}
% need this for including graphics (\includegraphics)
%\usepackage{graphicx}
% for neatly defining theorems and propositions
%\usepackage{amsthm}
% making logically defined graphics
%%%\usepackage{xypic}

% there are many more packages, add them here as you need them

% define commands here
\begin{document}
\begin{enumerate}

 \item If \,$f\!:\mathbb{C}\to\mathbb{C}$\, is an entire function and \,$z_0\in\mathbb{C}$, then $f(z)$ has the Taylor series \PMlinkescapetext{expansion}
$$f(z) = a_0\!+\!a_1(z\!-\!z_0)\!+\!a_2(z\!-\!z_0)^2\!+\cdots$$
which is valid in the whole complex plane.

 \item If, conversely, such a power series converges for every complex value $z$, then the \PMlinkname{sum of the series}{SumFunctionOfSeries} is an entire function.

 \item The entire functions may be divided in two disjoint \PMlinkescapetext{classes}:

a) The {\em entire rational functions}, i.e. polynomial functions; in their series \PMlinkescapetext{expansion} there is an $n_0$ such that\, $a_n = 0\,\,\forall n\geqq n_0$.

b) The {\em entire transcendental functions}; in their series \PMlinkescapetext{expansion} one has \,$a_n \neq 0$\, for infinitely many values of $n$.\, Examples are complex sine and cosine, complex exponential function, sine integral,  error function.

 \item A consequence of Liouville's theorem: \,If $f$ is a non-constant entire function and if $R$ and $M$ are two arbitrarily great positive numbers, then there exist such points $z$ that
        $$|z| > R \,\,\,\mathrm{and}\,\,\, |f(z)| > M.$$
This \PMlinkescapetext{means} that the non-constant entire functions are \PMlinkname{unbounded}{BoundedFunction}.

 \item The \PMlinkname{sum}{SumOfFunctions}, the \PMlinkname{product}{ProductOfFunctions} and the composition of two entire functions are entire functions.

 \item The ring of all entire functions is a Pr\"ufer domain.

\end{enumerate}

\begin{thebibliography}{9}
\bibitem{OH}{\sc O. Helmer:} ``{Divisibility properties of integral functions}''.\, -- {\em Duke Math. J.} \textbf{6} (1940), 345--356.
\end{thebibliography}
%%%%%
%%%%%
\end{document}
