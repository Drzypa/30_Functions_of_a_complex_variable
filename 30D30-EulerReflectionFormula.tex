\documentclass[12pt]{article}
\usepackage{pmmeta}
\pmcanonicalname{EulerReflectionFormula}
\pmcreated{2013-03-22 16:23:37}
\pmmodified{2013-03-22 16:23:37}
\pmowner{rm50}{10146}
\pmmodifier{rm50}{10146}
\pmtitle{Euler reflection formula}
\pmrecord{5}{38539}
\pmprivacy{1}
\pmauthor{rm50}{10146}
\pmtype{Theorem}
\pmcomment{trigger rebuild}
\pmclassification{msc}{30D30}
\pmclassification{msc}{33B15}

\endmetadata

% this is the default PlanetMath preamble.  as your knowledge
% of TeX increases, you will probably want to edit this, but
% it should be fine as is for beginners.

% almost certainly you want these
\usepackage{amssymb}
\usepackage{amsmath}
\usepackage{amsfonts}

% used for TeXing text within eps files
%\usepackage{psfrag}
% need this for including graphics (\includegraphics)
%\usepackage{graphicx}
% for neatly defining theorems and propositions
%\usepackage{amsthm}
% making logically defined graphics
%%%\usepackage{xypic}

% there are many more packages, add them here as you need them

% define commands here
\newtheorem{thm}{Theorem}
\newtheorem{defn}{Definition}
\newtheorem{prop}{Proposition}
\newtheorem{lemma}{Lemma}
\newtheorem{cor}{Corollary} 


\begin{document}
\begin{thm} (Euler Reflection Formula)
\[\Gamma(x)\Gamma(1-x)=\frac{\pi}{\sin(\pi x)}\]
\end{thm}

\textbf{Proof:}
We have
\[\frac{1}{\Gamma(x)}=xe^{\gamma x}\prod_{n=1}^{\infty} \left(\left(1+\frac{x}{n}\right)e^{-x/n}\right)\]
and thus
\[\frac{1}{\Gamma(x)}\frac{1}{\Gamma(-x)}=-x^2e^{\gamma x}e^{-\gamma x}\prod_{n=1}^{\infty}
\left(\left(1+\frac{x}{n}\right)e^{-x/n}\right)\left(\left(1-\frac{x}{n}\right)e^{x/n}\right)=-x^2\prod_{n=1}^{\infty}\left(1-\frac{x^2}{n^2}\right)\]
But $\Gamma(1-x)=-x\Gamma(-x)$ and thus
\[\frac{1}{\Gamma(x)}\frac{1}{\Gamma(1-x)}=x\prod_{n=1}^{\infty}\left(1-\frac{x^2}{n^2}\right)\]
Now, using the \PMlinkname{formula}{ExamplesOfInfiniteProducts} for $\sin x/x$, we have
\[\sin(\pi x)=\pi x\prod_{n=1}^{\infty}\left(1-\frac{x^2}{n^2}\right)\]
so that
\[\frac{1}{\Gamma(x)}\frac{1}{\Gamma(1-x)}=\frac{\sin(\pi x)}{\pi}\]
and the result follows.

%%%%%
%%%%%
\end{document}
