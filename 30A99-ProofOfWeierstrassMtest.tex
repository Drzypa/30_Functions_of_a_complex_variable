\documentclass[12pt]{article}
\usepackage{pmmeta}
\pmcanonicalname{ProofOfWeierstrassMtest}
\pmcreated{2013-03-22 12:58:01}
\pmmodified{2013-03-22 12:58:01}
\pmowner{CWoo}{3771}
\pmmodifier{CWoo}{3771}
\pmtitle{proof of Weierstrass M-test}
\pmrecord{5}{33336}
\pmprivacy{1}
\pmauthor{CWoo}{3771}
\pmtype{Proof}
\pmcomment{trigger rebuild}
\pmclassification{msc}{30A99}
\pmrelated{CauchySequence}

\endmetadata

% this is the default PlanetMath preamble.  as your knowledge
% of TeX increases, you will probably want to edit this, but
% it should be fine as is for beginners.

% almost certainly you want these
\usepackage{amssymb}
\usepackage{amsmath}
\usepackage{amsfonts}

% used for TeXing text within eps files
%\usepackage{psfrag}
% need this for including graphics (\includegraphics)
%\usepackage{graphicx}
% for neatly defining theorems and propositions
%\usepackage{amsthm}
% making logically defined graphics
%%%\usepackage{xypic}

% there are many more packages, add them here as you need them

% define commands here
\def\N{\mathbb{N}}
\def\eps{\epsilon}
\begin{document}
Consider the sequence of partial sums $s_n=\sum_{m=1}^n f_m$.
Take any $p,q\in\N$ such that $p\le q$,then, for every $x\in X$, we have
\begin{eqnarray*}
	|s_q(x)-s_p(x)| &=& \left|\sum_{m=p+1}^q f_m(x)\right| \\
		&\le& \sum_{m=p+1}^q |f_m(x)| \\
		&\le& \sum_{m=p+1}^q M_m
\end{eqnarray*}
But since $\sum_{n=1}^\infty M_n$ converges, for any $\eps>0$ we can
find an $N\in\N$ such that, for any $p,q>N$ and $x\in X$, we have
$|s_q(x)-s_p(x)|\le\sum_{m=p+1}^q M_m<\eps$. Hence the sequence
$s_n$ converges uniformly to $\sum_{n=1}^\infty f_n$.
%%%%%
%%%%%
\end{document}
