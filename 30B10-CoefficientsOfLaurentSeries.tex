\documentclass[12pt]{article}
\usepackage{pmmeta}
\pmcanonicalname{CoefficientsOfLaurentSeries}
\pmcreated{2013-03-22 15:19:22}
\pmmodified{2013-03-22 15:19:22}
\pmowner{pahio}{2872}
\pmmodifier{pahio}{2872}
\pmtitle{coefficients of Laurent series}
\pmrecord{15}{37130}
\pmprivacy{1}
\pmauthor{pahio}{2872}
\pmtype{Result}
\pmcomment{trigger rebuild}
\pmclassification{msc}{30B10}
\pmrelated{LaurentSeries}
\pmrelated{TechniqueForComputingResidues}
\pmrelated{UniquenessOfLaurentExpansion}

\endmetadata

% this is the default PlanetMath preamble.  as your knowledge
% of TeX increases, you will probably want to edit this, but
% it should be fine as is for beginners.

% almost certainly you want these
\usepackage{amssymb}
\usepackage{amsmath}
\usepackage{amsfonts}

% used for TeXing text within eps files
%\usepackage{psfrag}
% need this for including graphics (\includegraphics)
%\usepackage{graphicx}
% for neatly defining theorems and propositions
 \usepackage{amsthm}
% making logically defined graphics
%%%\usepackage{xypic}

% there are many more packages, add them here as you need them

% define commands here

\theoremstyle{definition}
\newtheorem*{thmplain}{Theorem}
\begin{document}
Suppose that $f$ is analytic in the annulus \,
$\{z\in\mathbb{C}\,\vdots\,\, R_1 < |z-a| < R_2 \}$,\, where $R_1$ may be 0 and $R_2$ may be $\infty$.\, Then the coefficients of the \PMlinkname{Laurent series}{LaurentSeries} \PMlinkescapetext{expansion}
           $$\sum_{n = -\infty}^\infty c_n (z-a)^n$$
of $f$ can be obtained from
\begin{align}
   c_n \;=\; \frac{1}{2\pi i}\oint_{\gamma} \frac{f(t)}{(t-a)^{n+1}}\,dt
           \quad (n = 0,\,\pm 1,\,\pm 2,\,\ldots),
\end{align}
where the \PMlinkname{path}{ContourIntegral} $\gamma$ goes anticlockwise once around the point \,$z = a$\, within the annulus.\, Especially, the residue of $f$ in the point $a$ is
\begin{align}
       c_{-1} \;=\; \frac{1}{2\pi i}\oint_{\gamma} f(t)\,dt.
\end{align}

\textbf{Remark.}\, Usually, the Laurent series of a function, i.e. the coefficients $c_n$, are not determined by using the integral formula (1), but directly from known series \PMlinkescapetext{expansions}.\, Often it is sufficient to know the value of $c_{-1}$ or the residue, which is used to compute integrals (see the Cauchy residue theorem ---\, cf. (2)).\, There is also the usable 

\textbf{Rule.}\, In the case that the limit \,
$\displaystyle\lim_{z\to a}(z-a)f(z)$\, exists and has a non-zero value $r$, the point\, $z = a$\, is a pole of the \PMlinkescapetext{order} 1 for the function $f$ and 
$$\operatorname{Res}(f;\,a) \;=\; r.$$

\textbf{Examples}
\begin{enumerate}
\item Let\, $f(z) := \frac{1}{\sin{z}}$,\, and\, $a = 0$.\, Using the Taylor series of the complex sine we obtain
    $$\lim_{z\to 0}z\frac{1}{\sin{z}} \;=\; 
      \lim_{z\to 0}\frac{1}{1-\frac{z^2}{3!}+-\ldots} \;=\; 1,$$
whence\, $\operatorname{Res}(\frac{1}{\sin{z}};\,0) = 1$.\, Thus we can write
$$\oint_{\gamma}\frac{dz}{\sin{z}} \;=\; 2\pi i,$$
where the \PMlinkescapetext{path} must be chosen such that it encloses only the pole $0$ of 
$\frac{1}{\sin{z}}$.
\item The Taylor series of the complex exponential function gives the Laurent series
   $$e^{\frac{1}{z}} \;\equiv\; 1+\frac{1}{z}+\frac{1}{2!z^2}+\frac{1}{3!z^3}+\ldots$$
which shows that\, $\operatorname{Res}(e^{\frac{1}{z}};\,0) = 1.$
\end{enumerate}
%%%%%
%%%%%
\end{document}
