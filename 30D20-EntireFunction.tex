\documentclass[12pt]{article}
\usepackage{pmmeta}
\pmcanonicalname{EntireFunction}
\pmcreated{2013-03-22 12:04:39}
\pmmodified{2013-03-22 12:04:39}
\pmowner{djao}{24}
\pmmodifier{djao}{24}
\pmtitle{entire function}
\pmrecord{10}{31148}
\pmprivacy{1}
\pmauthor{djao}{24}
\pmtype{Definition}
\pmcomment{trigger rebuild}
\pmclassification{msc}{30D20}
\pmsynonym{entire}{EntireFunction}

\endmetadata

\usepackage{amssymb}
\usepackage{amsmath}
\usepackage{amsfonts}
\usepackage{graphicx}
%%%\usepackage{xypic}
\begin{document}
An \emph{entire function} is a function $f: \mathbb{C} \longrightarrow \mathbb{C}$ which is holomorphic everywhere on the complex domain $\mathbb{C}$.

For example, a polynomial is holomorphic everywhere, as is the exponential function.  The function $z\mapsto 1/z$ is not holomorphic at zero, so it is not entire; it is meromorphic.
%%%%%
%%%%%
%%%%%
\end{document}
