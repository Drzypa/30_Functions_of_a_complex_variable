\documentclass[12pt]{article}
\usepackage{pmmeta}
\pmcanonicalname{ProofOfIdentityTheoremOfPowerSeries}
\pmcreated{2013-03-22 16:47:38}
\pmmodified{2013-03-22 16:47:38}
\pmowner{rspuzio}{6075}
\pmmodifier{rspuzio}{6075}
\pmtitle{proof of identity theorem of power series}
\pmrecord{7}{39027}
\pmprivacy{1}
\pmauthor{rspuzio}{6075}
\pmtype{Proof}
\pmcomment{trigger rebuild}
\pmclassification{msc}{30B10}
\pmclassification{msc}{40A30}

% this is the default PlanetMath preamble.  as your knowledge
% of TeX increases, you will probably want to edit this, but
% it should be fine as is for beginners.

% almost certainly you want these
\usepackage{amssymb}
\usepackage{amsmath}
\usepackage{amsfonts}

% used for TeXing text within eps files
%\usepackage{psfrag}
% need this for including graphics (\includegraphics)
%\usepackage{graphicx}
% for neatly defining theorems and propositions
%\usepackage{amsthm}
% making logically defined graphics
%%%\usepackage{xypic}

% there are many more packages, add them here as you need them

% define commands here

\begin{document}
We start by proving a more modest result.  Namely, we show that,
under the hypotheses of the theorem we are trying to prove, we
can conclude that $a_0 = b_0$. 

Let $R$ be chosen such that both series converge when $|z - z_0| < R$.
From the set of points at which the two power series are equal, we may
choose a sequence $\{ w_k \}_{k=0}^\infty$ such that
\begin{itemize}
\item $|w_k - z_0| < R/2$ for all $k$.
\item $\lim_{k \to \infty} w_k$ exists and equals $z_0$.
\item $w_k \neq z_0$ for all $k$.
\end{itemize}.

Since power series converge uniformly, we may interchange the
limit with the summation.
\begin{eqnarray*}
\lim_{k \to \infty} \sum_{n=0}^\infty
a_n (w_k - z_0)^n &=& 
\sum_{n=0}^\infty \lim_{k \to \infty}
a_n (w_k - z_0)^n = a_0 \\
\lim_{k \to \infty} \sum_{n=0}^\infty
b_n (w_k - z_0)^n &=& 
\sum_{n=0}^\infty \lim_{k \to \infty}
b_n (w_k - z_0)^n = b_0
\end{eqnarray*}
Because $\sum_{n=0}^\infty a_n (w_k - z_0)^n =
sum_{n=0}^\infty a_n (w_k - z_0)^n$ for all $k$,
this means that $a_0 = b_0$.

We will now prove that $a_n = b_n$ for all $n$ by
an induction argument.  The intial step with $n = 0$
is, of course, the result demonstrated above. 
Assume that $a_m = b_m$ for all $m$ less than 
some integer $N$.  Then we have
\[
\sum_{n=N}^\infty a_n (w - z_0)^n =
\sum_{n=N}^\infty b_n (w - z_0)^n
\]
for all $w \in S$.  Pulling out a common
factor and relabelling the index, we have
\[
(w - z_0)^N \sum_{n=0}^\infty a_{n+N} (w - z_0)^n =
(w - z_0)^N \sum_{n=0}^\infty b_{n+N} (w - z_0)^n.
\]
Because $z_0 \notin S$, the factor $w - z_0$ will
not equal zero, so we may cancel it:
\[
\sum_{n=0}^\infty a_{n+N} (w - z_0)^n =
\sum_{n=0}^\infty b_{n+N} (w - z_0)^n
\]
By our weaker result, we have $a_N = b_N$.
Hence, by induction, we have $a_n = b_n$ for all $n$.
%%%%%
%%%%%
\end{document}
