\documentclass[12pt]{article}
\usepackage{pmmeta}
\pmcanonicalname{SineIntegral}
\pmcreated{2015-02-04 12:58:26}
\pmmodified{2015-02-04 12:58:26}
\pmowner{pahio}{2872}
\pmmodifier{pahio}{2872}
\pmtitle{sine integral}
\pmrecord{17}{36844}
\pmprivacy{1}
\pmauthor{pahio}{2872}
\pmtype{Definition}
\pmcomment{trigger rebuild}
\pmclassification{msc}{30A99}
\pmsynonym{sinus integralis}{SineIntegral}
\pmsynonym{Si}{SineIntegral}
\pmrelated{SincFunction}
\pmrelated{SineIntegralInInfinity}
\pmrelated{LogarithmicIntegral2}
\pmrelated{CurvatureOfNielsensSpiral}
\pmrelated{LaplaceTransformOfIntegralSine}
\pmrelated{FresnelIntegrals}
\pmrelated{HyperbolicSineIntegral}
\pmdefines{sine integral}
\pmdefines{sinus integralis}
\pmdefines{cosine integral}

\endmetadata

% this is the default PlanetMath preamble.  as your knowledge
% of TeX increases, you will probably want to edit this, but
% it should be fine as is for beginners.

% almost certainly you want these
\usepackage{amssymb}
\usepackage{amsmath}
\usepackage{amsfonts}

% used for TeXing text within eps files
%\usepackage{psfrag}
% need this for including graphics (\includegraphics)
\usepackage{graphicx}
% for neatly defining theorems and propositions
%\usepackage{amsthm}
% making logically defined graphics
%%%\usepackage{xypic}

% there are many more packages, add them here as you need them

% define commands here
\DeclareMathOperator{\Si}{Si}
\DeclareMathOperator{\si}{si}
\DeclareMathOperator{\ci}{ci}
\DeclareMathOperator{\sinc}{sinc}
\begin{document}
The function {\em sine integral} (in Latin {\em sinus integralis}) from $\mathbb{R}$ to $\mathbb{R}$ is defined as
\begin{align}   
\mbox{Si }{x} \;:=\; \int_0^x\frac{\sin t}{t}\,dt = 
\int_0^x\mbox{sinc}(t)\;dt,
\end{align}
or alternatively as
$$\mbox{Si }{x} \,:=\,  \int_0^1\frac{\sin{tx}}{t}\,dt.$$

It isn't an elementary function.\, The equation (1) implies the Taylor series \PMlinkescapetext{expansion}
   $$\mbox{Si }{z} = z\!-\!\frac{z^3}{3\!\cdot\!3!}\!+\!\frac{z^5}{5\!\cdot\!5!}
                \!-\!\frac{z^7}{7\!\cdot\!7!}\!+-\ldots,$$
which converges for all complex values $z$ and thus defines an entire transcendental function.\\

$\mbox{Si }{x}$ satisfies the linear third \PMlinkescapetext{order} differential equation
          $$xf'''(x)\!+\!2f''(x)\!+\!xf'(x) =  0.$$

\begin{center}
\includegraphics[scale=0.4]{sinint}
\end{center}

\textbf{Remark 1.} \quad$\lim_{x\to\infty}\mbox{Si }{x} = \frac{\pi}{2}$\\

\textbf{Remark 2.}\, There is also another ``sine integral''
$$\mbox{si }{x}\; :=\; \int_\infty^x\frac{\sin t}{t}\,dt\; =\; \mbox{Si }{x}-\frac{\pi}{2}$$
and the corresponding {\em cosine integral}
$$\mbox{ci }{x} \;:=\; \int_\infty^x\frac{\cos t}{t}\,dt = \gamma\!+\ln{x}+\!\int_0^x\frac{\cos{t}\!-\!1}{t}\,dt$$
where $\gamma$ is the \PMlinkname{Euler--Mascheroni constant}{EulerMascheroniConstant}.

%%%%%
%%%%%
\end{document}
