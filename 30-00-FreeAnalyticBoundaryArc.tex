\documentclass[12pt]{article}
\usepackage{pmmeta}
\pmcanonicalname{FreeAnalyticBoundaryArc}
\pmcreated{2013-03-22 14:18:00}
\pmmodified{2013-03-22 14:18:00}
\pmowner{jirka}{4157}
\pmmodifier{jirka}{4157}
\pmtitle{free analytic boundary arc}
\pmrecord{7}{35758}
\pmprivacy{1}
\pmauthor{jirka}{4157}
\pmtype{Definition}
\pmcomment{trigger rebuild}
\pmclassification{msc}{30-00}
\pmclassification{msc}{54-00}
\pmrelated{AnalyticCurve}

% this is the default PlanetMath preamble.  as your knowledge
% of TeX increases, you will probably want to edit this, but
% it should be fine as is for beginners.

% almost certainly you want these
\usepackage{amssymb}
\usepackage{amsmath}
\usepackage{amsfonts}

% used for TeXing text within eps files
%\usepackage{psfrag}
% need this for including graphics (\includegraphics)
%\usepackage{graphicx}
% for neatly defining theorems and propositions
\usepackage{amsthm}
% making logically defined graphics
%%%\usepackage{xypic}

% there are many more packages, add them here as you need them

% define commands here
\theoremstyle{theorem}
\newtheorem*{thm}{Theorem}
\newtheorem*{lemma}{Lemma}
\newtheorem*{conj}{Conjecture}
\newtheorem*{cor}{Corollary}
\newtheorem*{example}{Example}
\theoremstyle{definition}
\newtheorem*{defn}{Definition}
\begin{document}
\begin{defn}
Let $G \subset \mathbb{C}$ be a region and let $\gamma$ be a connected subset of $\partial G$ (boundary of $G$), then $\gamma$ is a {\em free analytic boundary arc} of
$G$ if for every point $\zeta \in \gamma$ there is a neighbourhood $U$ of
$\zeta$ and
a conformal equivalence $h \colon {\mathbb{D}} \to U$ (where ${\mathbb{D}}$ is the unit disc) such that $h(0) = \zeta$, $h(-1,1) = \gamma \cap U$ and
$h({\mathbb{D}}_+) = G \cap U$ (where ${\mathbb{D}}_+$ is all the points
in the unit disc with non-negative imaginary part).
\end{defn}

\begin{thebibliography}{9}
\bibitem{Conway:complexII}
John~B. Conway.
{\em \PMlinkescapetext{Functions of One Complex Variable II}}.
Springer-Verlag, New York, New York, 1995.
\end{thebibliography}
%%%%%
%%%%%
\end{document}
