\documentclass[12pt]{article}
\usepackage{pmmeta}
\pmcanonicalname{ExamplesOfCauchyRiemannEquations}
\pmcreated{2013-03-22 17:38:05}
\pmmodified{2013-03-22 17:38:05}
\pmowner{rspuzio}{6075}
\pmmodifier{rspuzio}{6075}
\pmtitle{examples of Cauchy-Riemann equations}
\pmrecord{19}{40056}
\pmprivacy{1}
\pmauthor{rspuzio}{6075}
\pmtype{Example}
\pmcomment{trigger rebuild}
\pmclassification{msc}{30A99}

\endmetadata

% this is the default PlanetMath preamble.  as your knowledge
% of TeX increases, you will probably want to edit this, but
% it should be fine as is for beginners.

% almost certainly you want these
\usepackage{amssymb}
\usepackage{amsmath}
\usepackage{amsfonts}

% used for TeXing text within eps files
%\usepackage{psfrag}
% need this for including graphics (\includegraphics)
%\usepackage{graphicx}
% for neatly defining theorems and propositions
\usepackage{amsthm}
% making logically defined graphics
%%%\usepackage{xypic}

% there are many more packages, add them here as you need them

% define commands here

\begin{document}
To illustrate the Cauchy-Riemann equations, we may consider a few
examples.  Let $f$ be the squaring function, i.e. for any complex
number $z$, we have $f(z) = z^2$.

We now separate real from imaginary parts.  Letting $x$ and $y$ be
real variables we have
\[
 f(x + iy) = (x + iy)^2 
           = x^2 + 2ixy - y^2
           = (x^2 - y^2) + i (2xy) .
\]
Defining real functions $u$ and $v$ by $f(x+iy) = u(x,y) + i v (x,y)$ 
and taking derivatives, we have
\begin{align*}
u(x,y) & = x^2 - y^2  \\
v(x,y) &= 2xy \\
{\partial u \over \partial x} &= 2x \\
{\partial u \over \partial y} &= -2y \\
{\partial v \over \partial x} &= 2y \\
{\partial v \over \partial y} &= 2x
\end{align*}

Since $\partial u / \partial x = \partial v / \partial y$ and
$\partial u / \partial y = - \partial v / \partial x$. the
Cauchy-Riemann relations are seen to be satisfied.

Next, consider the complex conjugation function.  In this
case, $\overline{x + i y} = x - iy$, so we have $u(x,y) = x$ 
and $v(x,y) = y$.  Taking derivatives,
\begin{align*}
{\partial u \over \partial x} &= 1 \\
{\partial u \over \partial y} &= 0 \\
{\partial v \over \partial x} &= 0 \\
{\partial v \over \partial y} &= -1
\end{align*}
Because $\partial u / \partial x \neq \partial v / \partial y$,
the Cauchy-Riemann equations are \emph{not} satisfied do the
conjugation function is not holomorphic.  Likewise, one can show
that the functions $\Re, \Im, and | \cdot |$ which appear in
complex analysis are not holomorphic.

For our next example, we try a polynomial.  Let $f(Z) =
z^3 + 2 z + 5$.  Writing $z = x+iy$ and $f = u + iv$, 
we find that $u(x,y) = x^3 - 3 x y^2 + 2x + 5$ and
$v(x,y) = 3 x^2 y - y^3 + 2y$.  Taking partial derivatives,
one can confirm that the Cauchy-Riemann equations
are satisfied, so we have a holomorphic function.

More generally, we can show that \emph{all} complex
polynomials are holomorphic.  Since the Cauchy-Riemann
equations are linear, it suffices to check that integer 
powers are holomorphic.  We can do this by an induction
argument.  That $f(z) = z$ satisfies the equations is
trivial and we have shown that $f(z) = z^2$ also
satisfies them.  Let us assume that $f(z) = z^n$ 
happens to satisfy the Cauchy-Riemann equations for a 
particular value of $n$ and write
\begin{align*}
 (x + iy)^n &= u(x,y) + iv(x,y) \\
 (x + iy)^{n+1} &= {\tilde u} (x,y) + i {\tilde v} (x,y)
\end{align*}
By elementary algebra, we have
\begin{align*}
 {\tilde u} (x,y) &= x u(x,y) - y v(x,y) \\
 {\tilde v} (x,y) &= y u(x,y) + x v(x,y) .
\end{align*}
By elementary calculus, we have
\begin{align*}
 {\partial \tilde u \over \partial x} &=
   u + x {\partial u \over \partial x} - 
       y {\partial v \over \partial x} \\
 {\partial \tilde u \over \partial y} &=
 - v + x {\partial u \over \partial y} - 
       y {\partial v \over \partial y} \\
 {\partial \tilde v \over \partial x} &=
   v + y {\partial u \over \partial x} + 
       x {\partial v \over \partial x} \\
 {\partial \tilde v \over \partial y} &=
   u + y {\partial u \over \partial y} + 
       x {\partial v \over \partial y} 
\end{align*}
so
\begin{align*}
 {\partial \tilde u \over \partial x} - 
   {\partial \tilde v \over \partial y} &=
 x \left( {\partial u \over \partial x} - 
     {\partial v \over \partial y} \right) + 
 y \left( {\partial u \over \partial y} + 
     {\partial v \over \partial x} \right) \\
 {\partial \tilde u \over \partial y} + 
   {\partial \tilde v \over \partial x} &=
 x \left( {\partial u \over \partial y} + 
     {\partial v \over \partial x} \right) +
 y \left( {\partial u \over \partial x} - 
     {\partial v \over \partial y} \right)  
\end{align*}
Since the terms in parentheses are zero on account of
$u$ and $v$ satisfying the Cauchy-Riemann equations,
it follows that ${\tilde u}$ and ${\tilde v}$ also
satisfy the Cauchy-Riemann equations.  By induction,
$f(z) = z^n$ is holomorphic for all positive integers $n$.

As our next example, we consider the complex
square root.  As shown in the entry taking square 
root algebraically, we have the following equality:
\[
 \sqrt{x+iy} = \sqrt{\frac{\sqrt{x^2 + y^2} + x}{2}} +
  (\mbox{sign} \, {y}) i \sqrt{\frac{\sqrt{x^2 + y^2}- x}{2}}
\]
Differentiating and simplifying,
\begin{align*}
 u(x,y) &= \sqrt{\frac{\sqrt{x^2 + y^2} + x}{2}} \\
 v(x,y) &= (\mbox{sign} \, {y}) \sqrt{\frac{\sqrt{x^2 + y^2}- x}{2}} \\
 {\partial u \over \partial x} &=
  \frac{\sqrt{2}}{4} 
    \frac{\frac{x}
               {\sqrt {x^2 + y^2}} + 1}
         {\sqrt{ \sqrt {x^2 + y^2} + x}} = 
   \frac{\sqrt{2}}{4} 
   \frac{\sqrt{\sqrt {x^2 + y^2} + x}}
       {\sqrt{x^2 + y^2}} \\
 {\partial u \over \partial y} &=
  (\mbox{sign} \, {y}) \frac{\sqrt{2}}{4} 
    \frac{\frac{y}
               {\sqrt {x^2 + y^2}}}
         {\sqrt{ \sqrt {x^2 + y^2} + x}} =
  (\mbox{sign} \, {y}) \frac{\sqrt{2}}{4} 
  \frac{y}
       {{\sqrt{x^2 + y^2}}
        {\sqrt{ \sqrt {x^2 + y^2} + x}}} \\
 {\partial v \over \partial x} &=
  (\mbox{sign} \, {y}) \frac{\sqrt{2}}{4} 
    \frac{\frac{x}
               {\sqrt {x^2 + y^2}} - 1}
               {\sqrt{ \sqrt {x^2 + y^2} - x}} = 
  -(\mbox{sign} \, {y}) \frac{\sqrt{2}}{4} 
    \frac{\sqrt{ \sqrt {x^2 + y^2} - x}}
         {\sqrt{x^2 + y^2}} \\
 {\partial v \over \partial y} &=
  \frac{\sqrt{2}}{4} 
    \frac{\frac{y}
               {\sqrt {x^2 + y^2}}}
         {\sqrt{ \sqrt {x^2 + y^2} - x}} =
  \frac{\sqrt{2}}{4}
  \frac{y}
       {\sqrt{x^2 + y^2}
        \sqrt{ \sqrt {x^2 + y^2} - x}} .
\end{align*}
Pulling out a  common factor and placing over a common denominator,
\begin{align*}
 {\partial u \over \partial x} -
        {\partial v \over \partial y} &=
 \frac{\sqrt{2}}{4 \sqrt{x^2 + y^2}} 
  \left(
    \sqrt{ \sqrt {x^2 + y^2} + x} -
  \frac{y} {\sqrt{ \sqrt {x^2 + y^2} - x}}
  \right) \\ &=
 \frac{\sqrt{2}}{4 \sqrt{x^2 + y^2}} 
 \frac{\sqrt{ \sqrt {x^2 + y^2} + x} \cdot
       \sqrt{ \sqrt {x^2 + y^2} - x} - y}
      {\sqrt{ \sqrt {x^2 + y^2} - x}} \\ &=
 \frac{\sqrt{2}}{4 \sqrt{x^2 + y^2}} 
 \frac{\sqrt{(x^2 + y^2) - y^2} - y}
      {\sqrt{ \sqrt {x^2 + y^2} - x}} = 0 \\
 {\partial u \over \partial y} +
        {\partial v \over \partial x} &=
 (\mbox{sign} \, {y})
 \frac{\sqrt{2}}{4 \sqrt{x^2 + y^2}} 
 \left(
   \frac{y}{\sqrt{ \sqrt {x^2 + y^2} + x}} -
   \sqrt{ \sqrt {x^2 + y^2} - x}
 \right) \\ &=
 (\mbox{sign} \, {y})
 \frac{\sqrt{2}}{4 \sqrt{x^2 + y^2}} 
 \frac{y - \sqrt{ \sqrt {x^2 + y^2} + x} \cdot
           \sqrt{ \sqrt {x^2 + y^2} - x}}
      {\sqrt{ \sqrt {x^2 + y^2} + x}} \\ &=
 (\mbox{sign} \, {y})
 \frac{\sqrt{2}}{4 \sqrt{x^2 + y^2}} 
 \frac{\sqrt{(x^2 + y^2) - x^2} - y}
      {\sqrt{ \sqrt {x^2 + y^2} + x}} = 0 ,
\end{align*}
so the Cauchy-Riemann equations are satisfied.  More
generally, it can be shown that all complex algebraic
functions and fractional powers satisfy the
Cauchy-Riemann equations.  However, as suggested by
the above derivation, a direct verification could
be tedious, so it is better to use an indirect approach.

Finally, we finish up with two examples of transcendental
functions, the complex exponential and the complex logarithm,
The complex exponential is defined as
$\exp (x + iy) = \exp (x) (\cos y + i \sin y)$.  Hence
we have
\begin{align*}
 u (x,y) &= \exp (x) \cos (y) \\
 v (x,y) &= \exp (x) \sin (y) \\
 {\partial u \over \partial x} &= \exp (x) \cos (y) \\
 {\partial u \over \partial y} &= -\exp (x) \sin (y) \\
 {\partial v \over \partial x} &= \exp (x) \sin (y) \\
 {\partial v \over \partial y} &= \exp (x) \cos (y) 
\end{align*}
Thus we see that the complex exponential function is holomorphic.

The complex logarithm may be defined as $\log (x + iy) =
1/2 \log (x^2 + y^2) + i \arctan (y/x)$.  Hence we have
\begin{align*}
 u(x,y) &= 1/2 \log (x^2 + y^2) \\
 v(x,y) &= \arctan (y/x) \\
 {\partial u \over \partial x} &= {x \over x^2 + y^2} \\
 {\partial u \over \partial y} &= {y \over x^2 + y^2} \\
 {\partial v \over \partial x} &= {-y/x^2 \over 1 + (y/x)^2}
                                = - {y \over x^2 + y^2} \\
 {\partial v \over \partial y} &= {1/x \over 1 + (y/x)^2} 
                                = {x \over x^2 + y^2} 
\end{align*}
Hence the complex logarithm is holomorphic.
%%%%%
%%%%%
\end{document}
