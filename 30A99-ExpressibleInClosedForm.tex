\documentclass[12pt]{article}
\usepackage{pmmeta}
\pmcanonicalname{ExpressibleInClosedForm}
\pmcreated{2013-03-22 18:29:09}
\pmmodified{2013-03-22 18:29:09}
\pmowner{pahio}{2872}
\pmmodifier{pahio}{2872}
\pmtitle{expressible in closed form}
\pmrecord{8}{41162}
\pmprivacy{1}
\pmauthor{pahio}{2872}
\pmtype{Definition}
\pmcomment{trigger rebuild}
\pmclassification{msc}{30A99}
\pmclassification{msc}{26E99}
\pmrelated{ClosedForm}
\pmrelated{IrreducibilityOfBinomialsWithUnityCoefficients}
\pmrelated{ReductionOfEllipticIntegralsToStandardForm}
\pmdefines{closed form}

% this is the default PlanetMath preamble.  as your knowledge
% of TeX increases, you will probably want to edit this, but
% it should be fine as is for beginners.

% almost certainly you want these
\usepackage{amssymb}
\usepackage{amsmath}
\usepackage{amsfonts}

% used for TeXing text within eps files
%\usepackage{psfrag}
% need this for including graphics (\includegraphics)
%\usepackage{graphicx}
% for neatly defining theorems and propositions
 \usepackage{amsthm}
% making logically defined graphics
%%%\usepackage{xypic}

% there are many more packages, add them here as you need them

% define commands here

\theoremstyle{definition}
\newtheorem*{thmplain}{Theorem}

\begin{document}
An expression is \PMlinkescapetext{{\em expressible in a closed form}}, if it can be converted (simplified) into an expression containing only elementary functions, combined by a finite amount of rational operations and compositions.
Thus, such a closed form must not \PMlinkescapetext{contain} e.g. limit signs, integral signs, sum signs and ``\ldots''.

For example, 
$$\int\!\!\frac{dx}{x^4\!+\!1},$$
may be expressed in the closed form
$$\frac{1}{4\sqrt{2}}\ln\frac{x^2\!+\!x\sqrt{2}\!+\!1}{x^2\!-\!x\sqrt{2}\!+\!1}+
\frac{1}{2\sqrt{2}}\arctan\frac{x\sqrt{2}}{1\!-\!x^2}+C$$
but for
$$\int\!\!\frac{dx}{\sqrt{x^4\!+\!1}}\,dx,$$
there exists no closed form.\\

In certain contexts, the \PMlinkescapetext{scope} of the ``elementary functions'' may be enlarged by allowing in it some other functions, e.g. the error function.

%%%%%
%%%%%
\end{document}
