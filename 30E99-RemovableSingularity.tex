\documentclass[12pt]{article}
\usepackage{pmmeta}
\pmcanonicalname{RemovableSingularity}
\pmcreated{2013-03-22 12:56:01}
\pmmodified{2013-03-22 12:56:01}
\pmowner{rmilson}{146}
\pmmodifier{rmilson}{146}
\pmtitle{removable singularity}
\pmrecord{5}{33289}
\pmprivacy{1}
\pmauthor{rmilson}{146}
\pmtype{Definition}
\pmcomment{trigger rebuild}
\pmclassification{msc}{30E99}
\pmrelated{EssentialSingularity}

\endmetadata

\usepackage{amsmath}
\usepackage{amsfonts}
\usepackage{amssymb}
\newcommand{\reals}{\mathbb{R}}
\newcommand{\natnums}{\mathbb{N}}
\newcommand{\cnums}{\mathbb{C}}
\newcommand{\znums}{\mathbb{Z}}
\newcommand{\lp}{\left(}
\newcommand{\rp}{\right)}
\newcommand{\lb}{\left[}
\newcommand{\rb}{\right]}
\newcommand{\supth}{^{\text{th}}}
\newtheorem{proposition}{Proposition}
\newtheorem{definition}[proposition]{Definition}

\newtheorem{theorem}[proposition]{Theorem}

\newcommand{\nl}[1]{{\PMlinkescapetext{{#1}}}}
\newcommand{\pln}[2]{{\PMlinkname{{#1}}{#2}}}
\begin{document}
Let $U\subset\cnums$ be an open neighbourhood of a
point $a\in \cnums$.  We say that a function
$f:U\backslash\{a\}\rightarrow \cnums$ has a \emph{removable singularity} at
$a$, if the complex derivative $f'(z)$ exists for all $z\neq a$, and
if $f(z)$ is  bounded near $a$.

Removable singularities can, as the name suggests, be removed.
\begin{theorem}
  Suppose that $f:U\backslash\{a\}\rightarrow \cnums$ has a removable
  singularity at $a$.  Then, $f(z)$ can be holomorphically extended to
  all of $U$, i.e.
  there exists a holomorphic $g:U\rightarrow\cnums$ such that
  $g(z)=f(z)$ for all $z\neq a$.
\end{theorem}

\emph{Proof.}
Let $C$ be a circle centered at $a$, oriented counterclockwise, and
sufficiently small so that $C$ and its interior are contained in
$U$. For  $z$ in the interior of $C$, set
$$g(z) = \frac{1}{2\pi i} \oint_C \frac{f(\zeta)}{\zeta-z}d\zeta.$$
Since $C$ is a compact set, the defining limit for the derivative
$$\frac{d}{dz} \frac{f(\zeta)}{\zeta-z}= 
\frac{f(\zeta)}{(\zeta-z)^2}$$
converges uniformly for $\zeta\in C$.  Thanks to the uniform
convergence, the order of the derivative and the integral operations
can be interchanged.  Hence, we may deduce that $g'(z)$ exists 
for all $z$ in the interior of $C$.  Furthermore, by the Cauchy
integral formula we have that $f(z)=g(z)$ for all $z\neq a$, and therefore
$g(z)$ furnishes us with the desired extension.
%%%%%
%%%%%
\end{document}
