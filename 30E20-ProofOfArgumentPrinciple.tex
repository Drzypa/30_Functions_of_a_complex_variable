\documentclass[12pt]{article}
\usepackage{pmmeta}
\pmcanonicalname{ProofOfArgumentPrinciple}
\pmcreated{2013-03-22 14:34:32}
\pmmodified{2013-03-22 14:34:32}
\pmowner{rspuzio}{6075}
\pmmodifier{rspuzio}{6075}
\pmtitle{proof of argument principle}
\pmrecord{11}{36134}
\pmprivacy{1}
\pmauthor{rspuzio}{6075}
\pmtype{Proof}
\pmcomment{trigger rebuild}
\pmclassification{msc}{30E20}
\pmsynonym{Cauchy's argument principle}{ProofOfArgumentPrinciple}
%\pmkeywords{complex variables}
%\pmkeywords{complex analysis}
%\pmkeywords{complex integrals}
%\pmkeywords{contour integration}
%\pmkeywords{residues}

% this is the default PlanetMath preamble.  as your knowledge
% of TeX increases, you will probably want to edit this, but
% it should be fine as is for beginners.

% almost certainly you want these
\usepackage{amssymb}
\usepackage{amsmath}
\usepackage{amsfonts}

% used for TeXing text within eps files
%\usepackage{psfrag}
% need this for including graphics (\includegraphics)
%\usepackage{graphicx}
% for neatly defining theorems and propositions
%\usepackage{amsthm}
% making logically defined graphics
%%%\usepackage{xypic}

% there are many more packages, add them here as you need them

% define commands here

\begin{document}
Since $f$ is meromorphic, $f'$ is meromorphic, and hence $f'/f$ is meromorphic.  The singularities of $f'/f$ can only occur at the zeros and the poles of $f$.

I claim that all singularities of $f'/f$ are simple poles.  Furthermore, if $f$ has a zero at some point $p$, then the residue of the pole at $p$ is positive and equals the multiplicity of the zero of $f$ at $p$.  If $f$ has a pole at some point $p$, then the residue of the pole at $p$ is negative and equals minus the multiplicity of the pole of $f$ at $p$.

To prove these assertions, write\! $f(x) = (x\!-\!p)^n g(x)$\, with\, $g(p) \neq 0$.  Then
 $${f'(x) \over f(x)} = {n \over x\!-\!p} + {g'(x) \over g(x)}$$
Since $g(p) \neq 0$, the only singularity of $f'\!/\!f$ at $p$ comes from the first summand.  Since $n$ is either the order of the zero of $f$ at $p$ if $f$ has a zero at $p$ or minus the order of the pole of $f$ at $p$ if $f$ has a pole at $p$, the assertion is proven.

By the Cauchy residue theorem, the integral
 $${1 \over 2 \pi i} \int_C {f'(z) \over f(z)} dz$$
equals the sum of the residues of $f'/f$.  Combining this fact with the characterization of the poles of $f'/f$ and their residues given above, one deduces that this integral equals the number of zeros of $f$ minus the number of poles of $f$, counted with multiplicity.

%%%%%
%%%%%
\end{document}
