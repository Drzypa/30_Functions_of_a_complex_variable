\documentclass[12pt]{article}
\usepackage{pmmeta}
\pmcanonicalname{QuadraticEquationInmathbbC}
\pmcreated{2013-03-22 17:36:36}
\pmmodified{2013-03-22 17:36:36}
\pmowner{pahio}{2872}
\pmmodifier{pahio}{2872}
\pmtitle{quadratic equation in $\mathbb{C}$}
\pmrecord{11}{40026}
\pmprivacy{1}
\pmauthor{pahio}{2872}
\pmtype{Theorem}
\pmcomment{trigger rebuild}
\pmclassification{msc}{30-00}
\pmclassification{msc}{12D99}
\pmsynonym{quadratic equation}{QuadraticEquationInmathbbC}
\pmrelated{QuadraticFormula}
\pmrelated{DerivationOfQuadraticFormula}
\pmrelated{CardanosDerivationOfTheCubicFormula}

\endmetadata

% this is the default PlanetMath preamble.  as your knowledge
% of TeX increases, you will probably want to edit this, but
% it should be fine as is for beginners.

% almost certainly you want these
\usepackage{amssymb}
\usepackage{amsmath}
\usepackage{amsfonts}

% used for TeXing text within eps files
%\usepackage{psfrag}
% need this for including graphics (\includegraphics)
%\usepackage{graphicx}
% for neatly defining theorems and propositions
 \usepackage{amsthm}
% making logically defined graphics
%%%\usepackage{xypic}

% there are many more packages, add them here as you need them

% define commands here

\theoremstyle{definition}
\newtheorem*{thmplain}{Theorem}

\begin{document}
The quadratic formula
$$x \;=\; \frac{-b\!\pm\!\sqrt{b^2\!-\!4ac}}{2a}$$
for solving the quadratic equation
\begin{align}
ax^2\!+\!bx\!+\!c \;=\; 0
\end{align}
with real coefficients $a$, $b$, $c$ is valid as well for all complex values of these coefficients ($a \neq 0$), when the square root is determined as is presented in the \PMlinkname{parent entry}{TakingSquareRootAlgebraically}.\\

{\em Proof.}\; Multiplying (1) by $4a$ and adding $b^2$ to both sides gives an \PMlinkname{equivalent}{Equivalent3} equation
$$4a^2x^2\!+\!4abx\!+\!4ac\!+\!b^2 \;=\; b^2$$
or
$$(2ax)^2\!+\!2\!\cdot\!2ax\!\cdot\!{b}\!+\!b^2 \;=\; b^2\!-\!4ac$$
or furthermore
$$(2ax\!+\!b)^2 \;=\; b^2\!-\!4ac.$$
Taking square root algebraically yields
$$2ax\!+\!b \;=\; \pm\!\sqrt{b^2\!-\!4ac},$$
which implies the quadratic formula.\\


\textbf{Note.}\, A \PMlinkescapetext{similar} quadratic formula is meaningful besides $\mathbb{C}$ also in other fields with characteristic $\neq 2$\, if one can find the needed ``square root'' (this may require a field extension).



%%%%%
%%%%%
\end{document}
