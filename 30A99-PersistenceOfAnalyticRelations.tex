\documentclass[12pt]{article}
\usepackage{pmmeta}
\pmcanonicalname{PersistenceOfAnalyticRelations}
\pmcreated{2013-03-22 14:44:17}
\pmmodified{2013-03-22 14:44:17}
\pmowner{rspuzio}{6075}
\pmmodifier{rspuzio}{6075}
\pmtitle{persistence of analytic relations}
\pmrecord{9}{36373}
\pmprivacy{1}
\pmauthor{rspuzio}{6075}
\pmtype{Theorem}
\pmcomment{trigger rebuild}
\pmclassification{msc}{30A99}
\pmrelated{ComplexSineAndCosine}

\endmetadata

% this is the default PlanetMath preamble.  as your knowledge
% of TeX increases, you will probably want to edit this, but
% it should be fine as is for beginners.

% almost certainly you want these
\usepackage{amssymb}
\usepackage{amsmath}
\usepackage{amsfonts}

% used for TeXing text within eps files
%\usepackage{psfrag}
% need this for including graphics (\includegraphics)
%\usepackage{graphicx}
% for neatly defining theorems and propositions
%\usepackage{amsthm}
% making logically defined graphics
%%%\usepackage{xypic}

% there are many more packages, add them here as you need them

% define commands here
\begin{document}
The principle of persistence of analytic relations states that any algebraic relation between several analytic functions which holds on a sufficiently large set also holds wherever the functions are defined.

A more explicit statement of this principle is as follows:  Let $f_1, f_2, \ldots f_n$ be complex analytic functions.  Suppose that there exists an open set $D$ on which all these functions are defined and that there exists a polynomial $p$ of $n$ variables such that $p(f_1(z), f_2 (z), \ldots, f_n(z)) = 0$ whenever $z$ lies in a subset $X$ of $D$ which has a limit point in $D$.  Then $p (f_1 (z), f_2 (z), \ldots f_n (z)) = 0$ for all $z \in D$.  

This fact is a simple consequence of the rigidity theorem for analytic functions.  If $f_1, f_2, \ldots f_n$ are all analytic in $D$, then $p (f_1 (z), f_2 (z), \ldots f_n (z))$ is also analytic in $D$.  Hence, if $p (f_1 (z), f_2 (z), \ldots,  f_n (z)) = 0$ when $z$ in $X$, then $p (f_1 (z), f_2 (z), \ldots,  f_n (z)) = 0$ for all $z \in D$.

This principle is very useful in establishing identites involving analytic functions because it means that it suffices to show that the identity holds on a small subset.  For instance, from the fact that the familiar identity $\sin^2 x + \cos^2 x = 1$ holds for all real $x$, it automatically holds for all complex values of $x$.  This principle also means that it is unnecessary to specify for which values of the variable an algebraic relation between analytic functions holds since, if such a relation holds, it will hold for all values for which the functions appearing in the relation are defined.
%%%%%
%%%%%
\end{document}
