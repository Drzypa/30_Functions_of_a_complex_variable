\documentclass[12pt]{article}
\usepackage{pmmeta}
\pmcanonicalname{LimitForExpz}
\pmcreated{2013-03-22 14:34:54}
\pmmodified{2013-03-22 14:34:54}
\pmowner{mathcam}{2727}
\pmmodifier{mathcam}{2727}
\pmtitle{limit for exp(z)}
\pmrecord{11}{36144}
\pmprivacy{1}
\pmauthor{mathcam}{2727}
\pmtype{Theorem}
\pmcomment{trigger rebuild}
\pmclassification{msc}{30A99}

\endmetadata

% this is the default PlanetMath preamble.  as your knowledge
% of TeX increases, you will probably want to edit this, but
% it should be fine as is for beginners.

% almost certainly you want these
\usepackage{amssymb}
\usepackage{amsmath}
\usepackage{amsfonts}

% used for TeXing text within eps files
%\usepackage{psfrag}
% need this for including graphics (\includegraphics)
%\usepackage{graphicx}
% for neatly defining theorems and propositions
%\usepackage{amsthm}
% making logically defined graphics
%%%\usepackage{xypic}

% there are many more packages, add them here as you need them

% define commands here
\begin{document}
For any complex number $z$, we have
\begin{align*}
\lim\limits_{n\to\infty} \left(1+\frac{z}{n}+o\left(\frac{1}{n}\right)\right)^n=\exp{z},\end{align*} where $ \exp $ denotes the exponential function.
\newline
{\bf Proof:} 
For $\alpha\to 0$, we have 
\begin{align*} 
\ln(1+\alpha) &= \sum_{k=1}^\infty (-1)^{k-1}
\cdot\frac{\alpha^k}{k} \\
&= \alpha+O(\alpha^2). \end{align*}
Therefore 
\begin{eqnarray*}
\left(1+\frac{z}{n}+o\left(\frac{1}{n}\right)\right)^n &=& \exp\left(n\ln\left(1+\frac{z}{n}+o\left(\frac{1}{n}\right)\right)\right) \\
&=&\exp\left(n\left(\frac{z}{n}+o\left(\frac{1}{n}\right)+O(\frac{1}{n^2})\right)\right)\\
&=&\exp\left(z+o(1)+O(\frac{1}{n})\right)\to\exp{z} \quad \text{for} \quad n\to\infty. \quad \Box \end{eqnarray*}
%%%%%
%%%%%
\end{document}
