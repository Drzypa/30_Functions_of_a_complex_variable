\documentclass[12pt]{article}
\usepackage{pmmeta}
\pmcanonicalname{CalculationOfContourIntegral}
\pmcreated{2013-03-22 19:14:16}
\pmmodified{2013-03-22 19:14:16}
\pmowner{pahio}{2872}
\pmmodifier{pahio}{2872}
\pmtitle{calculation of contour integral}
\pmrecord{8}{42162}
\pmprivacy{1}
\pmauthor{pahio}{2872}
\pmtype{Example}
\pmcomment{trigger rebuild}
\pmclassification{msc}{30E20}
\pmclassification{msc}{30A99}
\pmrelated{AntiderivativeOfComplexFunction}
\pmrelated{SubstitutionNotation}
\pmrelated{ProofOfCauchyIntegralFormula}

% this is the default PlanetMath preamble.  as your knowledge
% of TeX increases, you will probably want to edit this, but
% it should be fine as is for beginners.

% almost certainly you want these
\usepackage{amssymb}
\usepackage{amsmath}
\usepackage{amsfonts}

% used for TeXing text within eps files
%\usepackage{psfrag}
% need this for including graphics (\includegraphics)
%\usepackage{graphicx}
% for neatly defining theorems and propositions
%\usepackage{amsthm}
% making logically defined graphics
%%%\usepackage{xypic}

% there are many more packages, add them here as you need them

% define commands here
\newcommand{\sijoitus}[2]%
{\operatornamewithlimits{\Big/}_{\!\!\!#1}^{\,#2}}

\begin{document}
We will determine the important complex integral
$$I \;:=\; \oint_C\!(z\!-\!z_0)^n\,dz$$
where $C$ is the circumference of the circle \,$|z\!-\!z_0| = \varrho$\, taken anticlockwise and $n$ an arbitrary integer.\\

Let's take the ``direction angle'' of the radius of $C$ as the parametre $t$, i.e. 
$$t \;:=\; \arg|z\!-\!z_0|.$$
Then on $C$, we have
$$z\!-\!z_0 \;=\; \varrho e^{it}, \quad 0 \;\leqq\; t \leqq 2\pi$$
and 
$$dz \;=\; i\varrho e^{it}\,dt, \quad (z\!-\!z_0)^n \;=\; \varrho^ne^{int},$$
whence
$$I \;=\; \int_0^{2\pi}\!\varrho^ne^{int}i\varrho e^{it}\,dt\;=\; i\varrho^{n+1}\!\int_0^{2\pi}\!e^{i(n+1)t}\,dt.$$
In the case \,$n = -1$\, one gets trivially\, $I = 2i\pi$.\, If\, $n \neq -1$,\, we obtain
$$I \;=\; i\varrho^{n+1}\!\!\sijoitus{t=0}{\quad 2\pi}\!\frac{e^{i(n+1)t}}{i(n\!+\!1)} 
\;=\; \frac{\varrho^{n+1}}{n\!+\!1}(1\!-\!1) \;=\; 0,$$
using the fact that $2i\pi$ is a \PMlinkname{period of the exponential function}{PeriodicityOfExponentialFunction}.

Hence we can write the result
\begin{align*}
 \oint_C(z\!-\!z_0)^n\,dz \;=\;
   \begin{cases}
     2i\pi \quad \mbox{if} \;\; n \;=\; -1, \\
     0    \qquad \mbox{if} \;\; n \;\in\; \mathbb{Z}\!\smallsetminus\!\{-1\}. \\
   \end{cases}
\end{align*}





%%%%%
%%%%%
\end{document}
