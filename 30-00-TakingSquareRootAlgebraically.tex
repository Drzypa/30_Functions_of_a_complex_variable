\documentclass[12pt]{article}
\usepackage{pmmeta}
\pmcanonicalname{TakingSquareRootAlgebraically}
\pmcreated{2015-06-14 16:31:35}
\pmmodified{2015-06-14 16:31:35}
\pmowner{pahio}{2872}
\pmmodifier{pahio}{2872}
\pmtitle{taking square root algebraically}
\pmrecord{17}{37175}
\pmprivacy{1}
\pmauthor{pahio}{2872}
\pmtype{Derivation}
\pmcomment{trigger rebuild}
\pmclassification{msc}{30-00}
\pmclassification{msc}{12D99}
\pmsynonym{square root of complex number}{TakingSquareRootAlgebraically}
\pmrelated{SquareRootOfSquareRootBinomial}
\pmrelated{CasusIrreducibilis}
\pmrelated{TopicEntryOnComplexAnalysis}
\pmrelated{ValuesOfComplexCosine}

% this is the default PlanetMath preamble.  as your knowledge
% of TeX increases, you will probably want to edit this, but
% it should be fine as is for beginners.

% almost certainly you want these
\usepackage{amssymb}
\usepackage{amsmath}
\usepackage{amsfonts}

% used for TeXing text within eps files
%\usepackage{psfrag}
% need this for including graphics (\includegraphics)
%\usepackage{graphicx}
% for neatly defining theorems and propositions
 \usepackage{amsthm}
% making logically defined graphics
%%%\usepackage{xypic}

% there are many more packages, add them here as you need them

% define commands here

\theoremstyle{definition}
\newtheorem*{thmplain}{Theorem}
\begin{document}
For getting the square root of the complex number\; $a\!+\!ib$\, 
($a, b\in \mathbb{R}$) purely algebraically, one should solve the real part $x$ and the imaginary part $y$ of\, $\sqrt{a\!+\!ib}$\, from the binomial equation
\begin{align}
  (x\!+\!iy)^2 = a\!+\!ib.
\end{align}
This gives 
$$a\!+\!ib \;=\; x^2\!+\!2ixy\!-\!y^2 \;=\; 
(x^2\!-\!y^2)\!+\!i\!\cdot\!2xy.$$
Comparing (see \PMlinkname{equality}{EqualityOfComplexNumbers}) the real parts and the imaginary parts yields the pair of real equations
$$x^2\!-\!y^2 = a,\qquad 2xy = b,$$
which may be written
$$x^2\!+\!(-y^2) = a, \qquad x^2\!\cdot\!(-y^2) = -\frac{b^2}{4}.$$
Note that the \PMlinkescapetext{signs of} $x$ and $y$ must be chosen such that their product ($= \frac{b}{2}$) has the same sign as $b$.\, Using the properties of quadratic equation, one infers that $x^2$ and $-y^2$ are the roots of the equation 
$$t^2\!-\!at\!-\!\frac{b^2}{4} = 0.$$
The quadratic formula gives
$$t = \frac{a\pm\sqrt{a^2\!+\!b^2}}{2},$$
and since $-y^2$ is the smaller root,\, $x^2 = \frac{a\!+\!\sqrt{a^2\!+\!b^2}}{2},\quad
 -y^2 = \frac{a\!-\!\sqrt{a^2\!+\!b^2}}{2}$.\, So we obtain the result
$$x = \sqrt{\frac{\sqrt{a^2\!+\!b^2}+\!a}{2}}, \qquad
y = (\mathrm{sign}\,b)\sqrt{\frac{\sqrt{a^2\!+\!b^2}-\!a}{2}}$$
(see the signum function).\, Because both may have also the \PMlinkescapetext{opposite signs, we have the formula}
\begin{align}
\sqrt{a\!+\!ib} \;=\; 
\pm\left(\sqrt{\frac{\sqrt{a^2\!+\!b^2}+\!a}{2}}+(\mathrm{sign}\,{b})i\sqrt{\frac{\sqrt{a^2\!+\!b^2}-\!a}{2}}\right).
\end{align}

The result shows that the real and imaginary parts of the square root of any complex number\; $a\!+\!ib$\, can be obtained from the real part $a$ and imaginary part $b$ of the number by using only algebraic operations, i.e. the rational operations and the \PMlinkescapetext{radicals}.\, Apparently, the same is true for all roots of a complex number with \PMlinkname{index}{NthRoot} an integer power of 2.

In practise, when determining the square root of a non-real complex number, one need not to remember the \PMlinkescapetext{formula} (2), but it's better to solve concretely the equation (1).

\textbf{Exercise.}\, Compute $\sqrt{i}$ and check it!
%%%%%
%%%%%
\end{document}
