\documentclass[12pt]{article}
\usepackage{pmmeta}
\pmcanonicalname{NaturalBoundary}
\pmcreated{2013-03-22 16:49:34}
\pmmodified{2013-03-22 16:49:34}
\pmowner{rspuzio}{6075}
\pmmodifier{rspuzio}{6075}
\pmtitle{natural boundary}
\pmrecord{7}{39065}
\pmprivacy{1}
\pmauthor{rspuzio}{6075}
\pmtype{Definition}
\pmcomment{trigger rebuild}
\pmclassification{msc}{30A99}
\pmclassification{msc}{30B40}

\endmetadata

% this is the default PlanetMath preamble.  as your knowledge
% of TeX increases, you will probably want to edit this, but
% it should be fine as is for beginners.

% almost certainly you want these
\usepackage{amssymb}
\usepackage{amsmath}
\usepackage{amsfonts}

% used for TeXing text within eps files
%\usepackage{psfrag}
% need this for including graphics (\includegraphics)
%\usepackage{graphicx}
% for neatly defining theorems and propositions
%\usepackage{amsthm}
% making logically defined graphics
%%%\usepackage{xypic}

% there are many more packages, add them here as you need them

% define commands here
\newtheorem{dfn}{Definition}
\begin{document}
It is not always possible to analytically continue a
function given in a certain region.  It might turn
out that, as one approaches the boundary of the region
(or a portion of the boundary), the function always blows 
up, so there is no way of extending it past that portion
of the boundary to a larger region.  When this happens,
we say that our function has a \emph{natural boundary}.  
More formally, we may make a definition as follows:

\begin{dfn}
Let $\mathbf{D}$ be an open subset of the complex plane
and let $f \colon \mathbf{D} \to \mathbb{C}$ be analytic.
Then the \emph{natural boundary} of $f$ is that subset
$B$ of $\partial \mathbf{D}$ such that, if $z\in B$,
then there exists no open neighborhood $\mathbf{N}$ of 
$z$ and no analytic function $g \colon \mathbf{N} \to
\mathbb{C}$ such that $f(w) = g(w)$ for all $w \in \mathbf{D}
\cup \mathbf{N}$.
\end{dfn}

As an example of this phenomenon, consider the power series
\[
\sum_{k=0}^\infty z^{k!}.
\]
By comparison with the geometric series, it is seen that this
series converges absolutely when $|z| < 1$:
\[
\sum_{k=0}^\infty |z|^{k!} <
\sum_{k=0}^\infty |z|^k =
{1 \over 1 - |z|}
\]
However, when we try to take the limit $|z| \to 1$, we find
that the series diverges.  Namely, let $p/q$ be a rational 
number and let $r$ be a positive real variable.  Then, if we 
set $z = r \exp (2 i \pi p / q)$, then, when $k \ge q$, we 
have that $q$ divides $k!$, so $z^{k!} = r^{k!}$.  However,
\[
\lim_{r \to 1} \sum_{k=q}^\infty r^{k!}
\]
diverges, so our powers series diverges when we try to take
the limit $z \to \exp (2 i \pi p / q)$.  Since numbers of the
form $\exp (2 i \pi p / q)$ are dense amongst complex numbers
with norm $1$, it follows that the limit $z \to z_0$ diverges
whenver $|z_0| = 1$.  Hence, the unit circle $|z| = 1$ forms
a natural boundary for the function defined by our power series.

Natural boundaries are not so familiar to beginners because
the functions which one encounters in the more elementary
part of the subject, such as algebraic functions, exponential
functions, and functions defined by linear differential
equations, do not have natural boundaries.  To be sure, one
could technically call a singular point a natural boundary, 
but this is usually not done, the term ``natural boundary''
being reserved for cases where the set on which the
function misbehaves consists of more than just isolated points,
as in the example above.

However, when one gains some more experience and studies more
advanced material, then natural boundaries arise rather
frequently.  For instance, theta functions, elliptic modular 
functions, and functions defined by non-linear differential 
equations have natural boundaries.  Natural boundaries also
play an important role in applications --- for instance, in
statistical mechanics, phase transitions (such as freezing
and boiling) are associated with natural boundaries of the
partition function.
%%%%%
%%%%%
\end{document}
