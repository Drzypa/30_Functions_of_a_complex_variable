\documentclass[12pt]{article}
\usepackage{pmmeta}
\pmcanonicalname{ProofOfConvergenceConditionOfInfiniteProduct}
\pmcreated{2013-03-22 17:22:27}
\pmmodified{2013-03-22 17:22:27}
\pmowner{fernsanz}{8869}
\pmmodifier{fernsanz}{8869}
\pmtitle{proof of convergence condition of infinite product}
\pmrecord{5}{39738}
\pmprivacy{1}
\pmauthor{fernsanz}{8869}
\pmtype{Proof}
\pmcomment{trigger rebuild}
\pmclassification{msc}{30E20}

\endmetadata

% this is the default PlanetMath preamble.  as your knowledge
% of TeX increases, you will probably want to edit this, but
% it should be fine as is for beginners.

% almost certainly you want these
\usepackage{amssymb}
\usepackage{amsmath}
\usepackage{amsfonts}
\usepackage{amsthm}

% define commands here
\newcommand{\abs}[1]{\left\vert#1\right\vert}
\begin{document}
\title{proof of theorem of convergence of infinite product}%
\author{Fernando Sanz Gamiz}%

\begin{proof}
Let $p_n=\prod_{i=1}^n u_i$. We have to study the convergence of the
sequence $\{p_n\}$. The sequence $\{p_n\}$ converges to a not null limit iff
$\{\log p_n\}$ ($\log$ is restricted to its principal branch) converges
to a finite limit. By the Cauchy criterion, this happens iff for
every $\epsilon^{\prime}>0$ there exist $N$ such that $\abs{\log
p_{n+k} - \log p_n} < \epsilon^{\prime}$ for all $n>N$ and all
$k=1,2,\ldots$, i.e, iff
$$\abs{\log \frac{p_{n+k}}{p_n}} = \abs{\log u_{n+1}u_{n+2}\cdots u_{n+k}} < \epsilon^{\prime};$$
as $\log(z)$ is an injective function and continuous at $z=1$ and
$\log(1)=0$ this will happen iff for every $\epsilon>0$
$$\abs{u_{n+1}u_{n+2}\cdots u_{n+k}-1} < \epsilon$$ for $n$ greater
than $N$ and $k=1,2,\ldots$
\end{proof}
%%%%%
%%%%%
\end{document}
