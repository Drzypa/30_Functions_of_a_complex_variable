\documentclass[12pt]{article}
\usepackage{pmmeta}
\pmcanonicalname{DilogarithmFunction}
\pmcreated{2013-03-22 19:34:58}
\pmmodified{2013-03-22 19:34:58}
\pmowner{pahio}{2872}
\pmmodifier{pahio}{2872}
\pmtitle{dilogarithm function}
\pmrecord{14}{42570}
\pmprivacy{1}
\pmauthor{pahio}{2872}
\pmtype{Definition}
\pmcomment{trigger rebuild}
\pmclassification{msc}{30D30}
\pmclassification{msc}{33B15}
\pmsynonym{Spence's function}{DilogarithmFunction}
%\pmkeywords{dilogarithm}
%\pmkeywords{polylogarithm}
\pmrelated{ApplicationOfLogarithmSeries}
\pmdefines{polylogarithm function}

\endmetadata

% this is the default PlanetMath preamble.  as your knowledge
% of TeX increases, you will probably want to edit this, but
% it should be fine as is for beginners.

% almost certainly you want these
\usepackage{amssymb}
\usepackage{amsmath}
\usepackage{amsfonts}

% used for TeXing text within eps files
%\usepackage{psfrag}
% need this for including graphics (\includegraphics)
%\usepackage{graphicx}
% for neatly defining theorems and propositions
 \usepackage{amsthm}
% making logically defined graphics
%%%\usepackage{xypic}

% there are many more packages, add them here as you need them

% define commands here

\theoremstyle{definition}
\newtheorem*{thmplain}{Theorem}

\begin{document}
The {\it dilogarithm function} 
\begin{align}
\mbox{Li}_2(x) \;=:\; \sum_{n=1}^{\infty}\frac{x^n}{n^2},
\end{align}
studied already by Leibniz, is a special case of the {\it polylogarithm function}
$$\mbox{Li}_s(x) \;=:\; \sum_{n=1}^{\infty}\frac{x^n}{n^s}.$$
The radius of convergence of the series (1) is 1, whence the definition (1) is valid also in the unit disc of the complex plane.\, For\, $0 \le x \le 1$,\, the equation (1) is apparently equivalent to
\begin{align}
\mbox{Li}_2(x) \;=:\; -\int_0^x\frac{\ln(1\!-\!t)}{t}\,dt,
\end{align}
(cf. logarithm series of $\ln(1\!-\!x)$).\, The analytic continuation of $\mbox{Li}_2$ for\, $|z| \ge 1$\, can be made by
\begin{align}
\mbox{Li}_2(z) \;=:\; -\int_0^z\frac{\log(1\!-\!t)}{t}\,dt.
\end{align}
Thus $\mbox{Li}_2(z)$ is a multivalued analytic function of $z$.\, Its {\it principal branch} is single-valued and is got by taking the principal branch of the complex logarithm; then
$$z \;\in\; \mathbb{C}\!\smallsetminus\![1,\,\infty[, \quad 0 < \arg(z\!-\!1) < 2\pi.$$\\

For real values of $x$ we have
\begin{align*}
\mbox{Im}(\mbox{Li}_2(x)) \;=\;
\begin{cases}
\;0 \qquad\mbox{for}\;\; x \le 1,\\
-\pi\ln{x} \;\;\mbox{for}\;\; x > 1.
\end{cases}
\end{align*}
According to (2), the derivative of the dilogarithm is
$$\mbox{Li}'_2(x) \;=\; \frac{-\ln(1\!-\!x)}{x}.$$
In terms of the Bernoulli numbers, the dilogarithm function has a series expansion more rapidly converging than (1):
\begin{align}
\mbox{Li}_2(x) \;=\; \sum_{n=1}^{\infty}B_{n-1}\frac{(-\ln(1\!-\!x))^n}{n!} 
\qquad (|\ln(1\!-\!x)| < 2\pi)
\end{align}


\textbf{Some functional equations and values}
$$\mbox{Li}_2(z)+\mbox{Li}_2(-z) \;=\; \frac{1}{2}\mbox{Li}_2(z^2),$$
$$\mbox{Li}_2(z)+\mbox{Li}_2\left(\frac{1}{z}\right) \;=\; -\frac{1}{2}(\log(-z))^2-\frac{\pi^2}{6},$$
$$\mbox{Li}_2(iz)-i\mbox{Li}_2(z) \;=\; \frac{1}{4}\mbox{Li}_2(-z^2),$$
$$\mbox{Li}_2(1) \;=\; \frac{\pi^2}{6}, \quad
\mbox{Li}_2(2) \;=\; \frac{\pi^2}{4}-i\pi\ln{2}, \quad
\mbox{Li}_2(i) \;=\; -\frac{\pi^2}{48}-i\,G$$
Here, $G$ is Catalan's constant.\\

\begin{thebibliography}{8}
\bibitem{K}{\sc Anatol N. Kirillov}: {\it Dilogarithm identities} (1994). Available \PMlinkexternal{here}{http://arxiv.org/pdf/hep-th/9408113v2.pdf}.
\bibitem{M}{\sc Leonard C. Maximon}:  The dilogarithm function for complex argument.\, -- {\it Proc. R. Soc. Lond. A} \textbf{459} (2003) 2807--2819.

\end{thebibliography}

%%%%%
%%%%%
\end{document}
