\documentclass[12pt]{article}
\usepackage{pmmeta}
\pmcanonicalname{ProofOfMobiusTransformationCrossratioPreservationTheorem}
\pmcreated{2013-03-22 14:08:20}
\pmmodified{2013-03-22 14:08:20}
\pmowner{Johan}{1032}
\pmmodifier{Johan}{1032}
\pmtitle{Proof of M\"obius transformation cross-ratio preservation theorem}
\pmrecord{5}{35553}
\pmprivacy{1}
\pmauthor{Johan}{1032}
\pmtype{Proof}
\pmcomment{trigger rebuild}
\pmclassification{msc}{30E20}

% this is the default PlanetMath preamble.  as your knowledge
% of TeX increases, you will probably want to edit this, but
% it should be fine as is for beginners.

% almost certainly you want these
\usepackage{amssymb}
\usepackage{amsmath}
\usepackage{amsfonts}

% used for TeXing text within eps files
%\usepackage{psfrag}
% need this for including graphics (\includegraphics)
%\usepackage{graphicx}
% for neatly defining theorems and propositions
%\usepackage{amsthm}
% making logically defined graphics
%%%\usepackage{xypic}

% there are many more packages, add them here as you need them

% define commands here
\begin{document}
From the definition of M\"obius transform we get
that the image $w_k$ of a point $z_k$ is
$$
w_k = \frac{az_k+b}{cz_k+d}
$$
From this we get
$$
w_i - w_j = \frac{az_i+b}{cz_i+d} - \frac{az_j+b}{cz_j+d}
= \frac{(ad-bc)(z_i-z_j)}{(cz_i+d)(cz_j+d)}
$$
and by inserting this into the cross-ratios
$$
\frac{(w_1-w_2)(w_3-w_4)}{(w_1-w_4)(w_3-w_2)} =
\frac{\frac{(ad-bc)(z_1-z_2)}{(cz_1+d)(cz_2+d)}\frac{(ad-bc)(z_3-z_4)}{(cz_3+d)(cz_4+d)}}{\frac{(ad-bc)(z_1-z_4)}{(cz_1+d)(cz_4+d)}\frac{(ad-bc)(z_3-z_2)}{(cz_3+d)(cz_2+d)}} =  \frac{(z_1-z_2)(z_3-z_4)}{(z_1-z_4)(z_3-z_2)}
$$
%%%%%
%%%%%
\end{document}
