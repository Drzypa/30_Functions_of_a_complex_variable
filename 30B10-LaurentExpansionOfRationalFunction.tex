\documentclass[12pt]{article}
\usepackage{pmmeta}
\pmcanonicalname{LaurentExpansionOfRationalFunction}
\pmcreated{2013-03-11 19:16:06}
\pmmodified{2013-03-11 19:16:06}
\pmowner{pahio}{2872}
\pmmodifier{}{0}
\pmtitle{Laurent expansion of rational function}
\pmrecord{5}{42619}
\pmprivacy{1}
\pmauthor{pahio}{0}
\pmtype{Example}
\pmclassification{msc}{30B10}
\pmsynonym{}{LaurentExpansionOfRationalFunction}
%\pmkeywords{}
\pmrelated{}
\pmdefines{}

% this is the default PlanetMath preamble.  as your knowledge
% of TeX increases, you will probably want to edit this, but
% it should be fine as is for beginners.

% almost certainly you want these
\usepackage{amssymb}
\usepackage{amsmath}
\usepackage{amsfonts}

% used for TeXing text within eps files
%\usepackage{psfrag}
% need this for including graphics (\includegraphics)
%\usepackage{graphicx}
% for neatly defining theorems and propositions
 \usepackage{amsthm}
% making logically defined graphics
%%\usepackage{xypic}

% there are many more packages, add them here as you need them

% define commands here

\theoremstyle{definition}
\newtheorem*{thmplain}{Theorem}

\begin{document}
The Laurent series expansion of a rational function may often be determined using the uniqueness of Laurent series coefficients in an annulus and applying geometric series.\, We will determine the expansion of 
$$f(z) \;:=\; \frac{2z}{1\!+\!z^2}$$
by the powers of $z\!-\!i$.

We first have the partial fraction decomposition 
\begin{align}
f(z) \;=\; \frac{1}{z\!-\!i}+\frac{1}{z\!+\!i}
\end{align}
whence the principal part of the Laurent expansion contains $\displaystyle\frac{1}{z\!-\!i}$.\, Taking into account the poles\, $z = \pm i$\, of $f$ we see that there are two possible annuli for the Laurent expansion:\\

a)\; The annulus\; $\displaystyle 0 \,<\, |z\!-\!i| \,<\, 2$.\, We can write
$$
\frac{1}{z\!+\!i} \;=\; \frac{1}{2i\!+\!(z\!-\!i)} \;=\; \frac{1}{2i}\cdot\frac{1}{1\!-\!\left(-\frac{z-i}{2i}\right)}
                  \;=\; \frac{1}{2i}-\frac{z\!-\!i}{(2i)^2}+\frac{(z\!-\!i)^2}{(2i)^3}-+\ldots
$$
Thus
$$\frac{2z}{1\!+\!z^2} \;=\; \frac{1}{z\!-\!i}-\sum_{n=0}^\infty\left(\frac{i}{2}\right)^{n+1}(z\!-\!i)^n
\quad \qquad (0 \,<\, |z\!-\!i| \,<\, 2).
$$\\


b)\; The annulus\; $\displaystyle 2 \,<\, |z\!-\!i| \,<\, \infty$.\, Now we write
$$\frac{1}{z\!+\!i} \;=\; 
\frac{1}{(z\!-\!i)\!+\!2i} \;=\; \frac{1}{z\!-\!i}\cdot\frac{1}{1\!-\!\left(-\frac{2i}{z-i}\right)}
\;=\; \frac{1}{z\!-\!i}-\frac{2i}{(z\!-\!i)^2}+\frac{(2i)^2}{(z\!-\!i)^3}-+\ldots
$$
Accordingly
$$\frac{2z}{1\!+\!z^2} \;=\; \frac{2}{z\!-\!i}+\sum_{n=2}^\infty\frac{(-2i)^{n-1}}{(z\!-\!i)^n}
\quad \qquad (2 \,<\, |z\!-\!i| \,<\, \infty).
$$
This latter Laurent expansion consists of negative powers only, but\, $z = i$\, isn't an essential singularity of $f$, though.

%%%%%
\end{document}
