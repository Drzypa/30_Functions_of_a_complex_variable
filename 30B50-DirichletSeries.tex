\documentclass[12pt]{article}
\usepackage{pmmeta}
\pmcanonicalname{DirichletSeries}
\pmcreated{2013-03-22 13:59:22}
\pmmodified{2013-03-22 13:59:22}
\pmowner{bbukh}{348}
\pmmodifier{bbukh}{348}
\pmtitle{Dirichlet series}
\pmrecord{8}{34764}
\pmprivacy{1}
\pmauthor{bbukh}{348}
\pmtype{Definition}
\pmcomment{trigger rebuild}
\pmclassification{msc}{30B50}
\pmrelated{DirichletLFunction}
\pmrelated{RiemannZetaFunction}
\pmrelated{DirichletLSeries}
\pmdefines{ordinary Dirichlet series}
\pmdefines{Stoltz angle}
\pmdefines{abscissa of convergence}
\pmdefines{abscissa of absolute convergence}

\endmetadata

\usepackage{amssymb}
\usepackage{amsmath}
\usepackage{amsfonts}

\newcommand*{\C}{\mathbb{C}}

\makeatletter
\@ifundefined{bibname}{}{\renewcommand{\bibname}{References}}
\makeatother
\begin{document}
\PMlinkescapeword{arithmetic}
\PMlinkescapeword{complex}
\PMlinkescapeword{analytic}
\PMlinkescapeword{theory}
\PMlinkescapeword{properties}

Let $(\lambda_n)_{n\ge 1}$ be an increasing sequence of
positive real numbers tending to $\infty$.
A \emph{Dirichlet series} with exponents $(\lambda_n)$ is
a series of the form
$$\sum_n a_n e^{-\lambda_nz}$$
where $z$ and all the $a_n$ are complex numbers.

An \emph{ordinary Dirichlet series} is one having $\lambda_n=\log n$
for all $n$.
It is written
$$\sum\frac{a_n}{n^z}\;.$$
The best-known examples are the Riemann zeta function (in which $a_n$
is the \PMlinkname{constant}{ConstantFunction} $1$) and the more general Dirichlet L-series
(in which the mapping $n\mapsto a_n$ is multiplicative and periodic).

When $\lambda_n=n$, the Dirichlet series is just a power series
in the variable $e^{-z}$.

The following are the basic convergence properties of Dirichlet series.
There is nothing profound about their proofs, which can be found
in \cite{cite:serre_arithm_course_vi} and in various other works on complex analysis and analytic
number theory.

Let $f(z)=\sum_n a_n e^{-\lambda_nz}$ be a Dirichlet series.
\begin{enumerate}
\item
If $f$ converges at $z=z_0$, then $f$ converges uniformly in the region
$$\Re(z-z_0)\ge 0\qquad -\alpha\le\arg(z-z_0)\le \alpha$$
where $\alpha$ is any real number such that $0<\alpha<\pi/2$.
(Such a region is known as a ``Stoltz angle''.)
\item Therefore, if $f$ converges at $z_0$, its sum defines a holomorphic
function on the region $\Re(z)>\Re(z_0)$, and moreover $f(z)\to f(z_0)$
as $z\to z_0$ within any Stoltz angle.
\item $f=0$ identically if and only if all the coefficients $a_n$ are zero.
\end{enumerate}
So, if $f$ converges somewhere but not everywhere in $\C$, then
the domain of its convergence is the region $\Re(z)>\rho$ for
some real number $\rho$, which is called the \emph{abscissa of convergence}
of the Dirichlet series.
The abscissa of convergence of the series
$f(z)=\sum_n |a_n| e^{-\lambda_nz}$, if it exists,
is called the \emph{abscissa of absolute convergence} of $f$.

Now suppose that the coefficients $a_n$ are all real and nonnegative.
If the series $f$ converges for $\Re(z)>\rho$, and the resulting function
admits an \PMlinkname{analytic extension}{AnalyticContinuation} to a neighbourhood of $\rho$,
then the series $f$ converges in a neighbourhood of $\rho$.
Consequently, the domain of convergence of $f$ (unless it is the whole
of $\C$) is bounded by a singularity at a point on the real axis.

Finally, return to the general case of any complex numbers $(a_n)$, but
suppose $\lambda_n=\log n$, so $f$ is an ordinary Dirichlet series
$\sum\frac{a_n}{n^z}$.
\begin{enumerate}
\item
If the sequence $(a_n)$ is bounded, then $f$ converges absolutely in the
region $\Re(z)>1$.
\item
If the partial sums $\sum_{n=k}^l a_n$ are bounded, then $f$ converges
(not necessarily absolutely) in the region $\Re(z)>0$.
\end{enumerate}

\begin{thebibliography}{1}

\bibitem{cite:serre_arithm_course_vi}
{Jean-Pierre} Serre.
\newblock {\em A Course in Arithmetic}, chapter~VI.
\newblock Springer-Verlag, 1973.
\newblock \PMlinkexternal{Zbl 0256.12001}{http://www.emis.de/cgi-bin/zmen/ZMATH/en/quick.html?type=html&an=0256.12001}.

\bibitem{cite:tichmarsh_theory_of_func}
E.~C. Titchmarsh.
\newblock {\em The Theory of Functions}.
\newblock Oxford Univ. Press, second edition, 1958.
\newblock \PMlinkexternal{Zbl 0336.30001}{http://www.emis.de/cgi-bin/zmen/ZMATH/en/quick.html?type=html&an=0336.30001}.

\end{thebibliography}
%%%%%
%%%%%
\end{document}
