\documentclass[12pt]{article}
\usepackage{pmmeta}
\pmcanonicalname{BlaschkeProduct}
\pmcreated{2013-03-22 14:19:35}
\pmmodified{2013-03-22 14:19:35}
\pmowner{jirka}{4157}
\pmmodifier{jirka}{4157}
\pmtitle{Blaschke product}
\pmrecord{8}{35795}
\pmprivacy{1}
\pmauthor{jirka}{4157}
\pmtype{Definition}
\pmcomment{trigger rebuild}
\pmclassification{msc}{30C45}
\pmdefines{Blascke factor}

% this is the default PlanetMath preamble.  as your knowledge
% of TeX increases, you will probably want to edit this, but
% it should be fine as is for beginners.

% almost certainly you want these
\usepackage{amssymb}
\usepackage{amsmath}
\usepackage{amsfonts}

% used for TeXing text within eps files
%\usepackage{psfrag}
% need this for including graphics (\includegraphics)
%\usepackage{graphicx}
% for neatly defining theorems and propositions
\usepackage{amsthm}
% making logically defined graphics
%%%\usepackage{xypic}

% there are many more packages, add them here as you need them

% define commands here
\theoremstyle{theorem}
\newtheorem*{thm}{Theorem}
\newtheorem*{lemma}{Lemma}
\newtheorem*{conj}{Conjecture}
\newtheorem*{cor}{Corollary}
\theoremstyle{definition}
\newtheorem*{defn}{Definition}
\begin{document}
\begin{defn}
Suppose that $\{ a_n \}$ is a sequence of complex numbers with $0 < \lvert a_n \rvert < 1$ and $\sum_{n=1}^\infty (1 - \lvert a_n \rvert) < \infty$, then
\begin{equation*}
B(z) := \prod_{n=1}^\infty \frac{\lvert a_n \rvert}{a_n} \left(
\frac{a_n - z}{1-\bar{a}_nz} \right)
\end{equation*}
is called the {\em Blaschke product}.
\end{defn}

This product converges uniformly on compact subsets of the unit disc, and thus $B$ is
a holomorhic function on the unit disc.
Further it is the function on the disc that has zeros exactly at $\{ a_n \}$.
And finally for $z$ in the unit
disc, $\left\lvert B(z) \right\rvert \leq 1$.


\begin{defn}
Sometimes $B_a(z) := \frac{z-a}{1-\bar{a}z}$ is called the {\em Blaschke factor}.
\end{defn}

With this definition, the Blascke product becomes $B(z) = \prod_{n=1}^\infty
\frac{\lvert a_n \rvert}{a_n} B_{a_n}(z)$.

%
% TODO: how is this used for the canonical factorization of a holomorphic
% function f ... inner functions
%

\begin{thebibliography}{9}
\bibitem{Conway:complexI}
John~B. Conway.
{\em \PMlinkescapetext{Functions of One Complex Variable I}}.
Springer-Verlag, New York, New York, 1978.
\bibitem{Krantz:several}
Steven~G.\@ Krantz.
{\em \PMlinkescapetext{Function Theory of Several Complex Variables}},
AMS Chelsea Publishing, Providence, Rhode Island, 1992.
\end{thebibliography}
%%%%%
%%%%%
\end{document}
