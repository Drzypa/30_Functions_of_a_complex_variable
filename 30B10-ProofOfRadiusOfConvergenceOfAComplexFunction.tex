\documentclass[12pt]{article}
\usepackage{pmmeta}
\pmcanonicalname{ProofOfRadiusOfConvergenceOfAComplexFunction}
\pmcreated{2013-03-22 14:40:35}
\pmmodified{2013-03-22 14:40:35}
\pmowner{rspuzio}{6075}
\pmmodifier{rspuzio}{6075}
\pmtitle{proof of radius of convergence of a complex function}
\pmrecord{9}{36279}
\pmprivacy{1}
\pmauthor{rspuzio}{6075}
\pmtype{Proof}
\pmcomment{trigger rebuild}
\pmclassification{msc}{30B10}

\endmetadata

% this is the default PlanetMath preamble.  as your knowledge
% of TeX increases, you will probably want to edit this, but
% it should be fine as is for beginners.

% almost certainly you want these
\usepackage{amssymb}
\usepackage{amsmath}
\usepackage{amsfonts}

% used for TeXing text within eps files
%\usepackage{psfrag}
% need this for including graphics (\includegraphics)
%\usepackage{graphicx}
% for neatly defining theorems and propositions
%\usepackage{amsthm}
% making logically defined graphics
%%%\usepackage{xypic}

% there are many more packages, add them here as you need them

% define commands here
\begin{document}
Without loss of generality, it may be assumed that $z_0 = 0$.

Let $c_n$ denote the coefficient of the $n$-th term in the Taylor series of $f$ about $0$.  Let $r$ be a real number such that $0 < r < R$.  Then $c_n$ may be expressed as an integral using the Cauchy integral formula.
 $$c_n = {1 \over 2 \pi i} \oint_{|z| = r} {f(z) \over z^{n+1}} \, dz = {1 \over 2 \pi r^n} \int_{-\pi}^{+\pi} e^{-n \theta} f(r e^{i \theta}) \, d \theta$$

Since $f$ is analytic, it is also continuous.  Since a continuous function on a compact set is bounded, $|f| < B$ for some constant $B > 0$ on the circle $|z| = r$.  Hence, we have
 $$|c_n| = {1 \over 2 \pi r^n} \left| \int_{-\pi}^{+\pi} e^{-n \theta} f(r e^{i \theta}) \, d \theta \right| \le {1 \over 2 \pi r^n} \int_{-\pi}^{+\pi} | e^{-n \theta} f(r e^{i \theta}) | \, d \theta \le {1 \over 2 \pi r^n} \int_{-\pi}^{+\pi} B d \theta = {B \over r^n}$$

Consequently, $\sqrt[n]{c_n} \le \sqrt[n]{B} / r$.  Since $\lim_{n \to \infty} \sqrt[n]{B} = 1$, the radius of convergence must be greater than or equal to $r$.  Since this is true for all $r < R$, it follows that the radius of convergence is greater than or equal to $R$.
%%%%%
%%%%%
\end{document}
