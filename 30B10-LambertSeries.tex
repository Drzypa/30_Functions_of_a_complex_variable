\documentclass[12pt]{article}
\usepackage{pmmeta}
\pmcanonicalname{LambertSeries}
\pmcreated{2013-03-22 18:46:42}
\pmmodified{2013-03-22 18:46:42}
\pmowner{pahio}{2872}
\pmmodifier{pahio}{2872}
\pmtitle{Lambert series}
\pmrecord{13}{41572}
\pmprivacy{1}
\pmauthor{pahio}{2872}
\pmtype{Example}
\pmcomment{trigger rebuild}
\pmclassification{msc}{30B10}
\pmclassification{msc}{40A05}
\pmrelated{NecessaryConditionOfConvergence}
\pmrelated{CauchysRootTest}
\pmrelated{TauFunction}

\endmetadata

% this is the default PlanetMath preamble.  as your knowledge
% of TeX increases, you will probably want to edit this, but
% it should be fine as is for beginners.

% almost certainly you want these
\usepackage{amssymb}
\usepackage{amsmath}
\usepackage{amsfonts}

% used for TeXing text within eps files
%\usepackage{psfrag}
% need this for including graphics (\includegraphics)
%\usepackage{graphicx}
% for neatly defining theorems and propositions
 \usepackage{amsthm}
% making logically defined graphics
%%%\usepackage{xypic}

% there are many more packages, add them here as you need them

% define commands here

\theoremstyle{definition}
\newtheorem*{thmplain}{Theorem}

\begin{document}
\PMlinkescapeword{terms}

The series
\begin{align}
\sum_{n=1}^\infty\frac{a_nz^n}{1\!-\!z^n} \;=\; \frac{a_1z}{1\!-\!z}+\frac{a_2z^2}{1\!-\!z^2}+\ldots
\end{align}
is called {\em Lambert series}.\, We here consider more closely only the special case
\begin{align}
\sum_{n=1}^\infty\frac{x^n}{1\!-\!x^n} \;=\; \frac{x}{1\!-\!x}+\frac{x^2}{1\!-\!x^2}+\ldots
\end{align}
for the real \PMlinkescapetext{variable} $x$.\\



\textbf{I.\; Convergence}\\

$1^\circ.$\; $x = \pm1$:\; The series is not defined.\\

$2^\circ.$\;  $|x| > 1$:\; We have

\[
\frac{x^n}{1\!-\!x^n} \;=\; \frac{1}{\frac{1}{x^n}\!-\!1} \;\to\; -1 \neq 0 
\quad \mbox{as}\;\; n \to \infty,
\]
whence the series (2) diverges.\\

$3^\circ.$\;  $0 \leqq x < 1$:\; The series with nonnegative terms converges, since
\[
\sqrt[n]{\frac{x^n}{1\!-\!x^n}} \;=\; \frac{x}{\sqrt[n]{1\!-\!x^n}} \;\to\; x < 1
\quad \mbox{as}\;\; n \to \infty.
\]

$4^\circ.$\;  $-1 < x < 0$:\; We get an alternating series with
\[
\left|\frac{x^n}{1\!-\!x^n}\right| \;=\; \frac{|x|^n}{|1\!-\!x^n|} \;\leqq\; \frac{|x|^n}{1\!-\!|x|^n}
\;\leqq\; \frac{|x|^n}{1\!-\!|x|} \;\to\; 0 \quad \mbox{as}\;\; n \to \infty,
\]
and by Leibniz theorem, the series converges.\\

Thus we have the result that the Lambert series (2) converges, \PMlinkescapetext{even} absolutely, when\, $|x| < 1$.\\


\PMlinkescapetext{\textbf{II.\; Power series expansion}}\\

Let\, $|x| < 1$.\, \PMlinkescapetext{Expand} the terms to geometric series:\\
\begin{tabular}{lcrrrrrrrrrrrrrrr}
$\displaystyle\frac{x}{1\!-\!x}$&$=$&$x$&$+$&$x^2$&$+$&$x^3$&$+$&$x^4$&$+$&$x^5$&$+$&$x^6$&$+$&$\ldots$\\
$\displaystyle\frac{x^2}{1\!-\!x^2}$&$=$&$$&$$&$x^2$&$$&$$&$+$&$x^4$&$$&$$&$+$&$x^6$&$+$&$\ldots$\\
$\displaystyle\frac{x^3}{1\!-\!x^3}$&$=$&$$&$$&$$&$$&$x^3$&$$&$$&$$&$$&$+$&$x^6$&$+$&$\ldots$\\
$\displaystyle\frac{x^4}{1\!-\!x^4}$&$=$&$$&$$&$$&$$&$$&$$&$x^4$&$$&$$&$$&$$&$+$&$\ldots$\\
$\displaystyle\frac{x^5}{1\!-\!x^5}$&$=$&$$&$$&$$&$$&$$&$$&$$&$$&$x^5$&$$&$$&$+$&$\ldots$\\
$\displaystyle\frac{x^6}{1\!-\!x^6}$&$=$&$$&$$&$$&$$&$$&$$&$$&$$&$$&$$&$x^6$&$+$&$\ldots$\\
$\displaystyle\;\;\ldots$&$$&$\ldots$&$$&$\ldots$&$$&$\ldots$&$$&$\ldots$&$$&$\ldots$&$$&$\ldots$&$$&$\ldots$\\
\end{tabular}

Those geometric series converge absolutely,
\[
|x^k|+|x^{2k}|+|x^{3k}|+\ldots \;=\; \frac{|x|^k}{1\!-\!|x|^k}
\]
and the series $\displaystyle\sum_{k=1}^\infty\frac{|x|^k}{1\!-\!|x|^k}$ converges.\, Thus we can sum the geometric series by the columns:
\[
\sum_{n=1}^\infty\frac{x^n}{1\!-\!x^n} \;=\; x+2x^2+2x^3+3x^4+2x^5+4x^6+\ldots
\]
Apparently, the coefficient of any $x^k$ in this power series expresses, by how many positive integers the number $k$ is divisible, i.e. the coefficient is given by the divisor function $\tau$.\, So we may write the power series form of the Lambert series as
\[
\sum_{n=1}^\infty\frac{x^n}{1\!-\!x^n} \;=\; \tau(1)x+\tau(2)x^2+\tau(3)x^3+\ldots
\]

%%%%%
%%%%%
\end{document}
