\documentclass[12pt]{article}
\usepackage{pmmeta}
\pmcanonicalname{AnalyticContinuationOfGammaFunction}
\pmcreated{2013-03-22 17:03:07}
\pmmodified{2013-03-22 17:03:07}
\pmowner{pahio}{2872}
\pmmodifier{pahio}{2872}
\pmtitle{analytic continuation of gamma function}
\pmrecord{10}{39342}
\pmprivacy{1}
\pmauthor{pahio}{2872}
\pmtype{Derivation}
\pmcomment{trigger rebuild}
\pmclassification{msc}{30D30}
\pmclassification{msc}{30B40}
\pmclassification{msc}{33B15}
\pmsynonym{residues of gamma function}{AnalyticContinuationOfGammaFunction}
\pmrelated{AnalyticContinuation}
\pmrelated{EmptyProduct}
\pmrelated{ResiduesOfTangentAndCotangent}
\pmrelated{RolfNevanlinna}

\endmetadata

% this is the default PlanetMath preamble.  as your knowledge
% of TeX increases, you will probably want to edit this, but
% it should be fine as is for beginners.

% almost certainly you want these
\usepackage{amssymb}
\usepackage{amsmath}
\usepackage{amsfonts}

% used for TeXing text within eps files
%\usepackage{psfrag}
% need this for including graphics (\includegraphics)
%\usepackage{graphicx}
% for neatly defining theorems and propositions
 \usepackage{amsthm}
% making logically defined graphics
%%%\usepackage{xypic}

% there are many more packages, add them here as you need them

% define commands here
\DeclareMathOperator{\Res}{Res}

\theoremstyle{definition}
\newtheorem*{thmplain}{Theorem}

\begin{document}
\PMlinkescapeword{side} \PMlinkescapeword{sides}

The last \PMlinkescapetext{formula} of the \PMlinkname{parent entry}{GammaFunction} may be expressed as
\begin{align}
 \Gamma(z) \;=\; \frac{\Gamma(z\!+\!n)}{z(z\!+\!1)(z\!+\!2)\cdots(z\!+\!n\!-\!1)}.
\end{align}
According to the standard definition
$$\Gamma(z) \;:=\; \int_0^\infty\!e^{-t}t^{z-1}\,dt,$$
the left hand side of (1) is defined only in the right half-plane\, $\Re{z} > 0$,\, 
whereas the expression $\Gamma(z+n)$ is defined and holomorphic for\, $\Re{z} > -n$\, 
and thus the right hand side of (1) is holomorphic in the half-plane\, $\Re{z} > -n$\, 
except the points
$$0,\,-1,\,-2,\,\ldots,\,-(n\!-\!1)$$
where it has the poles of order 1.\, Because the both sides of (1) are equal for\,  
$\Re{z} > 0$,\, the left side of (1) is the analytic continuation of $\Gamma(z)$ to 
the half-plane\, $\Re{z} > -n$.\, And since the positive integer $n$ can be chosen 
arbitrarily, the Euler's $\Gamma$-function has been defined analytically to the whole 
complex plane.

Accordingly, the gamma function is unambiguous and holomorphic everywhere in $\mathbb{C}$ 
except in the points
\begin{align}
0,\,-1,\,-2,\,-3,\,\ldots
\end{align}
which are poles of order 1 of the function.\, Hence, $\Gamma(z)$ 
is a meromorphic function.

For determining the residue of the function in the points (2), we rewrite the equation (1) as
$$\Gamma(z) \;=\; 
\frac{\Gamma(z\!+\!n\!+\!1)}{z(z\!+\!1)(z\!+\!2)\cdots(z\!+\!n)}.$$
In the point\, $z = -n$\, we have
$$\Gamma(z\!+\!n\!+\!1) \;=\; \Gamma(1) \;=\; 0! \;=\; 1,$$
which implies (see the rule in the entry coefficients of Laurent series) that
$$\Res(\Gamma;\,-n) \;=\; \frac{(-1)^n}{n!}.$$

\begin{thebibliography}{9}
\bibitem{NP}{\sc R. Nevanlinna \& V. Paatero}: {\em Funktioteoria}.\, Kustannusosakeyhti\"o Otava. Helsinki (1963).
\end{thebibliography}

%%%%%
%%%%%
\end{document}
