\documentclass[12pt]{article}
\usepackage{pmmeta}
\pmcanonicalname{ProofOfAdditionFormulaOfExp}
\pmcreated{2013-03-22 16:32:03}
\pmmodified{2013-03-22 16:32:03}
\pmowner{pahio}{2872}
\pmmodifier{pahio}{2872}
\pmtitle{proof of addition formula of exp}
\pmrecord{4}{38715}
\pmprivacy{1}
\pmauthor{pahio}{2872}
\pmtype{Proof}
\pmcomment{trigger rebuild}
\pmclassification{msc}{30D20}
\pmrelated{AdditionFormula}
\pmrelated{AdditionFormulas}
\pmdefines{addition formula of exponential function}

\endmetadata

% this is the default PlanetMath preamble.  as your knowledge
% of TeX increases, you will probably want to edit this, but
% it should be fine as is for beginners.

% almost certainly you want these
\usepackage{amssymb}
\usepackage{amsmath}
\usepackage{amsfonts}

% used for TeXing text within eps files
%\usepackage{psfrag}
% need this for including graphics (\includegraphics)
%\usepackage{graphicx}
% for neatly defining theorems and propositions
 \usepackage{amsthm}
% making logically defined graphics
%%%\usepackage{xypic}

% there are many more packages, add them here as you need them

% define commands here

\theoremstyle{definition}
\newtheorem*{thmplain}{Theorem}

\begin{document}
\PMlinkescapeword{constant} \PMlinkescapeword{product}

The addition formula 
$$e^{z_1+z_2} = e^{z_1}e^{z_2}$$
of the complex exponential function may be proven by applying Cauchy multiplication rule to the \PMlinkname{Taylor series expansions}{TaylorSeries} of the right side \PMlinkname{factors}{Product}.\, We present a proof which is based on the derivative of the exponential function.

Let $a$ be a complex constant.\, Denote\, $e^z = w(z)$.\, Then\, 
$w'(z) \equiv w(z)$.\, Using the product rule and the chain rule we calculate:
 $$\frac{d}{dz_1}[w(z_1)w(a-z_1)] = w'(z_1)w(a-z_1)+w(z_1)w'(a-z_1)(-1) = 
   e^{z_1}e^{a-z_1}-e^{z_1}e^{a-z_1} \equiv 0$$
Thus we see that the product\, $w(z_1)w(a-z_1) = e^{z_1}e^{a-z_1}$\, must be a constant $A$.\, If we choose specially\, $z_1 = 0$,\, we obtain:
   $$A = w(0)w(a-0) = e^0e^a = e^a$$
Therefore\, 
   $$e^{z_1}e^{a-z_1} \equiv e^a.$$
If we denote\, $a-z_1 := z_2$, the preceding equation reads\, $e^{z_1}e^{z_2} = e^{z_1+z_2}$. Q.E.D.




%%%%%
%%%%%
\end{document}
