\documentclass[12pt]{article}
\usepackage{pmmeta}
\pmcanonicalname{ProofToCauchyRiemannEquationspolarCoordinates}
\pmcreated{2013-03-22 14:06:13}
\pmmodified{2013-03-22 14:06:13}
\pmowner{Daume}{40}
\pmmodifier{Daume}{40}
\pmtitle{proof to Cauchy-Riemann equations (polar coordinates)}
\pmrecord{5}{35504}
\pmprivacy{1}
\pmauthor{Daume}{40}
\pmtype{Proof}
\pmcomment{trigger rebuild}
\pmclassification{msc}{30E99}

% this is the default PlanetMath preamble.  as your knowledge
% of TeX increases, you will probably want to edit this, but
% it should be fine as is for beginners.

% almost certainly you want these
\usepackage{amssymb}
\usepackage{amsmath}
\usepackage{amsfonts}

% used for TeXing text within eps files
%\usepackage{psfrag}
% need this for including graphics (\includegraphics)
%\usepackage{graphicx}
% for neatly defining theorems and propositions
%\usepackage{amsthm}
% making logically defined graphics
%%%\usepackage{xypic} 

% there are many more packages, add them here as you need them

% define commands here
\begin{document}
If $f(z)$ is differentialble at $z_0$ then the following limit
\begin{eqnarray*}
f'(z_0) & = & \lim_{\xi \to 0} \frac{f(z_0+\xi)-f(z_0)}{\xi}
\end{eqnarray*}
will remain the same approaching from any direction.  First we fix $\theta$ as $\theta_0$ then we take the limit along the ray where the argument is equal to $\theta_0$. Then
\begin{eqnarray*}
f'(z_0) & = &  \lim_{h \to 0} \frac{f(r_0e^{i\theta_0} + he^{i\theta_0})-f(r_0e^{i\theta_0})}{he^{i\theta_0}} \\
& = & \lim_{h\to 0} \frac{f((r_0+h)e^{i\theta_0})-f(r_0e^{i\theta_0})}{he^{i\theta_0}} \\
& = & \lim_{h\to 0} \frac{u(r_0+h,\theta_0) + iv(r_0+h,\theta_0) - u(r_0,\theta_0) - iv(r_0,\theta_0)}{he^{i\theta_0}}\\
& = & \frac{1}{e^{i\theta_0}} \Bigg[ \lim_{h\to 0} \frac{u(r_0+h,\theta_0) - u(r_0,\theta_0)}{h} + i \lim_{h\to 0} \frac{v(r_0+h,\theta_0) - v(r_0,\theta_0)}{h} \Bigg]\\
& = & \frac{1}{e^{i\theta_0}}\Bigg[ \frac{\partial u}{\partial r}(r_0,\theta_0) + i\frac{\partial v}{\partial r}(r_0,\theta_0) \Bigg]
\end{eqnarray*}

Similarly, if we take the limit along the circle with fixed $r$ equals $r_0$.  Then

\begin{eqnarray*}
f'(z_0) & = & \lim_{h\to 0} \frac{f(r_0e^{i\theta_0} + r_0e^{i(\theta_0+h)})-f(r_0e^{i\theta_0})}{r_0e^{i\theta_0}(e^{ih}-1)}\\
& = & \lim_{h\to 0} \frac{f(r_0e^{i(\theta_0+h)})-f(r_0e^{i\theta_0})}{he^{i\theta_0}}\\
& = & \lim_{h\to 0} \frac{u(r_0,\theta_0+h) + iv(r_0,\theta_0+h) -u(r_0,\theta_0) - iv(r_0,\theta_0)}{he^{i\theta_0}}\\
& = & \frac{1}{r_0e^{i\theta_0}} \Bigg[ \lim_{h\to 0} \frac{u(r_0+h,\theta_0) - u(r_0,\theta_0)}{h}\cdot \frac{h}{e^{ih}-1} + i \lim_{h\to 0} \frac{v(r_0+h,\theta_0) - v(r_0,\theta_0)}{h}\cdot \frac{h}{e^{ih}-1} \Bigg]\\
& = & \frac{1}{r_0e^{i\theta_0}} \Bigg[ \lim_{h\to 0} \frac{u(r_0+h,\theta_0) - u(r_0,\theta_0)}{h}\cdot \lim_{h\to 0}  \frac{h}{e^{ih}-1} + i \lim_{h\to 0} \frac{v(r_0+h,\theta_0) - v(r_0,\theta_0)}{h}\cdot \lim_{h\to 0} \frac{h}{e^{ih}-1} \Bigg]\\
& = & \frac{1}{r_0e^{i\theta_0}} \Bigg[ \frac{\partial u}{\partial \theta}(r_0,\theta_0)\frac{1}{i} + \frac{\partial v}{\partial \theta}(r_0,\theta_0) \Bigg]\\
& = & \frac{1}{r_0e^{i\theta_0}} \Bigg[ \frac{\partial v}{\partial \theta}(r_0,\theta_0) - i\frac{\partial u}{\partial \theta}(r_0,\theta_0) \Bigg]
\end{eqnarray*}

Note: We use l'H\^opital's rule to obtain the following result used above $\lim_{h\to 0} \frac{h}{e^{ih}-1} =  \frac{1}{i}$.

Now, since the limit is the same along the circle and the ray then they are equal:
\begin{eqnarray*}
\frac{1}{e^{i\theta_0}}\Bigg[ \frac{\partial u}{\partial r}(r_0,\theta_0) + i\frac{\partial v}{\partial r}(r_0,\theta_0) \Bigg]
& = & \frac{1}{r_0e^{i\theta_0}} \Bigg[ \frac{\partial v}{\partial \theta}(r_0,\theta_0) - i\frac{\partial u}{\partial \theta}(r_0,\theta_0) \Bigg]\\
\Bigg[ \frac{\partial u}{\partial r}(r_0,\theta_0) + i\frac{\partial v}{\partial r}(r_0,\theta_0) \Bigg]
& = & \frac{1}{r_0} \Bigg[ \frac{\partial v}{\partial \theta}(r_0,\theta_0) - i\frac{\partial u}{\partial \theta}(r_0,\theta_0) \Bigg]\\
\end{eqnarray*}
which implies that
\begin{eqnarray*}
\frac{\partial u}{\partial r} & = & \frac{1}{r}\frac{\partial v}{\partial \theta}\\
\frac{\partial v}{\partial r} & = & -\frac{1}{r}\frac{\partial u}{\partial \theta}
\end{eqnarray*}
QED
%%%%%
%%%%%
\end{document}
