\documentclass[12pt]{article}
\usepackage{pmmeta}
\pmcanonicalname{PlacesOfHolomorphicFunction}
\pmcreated{2013-03-22 18:54:18}
\pmmodified{2013-03-22 18:54:18}
\pmowner{pahio}{2872}
\pmmodifier{pahio}{2872}
\pmtitle{places of holomorphic function}
\pmrecord{8}{41753}
\pmprivacy{1}
\pmauthor{pahio}{2872}
\pmtype{Corollary}
\pmcomment{trigger rebuild}
\pmclassification{msc}{30A99}
\pmrelated{ZerosAndPolesOfRationalFunction}
\pmrelated{IdentityTheorem}
\pmrelated{TopologyOfTheComplexPlane}

\endmetadata

% this is the default PlanetMath preamble.  as your knowledge
% of TeX increases, you will probably want to edit this, but
% it should be fine as is for beginners.

% almost certainly you want these
\usepackage{amssymb}
\usepackage{amsmath}
\usepackage{amsfonts}

% used for TeXing text within eps files
%\usepackage{psfrag}
% need this for including graphics (\includegraphics)
%\usepackage{graphicx}
% for neatly defining theorems and propositions
 \usepackage{amsthm}
% making logically defined graphics
%%%\usepackage{xypic}

% there are many more packages, add them here as you need them

% define commands here

\theoremstyle{definition}
\newtheorem*{thmplain}{Theorem}

\begin{document}
If $c$ is a complex constant and $f$ a holomorphic function in a domain $D$ of $\mathbb{C}$, then $f$ has in every compact (\PMlinkname{closed}{TopologyOfTheComplexPlane} and \PMlinkname{bounded}{Bounded}) subdomain of $D$ at most a finite set of \PMlinkid{$c$-places}{9084}, i.e. the points $z$ where\, $f(z) = c$,\, except when\, $f(z) \equiv c$\, in the whole $D$.\\

\emph{Proof.}\, Let $A$ be a \PMlinkescapetext{closed and bounded} subdomain of $D$.\, Suppose that there is an infinite amount of $c$-places of $f$ in $A$.\, By \PMlinkid{Bolzano--Weierstrass theorem}{2125}, these $c$-places have an accumulation point $z_0$, which belongs to the closed set $A$.\, Define the constant function $g$ such that
$$g(z) \;=\; c$$
for all $z$ in $D$.\, Then $g$ is holomorphic in the domain $D$ and\, $g(z) = c$\, in an infinite subset of $D$ with the accumulation point $z_0$.\, Thus in the $c$-places of $f$ we have
$$g(z) \;=\; f(z).$$
Consequently, the identity theorem of holomorphic functions implies that
$$f(z) \;=\; g(z) \;=\; c$$
in the whole $D$.\, Q.E.D.
%%%%%
%%%%%
\end{document}
