\documentclass[12pt]{article}
\usepackage{pmmeta}
\pmcanonicalname{ProofOfRuffasFormulaForContinuousFunctions}
\pmcreated{2013-03-22 14:56:41}
\pmmodified{2013-03-22 14:56:41}
\pmowner{rspuzio}{6075}
\pmmodifier{rspuzio}{6075}
\pmtitle{proof of Ruffa's formula for continuous functions}
\pmrecord{8}{36636}
\pmprivacy{1}
\pmauthor{rspuzio}{6075}
\pmtype{Proof}
\pmcomment{trigger rebuild}
\pmclassification{msc}{30B99}
\pmclassification{msc}{26B15}
\pmclassification{msc}{78A45}

\endmetadata

% this is the default PlanetMath preamble.  as your knowledge
% of TeX increases, you will probably want to edit this, but
% it should be fine as is for beginners.

% almost certainly you want these
\usepackage{amssymb}
\usepackage{amsmath}
\usepackage{amsfonts}

% used for TeXing text within eps files
%\usepackage{psfrag}
% need this for including graphics (\includegraphics)
%\usepackage{graphicx}
% for neatly defining theorems and propositions
%\usepackage{amsthm}
% making logically defined graphics
%%%\usepackage{xypic}

% there are many more packages, add them here as you need them

% define commands here
\begin{document}
Define $s_n$ to be the following sum:
 $$s_n = \sum\limits_{m = 1}^{2^n - 1} 2^{ - n} f\left( {a + m(b - a)/2^n } \right)$$
Making the substitution $m' = 2m$ and using the fact that $1 + (-1)^{m'} = 0$ when $m'$ is odd to express the sum over even values of $m'$ as a sum over all values of $m'$, this becomes
 $$s_n = \sum\limits_{m = 1}^{2^{n+1} - 1} (1 + (-1)^{m'}) 2^{- 1 - n} f\left( {a + m(b - a)/2^n } \right)$$
Subtracting this sum from $s_{n+1}$ and simplifying gives
 $$s_{n+1} - s_n = \sum\limits_{m = 1}^{2^{n+1} - 1} (-1)^{m + 1} 2^{- n} f\left( {a + m(b - a)/2^n } \right)$$
Using the telescoping sum trick, we may write
 $$s_k = \sum_{n=1}^k (s_n - s_{n-1}) = \sum\limits_{n = 1}^k {\sum\limits_{m = 1}^{2^n - 1} {\left( { - 1} \right)^{m + 1} } } 2^{ - n} f\left( {a + m(b - a)/2^n } \right)$$

To complete the proof, we must investigate the limit as $k \to \infty$.  Since $f$ is assumed continuous and the interval $[a,b]$ is compact, $f$ is uniformly continuous.  This means that, for every $\epsilon > 0$, there exists a $\delta > 0$ such that $|x-y| < \delta$ implies $|f(x) - f(y)| < \epsilon$.  By the Archimedean property, there exists an integer $k > 0$ such that $2^k \delta > |a-b|$.  Hence, $|f(x) - f\left( {a + m(b - a)/2^n } \right)| \le \epsilon$ when $x$ lies in the interval $[a + (m-1) (b - a)/2^n, a + (m+1) (b - a)/2^n]$.  Thus, $(a - b) s_k + |a - b| \epsilon$ is a Darboux upper sum for the integral 
 $$\int_a^b f(x) \, dx$$
and $(b - a) s_k - |a - b| \epsilon$ is a Darboux lower sum.  (Darboux's definition of the integral may be thought of as a modern incarnation of the ancient method of exhaustion.)  Hence
 $$|\int_a^b f(x) \, dx - s_k| \le |a-b| \epsilon$$
Taking the limit $\epsilon \to 0$, we see that
\[
\int\limits_a^b {f(x)dx = \sum\limits_{n = 1}^\infty {A_n } = \left( {b - a} \right)} \sum\limits_{n = 1}^\infty {\sum\limits_{m = 1}^{2^n - 1} {\left( { - 1} \right)^{m + 1} } } 2^{ - n} f\left( {a + m(b - a)/2^n } \right)
\]
%%%%%
%%%%%
\end{document}
