\documentclass[12pt]{article}
\usepackage{pmmeta}
\pmcanonicalname{ProofOfRiemannsRemovableSingularityTheorem}
\pmcreated{2013-03-22 13:33:03}
\pmmodified{2013-03-22 13:33:03}
\pmowner{pbruin}{1001}
\pmmodifier{pbruin}{1001}
\pmtitle{proof of Riemann's removable singularity theorem}
\pmrecord{5}{34153}
\pmprivacy{1}
\pmauthor{pbruin}{1001}
\pmtype{Proof}
\pmcomment{trigger rebuild}
\pmclassification{msc}{30D30}

\endmetadata

% this is the default PlanetMath preamble.  as your knowledge
% of TeX increases, you will probably want to edit this, but
% it should be fine as is for beginners.

% almost certainly you want these
\usepackage{amssymb}
\usepackage{amsmath}
\usepackage{amsfonts}

% used for TeXing text within eps files
%\usepackage{psfrag}
% need this for including graphics (\includegraphics)
%\usepackage{graphicx}
% for neatly defining theorems and propositions
%\usepackage{amsthm}
% making logically defined graphics
%%%\usepackage{xypic}

% there are many more packages, add them here as you need them

% define commands here
\begin{document}
Suppose that $f$ is holomorphic on $U\setminus\{a\}$ and $\lim_{z\to a}(z-a)f(z)=0$.  Let
$$
f(z)=\sum_{k=-\infty}^{\infty}c_k (z-a)^k
$$
be the Laurent series of $f$ centered at $a$.  We will show that $c_k=0$ for $k<0$, so that $f$ can be holomorphically extended to all of $U$ by defining $f(a)=c_0$.

For any non-negative integer $n$, the residue of $(z-a)^n f(z)$ at $a$ is
$$
\operatorname{Res}((z-a)^n f(z),a)=\frac{1}{2\pi i}
\lim_{\delta\to 0^+}\oint_{|z-a|=\delta}(z-a)^n f(z)\mathrm{d}z.
$$
This is equal to zero, because
\begin{eqnarray*}
\left|\oint_{|z-a|=\delta}(z-a)^n f(z)\mathrm{d}z\right|
&\le&2\pi\delta\max_{|z-a|=\delta}|(z-a)^n f(z)|\\
&=&2\pi\delta^n\max_{|z-a|=\delta}|(z-a)f(z)|
\end{eqnarray*}
which, by our assumption, goes to zero as $\delta\to 0$.  Since the residue of $(z-a)^n f(z)$ at $a$ is also equal to $c_{-n-1}$, the coefficients of all negative powers of $z$ in the Laurent series vanish.

Conversely, if $a$ is a removable singularity of $f$, then $f$ can be expanded in a power series centered at $a$, so that
$$
\lim_{z\to a}(z-a)f(z)=0
$$
because the constant term in the power series of $(z-a)f(z)$ is zero.

A corollary of this theorem is the following: if $f$ is bounded near $a$, then
$$
|(z-a)f(z)|\le|z-a|M
$$
for some $M>0$.  This implies that $(z-a)f(z)\to 0$ as $z\to a$, so $a$ is a removable singularity of $f$.
%%%%%
%%%%%
\end{document}
