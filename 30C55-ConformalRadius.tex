\documentclass[12pt]{article}
\usepackage{pmmeta}
\pmcanonicalname{ConformalRadius}
\pmcreated{2013-03-22 14:18:33}
\pmmodified{2013-03-22 14:18:33}
\pmowner{jirka}{4157}
\pmmodifier{jirka}{4157}
\pmtitle{conformal radius}
\pmrecord{9}{35771}
\pmprivacy{1}
\pmauthor{jirka}{4157}
\pmtype{Definition}
\pmcomment{trigger rebuild}
\pmclassification{msc}{30C55}
\pmrelated{RiemannMappingTheorem}

\endmetadata

% this is the default PlanetMath preamble.  as your knowledge
% of TeX increases, you will probably want to edit this, but
% it should be fine as is for beginners.

% almost certainly you want these
\usepackage{amssymb}
\usepackage{amsmath}
\usepackage{amsfonts}

% used for TeXing text within eps files
%\usepackage{psfrag}
% need this for including graphics (\includegraphics)
%\usepackage{graphicx}
% for neatly defining theorems and propositions
\usepackage{amsthm}
% making logically defined graphics
%%%\usepackage{xypic}

% there are many more packages, add them here as you need them

% define commands here
\theoremstyle{theorem}
\newtheorem*{thm}{Theorem}
\newtheorem*{lemma}{Lemma}
\newtheorem*{conj}{Conjecture}
\newtheorem*{cor}{Corollary}
\newtheorem*{example}{Example}
\theoremstyle{definition}
\newtheorem*{defn}{Definition}
\begin{document}
\begin{defn}
Let $G \subset {\mathbb{C}}$ be a simply connected region that is not the whole
plane and let $a \in G$ be any point.
The Riemann mapping theorem tells us that there exists
a unique one-to-one and onto holomorphic map $f \colon {\mathbb{D}} \to G$ (where
${\mathbb{D}}$ is the unit disc) such that $f(0) = a$ and $f'(0) > 0$.  Then
define the {\em conformal radius} $r(G,a) = f'(0)$.
\end{defn}

\begin{example}
For example, take $G = B(0,\delta)$ (the open ball of radius $\delta$ around 0) for some $\delta > 0$, then $r(G,0) = \delta$ because
we have a map $f(z) = \delta \cdot z$ as our unique map.
And thus this definition
coincides with our \PMlinkescapetext{normal} definition of radius for this special case.
\end{example}

\begin{example}
For another example we look at how the conformal radius is affected by
the choice of the point $a$.  So suppose that we take $G$ to be the unit disc
(${\mathbb{D}}$)
itself and we take some point $a \in {\mathbb{D}}$.
The unique map that takes 0 to $a$ is
the map $f(z) = \frac{z+a}{1+\bar{a}z}$ (where $\bar{a}$ is the complex conjugate of $a$) and by the quotient rule we get that
$f'(z) = \frac{1-\lvert a \rvert^2}{(1+\bar{a}z)^2}$.  And so
$r({\mathbb{D}},a) = f'(0) = 1-\lvert a \rvert^2$, so the conformal radius
of the unit disc goes to 0 as we move the point $a$ towards the boundary
of the disc, and
it is largest (equal to 1) when $a = 0$.
\end{example}

From the first example we can now see another way of characterizing the conformal
radius.  Take the inverse map (inverses of holomorphic one-to-one functions are also always holomorphic) and call it $\varphi \colon 
G \to {\mathbb{D}}$ (the map such that $\varphi (f(z)) = z$).
We take the derivative (see the entry on \PMlinkname{univalent functions}{UnivalentFunction}) we get
$\varphi'(f(0)) = \frac{1}{f'(0)}$, that is
$\varphi'(a) = \frac{1}{r}$ (where we call $r = r(G,a)$ for brevity now).
If we multiply the map by
the conformal radius we get a map $\gamma\colon G \to B(0,r)$ such that
$\gamma (z) = r \cdot \varphi(z)$ and $\gamma'(a) = 1$.  By uniqueness of the
map arising from the Riemann mapping theorem we can see that $\gamma$ is
also unique.  Thus we could define the conformal radius as follows.

\begin{defn}
Let $G \subset {\mathbb{C}}$ be a region and let $a \in G$ be any
point.  By application of Riemann mapping
theorem there exists a unique map $\gamma \colon G \to B(0,r)$ for
some $r > 0$, such that $\gamma(a) = 0$ and $\gamma'(a) = 1$.  The
{\em conformal radius} is then defined as $r(G,a) = r$. 
\end{defn}

This definition gives more of an intuitive understanding of why we'd call this the conformal radius of $G$.  We look at the unique map with $\gamma'(a) = 1$, that is, the map that doesn't ``stretch'' the set.  So the radius of $G$
with respect to $a$ is really the radius of the unique ball around zero to which $G$ is conformally equivalent without any ``stretching'' needed.

\begin{thebibliography}{9}
\bibitem{ref1}
S. Rohde, M. Zinsmeister.  \PMlinkescapetext{Variation of the conformal radius}, \emph{Journal d'Analyse} (to appear).
Available at
\PMlinkexternal{http://www.math.washington.edu/~rohde/papers/rozi.ps}{http://www.math.washington.edu/~rohde/papers/rozi.ps}
\end{thebibliography}
%%%%%
%%%%%
\end{document}
