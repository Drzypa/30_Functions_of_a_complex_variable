\documentclass[12pt]{article}
\usepackage{pmmeta}
\pmcanonicalname{ProofOfGoursatsTheorem}
\pmcreated{2013-03-22 12:54:37}
\pmmodified{2013-03-22 12:54:37}
\pmowner{rmilson}{146}
\pmmodifier{rmilson}{146}
\pmtitle{proof of Goursat's theorem}
\pmrecord{13}{33261}
\pmprivacy{1}
\pmauthor{rmilson}{146}
\pmtype{Proof}
\pmcomment{trigger rebuild}
\pmclassification{msc}{30E20}

\endmetadata

\usepackage{graphicx}
\usepackage{amsmath}
\usepackage{amsfonts}
\usepackage{amssymb}
\newcommand{\reals}{\mathbb{R}}
\newcommand{\natnums}{\mathbb{N}}
\newcommand{\cnums}{\mathbb{C}}
\newcommand{\znums}{\mathbb{Z}}
\newcommand{\lp}{\left(}
\newcommand{\rp}{\right)}
\newcommand{\lb}{\left[}
\newcommand{\rb}{\right]}
\newcommand{\supth}{^{\text{th}}}
\newtheorem{proposition}{Proposition}
\newtheorem{definition}[proposition]{Definition}
\newcommand{\nl}[1]{{\PMlinkescapetext{{#1}}}}
\newcommand{\pln}[2]{{\PMlinkname{{#1}}{#2}}}
\newtheorem{lemma}[proposition]{Lemma}
\begin{document}
We argue by contradiction.  Set
$$\eta = \oint_{\partial R} f(z)\, dz,$$
and suppose that $\eta\neq
0$. Divide $R$ into four congruent rectangles $R_1, R_2, R_3, R_4$
(see Figure 1), and set
$$\eta_i = \oint_{\partial R_i} f(z)\, dz.$$

  \begin{center}
    \includegraphics[scale=.7]{ProofGoursatTheorem.eps}

    {\tiny Figure 1: subdivision of the rectangle contour.}
  \end{center}
  
  Now subdivide each of the four sub-rectangles, to get 16 congruent
  sub-sub-rectangles $R_{i_1i_2},\; i_1,i_2=1\ldots 4$, and then continue ad
  infinitum to obtain a sequence of nested families of rectangles
  $R_{i_1\ldots i_k}$, with $\eta_{i_1\ldots i_k}$ the values of $f(z)$
  integrated along the corresponding contour.

Orienting the boundary of
$R$ and all the sub-rectangles in the usual counter-clockwise fashion
we have
$$\eta = \eta_1+\eta_2+\eta_3+\eta_4,$$
and more generally
$$\eta_{i_1\ldots i_k} = \eta_{i_1\ldots i_k1}+\eta_{i_1\ldots i_k2}+\eta_{i_1\ldots i_k3}+\eta_{i_1\ldots i_k4}.$$
In as much as the integrals along oppositely oriented line
segments cancel, the contributions from the interior segments cancel,
and that is why the right-hand side reduces to the integrals along the
segments at the boundary of the composite rectangle.

Let $j_1\in\{1,2,3,4\}$ be such that $\vert\eta_{j_1}\vert$ is the maximum of
$\vert \eta_i \vert,\, i=1,\ldots,4$. By the triangle inequality we have
$$\vert\eta_1\vert+\vert\eta_2\vert+\vert\eta_3\vert+\vert\eta_4\vert \geq \vert\eta\vert,$$
and hence
$$\vert\eta_{j_1}\vert\geq 1/4 \vert\eta\vert.$$
Continuing inductively, let $j_{k+1}$ be such that 
$\vert\eta_{j_1\ldots j_k j_{k+1}}\vert$ is the maximum of
$\vert\eta_{j_1 \ldots j_k i}\vert,\, i=1,\ldots,4$.  We then have
\begin{equation}
  \label{eq:etaest}
\vert\eta_{j_1\ldots j_k j_{k+1}}\vert \geq 4^{-(k+1)} |\eta|.  
\end{equation}

Now the sequence of nested rectangles $R_{j_1 \ldots j_k}$
converges to some point $z_0\in R$; more formally
$$\{z_0\} = \bigcap_{k=1}^\infty R_{j_1\ldots j_k}.$$
The derivative
$f'(z_0)$ is assumed to exist, and hence for every $\epsilon>0$ there
exists a $k$ sufficiently large, so that for all $z\in R_{j_1\ldots j_k}$
we have
$$\vert f(z)-f'(z_0)(z-z_0) \vert \leq \epsilon \vert z-z_0\vert.$$
Now we make use of the following.
\begin{lemma}
  Let $Q\subset\cnums$ be a rectangle, let $a,b\in\cnums$, and let
  $f(z)$ be a continuous, complex valued function defined and bounded
  in a domain containing $Q$.  Then,
  \begin{eqnarray*}
    &&\oint_{\partial Q} (az+b) dz = 0 \\
    &&\left\vert \oint_{\partial Q} \!\!\!f(z) \right \vert \leq MP,
  \end{eqnarray*}
  where $M$ is an upper bound for $|f(z)|$ and where $P$ is the length
  of $\partial Q$.
\end{lemma}
The first of these assertions follows by the Fundamental Theorem of
Calculus; after all the function $az+b$ has an anti-derivative.  The
second assertion follows from the fact that the absolute value of an
integral is smaller than the integral of the absolute value of the
integrand --- a standard result in integration theory.

Using the Lemma and the fact that the perimeter of a rectangle is
greater than its diameter we infer that for every $\epsilon>0$ there
exists a $k$ sufficiently large that 
$$\eta_{j_1\ldots j_k} = \left\vert \oint_{\partial R_{j_1\ldots
      j_k}}\hskip -2em f(z)\, dz \right\vert \leq \epsilon
\vert\partial R_{j_1\ldots j_k}\vert^2 =  4^{-k}\, \vert
\partial R\vert^2 \epsilon.$$
where $\vert\partial R\vert$ denotes the
length of perimeter of the rectangle $R$.  This contradicts the
earlier estimate \eqref{eq:etaest}.  Therefore 
$\eta=0$.
%%%%%
%%%%%
\end{document}
