\documentclass[12pt]{article}
\usepackage{pmmeta}
\pmcanonicalname{SignumFunction}
\pmcreated{2013-03-22 13:36:41}
\pmmodified{2013-03-22 13:36:41}
\pmowner{yark}{2760}
\pmmodifier{yark}{2760}
\pmtitle{signum function}
\pmrecord{11}{34243}
\pmprivacy{1}
\pmauthor{yark}{2760}
\pmtype{Definition}
\pmcomment{trigger rebuild}
\pmclassification{msc}{30-00}
\pmclassification{msc}{26A06}
\pmrelated{ModulusOfComplexNumber}
\pmrelated{HeavisideStepFunction}
\pmrelated{PlusSign}
\pmrelated{SineIntegralInInfinity}
\pmrelated{ListOfImproperIntegrals}
\pmdefines{Heavyside step function}
\pmdefines{step function}

\usepackage{amssymb}
\usepackage{amsmath}
\usepackage{amsfonts}

\newcommand{\signum}[0]{\mathop{\mathrm{sgn}}}
\newcommand{\R}[0]{\mathbb{R}}

\begin{document}
\PMlinkescapeword{analysis}
\PMlinkescapeword{applications}
\PMlinkescapeword{arguments}
\PMlinkescapeword{calculate}
\PMlinkescapeword{clear}
\PMlinkescapeword{equation}
\PMlinkescapeword{onto}
\PMlinkescapeword{point}
\PMlinkescapeword{points}
\PMlinkescapeword{properties}
\PMlinkescapeword{relation}
\PMlinkescapeword{relations}

The \emph{signum function} is the function $\signum\colon\R\to \R$
 \begin{eqnarray*}
 \signum (x) &=& \left\{ \begin {array}{ll} 
 -1 & \mbox{when}\,\, x<0, \\
  0 & \mbox{when}\,\, x=0,\\
  1 & \mbox{when}\,\, x>0. \\ \end{array} \right.
 \end{eqnarray*}
 
The following properties hold:
 \begin{enumerate}
 \item For all $x\in \R$, $\signum(-x) = -\signum(x).$
 \item For all $x\in \R$, $|x|=\signum(x) x.$
 \item For all $x\neq 0$, $\frac{d}{dx}|x|=\signum(x)$.
 \end{enumerate}
 
Here, we should point out that the signum function
is often defined simply as $1$ for $x>0$ and $-1$ for $x<0$.
Thus, at $x=0$, it is left undefined. See for example \cite{kreyszig93}.
In applications such as the Laplace transform this definition is adequate,
since the value of a function at a single point does not change the analysis.
One could then, in fact, set $\signum(0)$ to any value.
However, setting $\signum(0)=0$ is motivated by the above relations.
On a related note, we can extend the definition to the extended real numbers
$\overline{\mathbb{R}}=\mathbb{R}\cup\{\infty,-\infty\}$
by defining $\signum(\infty)=1$ and $\signum(-\infty)=-1$.
 
A related function is the \emph{Heaviside step function}
 defined as
 \begin{eqnarray*}
 H(x) &=& \left\{ \begin {array}{ll} 0 & \mbox{when}\,\, x< 0, \\
  1/2 & \mbox{when}\,\, x= 0,\\
  1 & \mbox{when}\,\, x> 0.\\
 \end{array} \right.
 \end{eqnarray*}
 Again, this function is sometimes left undefined at $x=0$.
 The motivation for setting $H(0)=1/2$ is that
 for all $x\in\R$, we then have the relations
 \begin{eqnarray*}
 H (x) &=& \frac{1}{2}(\signum(x)+1), \\
 H(-x) &=& 1-H(x).
 \end{eqnarray*}
 This first relation is clear. For the second, we have
 \begin{eqnarray*}
 1-H(x) &=& 1-\frac{1}{2}(\signum(x)+1) \\
  &=& \frac{1}{2}(1-\signum(x)) \\
  &=& \frac{1}{2}(1+\signum(- x)) \\
  &=& H(-x).
 \end{eqnarray*}
 
{\bf Example} Let $a<b$ be real numbers, and let $f:\R\to\R$ be the
 piecewise defined function
 \begin{eqnarray*}
 f (x) &=& \left\{ \begin {array}{ll} 
 4 & \mbox{when}\,\, x\in(a,b), \\
  0 & \mbox{otherwise.} \\
 \end{array} \right.
 \end{eqnarray*}
 Using the Heaviside step function, we can write
 \begin{eqnarray}
 \label{almost}
  f(x) &=& 4\big(H(x-a) - H(x-b)\big)
 \end{eqnarray}
 almost everywhere.
 Indeed, if we calculate $f$ using equation \ref{almost} we obtain
 $f(x)=4$ for $x\in(a,b)$, $f(x)=0$ for $x\notin[a,b]$,
 and $f(a)=f(b)=2$. Therefore, equation \ref{almost}
 holds at all points except $a$ and $b$.
 $\Box$

\section{Signum function for complex arguments}
For a complex number $z$, the signum function is defined as \cite{bachman}
 \begin{eqnarray*}
 \signum (z) &=& \left\{ \begin {array}{ll} 
  0 & \mbox{when}\,\, z=0,\\
  {z}/{|z|} & \mbox{when}\,\, z\neq 0. \\ \end{array} \right.
 \end{eqnarray*}
In other words, if $z$ is non-zero, then $\signum z$ is the projection 
of $z$ onto the unit circle $\{z\in \mathbb{C} \mid |z| = 1\}$.
Clearly, the complex signum function reduces to the real signum function
for real arguments. 
For all $z\in \mathbb{C}$, we have
$$ z \signum \overline{z} = |z|,$$
where $\overline{z}$ is the complex conjugate of $z$. 

\begin{thebibliography}{9}
 \bibitem {kreyszig93} E. Kreyszig,
 \emph{Advanced Engineering Mathematics},
 John Wiley \& Sons, 1993, 7th ed.
 \bibitem{bachman} G. Bachman, L. Narici, 
\emph{Functional analysis}, Academic Press, 1966.
 \end{thebibliography}
%%%%%
%%%%%
\end{document}
