\documentclass[12pt]{article}
\usepackage{pmmeta}
\pmcanonicalname{MeromorphicExtension}
\pmcreated{2013-03-22 16:07:26}
\pmmodified{2013-03-22 16:07:26}
\pmowner{Wkbj79}{1863}
\pmmodifier{Wkbj79}{1863}
\pmtitle{meromorphic extension}
\pmrecord{10}{38193}
\pmprivacy{1}
\pmauthor{Wkbj79}{1863}
\pmtype{Definition}
\pmcomment{trigger rebuild}
\pmclassification{msc}{30D30}
\pmsynonym{meromorphic continuation}{MeromorphicExtension}
\pmrelated{AnalyticContinuationOfRiemannZeta}
\pmrelated{RestrictionOfAFunction}

% this is the default PlanetMath preamble.  as your knowledge
% of TeX increases, you will probably want to edit this, but
% it should be fine as is for beginners.

% almost certainly you want these
\usepackage{amssymb}
\usepackage{amsmath}
\usepackage{amsfonts}

% used for TeXing text within eps files
%\usepackage{psfrag}
% need this for including graphics (\includegraphics)
%\usepackage{graphicx}
% for neatly defining theorems and propositions
%\usepackage{amsthm}
% making logically defined graphics
%%%\usepackage{xypic}

% there are many more packages, add them here as you need them

% define commands here

\begin{document}
Let $A \subset B \subseteq \mathbb{C}$ and $f \colon A \to \mathbb{C}$ be analytic.  A {\em meromorphic extension of $f$} is a meromorphic function $g \colon B \to \mathbb{C}$ such that $g|_A=f$.

The meromorphic extension of an analytic function to a larger \PMlinkname{domain}{Domain} is unique; \PMlinkname{i.e.}{Ie}, using the above notation, if $h \colon B \to \mathbb{C}$ has the property that $h|_A=f$, then $g=h$ on $B$.

Occasionally, an analytic function and its meromorphic extension are denoted using the same notation.  A common example of this phenomenon is the Riemann zeta function.
%%%%%
%%%%%
\end{document}
