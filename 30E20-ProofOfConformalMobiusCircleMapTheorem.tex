\documentclass[12pt]{article}
\usepackage{pmmeta}
\pmcanonicalname{ProofOfConformalMobiusCircleMapTheorem}
\pmcreated{2013-03-22 13:36:45}
\pmmodified{2013-03-22 13:36:45}
\pmowner{brianbirgen}{2180}
\pmmodifier{brianbirgen}{2180}
\pmtitle{proof of conformal M\"obius circle map theorem}
\pmrecord{5}{34244}
\pmprivacy{1}
\pmauthor{brianbirgen}{2180}
\pmtype{Proof}
\pmcomment{trigger rebuild}
\pmclassification{msc}{30E20}
\pmrelated{SchwarzLemma}
\pmrelated{MobiusTransformation}
\pmrelated{AutomorphismsOfUnitDisk}

\endmetadata

% this is the default PlanetMath preamble.  as your knowledge
% of TeX increases, you will probably want to edit this, but
% it should be fine as is for beginners.

% almost certainly you want these
\usepackage{amssymb}
\usepackage{amsmath}
\usepackage{amsfonts}

% used for TeXing text within eps files
%\usepackage{psfrag}
% need this for including graphics (\includegraphics)
%\usepackage{graphicx}
% for neatly defining theorems and propositions
%\usepackage{amsthm}
% making logically defined graphics
%%%\usepackage{xypic}

% there are many more packages, add them here as you need them

% define commands here
\begin{document}
Let $f$ be a conformal map from the unit disk $\Delta$ onto itself. 
Let $a=f(0)$. Let $g_a(z) = \frac{z-a}{1-\overline{a}z}$. 
Then $g_a \circ f$ is a conformal map from $\Delta$ onto itself, 
with $g_a \circ f(0)=0$. 
Therefore, by Schwarz's Lemma for all $z \in \Delta$ $|g_a \circ f(z)| \le |z|$. 

Because $f$ is a conformal map onto $\Delta$, 
$f^{-1}$ is also a conformal map of $\Delta$ onto itself. 
$(g_a \circ f)^{-1}(0)=0$ so that by Schwarz's Lemma
$|(g_a \circ f)^{-1}(w)| \le |w|$ for all $w \in \Delta$. 
Writing $w=g_a \circ f(z)$ this becomes $|z| \le |g_a \circ f(z)|$. 

Therefore, for all $z \in \Delta$ $|g_a \circ f(z)| = |z|$. 
By Schwarz's Lemma, $g_a \circ f$ is a rotation. 
Write $g_a \circ f(z) = e^{i \theta} z$, or $f(z) = e^{i \theta} g_a^{-1}$. 

Therefore, $f$ is a M\"obius Transformation.
%%%%%
%%%%%
\end{document}
