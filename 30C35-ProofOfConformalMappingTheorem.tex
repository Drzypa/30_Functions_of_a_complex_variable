\documentclass[12pt]{article}
\usepackage{pmmeta}
\pmcanonicalname{ProofOfConformalMappingTheorem}
\pmcreated{2013-03-22 13:47:26}
\pmmodified{2013-03-22 13:47:26}
\pmowner{pbruin}{1001}
\pmmodifier{pbruin}{1001}
\pmtitle{proof of conformal mapping theorem}
\pmrecord{7}{34502}
\pmprivacy{1}
\pmauthor{pbruin}{1001}
\pmtype{Proof}
\pmcomment{trigger rebuild}
\pmclassification{msc}{30C35}
%\pmkeywords{analytic function}
%\pmkeywords{conformal mapping}

\endmetadata

% this is the default PlanetMath preamble.  as your knowledge
% of TeX increases, you will probably want to edit this, but
% it should be fine as is for beginners.

% almost certainly you want these
\usepackage{amssymb}
\usepackage{amsmath}
\usepackage{amsfonts}

% used for TeXing text within eps files
%\usepackage{psfrag}
% need this for including graphics (\includegraphics)
%\usepackage{graphicx}
% for neatly defining theorems and propositions
%\usepackage{amsthm}
% making logically defined graphics
%%%\usepackage{xypic}

% there are many more packages, add them here as you need them

% define commands here
\begin{document}
Let $D\subset\mathbb{C}$ be a domain, and let $f\colon D\to\mathbb{C}$ be an
analytic function.  By identifying the complex plane $\mathbb{C}$ with
$\mathbb{R}^2$, we can view $f$ as a function from $\mathbb{R}^2$ to
itself:
$$
\tilde f(x,y):=(\Re f(x+iy), \Im f(x+iy))=(u(x,y),v(x,y))
$$
with $u$ and $v$ real functions.  The Jacobian matrix of $\tilde f$ is
$$
J(x,y)=\frac{\partial(u,v)}{\partial(x,y)}=\begin{pmatrix}
u_x & u_y \\
v_x & v_y
\end{pmatrix}.
$$
As an analytic function, $f$ satisfies the Cauchy-Riemann equations,
so that $u_x=v_y$ and $u_y=-v_x$.  At a fixed point $z=x+iy\in D$, we
can therefore define $a=u_x(x,y)=v_y(x,y)$ and $b=u_y(x,y)=-v_x(x,y)$.
We write $(a,b)$ in polar coordinates as $(r\cos\theta,r\sin\theta)$
and get
$$
J(x,y)=\begin{pmatrix}
a & b \\
-b & a
\end{pmatrix} = r\begin{pmatrix}
\cos\theta & \sin\theta \\
-\sin\theta & \cos\theta
\end{pmatrix}.
$$

Now we consider two smooth curves through $(x,y)$, which we
parametrize by $\gamma_1(t)=(u_1(t),v_1(t))$ and
$\gamma_2(t)=(u_2(t),v_2(t))$.  We can choose the parametrization such
that $\gamma_1(0)=\gamma_2(0)=z$.  The images of these curves under
$\tilde f$ are $\tilde f\circ\gamma_1$ and $\tilde f\circ\gamma_2$,
respectively, and their derivatives at $t=0$ are
$$
(\tilde f\circ\gamma_1)'(0)=
\frac{\partial(u,v)}{\partial(x,y)}(\gamma_1(0))
\cdot \frac{{\rm d}\gamma_1}{{\rm d}t}(0)=
J(x,y)\begin{pmatrix}
\frac{{\rm d}u_1}{{\rm d}t} \\
\frac{{\rm d}v_1}{{\rm d}t}
\end{pmatrix}
$$
and, similarly,
$$
(\tilde f\circ\gamma_2)'(0)=J(x,y)\begin{pmatrix}
\frac{{\rm d}u_2}{{\rm d}t} \\
\frac{{\rm d}v_2}{{\rm d}t}
\end{pmatrix}
$$
by the chain rule.  We see that if $f'(z)\neq 0$, $f$ transforms the
tangent vectors to $\gamma_1$ and $\gamma_2$ at $t=0$ (and therefore
in $z$) by the orthogonal matrix
$$
J/r=\begin{pmatrix}
\cos\theta & \sin\theta \\
-\sin\theta & \cos\theta
\end{pmatrix}
$$
and scales them by a factor of $r$.  In particular, the transformation
by an orthogonal matrix implies that the angle between the tangent
vectors is preserved.  Since the determinant of $J/r$ is 1, the
transformation also preserves orientation (the direction of the angle
between the tangent vectors).  We conclude that $f$ is a conformal
mapping at each point where its derivative is nonzero.
%%%%%
%%%%%
\end{document}
