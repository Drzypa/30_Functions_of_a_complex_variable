\documentclass[12pt]{article}
\usepackage{pmmeta}
\pmcanonicalname{ProofOfWeakMaximumPrincipleForRealDomains}
\pmcreated{2013-03-22 14:35:21}
\pmmodified{2013-03-22 14:35:21}
\pmowner{rspuzio}{6075}
\pmmodifier{rspuzio}{6075}
\pmtitle{proof of weak maximum principle for real domains}
\pmrecord{6}{36152}
\pmprivacy{1}
\pmauthor{rspuzio}{6075}
\pmtype{Proof}
\pmcomment{trigger rebuild}
\pmclassification{msc}{30F15}
\pmclassification{msc}{31B05}
\pmclassification{msc}{31A05}
\pmclassification{msc}{30C80}

\endmetadata

% this is the default PlanetMath preamble.  as your knowledge
% of TeX increases, you will probably want to edit this, but
% it should be fine as is for beginners.

% almost certainly you want these
\usepackage{amssymb}
\usepackage{amsmath}
\usepackage{amsfonts}

% used for TeXing text within eps files
%\usepackage{psfrag}
% need this for including graphics (\includegraphics)
%\usepackage{graphicx}
% for neatly defining theorems and propositions
%\usepackage{amsthm}
% making logically defined graphics
%%%\usepackage{xypic}

% there are many more packages, add them here as you need them

% define commands here
\begin{document}
First, we show that, if $\Delta f > 0$ (where $\Delta$ denotes the Laplacian on $\mathbb{R}^d$) on $K$, then $f$ cannot attain a maximum on the interior of $K$.  Assume, to the contrary, that $f$ did attain a maximum at a point $p$ located on the interior of $K$.  By the second derivative test, the matrix of second partial derivatives of $f$ at $p$ would have to be negative semi-definite.  This would imply that the trace of the matrix is negative.  But the trace of this matrix is the Laplacian, which was assumed to be strictly positive on $K$, so it is impossible for $f$ to attain a maximum on the interior of $K$.

Next, suppose that $\Delta f = 0$ on $K$ but that $f$ does not attain its maximum on the boundary of $K$.  Since $K$ is compact, $f$ must attain its maximum somewhere, and hence there exists a point $p$ located in the interior of $K$ at which $f$ does attain its maximum.  Since $K$ is compact, the boundary of $K$ is also compact, and hence the image of the boundary of $K$ under $f$ is also compact.  Since every element of this image is strictly smaller than $f(p)$, there must exist a constant $C$ such that $f(x) < C < f(p)$ whenever $x$ lies on the boundary of $K$.  Furthermore Since $K$ is a compact subset of $\mathbb{R}^d$, it is bounded.  Hence, there exists a constant $R>0$ so that $|x - p| < R$ for all $x \in K$.

Consider the function $g$ defined as
 \[g(x) = f(x) + (f(p) - C) {|x-p|^2 \over R^2}\]
At any point $x \in K$,
 \[g(x) < f(x) + f(p) - C\]
In particular, if $x$ lies on the boundary of $K$, this implies that
 \[g(x) < f(p)\]
Since $g(p) = f(p)$ this inequality implies that $g$ cannot attain a maximum on the boundary of $K$.

This leads to a contradiction.  Note that, since $\Delta f = 0$ on $K$,
 \[\Delta g = {d (f(p) - C) \over R^2} > 0\]
which implies that $g$ cannot attain a maximum on the interior of $K$.  However, since $K$ is compact, $g$ must attain a maximum somewhere on $K$.  Since we have ruled out both the possibility that this maximum occurs in the interior and the possibility that it occurs on the boundary, we have a contradiction.  The only way out of this contradiction is to conclude that $f$ does attain its maximum on the boundary of $K$.
%%%%%
%%%%%
\end{document}
