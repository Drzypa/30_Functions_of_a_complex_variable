\documentclass[12pt]{article}
\usepackage{pmmeta}
\pmcanonicalname{Residue}
\pmcreated{2013-03-22 12:04:56}
\pmmodified{2013-03-22 12:04:56}
\pmowner{djao}{24}
\pmmodifier{djao}{24}
\pmtitle{residue}
\pmrecord{7}{31153}
\pmprivacy{1}
\pmauthor{djao}{24}
\pmtype{Definition}
\pmcomment{trigger rebuild}
\pmclassification{msc}{30D30}
\pmrelated{CauchyResidueTheorem}

\endmetadata

\usepackage{amssymb}
\usepackage{amsmath}
\usepackage{amsfonts}
\usepackage{graphicx}
%%%\usepackage{xypic}
\begin{document}
Let $U \subset \mathbb{C}$ be a domain and let $f: U \longrightarrow \mathbb{C}$ be a function represented by a Laurent series
$$
f(z) := \sum_{k=-\infty}^\infty c_k (z-a)^k
$$
centered about $a$. The coefficient $c_{-1}$ of the above Laurent series is called the \emph{residue} of $f$ at $a$, and denoted $\operatorname{Res}(f;a)$.
%%%%%
%%%%%
%%%%%
\end{document}
