\documentclass[12pt]{article}
\usepackage{pmmeta}
\pmcanonicalname{MonodromyTheorem}
\pmcreated{2013-03-22 14:44:35}
\pmmodified{2013-03-22 14:44:35}
\pmowner{rspuzio}{6075}
\pmmodifier{rspuzio}{6075}
\pmtitle{monodromy theorem}
\pmrecord{8}{36380}
\pmprivacy{1}
\pmauthor{rspuzio}{6075}
\pmtype{Theorem}
\pmcomment{trigger rebuild}
\pmclassification{msc}{30F99}

% this is the default PlanetMath preamble.  as your knowledge
% of TeX increases, you will probably want to edit this, but
% it should be fine as is for beginners.

% almost certainly you want these
\usepackage{amssymb}
\usepackage{amsmath}
\usepackage{amsfonts}

% used for TeXing text within eps files
%\usepackage{psfrag}
% need this for including graphics (\includegraphics)
%\usepackage{graphicx}
% for neatly defining theorems and propositions
%\usepackage{amsthm}
% making logically defined graphics
%%%\usepackage{xypic}

% there are many more packages, add them here as you need them

% define commands here
\begin{document}
Let $C(t)$ be a one-parameter family of smooth paths in the complex plane with common endpoints $z_0$ and $z_1$. (For definiteness, one may suppose that the parameter $t$ takes values in the interval $[0,1]$.)  Suppose that an analytic function $f$ is defined in a neighborhood of $z_0$ and that it is possible to analytically continue $f$ along every path in the family.  Then the result of analytic continuation does not depend on the choice of path.

Note that it is \emph{crucial} that it be possible to continue $f$ along all paths of the family.  As the following example shows, the result will no longer hold if it is impossible to analytically continue $f$ along even a single path.   Let the family of paths be the set of circular arcs (for the present purpose, the straight line is to be considered as a degenerate case of a circular arc) with endpoints $+1$ and $-1$ and let $f(z) = \sqrt{z}$.  It is possible to analytically continue $f$ along every arc in the family except the line segment passing through $0$.  The conclusion of the theorem does not hold in this case because continuing along arcs which lie above $0$ leads to $f(z_1) = +i$ whilst continuing along arcs which lie below $0$ leads to $f(z_1) = -i$.
%%%%%
%%%%%
\end{document}
