\documentclass[12pt]{article}
\usepackage{pmmeta}
\pmcanonicalname{VariantOfCauchyIntegralFormula}
\pmcreated{2013-03-22 18:54:15}
\pmmodified{2013-03-22 18:54:15}
\pmowner{pahio}{2872}
\pmmodifier{pahio}{2872}
\pmtitle{variant of Cauchy integral formula}
\pmrecord{6}{41752}
\pmprivacy{1}
\pmauthor{pahio}{2872}
\pmtype{Theorem}
\pmcomment{trigger rebuild}
\pmclassification{msc}{30E20}
\pmsynonym{Cauchy integral formula}{VariantOfCauchyIntegralFormula}
\pmrelated{CauchyIntegralFormula}
\pmrelated{CorollaryOfCauchyIntegralTheorem}
\pmrelated{ExampleOfFindingTheGeneratingFunction}
\pmrelated{GeneratingFunctionOfLaguerrePolynomials}
\pmrelated{GeneratingFunctionOfHermitePolynomials}

% this is the default PlanetMath preamble.  as your knowledge
% of TeX increases, you will probably want to edit this, but
% it should be fine as is for beginners.

% almost certainly you want these
\usepackage{amssymb}
\usepackage{amsmath}
\usepackage{amsfonts}

% used for TeXing text within eps files
%\usepackage{psfrag}
% need this for including graphics (\includegraphics)
%\usepackage{graphicx}
% for neatly defining theorems and propositions
 \usepackage{amsthm}
% making logically defined graphics
%%%\usepackage{xypic}

% there are many more packages, add them here as you need them

% define commands here

\theoremstyle{definition}
\newtheorem*{thmplain}{Theorem}

\begin{document}
\textbf{Theorem.}\; Let $f(z)$ be holomorphic in a domain $A$ of $\mathbb{C}$.\, If $C$ is a closed contour not intersecting itself which with its \PMlinkescapetext{inner} domain is contained in $A$ and if $z$ is an arbitrary point inside $C$, then
\begin{align}
f(z) \;=\; \frac{1}{2i\pi}\oint_C\frac{f(t)}{t\!-\!z}\,dt.
\end{align}


\emph{Proof.}\, Let $\varepsilon$ be any positive number.\, Denote by $C_r$ the circles with radius $r$ and centered in $z$.\, We have
$$\oint_C\frac{f(t)}{t\!-\!z}\,dt \;=\; \oint_C\frac{f(z)\!+\!(f(t)\!-\!f(z))}{t\!-\!z}\,dt
\;=\; \underbrace{\oint_C\frac{f(z)}{t\!-\!z}\,dt}_I+\underbrace{\oint_C\frac{f(t)\!-\!f(z)}{t\!-\!z}\,dt}_J.$$
According to the corollary of Cauchy integral theorem and its example, we may write
$$I \;=\; f(z)\oint_C\frac{dt}{t\!-\!z} \;=\; 2i\pi f(z).$$
If\, $0 < r < \mbox{ some } r_0$,\, we have
$$J \;=\; \oint_{C_r}\frac{f(t)\!-\!f(z)}{t\!-\!z}\,dt.$$
The continuity of $f$ in the point $z$ implies, that
$$|f(t)\!-\!f(z)| < \varepsilon$$
when\, $0 < |t\!-\!z| < \mbox{ some } \delta_\varepsilon$\, i.e. when
\begin{align} 
t \in C_r\, \mbox{ and }\, 0 < r < \mbox{ some } r_1.
\end{align}
If (2) is in \PMlinkescapetext{force}, we obtain first
$$\left|\frac{f(t)\!-\!f(z)}{t\!-\!z}\right| \;=\; \frac{|f(t)\!-\!f(z)|}{|t\!-\!z|} 
\;=\; \frac{|f(t)\!-\!f(z)|}{r} \;<\; \frac{\varepsilon}{r},$$
whence, by the estimation theorem of integral,
$$|J| \;\leqq\; \frac{\varepsilon}{r}\cdot2\pi r \;=\; 2\pi\varepsilon \quad 
\mbox{for} \quad 0 < r < \min\{r_0,\,r_1\},$$
and lastly
\begin{align}
\left|\frac{1}{2i\pi}\oint_C\frac{f(t)}{t\!-\!z}\,dt-f(z)\right| \;=\; \left|\frac{1}{2i\pi}J\right| \;\leqq\; \frac{1}{2\pi}\cdot2\pi\varepsilon \;=\; \varepsilon \quad \mbox{when  } 0 < r < \min\{r_0,\,r_1\}.
\end{align}
This result implies (1).


%%%%%
%%%%%
\end{document}
