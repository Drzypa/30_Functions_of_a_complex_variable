\documentclass[12pt]{article}
\usepackage{pmmeta}
\pmcanonicalname{IncompleteGammaFunction}
\pmcreated{2013-03-22 15:36:48}
\pmmodified{2013-03-22 15:36:48}
\pmowner{rspuzio}{6075}
\pmmodifier{rspuzio}{6075}
\pmtitle{incomplete gamma function}
\pmrecord{11}{37536}
\pmprivacy{1}
\pmauthor{rspuzio}{6075}
\pmtype{Definition}
\pmcomment{trigger rebuild}
\pmclassification{msc}{30D30}
\pmclassification{msc}{33B15}
\pmrelated{SineIntegralInInfinity}

\endmetadata

% this is the default PlanetMath preamble.  as your knowledge
% of TeX increases, you will probably want to edit this, but
% it should be fine as is for beginners.

% almost certainly you want these
\usepackage{amssymb}
\usepackage{amsmath}
\usepackage{amsfonts}

% used for TeXing text within eps files
%\usepackage{psfrag}
% need this for including graphics (\includegraphics)
%\usepackage{graphicx}
% for neatly defining theorems and propositions
%\usepackage{amsthm}
% making logically defined graphics
%%%\usepackage{xypic}

% there are many more packages, add them here as you need them

% define commands here
\begin{document}
The \emph{incomplete gamma function} is defined as the indefinite integral of the integrand of gamma integral.  There are several definitions which differ in details of normalization and constant of integration:

\begin{eqnarray*}
\gamma (a,x) &=& \int_0^x e^{-t} t^{a-1} \, dt \\
\Gamma (a,x) &=&  \int_x^\infty e^{-t} t^{a-1} \, dt = \Gamma(a) - \gamma (a,x) \\
P (a,x) &=& {1 \over \Gamma (a)} \int_0^x e^{-t} t^{a-1} \, dt = {\gamma (a,x) \over \Gamma(a)} \\
\gamma^* (a,x) &=& {x^{-a} \over \Gamma(a)} \int_0^x e^{-t} t^{a-1} \, dt = {\gamma (a,x) \over x^a \Gamma(a)} \\
I (a,x) &=&  {1 \over \Gamma (a+1)} \int_0^{x \sqrt{a+1}} e^{-t} t^a \, dt = {\gamma (a+1, x \sqrt{a+1}) \over  \Gamma(a+1)} \\
C (a,x) &=& \int_x^\infty t^{a-1} \cos t \, dt \\
S (a,x) &=& \int_x^\infty t^{a-1} \sin t \, dt \\
E_n (x) &=& \int_1^\infty e^{-xt} t^{-n} \, dt = x^{n-1} \Gamma (1-n) - x^{n-1} \gamma (1-n,x) \\
\alpha_n (x) &=& \int_1^\infty e^{-xt} t^n \, dt = x^{-n-1} \Gamma (1+n) - x^{-n-1} \gamma (1+n,x) \\
\end{eqnarray*}
For convenience of translating notations, these variants have been expressed in terms of $\gamma$.
%%%%%
%%%%%
\end{document}
