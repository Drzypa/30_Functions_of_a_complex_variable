\documentclass[12pt]{article}
\usepackage{pmmeta}
\pmcanonicalname{WeierstrassFactorizationTheorem}
\pmcreated{2013-03-22 14:19:31}
\pmmodified{2013-03-22 14:19:31}
\pmowner{jirka}{4157}
\pmmodifier{jirka}{4157}
\pmtitle{Weierstrass factorization theorem}
\pmrecord{8}{35794}
\pmprivacy{1}
\pmauthor{jirka}{4157}
\pmtype{Theorem}
\pmcomment{trigger rebuild}
\pmclassification{msc}{30C15}
\pmsynonym{Weierstrass product theorem}{WeierstrassFactorizationTheorem}
\pmrelated{MittagLefflersTheorem}
\pmdefines{elementary factor}

\endmetadata

% this is the default PlanetMath preamble.  as your knowledge
% of TeX increases, you will probably want to edit this, but
% it should be fine as is for beginners.

% almost certainly you want these
\usepackage{amssymb}
\usepackage{amsmath}
\usepackage{amsfonts}

% used for TeXing text within eps files
%\usepackage{psfrag}
% need this for including graphics (\includegraphics)
%\usepackage{graphicx}
% for neatly defining theorems and propositions
\usepackage{amsthm}
% making logically defined graphics
%%%\usepackage{xypic}

% there are many more packages, add them here as you need them

% define commands here
\theoremstyle{theorem}
\newtheorem*{thm}{Theorem}
\newtheorem*{lemma}{Lemma}
\newtheorem*{conj}{Conjecture}
\newtheorem*{cor}{Corollary}
\newtheorem*{example}{Example}
\newtheorem*{prop}{Proposition}
\theoremstyle{definition}
\newtheorem*{defn}{Definition}
\theoremstyle{remark}
\newtheorem*{rmk}{Remark}
\begin{document}
There are several different statements of this theorem, but in essence this theorem will allow us to prescribe zeros and their orders of a holomorphic function.  It also allows us to factor any holomorphic function into
a product of zeros and a non-zero holomorphic function.  We will need to know
here how an infinite product converges.  It can then be shown
that if $\prod_{k=1}^\infty f_k(z)$ converges uniformly and \PMlinkname{absolutely}{AbsoluteConvergenceOfInfiniteProduct}
on compact subsets, then it converges to a holomorphic function given that all the $f_k(z)$ are holomorphic.  This is what we will \PMlinkescapetext{mean} by the infinite product in what follows.

Note that once we can prescribe zeros of a function then we can also prescribe the poles as well and get a meromorphic function just by dividing
two holomorphic functions $f/h$ where $f$ will contribute zeros, and
$h$ will make poles at the points where $h(z) = 0$.  So let's start with the existence statement.

\begin{thm}[Weierstrass Product]
Let $G \subset {\mathbb{C}}$ be a domain, let $\{a_k\}$ be a sequence of
points in $G$ with no accumulation points in $G$, and let $\{n_k\}$ be any
sequence of non-zero integers (positive or negative).
Then there exists a function
$f$ meromorphic in $G$ whose poles and zeros are exactly at the points $a_k$
and the order of the pole or zero at $a_k$ is $n_k$
(a positive order stands for zero, negative stands for pole).
\end{thm}

Next let's look at a more specific statement with more \PMlinkescapetext{restrictions}.  For
one let's start looking at the whole complex plane and further let's forget about poles for now to make the following formulas simpler.

\begin{defn}
We call
\begin{align*}
E_0(z) & := 1-z,
\\
E_p(z) & := (1-z) e^{z+\frac{1}{2}z^2 + \cdots + \frac{1}{p} z^p} \qquad \text{for $p \geq 1$},
\end{align*}
an {\em elementary factor}.
\end{defn}

Now note that for some $a \in {\mathbb{C}} \backslash \{ 0 \}$, $E_p(z/a)$ has a \PMlinkescapetext{simple} zero (zero of order 1) at $a$.

\begin{thm}[Weierstrass Factorization]
Suppose $f$ be an entire function and let $\{ a_k \}$ be the zeros of $f$
such that $a_k \not= 0$ (the non-zero zeros of $f$).  Let $m$ be the order of
the zero of $f$ at $z=0$ ($m=0$ if $f$ does not have a zero at $z=0$).  Then
there exists an entire function $g$ and a sequence of non-negative
integers $\{ p_k \}$ such that
\begin{equation*}
f(z) = z^m e^{g(z)} \prod_{k=1}^\infty E_{p_k} \left( \frac{z}{a_k} \right) .
\end{equation*}
\end{thm}

Note that we can always choose $p_k = k-1$ and the product above will converge as needed, but we may be able to choose better $p_k$ for specific functions.

\begin{example}
As an example we can try to factorize the function $\sin (\pi z)$, which has zeros at all the integers.  Applying the Weierstrass factorization theorem directly we get that
\begin{equation*}
\sin (\pi z) =
z e^{g(z)} \prod_{k = -\infty, k \not= 0}^\infty \left( 1 - \frac{z}{k} \right) e^{z/k},
\end{equation*}
where $g(z)$ is some holomorphic function.  It turns out that $e^{g(z)} = \pi$,
and rearranging the product we get
\begin{equation*}
\sin (\pi z) =
z \pi \prod_{k = 1}^\infty \left( 1 - \frac{z^2}{k^2} \right) .
\end{equation*}
This is an example where we could choose the $p_k = 1$ for all $k$ and thus
we could then get rid of the ugly parts of the infinite product.  For \PMlinkescapetext{complete}
calculations in this example see Conway \cite{Conway:complexI}.
\end{example}


\begin{thebibliography}{9}
\bibitem{Conway:complexI}
John~B. Conway.
{\em \PMlinkescapetext{Functions of One Complex Variable I}}.
Springer-Verlag, New York, New York, 1978.
\bibitem{Gamelin:complex}
Theodore~B.\@ Gamelin.
{\em \PMlinkescapetext{Complex Analysis}}.
Springer-Verlag, New York, New York, 2001.
\end{thebibliography}
%%%%%
%%%%%
\end{document}
