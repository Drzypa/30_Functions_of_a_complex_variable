\documentclass[12pt]{article}
\usepackage{pmmeta}
\pmcanonicalname{QuasiconformalMapping}
\pmcreated{2013-03-22 14:06:43}
\pmmodified{2013-03-22 14:06:43}
\pmowner{jirka}{4157}
\pmmodifier{jirka}{4157}
\pmtitle{quasiconformal mapping}
\pmrecord{13}{35515}
\pmprivacy{1}
\pmauthor{jirka}{4157}
\pmtype{Definition}
\pmcomment{trigger rebuild}
\pmclassification{msc}{30C65}
\pmclassification{msc}{30C62}
\pmsynonym{K-quasiconformal mapping}{QuasiconformalMapping}
%\pmkeywords{quasiconformal}
%\pmkeywords{quasisymmetric}
%\pmkeywords{dilatation}
\pmrelated{QuasisymmetricMapping}
\pmrelated{BeurlingAhlforsQuasiconformalExtension}
\pmrelated{ConformalMapping}
\pmrelated{BeltramiDifferentialEquation}
\pmdefines{dilatation}
\pmdefines{small dilatation}
\pmdefines{maximal dilatation}
\pmdefines{complex dilatation}
\pmdefines{$K$-quasiconformal}
\pmdefines{K-quasiconformal}
\pmdefines{quasiconformal}

\endmetadata

% this is the default PlanetMath preamble.  as your knowledge
% of TeX increases, you will probably want to edit this, but
% it should be fine as is for beginners.

% almost certainly you want these
\usepackage{amssymb}
\usepackage{amsmath}
\usepackage{amsfonts}

% used for TeXing text within eps files
%\usepackage{psfrag}
% need this for including graphics (\includegraphics)
%\usepackage{graphicx}
% for neatly defining theorems and propositions
\usepackage{amsthm}
% making logically defined graphics
%%%\usepackage{xypic}

% there are many more packages, add them here as you need them

% define commands here
\theoremstyle{theorem}
\newtheorem*{thm}{Theorem}
\newtheorem*{lemma}{Lemma}
\newtheorem*{conj}{Conjecture}
\newtheorem*{cor}{Corollary}
\newtheorem*{example}{Example}
\newtheorem*{prop}{Proposition}
\theoremstyle{definition}
\newtheorem*{defn}{Definition}
\begin{document}
{\em Quasiconformal mappings} are mappings of the complex plane to itself that are ``almost'' conformal.  That is, they do not distort angles arbitrarily and this ``distortion'' is uniformly bounded throughout their domain of definition.  Alternatively one can think of quasiconformal mappings as mappings which take infinitesimal circles to infinitesimal ellipses.  For example invertible linear maps are quasiconformal.

More rigorously, suppose $f$ is a mapping of the complex plane to itself, and here we will only consider sense-preserving mappings, that is mappings with a positive jacobian.

\begin{defn}
Define the {\em dilatation} of the mapping $f$ at the point $z$ as
\begin{equation*}
D_f(z) := \frac{\lvert f_z \rvert + \lvert f_{\bar{z}} \rvert}{\lvert f_z \rvert-\lvert f_{\bar{z}}\rvert} \geq 1 ,
\end{equation*}
and define
the {\em maximal dilatation} of the mapping as
\begin{equation*}
K_f := \sup_z D_f(z) .
\end{equation*}
\end{defn}

Now we are ready to define what it means for $f$ to be quasiconformal.

\begin{defn}
For $f$ as above, we will call $f$ {\em quasiconformal} if the maximal
dilatation of $f$ is finite.  We will say that $f$
is $K$-quasiconformal mapping if the maximal dilatation of this mapping is $K$.
\end{defn}

Note that sometimes the \PMlinkescapetext{term} $K$-quasiconformal is used to \PMlinkescapetext{mean} that the dilatation is $K$ or lower.

It is easy to see that a conformal sense-preserving mapping has a dilatation of $1$ since $\lvert f_{\bar{z}} \rvert = 0$.  We can further define several
other related quantities

\begin{defn}
For $f$ as above, define the {\em small dilatation} as
\begin{equation*}
d_f(z) := \frac{\lvert f_{\bar{z}} \rvert}{\lvert f_z \rvert} .
\end{equation*}
\end{defn}

Again for sense-preserving maps this quantity is less then 1 and it is equal to 0 if the mapping is conformal.  Some authors call a map $k$-quasiconformal if the small dilatation is bounded by $k$.  It is however not ambiguous as the large dilatation is always greater then or equal to 1.  Furthermore this is related to the large dilatation by
\begin{equation*}
d_f := \frac{D_f-1}{D_f+1} .
\end{equation*}

\begin{defn}
For $f$ as above, define the {\em complex dilatation} as
\begin{equation*}
\mu_f(z) := \frac{f_{\bar{z}}}{f_z} .
\end{equation*}
\end{defn}

The complex dilatation now appears in the Beltrami differential equation
\begin{equation*}
f_{\bar{z}}(z) = \mu_f(z)f_z(z) .
\end{equation*}
This means that a quasiconformal mapping is a solution to the Beltrami equation where a non-negative measurable $\mu_f$ is uniformly bounded by some $k < 1$.

The above results are stated for $f\colon {\mathbb{C}} \to {\mathbb{C}}$, but
the statements are exactly the same if you take $f\colon G \subset {\mathbb{C}} \to {\mathbb{C}}$ for an open set $G$.

The theory generalizes to other dimensions as well.  For example in one real dimension, the analogous mappings are called quasisymmetric.  It is a well-known \PMlinkname{theorem of Beurling and Ahlfors}{BeurlingAhlforsQuasiconformalExtension} that an \PMlinkescapetext{extension} of a mapping of the real line to itself is quasiconformal if and only if the mapping is quasisymmetric.

\begin{thebibliography}{9}
\bibitem{ahlfors}
L.\@ V.\@ Ahlfors. \emph{\PMlinkescapetext{Lectures on Quasiconformal
Mappings}}. Van Nostrand-Reinhold, Princeton, New Jersey, 1966
\bibitem{lebl2003}
J.\@ Lebl. \emph{\PMlinkescapetext{Quasiconformal Extensions of Quasisymmetric
Mappings}}. \PMlinkescapetext{Masters thesis, San Diego State University, San
Diego, CA, May 2003}. Also available at
\PMlinkexternal{http://www.jirka.org/thesis.pdf}{http://www.jirka.org/thesis.pdf}
\end{thebibliography}
%%%%%
%%%%%
\end{document}
