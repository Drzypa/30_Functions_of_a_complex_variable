\documentclass[12pt]{article}
\usepackage{pmmeta}
\pmcanonicalname{EqualityOfComplexNumbers}
\pmcreated{2015-05-09 17:11:34}
\pmmodified{2015-05-09 17:11:34}
\pmowner{pahio}{2872}
\pmmodifier{pahio}{2872}
\pmtitle{equality of complex numbers}
\pmrecord{7}{40210}
\pmprivacy{1}
\pmauthor{pahio}{2872}
\pmtype{Topic}
\pmcomment{trigger rebuild}
\pmclassification{msc}{30-00}
\pmrelated{ModulusOfAComplexNumber}
\pmrelated{ArgumentOfProductAndQuotient}
\pmrelated{ComplexLogarithm}

\endmetadata

% this is the default PlanetMath preamble.  as your knowledge
% of TeX increases, you will probably want to edit this, but
% it should be fine as is for beginners.

% almost certainly you want these
\usepackage{amssymb}
\usepackage{amsmath}
\usepackage{amsfonts}

% used for TeXing text within eps files
%\usepackage{psfrag}
% need this for including graphics (\includegraphics)
%\usepackage{graphicx}
% for neatly defining theorems and propositions
 \usepackage{amsthm}
% making logically defined graphics
%%%\usepackage{xypic}

% there are many more packages, add them here as you need them

% define commands here

\theoremstyle{definition}
\newtheorem*{thmplain}{Theorem}

\begin{document}
The equality relation ``='' among the \PMlinkescapetext{complex numbers} is determined as consequence of the definition of the complex numbers as elements of the quotient ring $\mathbb{R}/(X^2\!+\!1)$, which enables the \PMlinkescapetext{interpretation} of the complex numbers as the ordered pairs \,$(a,\,b)$\, of real numbers and also as the sums $a\!+\!ib$ where\, $i^2 = -1$.
\begin{align}
a_1+ib_1 = a_2+ib_2 \quad \Longleftrightarrow \quad a_1 = a_2\; \wedge\; b_1 = b_2
\end{align}
This condition may as well be derived by using the field properties of $\mathbb{C}$ and the properties of the real numbers:
\begin{align*}
a_1+ib_1 = a_2+ib_2\; & \implies\; \;a_2-a_1 = -i(b_2-b_1)\\
& \implies\; (a_2-a_1)^2 = -(b_2-b_1)^2\\ 
& \implies\;  (a_2-a_1)^2+(b_2-b_1)^2 = 0\\
& \implies\; \;a_2-a_1 = 0, \;\; b_2-b_1= 0\\
& \implies\; \;a_1 = a_2, \;\;\; b_1 = b_2
\end{align*}
The implication \PMlinkescapetext{chain} in the reverse direction is apparent.\\


If\, $a+ib \neq 0$,\, then at least one of the real numbers $a$ and $b$ differs from 0.\, We can set
\begin{align}
a = r\cos\varphi, \qquad b = r\sin\varphi,
\end{align}
where $r$ is a uniquely determined positive number and $\varphi$ is an angle which is uniquely determined up to an integer multiple of $2\pi$.\, In fact, the equations (2) yield
$$a^2+b^2 = r^2(\cos^2\varphi+\sin^2\varphi) = r^2,$$
whence
\begin{align}
r = \sqrt{a^2+b^2}.
\end{align}
Thus (2) implies
\begin{align}
\cos\varphi = \frac{a}{\sqrt{a^2+b^2}}, \qquad \sin\varphi = \frac{b}{\sqrt{a^2+b^2}}.
\end{align}
The equations (4) are \PMlinkescapetext{compatible}, since the sum of the squares of their \PMlinkescapetext{right sides} is 1.\, So these equations determine the angle $\varphi$ up to a multiple of $2\pi$.\, We can write the\\

\textbf{Theorem.}\, Every complex number may be represented in the {\em polar form}
$$r(\cos\varphi+i\sin\varphi),$$
where $r$ is the modulus and $\varphi$ the argument of the number.\, Two complex numbers are equal if and only if they have equal moduli and, if the numbers do not vanish, their arguments differ by a multiple of $2\pi$.


%%%%%
%%%%%
\end{document}
