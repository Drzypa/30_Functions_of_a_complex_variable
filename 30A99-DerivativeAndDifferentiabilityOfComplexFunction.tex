\documentclass[12pt]{article}
\usepackage{pmmeta}
\pmcanonicalname{DerivativeAndDifferentiabilityOfComplexFunction}
\pmcreated{2014-02-23 18:20:58}
\pmmodified{2014-02-23 18:20:58}
\pmowner{pahio}{2872}
\pmmodifier{pahio}{2872}
\pmtitle{derivative and differentiability of complex function}
\pmrecord{3}{88037}
\pmprivacy{1}
\pmauthor{pahio}{2872}
\pmtype{Definition}
\pmclassification{msc}{30A99}

\endmetadata

% this is the default PlanetMath preamble.  as your knowledge
% of TeX increases, you will probably want to edit this, but
% it should be fine as is for beginners.

% almost certainly you want these
\usepackage{amssymb}
\usepackage{amsmath}
\usepackage{amsfonts}

% need this for including graphics (\includegraphics)
\usepackage{graphicx}
% for neatly defining theorems and propositions
\usepackage{amsthm}

% making logically defined graphics
%\usepackage{xypic}
% used for TeXing text within eps files
%\usepackage{psfrag}

% there are many more packages, add them here as you need them

% define commands here

\begin{document}
Let $f(z)$ be given uniquely in a neighborhood of the point 
$z$ in $\mathbb{C}$.\, If the difference quotient
$$\frac{\Delta f}{\Delta z} \;=\; \frac{f(z+\Delta z)-f(z)}{\Delta z}$$
tends to a finite limit $A$ as $\Delta z \to 0$,
then $A$ is the {\it derivative} of $f$ at the point $z$ and is denoted by 
\begin{align}
f'(z) = A = \lim_{\Delta z \to 0}\frac{\Delta f}{\Delta z}.
\end{align}
Thus the difference\, $\lambda = \frac{\Delta f}{\Delta z}-A$\, tends to 
zero simultaneously with $\Delta z$, and $\Delta f$ has the expansion
$$\Delta f = A\Delta z+\lambda\Delta z.$$
If we denote\, $|\Delta z| \;=:\; \varrho$,\, we have
$$\lambda\Delta z = \frac{\lambda\Delta z}{\varrho}\cdot\varrho
\;=\; \langle\varrho\rangle\varrho$$
where $\langle\varrho\rangle$ means a complex number vanishing when 
$|\Delta z| = \varrho \to 0$.\, Consequently, (1) implies
\begin{align}
\Delta f \;=\; A\Delta z+\langle\varrho\rangle\varrho
\end{align}
in which\, $A = f'(z)$\, and\, $\varrho = |\Delta z|$.\, It's easily seen that 
the conditions (1) and (2) are equivalent.\, The latter expresses the 
{\it differentiability} of $f$ at $z$.\, By it one can sayt that the increment of 
$f$ is ``locally proportional'' to the increment of $z$.\, Cf. the 
consideration of differential of real functions.\\

\begin{thebibliography}{9}
\bibitem{EL}{\sc E. Lindel\"of}: {\em Johdatus funktioteoriaan} (`Introduction to function theory').\, Mercatorin kirjapaino, Helsinki (1936).

\bibitem{NP}{\sc R. Nevanlinna \& V. Paatero}: {\em Funktioteoria}.\, Kustannusosakeyhti\"o Otava, Helsinki (1963).
\end{thebibliography}\\


\end{document}
