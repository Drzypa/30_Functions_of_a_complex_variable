\documentclass[12pt]{article}
\usepackage{pmmeta}
\pmcanonicalname{ExponentialIntegral}
\pmcreated{2013-03-22 18:44:17}
\pmmodified{2013-03-22 18:44:17}
\pmowner{pahio}{2872}
\pmmodifier{pahio}{2872}
\pmtitle{exponential integral}
\pmrecord{8}{41510}
\pmprivacy{1}
\pmauthor{pahio}{2872}
\pmtype{Definition}
\pmcomment{trigger rebuild}
\pmclassification{msc}{30A99}
\pmclassification{msc}{26A36}
\pmsynonym{Ei}{ExponentialIntegral}
\pmrelated{LogarithmicIntegral}
\pmrelated{TableOfLaplaceTransforms}
\pmrelated{IndexOfSpecialFunctions}

\endmetadata

% this is the default PlanetMath preamble.  as your knowledge
% of TeX increases, you will probably want to edit this, but
% it should be fine as is for beginners.

% almost certainly you want these
\usepackage{amssymb}
\usepackage{amsmath}
\usepackage{amsfonts}

% used for TeXing text within eps files
%\usepackage{psfrag}
% need this for including graphics (\includegraphics)
%\usepackage{graphicx}
% for neatly defining theorems and propositions
 \usepackage{amsthm}
% making logically defined graphics
%%%\usepackage{xypic}

% there are many more packages, add them here as you need them

% define commands here

\theoremstyle{definition}
\newtheorem*{thmplain}{Theorem}

\begin{document}
The antiderivative of the function
$$x \mapsto \frac{e^{-x}}{x}$$
is not expressible in closed form.\, Thus such \PMlinkname{integrals}{ImproperIntegral} as
$$\int_x^\infty\!\frac{e^{-t}}{t}\,dt \quad \mbox{and} \quad \int_\infty^{-x}\!\frac{e^{-t}}{t}\,dt,$$
define certain \PMlinkname{non-elementary}{ElementaryFunction} transcendental functions.\, They are called \emph{exponential integrals} and denoted usually
${\rm E}_1$ and ${\rm Ei}$, respectively.\, Accordingly,
$${\rm E}_1(x) \;:=\; \int_x^\infty\!\frac{e^{-t}}{t}\,dt$$
$${\rm Ei}\,x \;:=\; \int_\infty^{-x}\!\frac{e^{-t}}{t}\,dt \;=\; -\int_{-x}^\infty\!\frac{e^{-t}}{t}\,dt
\;:=\; \int_{-\infty}^x\!\frac{e^{-u}}{u}\,du.$$
Then one has the connection
$${\rm E}_1(x) \;=\; -{\rm Ei}\,(-x).$$
For positive values of $x$ the series expansion
$${\rm Ei}\,x \;=\; \gamma+\ln{x}+\sum_{j=1}^\infty\frac{x^j}{j!j},$$
where $\gamma$ is the \PMlinkid{Euler--Mascheroni constant}{1883}, is valid.\\

Note: Some authors use the convention\; ${\rm Ei}\,x \,:=\, \int_x^\infty\!\frac{e^{-t}}{t}\,dt$.\\

\subsection{Laplace transform of $\frac{1}{t+a}$}

By the definition of Laplace transform,
$$\mathcal{L}\{\frac{1}{t\!+\!a}\} \;=\; \int_0^\infty\frac{e^{-st}}{t\!+\!a}\,dt.$$
The \PMlinkname{substitution}{ChangeOfVariableInDefiniteIntegral}\, $t\!+\!a = u$\, gives
$$\mathcal{L}\{\frac{1}{t\!+\!a}\} \;=\; \int_a^\infty\frac{e^{as-su}}{u}\,du 
\;=\; e^{as}\int_a^\infty\frac{e^{-su}}{u}\,du,$$
from which the substitution\, $su = t$\, yields
$$\mathcal{L}\{\frac{1}{t\!+\!a}\} \;=\; e^{as}\int_{as}^\infty\frac{e^{-t}}{t}\,dt,$$
i.e. 
\begin{align}
\mathcal{L}\{\frac{1}{t\!+\!a}\} \;=\; e^{as}{\rm E}_1(as).
\end{align}
Using \PMlinkname{the rule}{LaplaceTransformOfDerivative}\, $\mathcal{L}\{f'(t)\} = sF(s)\!-\!f(0)$,\, one easily derives from (1) the \PMlinkescapetext{formula}
\begin{align}
\mathcal{L}\{\frac{1}{(t\!+\!a)^2}\} \;=\; \frac{1}{a}\!-\!se^{as}{\rm E}_1(as).
\end{align}




%%%%%
%%%%%
\end{document}
