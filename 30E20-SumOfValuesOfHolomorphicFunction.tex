\documentclass[12pt]{article}
\usepackage{pmmeta}
\pmcanonicalname{SumOfValuesOfHolomorphicFunction}
\pmcreated{2013-03-22 19:15:30}
\pmmodified{2013-03-22 19:15:30}
\pmowner{pahio}{2872}
\pmmodifier{pahio}{2872}
\pmtitle{sum of values of holomorphic function}
\pmrecord{6}{42186}
\pmprivacy{1}
\pmauthor{pahio}{2872}
\pmtype{Theorem}
\pmcomment{trigger rebuild}
\pmclassification{msc}{30E20}

% this is the default PlanetMath preamble.  as your knowledge
% of TeX increases, you will probably want to edit this, but
% it should be fine as is for beginners.

% almost certainly you want these
\usepackage{amssymb}
\usepackage{amsmath}
\usepackage{amsfonts}

% used for TeXing text within eps files
%\usepackage{psfrag}
% need this for including graphics (\includegraphics)
%\usepackage{graphicx}
% for neatly defining theorems and propositions
 \usepackage{amsthm}
% making logically defined graphics
%%%\usepackage{xypic}

% there are many more packages, add them here as you need them

% define commands here

\theoremstyle{definition}
\newtheorem*{thmplain}{Theorem}

\begin{document}
Let $w(z)$ be a holomorphic function on a simple closed curve $C$ and inside it.\, If $a_1,\,a_2,\,\ldots,\,a_m$ are inside $C$ the simple zeros of a function $f(z)$ holomorphic on $C$ and inside, then
\begin{align}
w(a_1)\!+\!w(a_2)\!+\ldots+\!w(a_m) \;=\; \frac{1}{2i\pi}\!\oint_C\!w(z)\frac{f'(z)}{f(z)}\,dz
\end{align}
where the contour integral is taken anticlockwise.

The \PMlinkescapetext{formula remains in force} if some of the zeros are multiple and are counted with \PMlinkname{multiplicities}{Pole}.

If the zeros $a_j$ of $f(z)$ have the multiplicities $\alpha_j$ and the function has inside $C$ also the poles 
$b_1,\,b_2,\,\ldots,\,b_n$ with the multiplicities $\beta_1,\,\beta_2,\,\ldots,\,\beta_n$,\, then (1) must be written
\begin{align}
\sum_j\alpha_jw(a_j)-\sum_k\beta_kw(b_k) \;=\; \frac{1}{2i\pi}\!\oint_C\!w(z)\frac{f'(z)}{f(z)}\,dz.
\end{align}


The special case \,$w(z) \equiv 1$\, gives from (2) the argument principle.

\begin{thebibliography}{8}
\bibitem{lindelof}{\sc Ernst Lindel\"of}: {\em Le calcul des r\'esidus et ses applications \`a la th\'eorie des fonctions}.\, Gauthier-Villars, Paris (1905).
\end{thebibliography}
%%%%%
%%%%%
\end{document}
