\documentclass[12pt]{article}
\usepackage{pmmeta}
\pmcanonicalname{AntiderivativeOfComplexFunction}
\pmcreated{2014-02-23 15:09:20}
\pmmodified{2014-02-23 15:09:20}
\pmowner{Wkbj79}{1863}
\pmmodifier{pahio}{2872}
\pmtitle{antiderivative of complex function}
\pmrecord{10}{37140}
\pmprivacy{1}
\pmauthor{Wkbj79}{2872}
\pmtype{Definition}
\pmcomment{trigger rebuild}
\pmclassification{msc}{30A99}
\pmclassification{msc}{03E20}
\pmsynonym{complex antiderivative}{AntiderivativeOfComplexFunction}
\pmrelated{Antiderivative}
\pmrelated{CalculationOfContourIntegral}

% this is the default PlanetMath preamble.  as your knowledge
% of TeX increases, you will probably want to edit this, but
% it should be fine as is for beginners.

% almost certainly you want these
\usepackage{amssymb}
\usepackage{amsmath}
\usepackage{amsfonts}

% used for TeXing text within eps files
%\usepackage{psfrag}
% need this for including graphics (\includegraphics)
%\usepackage{graphicx}
% for neatly defining theorems and propositions
 \usepackage{amsthm}
% making logically defined graphics
%%%\usepackage{xypic}

% there are many more packages, add them here as you need them

% define commands here

\theoremstyle{definition}
\newtheorem*{thmplain}{Theorem}
\begin{document}
By the {\em \PMlinkescapetext{antiderivative}} of a complex function $f$ in a domain $D$ of $\mathbb{C}$, we \PMlinkescapetext{mean} every complex function $F$ which in $D$ satisfies the condition
$$\frac{d}{dz}F(z) = f(z).$$

\begin{itemize}

\item If $f$ is a continuous complex function in a domain $D$ 
and if the integral
\begin{align}
F(z) := \int_{\gamma_z}f(t)\,dt
\end{align}
where the path ${\gamma_z}$ begins at a fixed point $z_0$ of 
$D$ and ends at the point $z$ of $D$, is independent of the 
path $\gamma_z$ for each value of $z$, then (1) defines an 
analytic function $F$ with domain $D$.\, This function is an 
antiderivative of $f$ in $D$, \PMlinkname{i.e.}{Ie} at all 
points of $D$, the condition
$$\frac{d}{dz}\int_{\gamma_z}f(t)\,dt = f(z)$$
is true.

\item If $f$ is an analytic function in a simply connected open domain $U$, then $f$ has an antiderivative in $U$, \PMlinkname{e.g.}{Eg} the function $F$ defined by (1) where the path $\gamma_z$ is within $U$.\, If $\gamma$ lies within $U$ and connects the points $z_0$ and $z_1$, then 
$$\int_{\gamma}f(z)\,dz = F(z_1)-F(z_0),$$
where $F$ is an arbitrary antiderivative of $f$ in $U$.

\end{itemize}
%%%%%
%%%%%
\end{document}
