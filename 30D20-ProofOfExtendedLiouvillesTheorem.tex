\documentclass[12pt]{article}
\usepackage{pmmeta}
\pmcanonicalname{ProofOfExtendedLiouvillesTheorem}
\pmcreated{2013-03-22 16:18:31}
\pmmodified{2013-03-22 16:18:31}
\pmowner{rm50}{10146}
\pmmodifier{rm50}{10146}
\pmtitle{proof of extended Liouville's theorem}
\pmrecord{8}{38431}
\pmprivacy{1}
\pmauthor{rm50}{10146}
\pmtype{Proof}
\pmcomment{trigger rebuild}
\pmclassification{msc}{30D20}

\endmetadata

% this is the default PlanetMath preamble.  as your knowledge
% of TeX increases, you will probably want to edit this, but
% it should be fine as is for beginners.

% almost certainly you want these
\usepackage{amssymb}
\usepackage{amsmath}
\usepackage{amsfonts}

% used for TeXing text within eps files
%\usepackage{psfrag}
% need this for including graphics (\includegraphics)
%\usepackage{graphicx}
% for neatly defining theorems and propositions
%\usepackage{amsthm}
% making logically defined graphics
%%%\usepackage{xypic}

% there are many more packages, add them here as you need them

% define commands here

\newcommand{\Reals}{\mathbb{R}}
\newcommand{\Complex}{\mathbb{C}}

\begin{document}
This is a proof of the second, more general, form of Liouville's theorem given in the \PMlinkname{parent}{LiouvillesTheorem2} article.

Let $f:\Complex\to\Complex$ be a holomorphic function such that
$$|f(z)| < c \cdot |z|^n$$
for some $c\in\Reals$ and for $z\in\Complex$ with $|z|$ sufficiently large. Consider
$$g(z)=\begin{cases}\frac{f(z)-f(0)}{z}&z\neq 0\\
f'(0)&z=0
\end{cases}
$$
Since $f$ is holomorphic, $g$ is as well, and by the bound on $f$, we have
\[|g(z)|< c_1 + c_2 \cdot |z|^{n-1} < c'\cdot |z|^{n-1}\]
again for $|z|$ sufficiently large.

By induction, $g$ is a polynomial of degree at most $n-1$, and thus $f$ is a polynomial of degree at most $n$.
%%%%%
%%%%%
\end{document}
