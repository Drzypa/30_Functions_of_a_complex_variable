\documentclass[12pt]{article}
\usepackage{pmmeta}
\pmcanonicalname{ProofOfGaussDigammaTheorem}
\pmcreated{2013-03-22 16:24:07}
\pmmodified{2013-03-22 16:24:07}
\pmowner{rm50}{10146}
\pmmodifier{rm50}{10146}
\pmtitle{proof of Gauss' digamma theorem}
\pmrecord{5}{38550}
\pmprivacy{1}
\pmauthor{rm50}{10146}
\pmtype{Proof}
\pmcomment{trigger rebuild}
\pmclassification{msc}{30D30}
\pmclassification{msc}{33B15}

\endmetadata

% this is the default PlanetMath preamble.  as your knowledge
% of TeX increases, you will probably want to edit this, but
% it should be fine as is for beginners.

% almost certainly you want these
\usepackage{amssymb}
\usepackage{amsmath}
\usepackage{amsfonts}

% used for TeXing text within eps files
%\usepackage{psfrag}
% need this for including graphics (\includegraphics)
%\usepackage{graphicx}
% for neatly defining theorems and propositions
%\usepackage{amsthm}
% making logically defined graphics
%%%\usepackage{xypic}

% there are many more packages, add them here as you need them

% define commands here

\begin{document}
\textbf{Proof.}
The proof follows the argument given in \cite{bib:andrews}, which in turn derives from that given in \cite{bib:jensen}.

The first formula is the logarithmic derivative of
\[\Gamma(x+n)=(x+n-1)(x+n-2)\cdots x\Gamma(x)\]

By the partial fraction decomposition satisfied by the $\psi$ function,
\[\psi\left(\frac{p}{q}\right)+\gamma=\sum_{n=0}^{\infty}\left(\frac{1}{n+1}-\frac{q}{p+nq}\right)=
\lim_{t\to 1^-}\sum_{n=0}^{\infty}\left(\frac{1}{n+1}-\frac{q}{p+nq}\right)t^{p+nq}\]
using Abel's limit theorem.

Now,
\[\sum_{n=0}^{\infty}\left(\frac{1}{n+1}-\frac{q}{p+nq}\right)t^{p+nq}=
\sum_{n=0}^{\infty}\frac{t^{p+nq}}{n+1}\ -\ \sum_{n=0}^{\infty}\frac{qt^{p+nq}}{p+nq}=
t^{p-q}\sum_{n=0}^{\infty}\frac{t^{(n+1)q}}{n+1}\ -\ q\sum_{n=0}^{\infty}\frac{t^{p+nq}}{p+nq}\]
Since
\[-\ln(1-t)=\sum_{n=1}^{\infty} \frac{t^n}{n}\]
the first term is
\[-t^{p-q}\ln(1-t^q)\]
Using the algorithm for \PMlinkname{extracting every $q^\mathrm{th}$ term of a series}{ExtractingEveryNthTermOfASeries}, the second term is
\[\sum_{n=0}^{q-1}\omega^{-np}\ln(1-\omega^n t)\]
and therefore
\begin{align*}\sum_{n=0}^{\infty}\left(\frac{1}{n+1}-\frac{q}{p+nq}\right)t^{p+nq}&=-t^{p-q}\ln(1-t^q)+\sum_{n=0}^{q-1}\omega^{-np}\ln(1-\omega^n t)\\
&=-t^{p-q}\ln\frac{1-t^q}{1-t}-(t^{p-1}-1)\ln(1-t)+\sum_{n=1}^{q-1}\omega^{-np}\ln(1-\omega^n t)
\end{align*}
Let $t\to 1^-$ to get
\[\psi\left(\frac{p}{q}\right)=-\gamma-\ln q+\sum_{n=1}^{q-1}\omega^{-np}\ln(1-\omega^n)\]
Replace $p$ by $q-p$ and add the two expressions to obtain
\begin{equation*}\psi\left(\frac{p}{q}\right)+\psi\left(\frac{q-p}{q}\right)=-2\gamma-2\ln q+2\sum_{n=1}^{q-1}\cos\left(\frac{2\pi n p}{q}\right)\ln(1-\omega^n)
\end{equation*}
The left side is real, so it is equal to the real part of the right side. But
\[\Re(\ln(1-\omega^n))=\ln\lvert 1-\omega^n\rvert^{1/2}=\ln\left\lvert \left(1-\cos\frac{2\pi n}{q}\right)^2+\sin^2\frac{2\pi n}{q}\right\rvert^{1/2}=\frac{1}{2}\ln\left(2-2\cos\frac{2\pi n}{q}\right)
\]
and so
\begin{equation}\label{eqn:one}\psi\left(\frac{p}{q}\right)+\psi\left(\frac{q-p}{q}\right)=-2\gamma-2\ln q+\sum_{n=1}^{q-1}\cos\left(\frac{2\pi n 
p}{q}\right)\ln(2-2\cos\frac{2\pi n}{q})\end{equation}
But
\[\psi(x)-\psi(1-x)=\frac{d}{dx}\ln(\Gamma(x)\Gamma(1-x))=-\pi\cot\pi x\]
by the Euler reflection formula and thus
\begin{equation}\label{eqn:two}\psi\left(\frac{p}{q}\right)-\psi\left(\frac{q-p}{q}\right)=-\pi\cot \frac{\pi p}{q}\end{equation}
Add equations (\ref{eqn:one}) and (\ref{eqn:two}) to get
\begin{align*}\psi\left(\frac{p}{q}\right)&=-\gamma-\frac{\pi}{2}\cot\frac{\pi p}{q}-\ln q+\frac{1}{2}\sum_{n=1}^{q-1}\cos\frac{2\pi n p}{q}\ln\left(2-2\cos\frac{2\pi n}{q}\right)\\
&=-\gamma-\frac{\pi}{2}\cot\frac{\pi p}{q}-\ln q+\sum_{n=1}^{q-1}\cos\frac{2\pi n p}{q}\ln\left(2\sin\frac{\pi n}{q}\right)
\end{align*}
where the last equality holds since
\[\ln(2-2\cos (2\theta))=\ln(2-2(1-2\sin^2\theta)=\ln(4\sin^2\theta)=2\ln(2\sin\theta)\]


\begin{thebibliography}{10}
\bibitem{bib:andrews}
G.E.~Andrews, R.~Askey, R.~Roy, \emph{Special Functions}, Cambridge University Press, 2001.
\bibitem{bib:jensen}
J.L.~Jensen [1915-1916], An elementary exposition of the theory of the gamma function, \emph{Ann. Math.} \textbf{17}, 124-166.
\end{thebibliography}
\include{Footer}
%%%%%
%%%%%
\end{document}
