\documentclass[12pt]{article}
\usepackage{pmmeta}
\pmcanonicalname{EstimatingTheoremOfContourIntegral}
\pmcreated{2013-03-22 15:19:36}
\pmmodified{2013-03-22 15:19:36}
\pmowner{pahio}{2872}
\pmmodifier{pahio}{2872}
\pmtitle{estimating theorem of contour integral}
\pmrecord{7}{37138}
\pmprivacy{1}
\pmauthor{pahio}{2872}
\pmtype{Theorem}
\pmcomment{trigger rebuild}
\pmclassification{msc}{30E20}
\pmclassification{msc}{30A99}
\pmsynonym{estimation theorem of integral}{EstimatingTheoremOfContourIntegral}
\pmsynonym{integral estimating theorem}{EstimatingTheoremOfContourIntegral}
\pmrelated{MinimalAndMaximalNumber}
\pmrelated{AnalyticContinuationOfRiemannZetaUsingIntegral}
\pmrelated{IntegralMeanValueTheorem}

\endmetadata

% this is the default PlanetMath preamble.  as your knowledge
% of TeX increases, you will probably want to edit this, but
% it should be fine as is for beginners.

% almost certainly you want these
\usepackage{amssymb}
\usepackage{amsmath}
\usepackage{amsfonts}

% used for TeXing text within eps files
%\usepackage{psfrag}
% need this for including graphics (\includegraphics)
%\usepackage{graphicx}
% for neatly defining theorems and propositions
 \usepackage{amsthm}
% making logically defined graphics
%%%\usepackage{xypic}

% there are many more packages, add them here as you need them

% define commands here

\theoremstyle{definition}
\newtheorem*{thmplain}{Theorem}
\begin{document}
\begin{thmplain}
If $f$ is a continuous complex function on the rectifiable curve $\gamma$ of the complex plane, then
\begin{align}
\left|\int_\gamma f(z)\,dz\right| \leqq \max_{z\in\gamma} |f(z)|\cdot l,
\end{align}
where 
$$l = \int_\gamma|dz|$$
is the \PMlinkescapetext{length} of $\gamma$.
\end{thmplain}

The form of (1) concerning the continuous real function $f$ on the interval $[a,\,b]$ is
$$\left|\int_a^b f(x)\,dx\right| \leqq \max_{a\leqq x\leqq b} |f(x)|\cdot(b\!-\!a).$$

For applications of this important theorem, see the example of using residue theorem.
%%%%%
%%%%%
\end{document}
