\documentclass[12pt]{article}
\usepackage{pmmeta}
\pmcanonicalname{TaylorSeriesOfHyperbolicFunctions}
\pmcreated{2013-03-22 19:07:04}
\pmmodified{2013-03-22 19:07:04}
\pmowner{pahio}{2872}
\pmmodifier{pahio}{2872}
\pmtitle{Taylor series of hyperbolic functions}
\pmrecord{7}{42012}
\pmprivacy{1}
\pmauthor{pahio}{2872}
\pmtype{Derivation}
\pmcomment{trigger rebuild}
\pmclassification{msc}{30B10}
\pmclassification{msc}{26A09}
\pmrelated{HyperbolicIdentities}
\pmrelated{HigherOrderDerivatives}

\endmetadata

% this is the default PlanetMath preamble.  as your knowledge
% of TeX increases, you will probably want to edit this, but
% it should be fine as is for beginners.

% almost certainly you want these
\usepackage{amssymb}
\usepackage{amsmath}
\usepackage{amsfonts}

% used for TeXing text within eps files
%\usepackage{psfrag}
% need this for including graphics (\includegraphics)
%\usepackage{graphicx}
% for neatly defining theorems and propositions
 \usepackage{amsthm}
% making logically defined graphics
%%%\usepackage{xypic}

% there are many more packages, add them here as you need them

% define commands here

\theoremstyle{definition}
\newtheorem*{thmplain}{Theorem}

\begin{document}
\PMlinkescapeword{expansion}
The differentiation rules
$$\frac{d}{dx}\cosh{x} \;=\; \sinh{x}, \quad \frac{d}{dx}\sinh{x} \;=\; \cosh{x}$$
 of the hyperbolic functions imply
$$\frac{d^{2n}}{dx^{2n}}\cosh{x} \;=\; \cosh{x}, \quad \frac{d^{2n+1}}{dx^{2n+1}}\cosh{x} \;=\; \sinh{x} 
\qquad (n = 0,\,1,\,2,\,\ldots).$$
In the origin \,$x = 0$,\, all \PMlinkname{even}{Even}-order derivatives of the hyperbolic cosine have the value 1, but the \PMlinkname{odd}{Odd}-order derivatives vanish.\, Thus the Taylor series expansion
$$f(x) \;=\; f(0)+\frac{f'(0)}{1!}x+\frac{f''(0)}{2!}x^2+\frac{f'''(0)}{3!}x^3+\ldots$$
of\, $f(x) := \cosh{x}$\, contains only the terms of even degree and writes simply
\begin{align}
\cosh{x} \;=\; 1+\frac{x^2}{2!}+\frac{x^4}{4!}+\ldots \;=\; \sum_{n=0}^\infty\frac{x^{2n}}{(2n)!}.
\end{align}


Similarly, one can derive for the hyperbolic sine the expansion
\begin{align}
\sinh{x} \;=\; x+\frac{x^3}{3!}+\frac{x^5}{5!}+\ldots \;=\; \sum_{n=0}^\infty\frac{x^{2n+1}}{(2n\!+\!1)!}.
\end{align}


Both series \PMlinkescapetext{presentations} \PMlinkname{converge}{AbsoluteConvergence} and \PMlinkescapetext{represent} the functions for all real (and complex) values of $x$.\, Comparing the expansions (1) and (2) with the corresponding ones of the circular functions cosine and sine, one sees easily that
$$\cosh{x} \;=\; \cos{ix}, \qquad \sinh{x} \;=\; -i\sin{ix}.$$\\

As for the Taylor expansion of the third important hyperbolic function \PMlinkname{tangens hyperbolica}{HyperbolicFunctions} tanh, it is obtained via \PMlinkname{division of the Taylor series}{TaylorSeriesViaDivision} (2) and (1); the begin of the quotient series is
\begin{align}
\tanh{x} \;=\; x-\frac{1}{3}x^3+\frac{2}{15}x^5-\frac{17}{315}x^7+-\ldots \qquad (|x| < \frac{\pi}{2}).
\end{align}
The coefficients of this power series may be expressed with the Bernoulli numbers.




%%%%%
%%%%%
\end{document}
