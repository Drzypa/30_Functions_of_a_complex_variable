\documentclass[12pt]{article}
\usepackage{pmmeta}
\pmcanonicalname{ComplexTangentAndCotangent}
\pmcreated{2013-03-22 16:49:56}
\pmmodified{2013-03-22 16:49:56}
\pmowner{pahio}{2872}
\pmmodifier{pahio}{2872}
\pmtitle{complex tangent and cotangent}
\pmrecord{10}{39074}
\pmprivacy{1}
\pmauthor{pahio}{2872}
\pmtype{Definition}
\pmcomment{trigger rebuild}
\pmclassification{msc}{30A99}
\pmclassification{msc}{30D10}
\pmclassification{msc}{33B10}
\pmrelated{ExamplesOfInfiniteProducts}
\pmrelated{QuasiPeriodicFunction}
\pmrelated{HyperbolicFunctions}
\pmrelated{QuasiperiodicFunction}

% this is the default PlanetMath preamble.  as your knowledge
% of TeX increases, you will probably want to edit this, but
% it should be fine as is for beginners.

% almost certainly you want these
\usepackage{amssymb}
\usepackage{amsmath}
\usepackage{amsfonts}

% used for TeXing text within eps files
%\usepackage{psfrag}
% need this for including graphics (\includegraphics)
%\usepackage{graphicx}
% for neatly defining theorems and propositions
 \usepackage{amsthm}
% making logically defined graphics
%%%\usepackage{xypic}

% there are many more packages, add them here as you need them

% define commands here

\theoremstyle{definition}
\newtheorem*{thmplain}{Theorem}

\begin{document}
The tangent and the cotangent function for complex values of the \PMlinkescapetext{argument} $z$ are defined with the equations
$$\tan{z} := \frac{\sin{z}}{\cos{z}},\quad \cot{z} := \frac{\cos{z}}{\sin{z}}.$$
Using the \PMlinkname{Euler's formulae}{ComplexSineAndCosine}, one also can define
\begin{align}
\tan{z} := -i\frac{e^{iz}-e^{-iz}}{e^{iz}+e^{-iz}},\quad 
\cot{z} := i\frac{e^{iz}+e^{-iz}}{e^{iz}-e^{-iz}}.
\end{align}
The subtraction formulae of \PMlinkname{cosine and sine}{ComplexSineAndCosine} yield an additional \PMlinkescapetext{connection} between the cotangent and tangent:
$$\cot{(\frac{\pi}{2}-z)} =
\frac{\cos{(\frac{\pi}{2}-z)}}{\sin{(\frac{\pi}{2}-z)}} =
\frac{\cos{\frac{\pi}{2}}\cos{z}+\sin{\frac{\pi}{2}}\sin{z}}
     {\sin{\frac{\pi}{2}}\cos{z}-\cos{\frac{\pi}{2}}\sin{z}} = 
\frac{\sin{z}}{\cos{z}} = \tan{z}.$$
Thus the properties of the tangent are easily derived from the corresponding properties of the cotangent.

Because of the identic equation\, $\cos^2{z}+\sin^2{z} = 1$\, the cosine and sine do not vanish simultaneously, and so their quotient $\cot{z}$ is finite in all finite points $z$ of the complex plane except in the zeros\, $z = n\pi$\, ($n = 0,\,\pm1,\,\pm2,\,\ldots$) of $\sin{z}$, where $\cot{z}$ becomes infinite.\, We shall see that these multiples of $\pi$ are simple poles of $\cot{z}$.

If one moves from $z$ to $z\!+\!\pi$, then both $\cos{z}$ and $\sin{z}$ change their signs (cf. antiperiodic function), and therefore their quotient remains unchanged.\, Accordingly, $\pi$ is a period of $\cot{z}$.\, But if $\omega$ is an arbitrary period of $\cot{z}$, we have\, $\cot{(z\!+\!\omega)} = \cot{z}$,\, and especially\, $z = 0$ gives\, $\cot{\omega} = \infty$;\, then (1) says that\, 
$e^{i\omega} = e^{-i\omega}$,\, i.e.\, $e^{2i\omega} = 1$.\, Since the prime period of the complex exponential function is $2i\pi$, the last equation is valid only for the values\, $\omega = n\pi$\, ($n = 0,\,\pm1,\,\pm2,\,\ldots$).\, Thus we have shown that the prime period of $\cot{z}$ is $\pi$.

We know that
$$\frac{\sin{z}}{z} = \frac{\sin{z}-\sin{0}}{z} \to \cos{0} = 1
\quad \mathrm{as} \quad z\to 0;$$
therefore
$$z\cot{z} = \frac{z}{\sin{z}}\cdot\cos{z} \to 1\cdot\cos{0} = 1
\quad \mathrm{as} \quad z\to 0.$$
This result, together with
$$\cot{z} \to \infty \quad \mathrm{as} \quad z\to 0,$$
means that\, $z = 0$\, is a simple pole of $\cot{z}$.\, 

Because of the periodicity, $\cot{z}$ has the simple poles in the points 
$z = 0,\,\pm\pi,\,\pm 2\pi,\,\ldots$.\, Since one has the derivative
        $$\frac{d\cot{z}}{dz} = -\frac{1}{\sin^2{z}},$$
$\cot{z}$ is holomorphic in all finite points except those poles, which accumulate only to the point\, $z = \infty$.\, Thus the cotangent is a meromorphic function.\, The same concerns naturally the tangent function.

As all meromorphic functions, the cotangent may be expressed as a series with  the \PMlinkname{partial fraction}{PartialFractionsOfExpressions} terms of the form $\frac{a_{jk}}{(z-p_j)^k}$, where $p_j$'s are the poles --- see \PMlinkname{this entry}{ExamplesOfInfiniteProducts}.

The \PMlinkname{real}{CmplexFunction} and imaginary parts of tangent and cotangent are seen from the formulae
$$\tan(x+iy) = \frac{\sin{x}\cos{x}+i\sinh{y}\cosh{y}}{\cos^2{x}+\sinh^2{y}},$$
$$\cot(x+iy) = \frac{\sin{x}\cos{x}-i\sinh{y}\cosh{y}}{\sin^2{x}+\sinh^2{y}},$$
which may be derived from (1) by substituting\, $z := x\!+\!iy$ 
($x,\,y \in\mathbb{R}$).

\begin{thebibliography}{9}
\bibitem{NP}{\sc R. Nevanlinna \& V. Paatero}: {\em Funktioteoria}.\, Kustannusosakeyhti\"o Otava. Helsinki (1963).
\end{thebibliography}
%%%%%
%%%%%
\end{document}
