\documentclass[12pt]{article}
\usepackage{pmmeta}
\pmcanonicalname{ArgumentPrinciple}
\pmcreated{2013-03-22 14:34:28}
\pmmodified{2013-03-22 14:34:28}
\pmowner{rspuzio}{6075}
\pmmodifier{rspuzio}{6075}
\pmtitle{argument principle}
\pmrecord{10}{36133}
\pmprivacy{1}
\pmauthor{rspuzio}{6075}
\pmtype{Algorithm}
\pmcomment{trigger rebuild}
\pmclassification{msc}{30E20}
\pmsynonym{Cauchy's argument principle}{ArgumentPrinciple}
%\pmkeywords{argument}
%\pmkeywords{complex anaysis}
%\pmkeywords{contour integration}
\pmdefines{argument principle}

\endmetadata

% this is the default PlanetMath preamble.  as your knowledge
% of TeX increases, you will probably want to edit this, but
% it should be fine as is for beginners.

% almost certainly you want these
\usepackage{amssymb}
\usepackage{amsmath}
\usepackage{amsfonts}

% used for TeXing text within eps files
%\usepackage{psfrag}
% need this for including graphics (\includegraphics)
%\usepackage{graphicx}
% for neatly defining theorems and propositions
%\usepackage{amsthm}
% making logically defined graphics
%%%\usepackage{xypic}

% there are many more packages, add them here as you need them

% define commands here
\begin{document}
\PMlinkescapeword{origin}
\PMlinkescapeword{side}
\PMlinkescapeword{sides}
\PMlinkescapeword{formula}
\PMlinkescapeword{relation}

If a function $f$ is meromorphic on the interior of a rectifiable simple closed curve $C$, then
\begin{align}\label{eq:arg-princ}
{1 \over 2 \pi i} \oint_C {f'(z) \over f(z)} dz
\end{align}
equals the difference between the number of zeros and the number of poles of $f$ counted with multiplicity. (For example, a zero of order two counts as two zeros; a pole of order three counts as three poles.)
This fact is known as the \emph{argument principle}.

The principle may be stated in another form which makes the origin of the name apparent:  If a function $f$ is meromorphic on the interior of a rectifiable simple closed curve $C$ and has $m$ poles and $n$ zeros on the interior of $C$, then the argument of $f$ increases by $2 \pi (n - m)$ upon traversing $C$.  The relation of this statement to the previous statement is easy to see.  Note that $f'/f = (\log f)'$ and that $\log (z) = \log |z| + i \arg z$.  Substituting this into formula \eqref{eq:arg-princ}, we find
\[
2 \pi i (n - m) = \oint_C {f'(z) \over f(z)} dz = \oint_C d \log |f(z)| +  i \oint_C d \arg (f(z))\,.
\]
The first integral on the rightmost side of this equation equals zero because $\log|f|$ is single-valued.  The second integral on the rightmost side equals the change in the argument as one traverses $C$.  Cancelling the $i$ from both sides, we conclude that the change in the argument equals $2 \pi (n - m)$.

Note also that the integral \eqref{eq:arg-princ}
is the winding number, about zero, of the image curve $f \circ C$.


%%%%%
%%%%%
\end{document}
