\documentclass[12pt]{article}
\usepackage{pmmeta}
\pmcanonicalname{ProofOfGaussMeanValueTheorem}
\pmcreated{2013-03-22 13:35:36}
\pmmodified{2013-03-22 13:35:36}
\pmowner{yark}{2760}
\pmmodifier{yark}{2760}
\pmtitle{proof of Gauss' mean value theorem}
\pmrecord{18}{34217}
\pmprivacy{1}
\pmauthor{yark}{2760}
\pmtype{Proof}
\pmcomment{trigger rebuild}
\pmclassification{msc}{30E20}
\pmrelated{CauchyIntegralFormula}

% this is the default PlanetMath preamble.  as your knowledge
% of TeX increases, you will probably want to edit this, but
% it should be fine as is for beginners.

% almost certainly you want these
\usepackage{amssymb}
\usepackage{amsmath}
\usepackage{amsfonts}

% used for TeXing text within eps files
%\usepackage{psfrag}
% need this for including graphics (\includegraphics)
%\usepackage{graphicx}
% for neatly defining theorems and propositions
%\usepackage{amsthm}
% making logically defined graphics
%%%\usepackage{xypic}

% there are many more packages, add them here as you need them

% define commands here
\begin{document}
We can parameterize the circle by letting $z=z_0 + r e^{i\phi}$.
Then $dz=ir e^{i\phi}d\phi$. Using the Cauchy integral formula we can express $f(z_0)$ in the following way:
\begin{eqnarray*}
f(z_0)&=&\frac{1}{2\pi i} \oint_{C} \frac{f(z)}{z-z_0}dz\\
&=&\frac{1}{2\pi i} \int_{0}^{2\pi} \frac{f(z_0 + r e^{i\phi})}{r e^{i\phi}} ir e^{i\phi} d\phi\\
&=&\frac{1}{2\pi} \int_{0}^{2\pi} f(z_0 + r e^{i\phi}) d\phi .
\end{eqnarray*}

%%%%%
%%%%%
\end{document}
