\documentclass[12pt]{article}
\usepackage{pmmeta}
\pmcanonicalname{LoewnerChain}
\pmcreated{2013-03-22 14:12:02}
\pmmodified{2013-03-22 14:12:02}
\pmowner{jirka}{4157}
\pmmodifier{jirka}{4157}
\pmtitle{Loewner chain}
\pmrecord{4}{35632}
\pmprivacy{1}
\pmauthor{jirka}{4157}
\pmtype{Definition}
\pmcomment{trigger rebuild}
\pmclassification{msc}{30C55}

\endmetadata

% this is the default PlanetMath preamble.  as your knowledge
% of TeX increases, you will probably want to edit this, but
% it should be fine as is for beginners.

% almost certainly you want these
\usepackage{amssymb}
\usepackage{amsmath}
\usepackage{amsfonts}

% used for TeXing text within eps files
%\usepackage{psfrag}
% need this for including graphics (\includegraphics)
%\usepackage{graphicx}
% for neatly defining theorems and propositions
%\usepackage{amsthm}
% making logically defined graphics
%%%\usepackage{xypic}

% there are many more packages, add them here as you need them

% define commands here
\begin{document}
A {\em Loewner chain} is a continuous function $f: {\mathbb{D}} \times [0,\infty)
\rightarrow {\mathbb{C}}$ (where ${\mathbb{D}}$ refers to the open unit disc) with the following properties:
\begin{enumerate}
\item
the function $z \mapsto f(z,t)$ is analytic and univalent for all $t \in [0,\infty)$
\item $f(0,t) = 0$ and $f'(0,t) = e^t$
\item $f({\mathbb{D}},s) \subseteq f({\mathbb{D}},t)$ for $0 \leq s < t < \infty$
\end{enumerate}

{\bf Bibliography}

\begin{itemize}
\item
John B. Conway  \emph{\PMlinkescapetext{Functions of One Complex Variable
II}}.  Springer-Verlag, New York, Inc, 1995
\end{itemize}
%%%%%
%%%%%
\end{document}
