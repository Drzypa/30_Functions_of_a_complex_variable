\documentclass[12pt]{article}
\usepackage{pmmeta}
\pmcanonicalname{PolynomiallyConvexHull}
\pmcreated{2013-03-22 14:21:15}
\pmmodified{2013-03-22 14:21:15}
\pmowner{jirka}{4157}
\pmmodifier{jirka}{4157}
\pmtitle{polynomially convex hull}
\pmrecord{6}{35833}
\pmprivacy{1}
\pmauthor{jirka}{4157}
\pmtype{Definition}
\pmcomment{trigger rebuild}
\pmclassification{msc}{30C10}
\pmclassification{msc}{52A01}
\pmrelated{HolomorphicallyConvex}
\pmdefines{polynomially convex}

\endmetadata

% this is the default PlanetMath preamble.  as your knowledge
% of TeX increases, you will probably want to edit this, but
% it should be fine as is for beginners.

% almost certainly you want these
\usepackage{amssymb}
\usepackage{amsmath}
\usepackage{amsfonts}

% used for TeXing text within eps files
%\usepackage{psfrag}
% need this for including graphics (\includegraphics)
%\usepackage{graphicx}
% for neatly defining theorems and propositions
\usepackage{amsthm}
% making logically defined graphics
%%%\usepackage{xypic}

% there are many more packages, add them here as you need them

% define commands here
\theoremstyle{theorem}
\newtheorem*{thm}{Theorem}
\newtheorem*{lemma}{Lemma}
\newtheorem*{conj}{Conjecture}
\newtheorem*{cor}{Corollary}
\newtheorem*{example}{Example}
\newtheorem*{prop}{Proposition}
\theoremstyle{definition}
\newtheorem*{defn}{Definition}
\begin{document}
\begin{defn}
Let $K \subset {\mathbb{C}}$ be a compact subset of the complex plane,
then the {\em polynomially convex hull} of $K$, denoted $\hat{K}$, is defined as
\begin{equation*}
\hat{K} :=
\{ z \in {\mathbb{C}} : \lvert p(z) \rvert \leq \max_{\zeta \in K} \lvert p(\zeta) \rvert \text{ for all polynomials $p$ } \}.
\end{equation*}
A compact set $K$ is said to be {\em polynomially convex} if
$K = \hat{K}$
\end{defn}

Obviously $K \subset \hat{K}$.  The intuitive idea behind this definition is that the polynomially convex hull of $K$ fills in any ``holes'' that may exist in $K$.
The following proposition makes that precise.

\begin{prop}
If $K \subset {\mathbb{C}}$ is a polynomially convex set, then
all the components of the interior of $K$ are simply connected.
\end{prop}

One of the reasons for this definition is the following result.

\begin{prop}
Let $f$ be a function analytic in an open neighbourhood $N$ of a compact set
$K \subset
{\mathbb{C}}$, and suppose that $f$ can be approximated by polynomials uniformly
on compact subsets of $N$.  Then $f$ can be extended analytically to a neighbourhood of $\hat{K}$.
\end{prop}

For example if we take $K = \{ z \in {\mathbb{C}} : \lvert z \rvert = 1 \}$
(the unit circle)
then $\hat{K} = \{ z \in {\mathbb{C}} : \lvert z \rvert \leq 1 \}$ (the
closed unit disc).  The fact that the inside of the disc belongs to
$\hat{K}$ follows from the maximum modulus principle as polynomials are analytic functions.  The fact that $\hat{K}$ does not contain anything outside the
closed unit disc follows by looking at the polynomial $p(z) = z$ which has
always greater modulus outside of the unit disc then anywhere on the unit circle.  So if we have a function defined on a neighbourhood of the unit circle
and which we can approximate uniformly on compact subsets of this
neighbourhood
by polynomials, then we can extend this function analytically
to the whole unit disc.  So this for example implies that $f(z) := \frac{1}{z}$
cannot be approximated uniformly on compact subsets by polynomials on a
neighbourhood of the unit circle.

The reason why we call $\hat{K}$ a ``hull'' of some \PMlinkescapetext{sort} is that the conventional convex hull of $K \subset {\mathbb{R}}^n$ can be defined as the set of points $x$ such that for all \emph{linear functions} $f \colon {\mathbb{R}}^n\to \mathbb{R}$ we have $\lvert f(x) \rvert \leq \sup_{y\in K} \lvert f(y) \rvert$.  This coincides with conventional definition because if $x$ is not in the conventional convex hull, then there is a linear functional that separates $x$ from the hull (by Hahn-Banach theorem in general, or more elementarily in ${\mathbb{R}}^n$ by Farkas's lemma), and conversely if $x$ is in the convex hull of $K$ then such linear function does not exist for the same reason.  So, intuitively the conventional convex hull is set of point that are inseparable from from $K$ by linear functions.  Polynomially convex hull is the same thing, but with polynomials.
Of course similar definitions can be made with respect to other classes of functions.  For example, hulls with respect to plurisubharmonic functions are very useful in multivariate complex analysis.

\begin{thebibliography}{9}
\bibitem{Conway:complexI}
John~B. Conway.
{\em \PMlinkescapetext{Functions of One Complex Variable I}}.
Springer-Verlag, New York, New York, 1978.
\end{thebibliography}
%%%%%
%%%%%
\end{document}
