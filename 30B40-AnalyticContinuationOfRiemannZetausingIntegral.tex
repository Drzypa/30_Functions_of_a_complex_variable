\documentclass[12pt]{article}
\usepackage{pmmeta}
\pmcanonicalname{AnalyticContinuationOfRiemannZetausingIntegral}
\pmcreated{2013-03-22 16:53:59}
\pmmodified{2013-03-22 16:53:59}
\pmowner{rspuzio}{6075}
\pmmodifier{rspuzio}{6075}
\pmtitle{analytic continuation of Riemann zeta (using integral)}
\pmrecord{21}{39157}
\pmprivacy{1}
\pmauthor{rspuzio}{6075}
\pmtype{Example}
\pmcomment{trigger rebuild}
\pmclassification{msc}{30B40}
\pmclassification{msc}{30A99}
\pmrelated{EstimatingTheoremOfContourIntegral}
\pmrelated{PeriodicityOfExponentialFunction}
\pmrelated{AnalyticContinuationOfRiemannZeta}

\endmetadata

% this is the default PlanetMath preamble.  as your knowledge
% of TeX increases, you will probably want to edit this, but
% it should be fine as is for beginners.

% almost certainly you want these
\usepackage{amssymb}
\usepackage{amsmath}
\usepackage{amsfonts}

% used for TeXing text within eps files
%\usepackage{psfrag}
% need this for including graphics (\includegraphics)
%\usepackage{graphicx}
% for neatly defining theorems and propositions
%\usepackage{amsthm}
% making logically defined graphics
%%%\usepackage{xypic}

% there are many more packages, add them here as you need them

% define commands here

\begin{document}
The Riemann zeta function can be analytically 
continued to the whole complex plane minus the point 1
by means of an integral representation.  Remember that the zeta functon
is defined by the series
\[
\zeta (s) = \sum_{n=1}^\infty {1 \over n^s}.
\]
When $\Re s > 1$, this series converges; furthermore, this convergence is 
uniform on compact subsets of this half-plane, hence the series converges to
an analytic function on this half plane.  However, the series diverges when 
we have $\Re s < 1$, so this series cannot be used to define the zeta function
in the whole complex plane, which is why we must make an analytic continuation.

To make this continuation, we start by changing the variable in an integration:
\[
n^s \int_0^{\infty} e^{-nx} x^{s-1} \, dx =
\int_0^{\infty} e^{-y} y^{s-1} \, dy = 
\Gamma (s)
\]
This provides us with an integral representation of our summand.  
Substituting this into the series, we find that
\[
\zeta (s) = \sum_{n=1}^\infty
{1 \over \Gamma(s)} \int_0^{\infty} e^{-nx} x^{s-1} \, dx =
{1 \over \Gamma(s)}
\sum_{n=1}^\infty \int_0^{\infty}
e^{-nx} x^{s-1} \, dx.
\]
We note that
\[
\sum_{n=1}^\infty \int_0^{\infty}
\left| e^{-nx} x^{s-1} \right| \, dx =
\sum_{n=1}^\infty \int_0^{\infty}
e^{-nx} x^{|s-1|} \, dx =
\sum_{n=1}^\infty {\Gamma (|s - 1| + 1) \over n^s};
\]
because the series converges, hence it is possible
to interchange integration and summation and
subsequently sum a geometric series.
\[
\zeta(s) = {1 \over \Gamma (s)}
\sum_{n=1}^\infty \int_0^{\infty}
e^{-nx} x^{s-1} \, dx =
{1 \over \Gamma (s)}
\int_0^{\infty} \sum_{n=1}^\infty
e^{-nx} x^{s-1} \, dx =
{1 \over \Gamma (s)}
\int_0^{\infty} 
{x^{s-1} \over e^x - 1} \, dx
\]

As it stands, the integral representation we have is not of
much use for analytically continuing the zeta function 
because the integral diverges when $\Re s < 1$ on account of 
the fact that the integrand behaves like $x^{-s}$ when $x$
is close to zero.  However, it is possible to make use of the
theorem of Cauchy to move the path of integration away from
zero.

Given a real number $r > 0$, define the contour $C_r$ on the
Riemann surface of $z^{s-1}$ as follows: $C_r$ passes from
$+\infty$ to $r$ along a lift of the real axis, then continues 
along the circle of radius $r$ clockwise, and finally goes
from $r$ to $+\infty$.  

We now examine the integral over such a contour by breaking it 
into three pieces.
\[
\int_{C_r}
{x^{s-1} \over e^x - 1} \, dx =
\int_r^\infty 
{x^{s-1} \over e^x - 1} \, dx +
\int_{|x| = r}
{x^{s-1} \over e^x - 1} \, dx -
\int_r^\infty 
{(e^{-2 \pi i}x)^{s-1} \over e^x - 1} \, dx.
\]
We may estimate the third integral in absolute value like so:
\[
\left| \int_{|x| = r}
{x^{s-1} \over e^x - 1} \, dx \right| \le
2 \pi  r \sup_{|x| = r}
\left| {x^{s-1} \over e^x - 1} \right|
\]

The expression $x / (e^x - 1)$ represents an analytic function of
$x$, and hence a bounded function of $x$ in a neighborhood of $0$.
When $\Re s > 1$, it happens that $\lim_{x \to 0} |x| x^{s-2} = 0$, so
\[
\lim_{r \to 0} \left| \int_{|x| = r}
{x^{s-1} \over e^x - 1} \, dx \right| = 0.
\]

The third integral differs from the first integral by a phase, so
they may be combined by pulling out this common factor.  When 
$\Re s > 0$, we may take the limit as $r$ approaches $0$ after doing
so to obtain the following:
\[
\lim_{r \to 0}
\int_{C_r}
{x^{s-1} \over e^x - 1} \, dx =
\left( 1 + e^{2 \pi i (1 - s)} \right)
\int_0^\infty 
{x^{s-1} \over e^x - 1} \, dx
\]

Since, aside from the branch point at
$0$, the only singularities of our integrand occur at multiples
of $2 \pi i$, it follows from Cauchy's theorem that
\[
\int_{C_a} 
{x^{s-1} \over e^x - 1} \, dx =
\int_{C_b} 
{x^{s-1} \over e^x - 1} \, dx
\]
whenever $0 < a < 2 \pi$ and $0 < b < 2 \pi$, which trivially implies that
\[
\lim_{r \to 0}
\int_{C_r}
{x^{s-1} \over e^x - 1} \, dx =
\int_{C_r}
{x^{s-1} \over e^x - 1} \, dx
\]
for any $r$ between $0$ and $2 \pi$.  Therefore,
\[
\zeta (s) = 
{1 \over \left( 1 + e^{2 \pi i (1 - s)} \right) \Gamma(s)}
\int_{C_\pi} {x^{s-1} \over e^x - 1} \, dx
\]
when $\Re z > 1$.  This integral converges for all complex $s$ because
the exponential grows more rapidly than the power.  Furthermore, this
integral defines an analytic function of $s$, so we have an analytic
continuation of the zeta function to the whole complex plane minus the point 1.


%%%%%
%%%%%
\end{document}
