\documentclass[12pt]{article}
\usepackage{pmmeta}
\pmcanonicalname{WeierstrassDoubleSeriesTheorem}
\pmcreated{2013-03-22 16:48:15}
\pmmodified{2013-03-22 16:48:15}
\pmowner{pahio}{2872}
\pmmodifier{pahio}{2872}
\pmtitle{Weierstrass double series theorem}
\pmrecord{7}{39038}
\pmprivacy{1}
\pmauthor{pahio}{2872}
\pmtype{Theorem}
\pmcomment{trigger rebuild}
\pmclassification{msc}{30B10}
\pmclassification{msc}{40A05}
\pmclassification{msc}{30D30}
\pmrelated{TheoremsOnComplexFunctionSeries}

\endmetadata

% this is the default PlanetMath preamble.  as your knowledge
% of TeX increases, you will probably want to edit this, but
% it should be fine as is for beginners.

% almost certainly you want these
\usepackage{amssymb}
\usepackage{amsmath}
\usepackage{amsfonts}

% used for TeXing text within eps files
%\usepackage{psfrag}
% need this for including graphics (\includegraphics)
%\usepackage{graphicx}
% for neatly defining theorems and propositions
 \usepackage{amsthm}
% making logically defined graphics
%%%\usepackage{xypic}

% there are many more packages, add them here as you need them

% define commands here

\theoremstyle{definition}
\newtheorem*{thmplain}{Theorem}

\begin{document}
If the complex functions\, $f_0,\,f_1,\,f_2,\,\ldots$\, are holomorphic in the disc\, $\vert z-z_0\vert < r$\, and thus
\begin{align}
  f_n(z) = \sum_{\nu=0}^\infty a_{n\nu}(z-z_0)^\nu, \quad a_{n\nu} = \frac{f_n^{(\nu)}(z_0)}{\nu!}\quad \forall\,n,\,\nu
\end{align}
in this disc, and if the function series
\begin{align}
 \sum_{n=0}^\infty f_n = f_0+f_1+f_2+\ldots
\end{align}
converges uniformly to the function $F$ in each disc\, $|z-z_0| \leqq \varrho$\, where\, $0 < \varrho < r$,\, then also all the series 
\begin{align}
 \sum_{n=0}^\infty a_{n\nu} = a_{0\nu}+a_{1\nu}+a_{2\nu}+\ldots \quad 
(\nu = 0,\,1,\,2,\,\ldots)
\end{align}
converge, and in the disc\, $|z-z_0| < r$\, one has
\begin{align}
 F(z) = \sum_{\nu=0}^\infty A_\nu(z-z_0)^\nu
\end{align}
where the $A_\nu$s are the sums of the series (3).\\

{\em Proof.}\, Apparently, the series (2) converges uniformly also in every closed sub-disc of the open disc \, $|z-z_0| < r$.\, Therefore the theorem 2 in the entry ``\PMlinkname{theorems on complex function series}{TheoremsOnComplexFunctionSeries}'' says that the sum $F(z)$ is holomorphic in\, $|z-z_0| < r$\, and
$$F^{(\nu)}(z) = f_0^{(\nu)}(z_0)+f_1^{(\nu)}(z_0)+f_2^{(\nu)}(z_0)+\ldots
\quad (\nu = 0,\,1,\,2,\,\ldots).$$
Theorem 3 in the same entry thus guarantees that $F(z)$ has the Taylor expansion of the form (4) wherein
$$A_\nu = \frac{1}{\nu!}F^{(\nu)}(z_0) \quad (\nu = 0,\,1,\,2,\,\ldots).$$
According to theorem 2 in the same entry the series (2) may be differentiated termwise,
$$A_\nu = \frac{1}{\nu!}\sum_{n=0}^\infty f_n^{(\nu)}(z_0) = 
\sum_{n=0}^\infty \frac{1}{\nu!}f_n^{(\nu)}(z_0) = \sum_{n=0}^\infty a_{n\nu}$$
Q.E.D.\\

\textbf{Note.}\, In Weierstrass double series theorem it's a question of changing the summing \PMlinkescapetext{order}:
$$
\begin{array}{l}
F(z) = f_0(z)+f_1(z)+\ldots+f_n(z)+\ldots =\\
= [a_{00}+a_{01}(z-z_0)+a_{02}(z-z_0)^2+\ldots+a_{0\nu}(z-z_0)^\nu+\ldots]\\
\,+[a_{10}+a_{11}(z-z_0)+a_{12}(z-z_0)^2+\ldots+a_{1\nu}(z-z_0)^\nu+\ldots]\\
\,+[a_{20}+a_{21}(z-z_0)+a_{22}(z-z_0)^2+\ldots+a_{2\nu}(z-z_0)^\nu+\ldots]\\
\qquad \qquad \ldots\ldots\\
\,+[a_{n0}+a_{n1}(z-z_0)+a_{n2}(z-z_0)^2+\ldots+a_{n\nu}(z-z_0)^\nu+\ldots]\\
\underline{\qquad\qquad\cdots\cdots\qquad\qquad\qquad\qquad\qquad\qquad\qquad\qquad\qquad\qquad}\\
= A_0+A_1(z-z_0)+A_2(z-z_0)^2+\ldots+A_\nu(z-z_0)^\nu+\ldots
\end{array}
$$



%%%%%
%%%%%
\end{document}
