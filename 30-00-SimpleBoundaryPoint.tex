\documentclass[12pt]{article}
\usepackage{pmmeta}
\pmcanonicalname{SimpleBoundaryPoint}
\pmcreated{2013-03-22 14:23:23}
\pmmodified{2013-03-22 14:23:23}
\pmowner{jirka}{4157}
\pmmodifier{jirka}{4157}
\pmtitle{simple boundary point}
\pmrecord{5}{35885}
\pmprivacy{1}
\pmauthor{jirka}{4157}
\pmtype{Definition}
\pmcomment{trigger rebuild}
\pmclassification{msc}{30-00}
\pmclassification{msc}{54-00}

\endmetadata

% this is the default PlanetMath preamble.  as your knowledge
% of TeX increases, you will probably want to edit this, but
% it should be fine as is for beginners.

% almost certainly you want these
\usepackage{amssymb}
\usepackage{amsmath}
\usepackage{amsfonts}

% used for TeXing text within eps files
%\usepackage{psfrag}
% need this for including graphics (\includegraphics)
%\usepackage{graphicx}
% for neatly defining theorems and propositions
\usepackage{amsthm}
% making logically defined graphics
%%%\usepackage{xypic}

% there are many more packages, add them here as you need them

% define commands here
\theoremstyle{theorem}
\newtheorem*{thm}{Theorem}
\newtheorem*{lemma}{Lemma}
\newtheorem*{conj}{Conjecture}
\newtheorem*{cor}{Corollary}
\newtheorem*{example}{Example}
\newtheorem*{prop}{Proposition}
\theoremstyle{definition}
\newtheorem*{defn}{Definition}
\begin{document}
\begin{defn}
Let $G \subset {\mathbb{C}}$ be a region and $\omega \in \partial G$ (the boundary of $G$).  Then
we call $\omega$ a {\em simple boundary point} if whenever
$\{ \omega_n \} \subset G$ is a sequence converging to $\omega$ there is a path
$\gamma \colon [0,1] \to {\mathbb{C}}$ such that
$\gamma(t) \in G$ for $0 \leq t < 1$, $\gamma(1) = \omega$ and there is a sequence
$\{ t_n \} \in [0,1)$ such that $t_n \to 1$ and $\gamma(t_n) = \omega_n$ for all
$n$.
\end{defn}

For example if we let $G$ be the open unit disc, then every boundary point is a simple boundary point.  This definition is useful for studying boundary behaviour of Riemann maps (maps arising from the Riemann mapping theorem), and one can prove for example the following theorem.

\begin{thm}
Suppose that $G \subset {\mathbb{C}}$ is a bounded simply connected region such
that every point in the boundary of $G$ is a simple boundary point, then $\partial G$ is a Jordan curve.
\end{thm}

\begin{thebibliography}{9}
\bibitem{Conway:complexII}
John~B. Conway.
{\em \PMlinkescapetext{Functions of One Complex Variable II}}.
Springer-Verlag, New York, New York, 1995.
\end{thebibliography}
%%%%%
%%%%%
\end{document}
